   \documentclass[a4paper,french,bookmarks]{article}
\usepackage{./Structure/4PE18TEXTB}

\newboxans

\begin{document}

    \stylizeDoc{Physique}{Compte-rendu du TP 25}{Mesures de champs magnétiques}

    \boxhtp{
        \itstar L'objectif de ce TP est de réaliser quelques mesures de champs magnétiques : champ magnétique créé par un solénoïde, champ magnétique créé par une ou plusieurs spires, champ magnétique terrestre.
    }{
        \itstar Bobines, teslamètre, \dots.
    }
   
    \begin{warning}{}{}
        Dans tout ce TP, on sera amené à utiliser des courants relativement importants. Pour éviter d'endommager le matériel, on prendra soin de \hg{ne jamais dépasser les valeurs nominales} de courant indiquées sur les bobines. On réglera toujours la tension et l'intensité des générateurs de courant à \hg{zéro} avant de les \hg{connecter} à une bobine. De même, on réglera toujours la tension et l'intensité des générateurs de courant à \hg{zéro} avant de les \hg{déconnecter}, afin d'éviter la surtension qui se produit dans un circuit inductif lorsqu'on tente de faire varier brusquement l'intensité.
    \end{warning}
    
    \section{Réglage du matérel - s'y reporter en cours de TP}
    
    Les champs magnétiques seront mesurés avec un teslamètre, appareil de mesure de champs magnétiques dont la sonde est une sonde de \textsc{Hall}. La présence de champs magnétiques parasites (champ magnétique terrestre, aimant posé non loin, armature en fer dans la paillasse, etc.) peut perturber les mesures de champ magnétique. Le comportement de la sonde de \textsc{Hall} dépend également de la température.
    
    \subsection{Le teslamètre}
    
    \noafter
    %
    \boxans{
        \begin{experience}{Réglage du teslamètre}{}
            \begin{enumerate}
                \ithand Placer le teslamètre au centre de la zone de mesure, les appareils étant éteints.
                
                \ithand Régler le zéro (bouton gravé d'une flèche à côté de l'écran).
            \end{enumerate}
        \end{experience}
    }
    %
    \nobefore\yesafter
    %
    \begin{expcom}
        Effectué.
    \end{expcom}
    %
    \yesbefore
    
    \subsection{Les générateurs de courant}\label{subsec:1.2}
    
    Les générateurs de courant qu'on utilise sont des générateurs limités à la fois en courant et en tension. Leur point de fonctionnement est tel que ni la tension délivrée ni le courant débité ne dépassent les valeurs demandées.\medskip
    
    Puisqu'on souhaite alimenter les bobines en contrôlant le courant, la bonne manière de les régler est la suivante :
    
    \noafter
    %
    \boxans{
        \begin{experience}{Réglage d'un générateur de courant}{}
            \begin{enumerate}
                \ithand Allumer le générateur et positionner les boutons de réglage de la tension et du courant à zéro.
                
                \ithand Connecter le générateur à la bobine.
                
                \ithand Régler la tension au maximum en maintenant l'intensité à zéro.
                
                \ithand Augmenter ensuite l'intensité à la valeur souhaitée : le générateur fonctionne en mode source de courant variable.
            \end{enumerate}
        \end{experience}
    }
    %
    \nobefore\yesafter
    %
    \begin{expcom}
        Effectué.
    \end{expcom}
    %
    \yesbefore
    
    \section{Étude du champ magnétique sur l'axe d'un solénoïde}
    
    On dispose d'un solénoïde de longueur $L = \SI{40}{\cm}$, de rayon $\alpha = \SI{2.5}{\cm}$, constitué de deux enroulements indépendants :
    %
    \begin{enumerate}
        \itt Un enroulement de $200$ spires entre les deux bornes noires marquées chacune \guill{$100$}.
        
        \itt Un enroulement de $200$ spires auquel sont connectées les bornes rouges. On peut se connecter sur cet enroulement entre les deux bornes extrémales (marquées $100$ et $100$), pour faire circuler un courant dans $200$ spires. On peut aussi connecter à d'autres bornes (marquées $5$, $10$, $20$, $30$, $50$, ou $70$) pour faire circuler du courant dans une fraction des spires (par exemple $140$ spires si l'on se connecte aux bornes rouges marquées $70$ et $70$).
    \end{enumerate}
    
    \noafter
    %
    \boxans{
        \begin{experience}{}{}
            \begin{enumerate}
                \ithand En procédant comme expliqué au \ref{subsec:1.2}, alimenter le solénoïde avec un courant d'intensité de $I = \SI{1}{\ampere}$.
                
                \ithand Placer une aiguille aimantée au bord du solénoïde, en déduire le sens du champ créé par le solénoïde puis le sens du courant parcourant les spires.
                
                \ithand Relever les valeurs du champ magnétique le long de l'axe du solénoïde et représenter sur un graphe l'intensité du champ magnétique en fonction de la position dans le solénoïde.
            \end{enumerate}
        \end{experience}
    }
    %
    \nobefore\yesafter
    %
    \begin{expcom}\text{}\\
        \begin{enumerate}
            \itt On alimente en suivant le protocole expliqué au \ref{subsec:1.2}.
            
            \itt Face à la bobine, on branche la borne + à droite et la borne - à gauche. D'après les aiguilles, le champ créé \guill{sort} de la bobine par la gauche et y \guill{rentre} par la droite.
            
            \itt Graphique : TODO.
        \end{enumerate}
    \end{expcom}
    %
    \yesbefore
    
    On rappelle que l'intensité du champ magnétique dans un solénoïde infini, comportant n spires par unité de longueur et parcouru par un courant d'intensité $I$ vaut $B = \mu_0 n I$.
    
    \begin{enumerate}
        \item Le champ au centre du solénoïde peut-il être assimilé au champ créé par un solénoïde infini ? À quelle précision ?
        
        \boxans{
            Comme on le voit ci-dessus, le champ au centre d'un solénoïde est relativement \guill{plat} : on peut l'assimiler à une ligne droite. Par ailleurs,le champ créé par un solénoïde infini est constitué de lignes droites. Pour la région \guill{au centre} du solénoïde, on peut donc assimiler le champ à celui créé par un solénoïde infini.
        }
    \end{enumerate}
    
    On cherche à montrer que le champ au centre du solénoïde est bien proportionnel à l'intensité du courant $I$.

        \noafter
    %
    \boxans{
        \begin{experience}{}{}
            \begin{enumerate}
                \ithand Faire varier l'intensité $I$ entre $\SI{0}{\ampere}$ et $\SI{1}{\ampere}$ et relever à chaque fois la valeur du champ.
                
                \ithand Utiliser une régression linéaire pour déterminer une estimation de $μ0n$ ; le script est disponible sur l'ENT via le lien \texttt{https://capytale2.ac-paris.fr/web/c/7186-611329}.
            \end{enumerate}
        \end{experience}
    }
    %
    \nobefore\yesafter
    %
    \begin{expcom}\text{}\\
        \begin{enumerate}
            \itt On alimente en suivant le protocole expliqué au \ref{subsec:1.2}.
            
            \itt On obtient bien une fonction linéaire. Graphique : TODO.
        \end{enumerate}
    \end{expcom}
    %
    \yesbefore
    
    \section{Étude du dispositif des bobines de Helmholtz}
    
    On dispose d’un dispositif constitué de deux bobines circulaires plates, coaxiales, de rayon $a = \SI{6.5}{\cm}$ constitués chacune de $N = 95$ spires.
    
\end{document}