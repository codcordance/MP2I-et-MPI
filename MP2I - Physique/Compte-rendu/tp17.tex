\documentclass[a4paper,french,bookmarks]{article}
\usepackage{./Structure/4PE18TEXTB}
    
\newboxans 

\begin{document}
\stylizeDoc{Physique}{Compte-rendu du TP 17}{Étude du régime sinusoïdal forcé dans un RLC}

\boxhtp{
    \itstar Mettre en œuvre un dispositif expérimental autour du phénomène de résonance.
}{
    \itstar un GBF, un oscilloscope, boîte à décades de résistances, de capacités, une bobine.
}

\text{}\newline
On étudie dans ce TP deux types de résonance d’un circuit RLC : d’abord en intensité, puis en tension.

\section{Résonance de type intensité}

\noafter
%
\boxans{
    \begin{experience}{Mise en place du circuit}{}
        \begin{enumerate}
        \ithand Monter un circuit RLC série en régime sinusoïdal forcé, alimenté par un GBF d’amplitude $\SI{3,5}{\volt}$.
        
        \ithand Prendre $R = \SI{1}{k\Omega}$, la bobine mise à disposition sur la paillasse, et $C = \SI{10}{nF}$.
        
        \ithand Visualiser en \texttt{Voie 1} le signal délivré par le GBF, et en \texttt{Voie 2} l’allure de l’intensité du courant dans le circuit.
        
        Vérifier que les deux signaux sont \textit{synchrones} mais pas forcément en     \textit{phase}.
        \end{enumerate}
    \end{experience}
}
%
\yesafter\nobefore
%
\begin{expcom}
    On mesure l'intensité à l'aide de la résistance. En effet, la loi d'Ohm livre :
    
    \[ u_R(t) = R\cdot i(t) \qquad \text{donc} \qquad i(t) = \dfrac{1}{R}\cdot u_R(t) \qquad \text{soit} \qquad i(t) \propto u_R(t)\]
    
    On obtient ainsi un signal proportionnel que l'on peut redimensionner sur l'oscilloscope. 
    
    On observe bien que les deux signaux sont \textit{synchrones} mais pas forcément en \textit{phase}.
\end{expcom}
%
\yesbefore

\begin{enumerate}
    \item Faire un schéma du montage en indiquant clairement la masse du circuit.
    
    \boxans{
        \begin{center}
            \begin{circuitikz}
                \draw (0, 0) 
                    to[vsourcesin, v=\SI{3,5}{\volt}] ++(0, 2.4) coordinate (m0)
                    to[L, l=$L$] ++(2.5,0)
                    to[C, l=$C$] ++(2.5,0) coordinate (m1)
                    to[R, v=$u_R(t)$, l=$R \eq \SI{1}{k\Omega}$] ++(0,-2.4)
                    to[short, i=$i$] ++(-5,0) node[eground, rotate = -45]{}
                    to[short, -o] ++(0,0);
                --(0,0);
                
                \draw[->] (m0) to[short, o-]  ++(-1,1) node[label={west:\texttt{Voie 1}}]{};
                
                \draw[->] (m1) to[short, o-]  ++(1,1) node[label={east:\texttt{Voie 2}}]{};
            \end{circuitikz}
        \end{center}
    }
    
    \item Rappeler (sans refaire le calcul pendant la séance) la fréquence théorique 
    $f_0$ pour laquelle l’intensité est maximale. Faire l’application numérique pour les composants choisis dans le TP.
    
    \boxans{
        Pour une résonance de type intensité, l'intensité est maximale lorsque $\omega = \omega_0$. 
        
        Or $\omega_0 = \dfrac{1}{\sqrt{LC}}$ pour un circuit RLC, et de plus $\omega = 2\pi f$ donc $2\pi f_0 = \dfrac{1}{\sqrt{LC}}$, donc \boxsol{$f_0 = \dfrac{1}{2\pi\sqrt{LC}}$}.
        
        L'application numérique donne \boxsol{$f_0 = \SI{7605}{Hz}$}.
    }
\end{enumerate}

\noafter
%
\boxans{
    \begin{experience}{Premières observations}{}
        \begin{enumerate}
            \ithand Faire varier la fréquence f du signal délivré par le GBF et observer l’allure des deux voies. Faire des schémas (qualitatifs) des signaux observés pour plusieurs fréquences.
        \end{enumerate}
    \end{experience}
}
%
\nobefore\yesafter
%
\begin{expcom}
    \pgfplotsset{width=8cm}
    \begin{minipage}{0.5\linewidth}
        \begin{center}
            \underline{$f = f_0$}
        \end{center}
        \begin{tikzpicture}
        \begin{axis}[
            axis lines = middle,
            xlabel=$t$,
            ylabel=$u(t)$,
            domain=0:6,
            xmin=0,
            xmax=6,
            ymin=-6,
            ymax=6,
            trig format plots=rad,
            xticklabels={,,},
            yticklabels={,,},
            font=\footnotesize,
                grid = both,
                grid style = {line width = .1pt, draw = gray!30},
                major grid style = {line width=.2pt,draw=gray!50},
        ]
            \addplot[samples=500,color=colexp, line width=0.6mm] {5*cos(3*x)};
            \addlegendentry{$u_G(t)$}
            \addplot[samples=500,color=main1, line width=0.6mm] {4.5*cos(3*x)};
            \addlegendentry{$i(t)$}
        \end{axis}
        \end{tikzpicture}
    \end{minipage}
    \begin{minipage}{0.5\linewidth}
        \begin{center}
            \underline{$f = \sfrac{f_0}{2}$ et $f=2f_0$}
        \end{center}
        \begin{tikzpicture}
        \begin{axis}[
            axis lines = middle,
            xlabel=$t$,
            ylabel=$u(t)$,
            domain=0:6,
            xmin=0,
            xmax=6,
            ymin=-6,
            ymax=6,
            trig format plots=rad,
            xticklabels={,,},
            yticklabels={,,},
            font=\footnotesize,
                grid = both,
                grid style = {line width = .1pt, draw = gray!30},
                major grid style = {line width=.2pt,draw=gray!50},
        ]
            \addplot[samples=500,color=colexp, line width=0.6mm] {5*cos(3*x)};
            \addlegendentry{$u_G(t)$}
            \addplot[samples=500,color=main1, line width=0.6mm] {3*cos(3*x+pi)};
            \addlegendentry{$i(t)$}
        \end{axis}
        \end{tikzpicture}
    \end{minipage}
\end{expcom}
%
\yesbefore

\subsection{Amplitude du signal de sortie}

\noafter
%
\boxans{
    \begin{experience}{Détermination de l’amplitude de l’intensité}{}
    \begin{enumerate}
        \ithand Relever la valeur de l’amplitude $I_m$ de l’intensité $i(t)$ pour différentes fréquences, en prenant plus de points autour de la résonance. \bf{Attention}, l’amplitude du signal généré par le GBF ne doit pas varier.
        
        \ithand  Tracer $I_m$ en fonction de la fréquence $f$.
    \end{enumerate}
    \end{experience}
}
%
\nobefore\yesafter
%
\begin{expcom}

    \begin{center}
         \begin{tabular}{|>{\centering\columncolor{colexp!20}}p{0.1\linewidth}|c!{\color{colexp!50!gray}\vrule}c!{\color{colexp!50!gray}\vrule}c!{\color{colexp!50!gray}\vrule}c!{\color{colexp!50!gray}\vrule}c!{\color{colexp!50!gray}\vrule}c!{\color{colexp!50!gray}\vrule}c!{\color{colexp!50!gray}\vrule}c!{\color{colexp!50!gray}\vrule}c|}\hline
        \rowcolor{colexp!20} $f$ & $\sfrac{f_0}{4}$ & $\vphantom{\dfrac{1}{2}}\sfrac{f_0}{3}$ & $\sfrac{f_0}{2}$ & $\sfrac{2f_0}{3}$ & $f_0$ & $\sfrac{3f_0}{2}$ & $2f_0$ & $3f_0$ & $4f_0$\\\hline
        $I_m$ & $\SI{700}{mV}$ & $\SI{900}{mV}$ & $\SI{1.3}{V}$ & $\SI{1.9}{V}$ & $\SI{3.2}{V}$ & $\SI{1.8}{V}$ &$\vphantom{\dfrac{1}{2}}\SI{1.2}{V}$ & $\SI{800}{mV}$ & $\SI{600}{mV}$\\\hline
    \end{tabular}
    
    \begin{tikzpicture}
        \begin{axis}[
            axis lines = middle,
            xlabel=$f$,
            ylabel=$I_m$,
            domain=0:6,
            xmin=0,
            xmax=3,
            ymin=0,
            ymax=4,
            trig format plots=rad,
            xticklabels={,,},
            yticklabels={,,},
            font=\footnotesize,
                grid = both,
                grid style = {line width = .1pt, draw = gray!30},
                major grid style = {line width=.2pt,draw=gray!50},
        ]
            \addplot[samples=700,color=colexp, line width=0.6mm] {3.05/sqrt(1+214*(x-1/x)^2)};
            \addlegendentry{$I_m(f)$}
        \end{axis}
    \end{tikzpicture}
    \end{center}
\end{expcom}
%
\yesbefore

\begin{enumerate}[resume]
    \item Déduire du graphe précédent la fréquence $f_0$ de résonance en intensité du circuit RLC, et comparer avec la valeur attendue par rapport aux caractéristiques des composants. On prendra pour $L$ la valeur indiquée sur la bobine.
    
    \boxans{
        On retrouve bien le $f_0$ calculé précédemment. 
    }
\end{enumerate}

\subsection{Facteur de qualité et acuité}

\begin{enumerate}[resume]
    \item Rappeler la définition de la bande passante $\Delta \omega$ et déterminer sa valeur à partir de la courbe $I_m(f)$.
    
    \boxans{
        La bande passante est définie par l'ensemble des $f$ tels que \boxsol{$I_m(x) \geq \dfrac{I_{m,\ max}}{\sqrt{2}}$}.
        
        On trouve alors \boxsol{$\Delta \omega = \SI{0,683}{s^{-1}}$}.
    }
    
    \item  Après avoir estimé toutes les résistances dans le circuit, déterminer la valeur attendue du facteur de qualité
    
    \begin{equation}
        Q = \dfrac{1}{R_{tot}}\sqrt{\dfrac{L}{C}}
    \end{equation}
    
    où l’on estimera $L$ à partir de $C$ et de la fréquence $f_0$ déterminée plus haut. Donner l’incertitude sur cette valeur.
    
    \boxans{
        Manque de temps.
    }
    
    \item On rappelle par ailleurs que
    
    \begin{equation}
        Q = \dfrac{\omega_0}{\Delta \omega}
    \end{equation}
    
    Vérifier cette relation avec les valeurs obtenues aux deux questions précédentes. On déterminera précisément les incertitudes sur les paramètres avant de conclure.
    
    \boxans{
        Manque de temps.
    }
\end{enumerate}

\subsection{Étude de la phase}

\noafter
%
\boxans{
    \begin{experience}{Influence de la phase}{}
    \begin{enumerate}
        \ithand Vérifier que la valeur du déphasage est conforme à ce qui a été vu en cours pour la résonance en intensité.
        
        On s’intéressera en particulier aux cas $f \ll f_0$, $f = f_0$ et $f \gg f_0$.
    \end{enumerate}
    \end{experience}
}
%
\nobefore\yesafter
%
\begin{expcom}
    On retrouve bien que pour des valeurs de $f$ éloignées de $f_0$ (donc $f \ll f_0$ et $f \gg f_0$), le déphasage tends vers $\dfrac{\pi}{2}$.
\end{expcom}
%
\yesbefore

\section{Résonance de type tension}

On s’intéresse maintenant à la résonance en tension aux bornes du condensateur.

\begin{enumerate}
    \item Faut-il modifier le montage précédent ? Pourquoi ?
    
    \boxans{
        Manque de temps.
    }
    
    \item Reprendre le protocole précédent afin d’étudier le phénomène de résonance de type tension/élongation.
    
    \boxans{
        Manque de temps.
    }
\end{enumerate}
\end{document}