\documentclass[a4paper,french,bookmarks]{article}
\usepackage{./Structure/4PE18TEXTB}
    
\begin{document}
\stylizeDoc{Physique}{Devoir Maison 5}{Diode à effet tunnel}

\section{Équations impression}

\subsection{Cas apériodique}
\[ \resizebox{1\hsize}{!}{$
                u_C(t) = \dfrac{\frac{E_0\frac{x}{Q}}{(1-x^2)^2 + (\frac{x}{Q})^2} + \omega_0^2\tau\left(E_0 + \frac{E_0(1-x^2)}{(1-x^2)^2 + (\frac{x}{Q})^2}\right)}{\frac{1}{\tau} - \omega_0^2\tau}e^{-\frac{t}{\tau}} - \left(\dfrac{\frac{E_0\frac{x}{Q}}{(1-x^2)^2 + (\frac{x}{Q})^2} + \omega_0^2\tau\left(E_0 + \frac{E_0(1-x^2)}{(1-x^2)^2 + (\frac{x}{Q})^2}\right)}{\frac{1}{\tau} - \omega_0^2\tau}+ E_0 + \dfrac{E_0(1-x^2)}{(1-x^2)^2 + (\frac{x}{Q})^2} \right)e^{-\omega_0^2\tau t} + \dfrac{E_0}{(1-x^2)^2 + (\frac{x}{Q})^2}\left((1-x^2)\cos{\omega t} + \dfrac{x}{Q}\sin{\omega t}\right) + E_0
            $}\]

\subsection{Cas critique}
\[ \resizebox{\hsize}{!}{$
                u_C(t) = \left(t\dfrac{\frac{E_0(1-x^2)}{(1-x^2)^2 + (\frac{x}{Q})^2} + \alpha \frac{E_0\frac{x}{Q}}{(1-x^2)^2 + (\frac{x}{Q})^2} + \alpha E_0}{\alpha -1}-\dfrac{E_0(1-x^2)}{(1-x^2)^2 + (\frac{x}{Q})^2} - E_0\right)e^{-\alpha t} + \dfrac{E_0}{(1-x^2)^2 + (\frac{x}{Q})^2}\left((1-x^2)\cos{\omega t} + \dfrac{x}{Q}\sin{\omega t}\right) + E_0
            $}\]
            
\subsection{Cas pseudo-périodique}
\[ \resizebox{\hsize}{!}{$u_C(t) = -\left(\dfrac{E_0(1-x^2)}{(1-x^2)^2 + (\frac{x}{Q})^2} + E_0\right)\cos(\omega t) -\dfrac{E_0\frac{x}{\omega Q}}{(1-x^2)^2 + (\frac{x}{Q})^2}\sin(\omega t)
                + \dfrac{E_0}{(1-x^2)^2 + (\frac{x}{Q})^2}\left((1-x^2)\cos{\omega t} + \dfrac{x}{Q}\sin{\omega t}\right) + E_0
            $} \]

\end{document}