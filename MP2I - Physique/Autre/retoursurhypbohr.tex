\documentclass[a4paper,french,bookmarks]{article}
\usepackage{./Structure/4PE18TEXTB}

\newboxans

\begin{document}

\stylizeDoc{Physique}{Autre}{Retour sur l'hypothèse de Bohr}

\section*{\centering\EBGaramond\itshape\Large Retour sur l'hypothèse de Bohr}
    
\begin{minipage}{0.48\linewidth}

\begin{form}{Hypothèse de Bohr}{}{}
    \begin{center}
        \begin{tikzpicture}
            \draw[main3, dashed] circle (1.5);

            \draw[main2, fill=main2!10] circle (0.1) node[right=3pt] {\textit{\EBGaramond proton}};
    
            \draw[main7, ->] (0, 1.5)  --node[midway, above] {$\vec{u_\theta}$} (-1, 1.5);
        
            \draw[main7, ->] (0, 1.5)  --node[midway, right] {$\vec{u_r}$} (0, 2.5);
    
            \draw[main5, fill=main2!10] (0, 1.5) circle (0.1) node[below=3pt] {\textit{\EBGaramond électron}};

        \end{tikzpicture}
    \end{center}
    L'hyptohèse de Bohr est la suivante :
    %
    \[ \boxedcol{\hg{\norm{\vec{\sigma_O}} = n\hbar}}\]
    %
    Ce qui amène $\left\lbrace\begin{array}{rl}
        R_n &= R_0n^2  \\
        E_n &= - \dfrac{E_0}{n^2} 
        \end{array}\right.$
\end{form}
\end{minipage}
%
\hfill
%
\begin{minipage}{0.48\linewidth}
    \begin{align*}
        &&\vec{\sigma_0} &= m_\text{e} \vec{OM} \wedge \vec{v}\\
        \text{donc} && &= mR\vec{u_R} \wedge R\dot \theta \vec{u_\theta}\\
        \text{donc} &&     &= m\underbrace{R\dot\theta}_{v} \vec{u_z}\\
    \text{donc} && n \hbar &= mRv = Rp\\
    \text{soit} && n\dfrac{h}{2\pi} &= Rp\\
    \text{donc} && n\dfrac{h}{p} &= 2\pi R\\
    \text{soit} && n\lambda_\text{dB} &= 2\pi R
    \end{align*}
    
    Ainsi, l'onde associée à $e^-$ doit être en phase avec elle-même au bout d'un tour.
\end{minipage}

\end{document}