\documentclass[a4paper,french,bookmarks]{article}
\usepackage{./Structure/4PE18TEXTB}
\usepackage{marginnote}

\begin{document}

\stylizeDoc{Francais}{Résumé n° 2}{A rendre pour le 17/02/2022}

\newcommand{\wordcount}[1]{\reversemarginpar\marginnote{\color{white5}#1}\normalmarginpar}

{
\section*{\centering\EBGaramond\Large Résumé n° 2}

\reversemarginpar
%{\fontfamily{lmr}\selectfont
\wordcount{0} L'entreprise cartésienne, quête de vérité, identifie premièrement ce qui la contrarie : nos préjugés, illusions de la connaître \wordcount{17}déjà. %(20)
Mais pourquoi est-ce si difficile de s'en séparer ? Parce-que \guill{nous avons tous été enfant avant que d'être \wordcount{38}homme}.\\ % (21 -> 41

\wordcount{40} En effet l'enfant, égocentrique, fait prévaloir sa survie, ignore la vérité pour retenir une idée utile et se fie à ses sens. %(23 -> 64)
Naissant sans pouvoir utiliser sa \wordcount{63}raison, et devant se fier aux autres, il reçoit alors les préjugés de ses précepteurs lors de son éducation et devient\wordcount{83} inapte à penser par lui-même.\\ %(31 -> 95)

\wordcount{95} L'enfance est donc pour divers motifs le siège de nos préjugés. %(12 -> 107)
Ainsi, contre Platon accédant aux idées par la maïeutique, Descartes \wordcount{116}préconise une rupture avec l'enfant en nous, un infanticide. Cette rupture avec l'enfance s'opère chez Descartes par le doute radical de toutes nos idées. %(37 -> 144)
\wordcount{134}Mais l'enfant ne se détache-t-il pas ainsi de l'enfance, pris de désillusion pour le monde qui \wordcount{157}l'entoure ?\\[10pt] %(21 -> 165)
%}

\hfill{\EBGaramond \textit{\color{white5}165 mots.}}

}

\section*{\centering\EBGaramond\Large Correction}

Relecture du texte en début de cours. 

\begin{enumerate}
    \ithand Thème : Enfance et préjugés.
    
    \ithand Thèse : Pour se délivrer des préjuger apparus avec l'enfance, il faut apprendre à douter.
    
    \ithand Mots clés : préjugés, doute, vérité, enfance, éducation, égocentré, Descartes
\end{enumerate}

\underline{Schéma argumentatif simple :}

\begin{enumerate}
    \ithand Découpage : premier et deuxième paragraphe regroupés en une première partie, troisième et quatrième partie regroupées en une deuxième, et dernier paragraphe en une troisième partie. On a donc un résumé en trois parties.
    
    \ithand Équilibre des parties : La première partie fait 26\% du texte original donc environ 37 mots, la deuxième fait 50\% du texte original donc environ 75 mots, et la dernière fait 24\%, donc 37 mots.
\end{enumerate}

\begin{warning}{}{}
    Ne \hg{pas reproduire les citations} !!
\end{warning}

\underline{Schéma argumentatif détaillé :}

\begin{enumerate}
    \item[$\boxed{\textbf{\makebox[0pt][c]{\centering I.}\phantom{IIII}}}$] \begin{enumerate}
            \ithand Decscartes recherche la vérité
            \ithand Les préjugés sont un obstacle
            \ithand Ils naissent avec l'enfance
            \ithand L'enfant pense d'abord à survivre qu'à chercher la vérité
        \end{enumerate}
        
        \item[$\boxed{\textbf{\makebox[0pt][l]{\centering II.}\phantom{IIII}}}$] \begin{enumerate}
            \ithand L'enfant se fie exagérément à ses sens et rapporte tout à lui, ce qui forge ses préjugés.
            \ithand 
        \end{enumerate}
\end{enumerate}

\end{document}