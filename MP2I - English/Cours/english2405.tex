\documentclass[a4paper,english,bookmarks]{article}
\usepackage{./Structure/4PE18TEXTB}
\usepackage{marginnote}

\begin{document}

\stylizeDoc{English}{Lesson Notetaking}{Tuesday 10 May 2022}

\newboxans

\section{Amanda Gorman}

\subsection{Poem VIdeo}

Inoguration poem. Idea : Two is a couple

\subsection{Controversy}

\subsection{No Go Zone}

\begin{itemize}
    \item Don't act, don't imagine that you're someone fictional. You can ither adopt the neutral stance of The Economist or you cn use the "I" pronoun ("I believe \dots", \dots).
    
    \item Don't take for granted the fact that the jry has read the article. You need to indeitfy the event and the opinion of the editorialist.
    
    \item Absence of opinion is something not to do ! You must express your points forcefully. Don't hold your punche. Don't be rude, but be formal.
    
    \item Don't write a single pargraph, many paragraphs offer better reading experience.
    
    \item Do not write a "dissert" on the subject. You need to adress a specific elements from the article (content, tone, structure).
    
    \item Headline ? Il n'y a rien là-dessus dans le rapport, donc comme vous voulez.
\end{itemize}

\subsection{Example}

\textbf{Good translation matters}\medskip

The culture wars are raging. Increasingly, both conservatives and liberals are apt to consider all cultural productions as either friends or foes. Even such a bland and cautious instution as Disney fails to remain above the fray - witness governor DeSantis's move to strip it of its exceptional tax status, for alleged collusion with the "woke" foe.\\

It is no suprise, then, that \textit{The Daily Wire}'s Ian Haworth should find fault with the decision that only "a woman, young, activist and preferably black" sjould translate Amanda Gorman's "The Hill We Climb". Rather plausibly, he stresses that the very premise of translation is that one person's experience can be understood by someone from another culture. Hence (and here is where Haworth and I part ways), it seems to him preposturous to insist that the translator should have an experience that is as close as possible to the original writer's : this, to Haworth, is both racism and a failure to understand the power of translation.\\

Such a stance seems to me somewhat blinkered.


\end{document}