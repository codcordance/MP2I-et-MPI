\documentclass[a4paper,french,bookmarks]{article}
\usepackage{./Structure/4PE18TEXTB}

\begin{document}
\stylizeDoc{Mathématiques}{Chapitre 21}{Matrices et applications linéaires}

Les chapitres précédents d'algèbre linéaire nous ont permis d'introduire
la structure algébrique d'espace vectoriel et d'y développer une théorie
de la dimension. On s'intéresse ici plus particulièrement aux
homomorphismes entre ces structures, les \textit{applications linéaires}. Muni de la notion de dimension, on s'intéressera à leur lien avec les matrices et les bases. En effet, bien que la notion de \textit{matrice} remonte 

\initcours{}

\subsection{Matrice d'un vecteur}

On peut tout d'abord considérer un corps commutatif $\bdK$, et $E$ un $\bdK$-espace vectoriel qu'on prendra de dimension $n \in \bdN^*$ finie. On considérera de plus une base $\bcB = \left(e_1, e_2, \dots, e_n\right)$ de $E$. On a montré dans un chapitre précédent que :
%
\[ \forall x \in E,\qquad \exists ! \left(x_i\right)_{i \in \left\llbracket1, n\right\rrbracket} \in \bdK^n,\qquad x = \sum_{i=1}^n x_i e_i\]
%
Cette famille de scalaires $\left(x_1, x_2, \dots, x_n\right)$ sont les \textit{coordonnées} du vecteur $x$ de $E$ dans la base $\bcB$. On peut alors considérer la matrice colonne faite de ces coordonnées

\begin{definition}{Matrice colonne des coordonnées d'un vecteur}{}
    Soient $E$ un $\bdK$-espace vectoriel de dimension finie $n \in
    \bdN^*$ et de base $\bcB$. Pour tout vecteur $x \in E$, on appelle
    \hg{matrice colonne des coordonnées du vecteur $x$ dans la base $\bcB$} la \hg{matrice colonne à $n$ lignes de $\bcM_{n, 1}\left(\bdK\right)$} dont les coefficients sont de haut en bas \hg{les coordonnées de $x$ dans la base $\bcB$}.
\end{definition}

Cette matrice est généralement notée $\Matc_\bcB\left(x\right)$ ou encore $\Coord_\bcB\left(x\right)$. Dans le cas du vecteur $x$ ci-dessous, on a alors :
%
\[ \Matc_\bcB\left(x\right) = \Coord_\bcB\left(x\right) = \begin{pnm}
    x_1\\
    x_2\\
    \Vdots \\
    x_n
\end{pnm}\]



\begin{theorem}{}{}
     Soit $E$ un $\bdK$-espace vectoriel de dimension finie $n \in \bdN^*$ et de base $\bcB_E$.
     
     \[ \hg{\begin{array}[t]{rcl}
         E &\to& \bcM_{n, 1}\left(\bdK\right)  \\
         x &\mapsto& \Matc_{\bcB_E}\left(x\right) 
     \end{array} \ \text{est un isomorphisme}}.\]
\end{theorem}

\subsection{Matrice d'une famille de vecteurs}

\begin{definition}
     Soit $E$ un $\bdK$-espace vectoriel de dimension finie $n \in \bdN^*$ et de base $\bcB_E$.
     
     \[ \hg{\begin{array}[t]{rcl}
         E &\to& \bcM_{n, 1}(\bdK]  \\
         x &\mapsto& \Mat_{\bcB_E}\left(x\right) 
     \end{array} \ \text{est un isomorphisme}}.\]
\end{definition}

\end{document}