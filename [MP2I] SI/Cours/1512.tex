\documentclass[a4paper,french,bookmarks]{article}
\usepackage{./Structure/4PE18TEXTB}

\begin{document}
\stylize{Sciences de l'ingénieur}{C3 - Cinématique du Solide}
\initcours

\section{Définitions}


\section{Comment on dérive un vecteur ?}

\section{Dérivée d'un vecteur quelconque}

\section{Cinématique des solides}

\begin{definition}{Solide}
    Un solide est un ensemble $S$ de points tels que :
    \[ \forall t,\ \forall (A, B) \in S,\ \exists! \lambda \in \bdR_+ \mod{\mod{\vec{AB}(t)}} = \lambda \]
\end{definition}

On prend un tel vecteur $\vec{AB}$ et $\lambda \in \bdR$.
On a $\vec{AB}^2 = \lambda^2$ donc $\vec{AB}\cdot\left.\dfrac{\text d\vec{AB}}{dt}\right|_{R_0}=0$. Donc :
\[ \vec{AB}\left.\dfrac{\text d\vec{O_OB}}{\textdt}\right|_{R_0} - \vec{AB}\left.\dfrac{\text d\vec{O_OA}}{\text dt}\right|_{R_0} \]
\end{document}