\documentclass[a4paper,french,bookmarks]{article}
\usepackage{./Structure/4PE18TEXTB}

\begin{document}
    \stylizeDoc{Mathématiques}{Chapitre 25}{Espaces préhilbertiens}

    \initcours{}

    \section{Projection orthogonale}
    
    Pour rappel, si $E$ est un espace préhilbertien réel, et si $F$ est un sous-espace vectoriel de $E$ avec $F$ de dimension finie, alors $F$ possède un supplémentaire orthogonal : $F \oplus F^\bot = E$. 
    
    \begin{definition}{Projetion orthogonale}
        On appelle \hg{projection orthogonale sur $F$} la projection vectorielle $p \in \bcL\p{E}$ sur $F$ parallèlement à $F^\bot$, \ie :
        %
        \begin{psse}
            \item $\forall x \in F$,\qquad $p\p{x} = x$.
            \item $\forall x \in F^\bot$,\qquad $p\p{x} = 0_E$.
            \item $\forall x \in E$,\qquad $x = p\p{x} + \p{x - p\p{x}}$.
        \end{psse}
    \end{definition}
    
    On a alors $F = \Imm p = \ker\p{p - \Id}$ et $F^\bot = \Ker p$.
    
    \begin{tikzpicture}
        \draw (-2, 0) -- (2, 0);
        \draw (0, -3) -- (0, 0);
        \draw[dashed] (0, 0) -- (0, 1);
        \draw (0, 1) -- (0, 4);
        \draw (-5, 0) -- (-1, 4);
    \end{tikzpicture}
    
    Ainsi, $\forall x \in E$, $p\p{x} \bot \p{x - p\p{x}}$.
    
    \underline{Expression du projeté orthogonal dans une B.O.N :}
    
    Fixons $m = \dim F$ et $\bcB = \p{e_1, \dots, e_m}$ une B.O.N de $F$. Le projeté orthogonal sur $F$ d'un vecteur $x \in E$ est :
    %
    \[ p\p{x} = \sum_{i=1}^m \underbrace{\phyavg{x, e_i}}_{\lambda_i}e_i\]
    %
    En effet, $p\p{x} \in F$, donc $\exists ! \p{\lambda_1, \lambda_2, \dots, \lambda_m} \in \bdR^m$ tq $p\p{x} = \sum_{i=1}^m \lambda_i e_i$.
    
    Pour $j \in \iint{1, m}$, on a donc $\phyavg{p\p{x}, e_j} = \sum_{i=1}^m \lambda_i \underbrace{\phyavg{e_i, e_j}}_{\delta_{i, j}} = \lambda_j$.
    
    Or $x = p\p{x} + x - p\p{x}$ donc $\phyavg{x, e_j} = \underbrace{\phyavg{p\p{x}, e_j}}_{\lambda_j} + \underbrace{\phyavg{\underbrace{x - p\p{x}}_{\in F^\bot}, \underbrace{e_j}_{\in F}}}_{= 0}$.
    
    \begin{example}{}{}
        \begin{enumerate}
            \itt \hgu{Projection orthogonale sur une droite :} Soit $\bca \in E$ avec $\bca \neq 0_E$ et $\bsD = \Vect{\bca}$.
            
            Calculons la projection orthogonale sur $\bsD$. Prenons d'abord une B.O.N de $\bsD$ : $\p{\dfrac{\bca}{\norm{\bca}}}$.
            
            Pour tout $x \in \bsE$, $p\p{x} = \phyavg{x, \dfrac{\bca}{\norm{\bca}}}\dfrac{\bca}{\norm{\bca}}$
            
            $\mathscr{E} \mathscr{e}$.
        \end{enumerate}
    \end{example}
    
    Exercice de calcul de distance :
    
\end{document}