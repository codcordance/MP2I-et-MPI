\documentclass[a4paper,french,bookmarks]{article}

\usepackage{./Structure/4PE18TEXTB}

\newboxans

\begin{document}

    \stylizeDoc{Mathématiques}{Programme de khôlle 23}{Énoncés et résolutions}

    \section*{Déterminants}

    \subsection*{Groupe symétrique}

    \begin{enumerate}
        \ithand Généralités, cardinal de $\bfS_n$. Cycles et transpositions. Décomposition d'un
        cycle en produit de transpositions.
        
        \ithand Décomposition d'une permutation en produit de cycles à supports disjoints.
        
        \ithand Signature d'une permutation : définition via les inversions. Signature d'une
        transposition.
        
        \ithand La signature est un morphisme de groupes de $\p{\bfS_n, \circ}$ dans
        $\p{\ens{\pm 1}, \times}$.
        
        \ithand Signature d'un cycle, calcul de la signature d'une permutation. Groupe alterné
        $\bfA_n$.
    \end{enumerate}

    \subsection*{Formes multilinéaires}

    \begin{enumerate}
        \ithand Formes $n$-linéaires, formes $n$-linéaires alternées, antisymétriques.
        Propriétés.
        
        \ithand Formes $n$-linéaires alternées sur un espace vectoriel $E$ de dimension $n$.
    \end{enumerate}

    \subsection*{Déterminants}

    \begin{enumerate}
        \ithand Déterminant d'une famille de vecteurs : définition, propriétés. Caractérisation
        des bases.
        
        \ithand Déterminant d'un endomorphisme. Lien avec le déterminant d'une famille de
        vecteurs. Propriétés. Caractérisation des automorphisme.
        
        \ithand Déterminant d'une matrice carrée. Définition, lien avec les déterminants
        précédents. Propriétés. Caractérisation des matrices inversibles.
    \end{enumerate}

    \subsection*{Calcul du déterminant}

    \begin{enumerate}
        \ithand Effet des opérations élémentaires sur le déterminant.
        Développement selon une ligne ou une colonne.
        
        \ithand Déterminant d'une matrice triangulaire, triangulaire par blocs. Cofacteur,
        comatrice.
        
        \ithand Formule de l'inverse d'une matrice inversible.
        
        \ithand Déterminant de \textsc{Vandermonde}. Orientation d'un $\bdR$-espace vectoriel
        de dimension finie.
    \end{enumerate}

    \questionsdecours

    \begin{enumerate}
        \item Savoir calculer la signature d'une permutation, en la décomposant en produit de
        cycles à supports disjoints et en produit de transpositions.
        
        \noafter
        %
        \boxans{
            \begin{definition}{Inversion}{}
                Soit $n \in \bdN^*$, une permutation $\sigma \in \bfS_n$ et $\p{i, j} \in
                \iint{1, n}^2$. On dit que \hg{$\sigma$ réalise une inversion sur $\p{i, j}$}
                (ou que \hg{$\p{i, j}$ est une inversion de $\sigma$}) lorsque \hg{$i < j$ et
                $\sigma\p{i} > \sigma\p{j}$}.
            \end{definition}
        }
        %
        \nobefore
        %
        \begin{notation}
            On note généralement :
            %
            \[ \hg{\Inv{\sigma} = \mod{\ens{\vphantom{\displaystyle\sum} \p{i, j} \in 
            \iint{1, n}^2 \;\middle\vert\; i < j \et \sigma\p{i} > \sigma\p{j}}}} \]
            %
            le \hg{nombre d'inversions} d'une permutation $\sigma \in \bfS_n$.
        \end{notation}
        %
        \boxans{
            \begin{definition}{Signature d'une permutation}{}
                Soit $n \in \bdN^*$ et une permutation $\sigma \in \bfS_n$. On alors
                \hg{signature de $\sigma$} la quantité $\hg{\left(-1\right)^{\Inv\p{\sigma}}}$.
            \end{definition}
        }
        %
        \begin{notation}
            On note généralement \hg{$\epsilon\p{\sigma}$} la signature d'une permutation
            $\sigma \in \bfS_n$.
        \end{notation}
        %
        \yesafter
        %
        \boxans{
            \begin{form}{Méthode}{}
                Pour calculer la signature d'une permutation $\sigma \in \bfS_n$ :
                \begin{enumerate}
                    \itt Déterminer les orbites $p$ de $\sigma$, \ie les cycles de supports
                    disjoints (\hg{support cyclique de $\sigma$}).
                    
                    \itt Chaque orbite $\p{C_i}_{1 \leq i \leq p}$ est un $k_i$-cycle  de
                    signature $\epsilon\p{C_i} = \p{-1}^{k_i - 1}$.
                    
                    Les $k_i$, rangés dans l'ordre croissent forment d'ailleurs le \hg{type
                    cyclique de $\sigma$}.
                    
                    \itt La signature de $\sigma$ est le produit des signatures des cycles :
                    %
                    \[ \prod_{i=1}^p \epsilon\p{C_i} = \prod_{i=1}^p \p{-1}^{k_i - 1} =
                    \p{-1}^{\p{\sum_{i=1}^p k_i} - p} = \p{-1}^{n-p} \]
                    %
                    \bf{\itstar Donc pour $n \in \bdN$ et $\sigma \in \bfS_n$, on a
                    $\epsilon\p{\sigma} = \p{-1}^{n - p}$ où $p$ est le nombre de
                    cycles/d'orbites de $\sigma$.}
                \end{enumerate}
            \end{form}
        }
        %
        \yesbefore
        
        \item Donner (sans preuve) quelques propriétés du déterminant d'une matrice :
        
        $\det{AB}$, $\det{A^n}$, $\det{\lambda A}$, $\det{\mtrans{A}}$, $\det{A^{-1}}$ lorsque
        $A$ est inversible.
        
        \boxans{
            \begin{property}{Propriétés du déterminant}{}
                    Soient un entier non nul $n \in \bdN^*$ et deux matrices $\p{A, B} \in
                    \bcM_n\p{\bdK}$. On a :
                    %
                    \begin{psse}
                        \item \hg{$\det{AB} = \det{A}\det{B}$} ;
                        \item \hg{$\forall p \in \bdN,\qquad \det{A^p} = \det{A}^p$} ;
                        \item \hg{$\forall \lambda \in \bdK,\qquad \det{\lambda A} =
                        \lambda^n\det{A}$} ;
                        \item \hg{$A \in \GL_n\p{\bdK} \implies \det{A^{-1}} =
                        \dfrac{1}{\det{A}}$}.
                    \end{psse}
            \end{property}
        }
        
        \item Effet des opérations élémentaires sur les colonnes sur le déterminant : échange $C_i \leftrightarrow C_j$, multiplication par un scalaire $C_j
        \leftarrow \lambda C_j$ (avec $\lambda \neq 0$), ajout d'une combinaison
        linéaire de colonnes $C_j \leftarrow C_j + \sum\limits_{i \neq j} \lambda_i C_i$.
        
        \boxans{
            \begin{theorem}{Effet des opérations élémentaires sur le déterminant}{}
                Soit un entier non nul $n \in \bdN^*$, une matrice $A \in \bcM_n\p{\bdK}$ de
                colonnes $C_1, C_2, \dots C_n$ (de la gauche vers la droite) et un indice $j \in \iint{1, n}$. \hg{Les opérations élémentaires ont pour effet sur $\det A$} :
                
                \begin{psse}
                    \item Échanger deux colonnes (\hg{$\forall i \in \iint{1, n},\ C_j \leftrightarrow C_i$})
                    \hg{multiplie le déterminant par $-1$}.
                    
                    \item Multiplier une ligne par un scalaire non nul $\lambda \neq 0$
                    (\hg{$C_j \leftarrow \lambda C_j$}) \hg{multiplie le déterminant par $\lambda$}.
                    
                    \item Faire une combinaison linéaire des colonnes $\left(\hg{C_j \leftarrow C_j - \sum\limits^n_{\substack{i = 1\\ i \neq j}} \lambda_i C_i}\right)$ \hg{ne modifie pas le déterminant}.
                    
                    \item Permuter les colonnes selon $\sigma \in \bfS_n$ (\hg{$\forall i \in \iint{1, n},\ C_i \leftrightarrow C_{\sigma\p{i}}$}) \hg{multiplie le déterminant par $\epsilon\p{\sigma}$}.
                    
                    \item De même pour les opérations sur les lignes.
                \end{psse}
            \end{theorem}
        }
        
        \item Déterminant de Vandermonde. Énoncé et preuve.
        
        \noafter
        %
        \boxans{
            \begin{definition}{Matrice de Vandermonde}{}
                Soit $n \in \bdN^*$ et une famille de $n$ scalaires $\p{\alpha_i}_{i \in \iint{1, n}} \in \bdK^n$. On appelle \hg{matrice de \textsc{Vandermonde}} de coefficients $\p{\alpha_i}_{i \in \iint{1, n}}$ \hg{la matrice $\p{{\alpha_i}^{j-1}}_{ \p{i, j} \in \iint{1, n}^2}$}, \ie la matrice :
                %
                \[\hg{  \begin{pNiceMatrix}
                            1 & \alpha_1 & {\alpha_1}^2 & \Cdots & {\alpha_1}^{n-1} \\
                            1 & \alpha_2 & {\alpha_2}^2 & \Cdots & {\alpha_2}^{n-1} \\
                            1 & \alpha_3 & {\alpha_3}^2 & \Cdots & {\alpha_3}^{n-1} \\
                            \Vdots & \Vdots & \Vdots & & \Vdots \\
                            1 & \alpha_n & {\alpha_n}^2 & \Cdots & {\alpha_n}^{n-1} 
                        \end{pNiceMatrix}_{\left[n\right]} }\]
            \end{definition}
        }
        %
        \nobefore
        %
        \begin{notation}
            Le déterminant d'une telle matrice est appelé \hg{déterminant de \textsc{Vandermonde}} est est noté, avec les mêmes coefficients, $\hg{V\p{\alpha_1, \alpha_2, \dots, \alpha_n}}$.
        \end{notation}
        %
        \boxans{
            \begin{property}{Déterminant de Vandermonde}{}
                Soit $n \in \bdN^*$ et une famille de $n$ scalaires $\p{\alpha_i}_{i \in \iint{1, n}} \in \bdK^n$. On a :
                %
                \[ \hg{V\p{\alpha_1, \alpha_2, \dots, \alpha_n} = \begin{vNiceMatrix}
                            1 & \alpha_1 & {\alpha_1}^2 & \Cdots & {\alpha_1}^{n-1} \\
                            1 & \alpha_2 & {\alpha_2}^2 & \Cdots & {\alpha_2}^{n-1} \\
                            1 & \alpha_3 & {\alpha_3}^2 & \Cdots & {\alpha_3}^{n-1} \\
                            \Vdots & \Vdots & \Vdots & & \Vdots \\
                            1 & \alpha_n & {\alpha_n}^2 & \Cdots & {\alpha_n}^{n-1} 
                        \end{vNiceMatrix}_{\left[n\right]} = \prod_{1 \leq i < j \leq n} \p{\alpha_j - \alpha_i} }\]
            \end{property}
        }
        %
        \yesafter
        %
        \begin{nproof}
            Soit $n \in \bdN^*$ et une famille de $n$ scalaires $\p{\alpha_i}_{i \in \iint{1, n}} \in \bdK^n$.
            
            \begin{enumerate}
                \itt Soit $M = \begin{pNiceMatrix}
                            1 & \alpha_1 & {\alpha_1}^2 & \Cdots & {\alpha_1}^{n-1} \\
                            1 & \alpha_2 & {\alpha_2}^2 & \Cdots & {\alpha_2}^{n-1} \\
                            1 & \alpha_3 & {\alpha_3}^2 & \Cdots & {\alpha_3}^{n-1} \\
                            \Vdots & \Vdots & \Vdots & & \Vdots \\
                            1 & X & X^2 & \Cdots & X^{n-1} 
                        \end{pNiceMatrix}_{\left[n\right]}$ de mineurs de positions $\Delta_{i, j}$. On pose le polynôme $P\p{X} = V\p{\alpha_1, \alpha_2, \dots, X} = \det{M}$ d'indéterminée $X$, et on développe suivant la dernière ligne (d'indice $n$):
                %
                \[ P\p{X} = \sum_{j=1}^n \left(-1\right)^{n+j}\left[M\right]_{n,j} \Delta_{n,j} = \sum_{j=1}^n \left(-1\right)^{n+j} X^{j-1} \Delta_{n,j}\]
                %
                On a donc $P\p{X}$ de la forme :
                %
                \[P\p{X} = X^{n-1}\underbrace{\begin{vNiceMatrix}
                            1 & \alpha_1 & {\alpha_1}^2 & \Cdots & {\alpha_1}^{n-1} \\
                            1 & \alpha_2 & {\alpha_2}^2 & \Cdots & {\alpha_2}^{n-1} \\
                            1 & \alpha_3 & {\alpha_3}^2 & \Cdots & {\alpha_3}^{n-1} \\
                            \Vdots & \Vdots & \Vdots & & \Vdots \\
                            1 & \alpha_{n-1} & {\alpha_{n-1}}^2 & \Cdots & {\alpha_{n-1}}^{n-1} 
                        \end{vNiceMatrix}}_{V\p{\alpha_1, \alpha_2, \dots, \alpha_{n-1}} = cst} - X^{n-1}\underbrace{\begin{vNiceMatrix}
                            1 & \alpha_1 & \Cdots & {\alpha_1}^{n-1} & {\alpha_1}^n \\
                            1 & \alpha_2 & \Cdots & {\alpha_2}^{n-2} & {\alpha_2}^n \\
                            1 & \alpha_3 & \Cdots & {\alpha_3}^{n-2} & {\alpha_3}^n \\
                            \Vdots & \Vdots & & \Vdots & \Vdots \\
                            1 & \alpha_{n-1} & \Cdots & {\alpha_{n-1}}^{n-2} & {\alpha_{n-1}}^n 
                        \end{vNiceMatrix}}_{cst} + \dots\]
                
                \itt Puisque dans chaque $\Delta_{n, j}$ on retire la dernière ligne, la seule à contenir l'indéterminée $X$, chaque $\Delta{n, j}$ est constant donc $P \in \bdK\left[X\right]$ est de degré $n-1$. On vient par ailleurs de montrer que son coefficient dominant est égal à $\Delta_{n, n} = V\p{\alpha_1, \alpha_2, \dots, \alpha_{n-1}}$.
                
                \itt Lorsque $X = \alpha_i$ pour $i \in \iint{1, n-1}$, on a deux lignes égales dans la matrice $M$ (les lignes $i$ et $n$) donc le déterminant est nul. Ainsi pour chaque $i \in \iint{1, n-1}$ on a $P\p{\alpha_i} = 0$ donc $\alpha_i$ est racine de $P$. Par suite, on a la factorisation $\displaystyle P\left(X\right) = V\p{\alpha_1, \alpha_2, \dots, \alpha_{n-1}}\prod_{i = 1}^{n-1}\p{X - \alpha_i}$. Donc en évaluant en $\alpha_n$, on a :
                %
                \[V\p{\alpha_1, \alpha_2, \dots, \alpha_n} = P\left(\alpha_n\right) = V\p{\alpha_1, \alpha_2, \dots, \alpha_{n-1}}\prod_{i=1}^{n-1} \p{\alpha_n - \alpha_i}\]
                
                \itt On procède par récurrence sur $k \in \iint{2, n}$ selon le prédicat $H\p{k} : V\p{\alpha_1, \alpha_2, \dots, \alpha_k} = \displaystyle \prod_{1 \leq i < j \leq k} \p{\alpha_j - \alpha_i}$.
                
                \itstar \underline{Initialisation :} On a $V\p{\alpha_1, \alpha_2} = \begin{vNiceMatrix}
                    1 & \alpha_0\\
                    1 & \alpha_1
                \end{vNiceMatrix} = \alpha_1 - \alpha_0$ donc $V\p{2}$ est vrai.
                
                \itstar \underline{Hérédité :} Soit $k \in \iint{1, k-1}$ tel que $V\p{k}$ est vrai, \ie $ V\p{\alpha_1, \alpha_2, \dots, \alpha_k} = \prod\limits_{1 \leq i < j \leq k} \p{\alpha_j - \alpha_i}$.
                
                Par les résultats précédents, on a $ V\p{\alpha_1, \alpha_2, \dots, \alpha_{k+1}} = V\p{\alpha_1, \alpha_2, \dots, \alpha_k}\prod\limits_{i=1}^{k} \p{\alpha_{k+1} - \alpha_i}$, donc :
                %
                \[ V\p{\alpha_1, \alpha_2, \dots, \alpha_{k+1}} = \prod_{1 \leq i < j \leq k} \p{\alpha_j - \alpha_i}\prod_{i=1}^{k} \p{\alpha_{k+1} - \alpha_i} = \prod_{1 \leq i < j \leq k+1} \p{\alpha_j - \alpha_i} \qquad\text{donc} \ V\p{k+1} \ \text{est vrai.}\]
                %
                \itstar \underline{Conclusion :} Par principe de récurrence, on a $V\p{\alpha_1, \alpha_2, \dots, \alpha_n} = \prod\limits_{1 \leq i < j \leq n} \p{\alpha_j - \alpha_i}$.
            \end{enumerate}
            
            
        \end{nproof}
        %
        \yesbefore
        
        \item Calculer le déterminant suivant où $\p{a, b} \in \bdC^2$ et $n \in \bdN^*$ :
        \qquad $\begin{vNiceMatrix}
                    a       &   b       &   \Cdots  &   b       \\
                    b       &   \Ddots  &   \Ddots  &   \Vdots  \\
                    \Vdots  &   \Ddots  &           &   b       \\
                    b       &   \Cdots  &   b\hphantom{-}    &   a
                \end{vNiceMatrix}_{\left[n\right]}$
                
        \boxans{
            \begin{align*}
                \begin{vNiceMatrix}
                    a       &   b       &   \Cdots  &   b       \\
                    b       &   \Ddots  &   \Ddots  &   \Vdots  \\
                    \Vdots  &   \Ddots  &           &   b       \\
                    b       &   \Cdots  &   b\hphantom{-}       &   a
                \end{vNiceMatrix} &\eq{C_1 \leftarrow C_1 + \sum\limits_{i=2}^n C_i}
                \begin{vNiceMatrix}
                    a + \p{n-1}b & b       &   \Cdots  &        & b       \\
                    \Vdots       & a       &   \Ddots  &        & \Vdots  \\
                                 & b       &   \Ddots  &        & \Vdots       \\
                                 & \Vdots  &   \Ddots  &        & b       \\
                    a + \p{n-1}b & b       &   \Cdots  &    b\hphantom{--}   & a      
                \end{vNiceMatrix} =
                \p{a + \p{n-1}b}\begin{vNiceMatrix}
                    1 & b       &   \Cdots  &        & b       \\
                    \Vdots       & a       &   \Ddots  &        & \Vdots  \\
                                 & b       &   \Ddots  &        & \Vdots       \\
                                 & \Vdots  &   \Ddots  &        & b       \\
                    1 & b       &   \Cdots  &    b\hphantom{--}   & a      
                \end{vNiceMatrix}\\
                &\eq{\left\lbrace\substack{L_2 \leftarrow L_2 - L_1\\
                L_3 \leftarrow L_3 - L_1\\
                \cdots \\
                L_n \leftarrow L_n - L_1}\right.}\p{a + \p{n-1}b}\begin{vNiceMatrix}
                    1       & b       &   \Cdots  &        & b       \\
                    0       & a-b     &   0       & \Cdots & 0  \\
                    \Vdots  & \Ddots  &   \Ddots  & \Ddots & \Vdots       \\
                            &         &           &        & 0       \\
                    0       & \Cdots  &           &    0   & a-b      
                \end{vNiceMatrix} = \p{a + \p{n-1}b}\p{a-b}^{n-1}
            \end{align*}
        }
                
        \item Calculer le déterminant suivant :
        \qquad $\begin{vNiceMatrix}
                    2       &   1       &   0       &   \Cdots  &   0       \\
                    1       &   \Ddots  &  \Ddots   &   \Ddots  &   \Vdots  \\
                    0       &   \Ddots  &           &           &   0       \\
                    \Vdots  &   \Ddots  &  \Ddots   &           &   1       \\
                    0       &   \Cdots &0\phantom{-}&   1       &   2
                \end{vNiceMatrix}_{\left[n\right]}$
                
        \boxans{
            On pose pour $n \in \bdN^*$ la matrice $D_n = \begin{pNiceMatrix}
                    2       &   1       &   0       &   \Cdots  &   0       \\
                    1       &   \Ddots  &  \Ddots   &   \Ddots  &   \Vdots  \\
                    0       &   \Ddots  &           &           &   0       \\
                    \Vdots  &   \Ddots  &  \Ddots   &           &   1       \\
                    0       &   \Cdots &0\phantom{-}&   1       &   2
                \end{pNiceMatrix}_{\left[n\right]}$. On développe $\det{D_n}$ selon la ligne 1 :
                %
                \[ \det{D_n} = \sum_{j=1}^n \p{-1}^{1+j}\left[D\right]_{1, j}\Delta_{1,j}\p{D_n} = \begin{vNiceMatrix}
                    2       &   1       &   0       &   \Cdots  &   0       \\
                    1       &   \Ddots  &  \Ddots   &   \Ddots  &   \Vdots  \\
                    0       &   \Ddots  &           &           &   0       \\
                    \Vdots  &   \Ddots  &  \Ddots   &           &   1       \\
                    0       &   \Cdots &0\phantom{-}&   1       &   2
                \end{vNiceMatrix}_{\left[n-1\right]} - \begin{vNiceMatrix}
                    1       &   1       &   0       &   \Cdots  &   0       \\
                    0       &   2       &  \Ddots   &   \Ddots  &   \Vdots  \\
                    \Vdots  &   1       &  \Ddots   &           &   0       \\
                            &   0       &  \Ddots   &           &   1       \\
                            &   \Vdots  &  \Ddots   &           &   2        \\
                    0       &   0       &   \Cdots  &0\hphantom{--}&   1\vphantom{\dfrac{-}{-}}
                \end{vNiceMatrix}_{\left[n-1\right]}\]
                %
                On reconnaît le premier terme comme étant $\det{D_{n-1}}$. En développant le second terme selon la première ligne, on reconnaît également $\det{D_{n-2}}$. On a donc pour entier $n \in \bdN$ tel que $n \geq 3$, $\det{D_n} = 2\det{D_{n-1}} - \det{D_{n-2}}$.
                
               On a suite récurrente linéaire d'ordre 2, d'équation caractéristique $r^2 = 2r - 1$ soit $\p{r-1}^2 = 0$. On a la racine double $r = 1$, ainsi :
                %
                \[ \exists ! \p{\lambda, \mu} \in \bdR^2,\qquad D_n = \p{\lambda + n\mu}\times 1^n = \lambda + n\mu\]
                %
                Puisque  $\lambda + \mu = \det{D_1} = \mod{2} = 2$ et $\lambda + 2\mu = \det{D_2} = \begin{vNiceMatrix}
                    2 & 1\\ 1 & 2
                \end{vNiceMatrix} = 3$, on a $\lambda = \mu = 1$. Donc finalement :
                %
                \[ \forall n \in \bdN^*,\qquad \begin{vNiceMatrix}
                    2       &   1       &   0       &   \Cdots  &   0       \\
                    1       &   \Ddots  &  \Ddots   &   \Ddots  &   \Vdots  \\
                    0       &   \Ddots  &           &           &   0       \\
                    \Vdots  &   \Ddots  &  \Ddots   &           &   1       \\
                    0       &   \Cdots &0\phantom{-}&   1       &   2
                \end{vNiceMatrix}_{\left[n\right]} = n + 1\]
        }
    \end{enumerate}
\end{document}