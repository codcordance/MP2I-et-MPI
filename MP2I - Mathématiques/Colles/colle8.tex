\documentclass[a4paper,french,bookmarks]{article}
\usepackage{./Structure/4PE18TEXTB}

\begin{document}
\stylizeDoc{Mathématiques}{Programme de khôlle 8}{Énoncés et résolutions}

\section*{Calculs de primitives et intégrales}

\subsection*{Primitives}
Définition d’une primitive. Ensemble des primitives d’une fonction $f$ continue sur un intervalle $I$, à valeurs dans $\bdK$ ($\bdR$ ou $\bdC$). Rappel des primitives des fonctions usuelles, avec l’intervalle de validité. \textbf{Méthodes :}
\begin{enumerate}
    \itarr Changements affine : primitive de $x \mapsto \lambda f(ax + b)$.
    \itarr Linéarité : primitive de $x \mapsto \lambda f + \mu g$.
    \itarr Primitive de $x \mapsto \dfrac{1}{(x-a)(x-b)}$ avec $a$ et $b$ deux réels distincts.
    \itarr Primitive de $x \mapsto u'(x)f(u'(x))$.
    \itarr Rappels de trigonométrie : linéarisation.
    \itarr Primitive de $x \mapsto \dfrac{1}{x^2+px+q}$ avec $q$ et $p$ deux réels.
    \itarr Primitive de $x \mapsto \dfrac{1}{x + \alpha}$ avec $\alpha \in \bdC$.
\end{enumerate}

\subsection*{Intégrales}
Intégrale d’une fonction continue sur un intervalle $I$, à valeurs dans $\bdR$ ou $\bdC$.

Théorème fondamental de l’analyse. Intégration par parties. Changement de variables.

\section*{Équations différentielles linéaires d’ordre 1}

Définitions. Structure des solutions. Résolution de l’équation homogène. Recherche d’une solution particulière de l’équation avec second membre : solution évidente, méthode de variation de la constante, principe
de superposition.

Cas des équations différentielles à coefficients constants. Problème de Cauchy.

Cas où l’équation n’est pas normalisée : recollement.

\section*{Équations différentielles linéaires d’ordre 2 [Début]}

Définitions. Structure des solutions. Résolution de l’équation homogène : cas complexe et cas réel.

\section*{Questions / Exercices de cours / Savoir faire}

Savoir calculer une intégrale par intégration par parties.
Savoir calculer une intégrale par changement de variable.

Savoir résoudre une Équation Différentielle Linéaire d’ordre 1, avec second membre.

Savoir résoudre une Équation Différentielle Linéaire d’ordre 2 homogène à coefficients constants.

\begin{enumerate}
    \item Déterminer les primitives de $x \mapsto \dfrac{1}{\sh x}$ sur $\bdR_+^*$. \quad \textit{(on posera $u = e^t$)}
    
    \boxans{
        Soit $x \in \bdR_+^*$. On cherche $\displaystyle \int^x \dfrac{\text dt}{\sh t} = \int^x \dfrac{2\text dt}{e^t - e^{-t}}$. On pose $u = e^t$, donc $t = \ln u$ d'où $\text dt = \dfrac{\text du}{u}$.
        
        Lorsque $t = x$, $u = e^x$ donc $\displaystyle \int^x \dfrac{\text dt}{\sh t} = \int^{e^x} \dfrac{2\text du}{u - \dfrac{1}{u}} = \int^{e^x} \dfrac{2\text du}{u^2 -1} = \int^{e^x} \left(\dfrac{1}{u-1}-\dfrac{1}{u+1}\right)\text du$. Donc :
        \[ \int^x \dfrac{\text dt}{\sh t} = \ln\left(e^x - 1\right) - \ln\left(e^x + 1\right) = \ln\left(\dfrac{e^x - 1}{e^x + 1}\right) = \ln\left(\th{\dfrac{x}{2}}\right)\]
        Donc \boxsol{les primitives de $x \mapsto \dfrac{1}{\sh x}$ sur $\bdR_+^*$ sont $\left\{x \mapsto \ln\left(\th{\dfrac{x}{2}}\right) + c, c \in \bdR \right\}$}.
    }
    
    \item Déterminer les primitives de $x \mapsto \sqrt{1-x^2}$ sur $\left]-1,1\right[$. \quad \textit{(on posera $x = \sin \theta$)}
    
    \boxans{
        Soit $x \in \left]-1,1\right[$. On pose $t = \sin \theta$, d'où $\text dt = \cos \theta \; \text d\theta$. Lorsque $t = \sin \theta = x$, $\theta = \arcsin x$ donc $\displaystyle \int^x \sqrt{1-t^2}\text dt = \int^{\arcsin x} \sqrt{1-\sin^2 \theta}\cos \theta \; \text d\theta = \int^{\arcsin x} \mod{\cos \theta}\cos \theta \; \text d\theta$. Or $t \in \left]-1,1\right[$ donc $\theta \in \left]-\dfrac{\pi}{2},\dfrac{\pi}{2}\right[$ donc $\cos \theta \geq 0$. On cherche donc $\displaystyle \int^{\arcsin x} \cos^2 \theta \; \text d\theta = \dfrac{1}{2}\int^{\arcsin x} \left(\cos(2t) +1\right)\text d\theta = \dfrac{1}{4}\sin{2\arcsin x} - \dfrac{1}{2}\arcsin{x}$. Or $\sin{2\arcsin x} = 2\sin{\arcsin x}\cos{\arcsin x} = 2x\sqrt{1-x^2}$
        \[\text{Donc} \qquad \displaystyle \int^x \sqrt{1-t^2}\text dt = \dfrac{1}{4}2x\sqrt{1-x^2} + \dfrac{1}{2}\arcsin x = \dfrac{1}{2}\left(x\sqrt{1-x^2}+\arcsin x\right) \]
        Donc \boxsol{les primitives de $x \mapsto \sqrt{1-x^2}$ sur $\left]-1,1\right[$ sont $\left\{x \mapsto \dfrac{1}{2}\left(x\sqrt{1-x^2}+\arcsin x\right) + c, c \in \bdR \right\}$}.
    }
    
    \item  Déterminer les primitives de $x \mapsto \dfrac{1}{\sin x}$ sur $\left]0,\pi\right[$. \quad \textit{(on posera $u = \tan{\frac{t}{2}}$)}
    
    \boxans{
        Soit $x \in \left]0,\pi\right[$. On pose $u = \tan{\frac{t}{2}}$, donc $\text du = \dfrac{1+u^2}{2}\text dt$ d'où $\text dt = \dfrac{2\text du}{1+u^2}$. Les formules de l'angle moitié livrent $\sin t = \dfrac{2u}{1+u^2}$ donc $\dfrac{1}{\sin t} = \dfrac{1+u^2}{2\text du}$. Lorsque $t = x$, on a $u = \tan{\dfrac{x}{2}}$. 
        
        Donc $\displaystyle \int^x \dfrac{\text dt}{\sin t} = \int^{\tan{\dfrac{x}{2}}} \dfrac{1+u^2}{2u}\dfrac{2\text du}{1+u^2} = \int^{\tan{\dfrac{x}{2}}} \dfrac{\text du}{u} = \ln\mod{\tan \dfrac{x}{2}}$. Or $\tan$ est positive sur $\left]0,\dfrac{\pi}{2}\right[$. 
        
        Donc \boxsol{les primitives de  $x \mapsto \dfrac{1}{\sin x}$ sur $\left]0,\pi\right[$ sont $\left\{\ln\left(\tan{\dfrac{x}{2}}\right) + c, c \in \bdR \right\}$}.
    }
    \item Soit $\mathbf{(H)}: y' + a(x)y = 0$ avec $a$ continue sur un intervalle $I$.
    
    Démontrer que $S_H = \{ x \mapsto \lambda e^{-A(x)}, \lambda \in \bdK\}$ où $A$ est une primitive de $a$ sur $I$.
    
    \boxans{
        Soit $a \in \cc{0}{I}{\bdK}$ de primitive $A$ et $f$ une solution de $\mathbf{(H)}$ sur $I$. Posons $y: \begin{array}{ll}
            I &\to \bdK  \\
            x &\mapsto f(x)\exp{A(x)} 
        \end{array}$.
        
        $\begin{array}{lll}
           y \ \text{est dérivable sur} \ I \ \text{telle que} \ \forall x \in I, y'(x) &= f'(x)\exp{A(x)} + f(x)A'(x)\exp{A(x)}  \\
            &= \exp{A(x)}\left[f'(x) + a(x)f(x)\right]\\
            &= \exp{A(x))}\times0 & \text{donc} \ y'(x) = 0.
        \end{array}$
        $y'$ est nulle donc $y$ est constante, donc $\exists \lambda \in \bdK$, $\forall x \in I$, $y(x) = f(x)\exp{A(x)} = \lambda$ donc $f(x)=\lambda e^{-A(x)}$.
        \[ \text{Donc} \ S_H \subset \left\{x \mapsto \lambda e^{-A(x)}, \lambda \in \bdK \right\}\]
        
        Réciproquement, soit $f \in \left\{x \mapsto \lambda e^{-A(x)}, \lambda \in \bdK \right\}$. Montrons que $f \in S_H$.
        \[ \forall x \in I, \quad f'(x) + a(x)f(x) = -\lambda a(x)e^{-A(x)} + \lambda a(x)e^{-A(x)}\]
        $f$ est solution de $\mathbf{(H)}$ donc $f$ est bien dans $S_H$, donc $\left\{x \mapsto \lambda e^{-A(x)}, \lambda \in \bdK \right\} \subset S_H$.
        
        On en déduit donc \boxsol{$S_H = \left\{x \mapsto \lambda e^{-A(x)}, \lambda \in \bdK \right\}$}
    }
    
    \item  Résoudre $y' + \dfrac{1}{x}y = \dfrac{x}{x-1}$ sur $\left]1;+\infty\right[$.
    
    \boxans{
    On résout l'équation homogène $\mathbf{(H)}: y' + \dfrac{1}{x}y = 0$ sur $I = \left]1;+\infty\right[$.
    La fonction inverse $x \mapsto \dfrac{1}{x}$ est primitivable sur $I$ en la fonction logarithme népérien $x \mapsto \ln x$. On a donc $f(x) = \lambda e^{-\ln x} = \dfrac{\lambda}{x}$ une solution de $\mathbf{(H)}$ sur $I$ avec $\lambda \in \bdR$, donc $ S_H = \left\{x\mapsto \dfrac{\lambda}{x}, \lambda \in \bdR\right\}$. \qquad Notons $h_0 \in S_H$, avec $\forall x \in I$, $h_0(x) = \dfrac{1}{x}$.
    
    Soit $\lambda \in \cc{1}{I}{\bdR}$ et $f_p(x) = \lambda(x)h_0(x)$ une solution de l'équation $\mathbf{(E)}: y' + \dfrac{1}{x}y = \dfrac{x}{x-1}$ sur $I$. Or :
    \[ f_p'(x) + \dfrac{f_p(x)}{x} = \lambda'(x)h_0(x) + \lambda(x){h_0}'(x) + \dfrac{\lambda(x)h_0(x)}{x} = \lambda'(x)h_0(x) + \lambda(x)\left[{h_0}'(x) + \dfrac{1}{x}h_0(x)\right] = \lambda'(x)h_0(x)\]
    
    Donc $\lambda'(x)h_0(x) = \dfrac{x}{x-1}$ donc $\lambda'(x) = \dfrac{x}{(x-1)h_0(x)} = \dfrac{x^2}{x-1} = \dfrac{x^2 - 1 + 1}{x-1} = x + 1 + \dfrac{1}{x-1}$
    
    Donc $\exists \ c \in \bdR$ tel que $\lambda(x) = \dfrac{x^2}{2} + x + \ln{x-1} + c$ d'où $f_p(x) = \dfrac{x}{2} + 1 + \dfrac{\ln{x-1}}{x} + \dfrac{c}{x}$.
    
    Finalement, \boxsol{$S_E = \left\{x \mapsto \dfrac{\lambda+\ln{x-1}}{x} + \dfrac{x}{2}+1, \lambda \in \bdR \right\}$}
    }
    
    \item Résoudre $2y' - 3y = \sin^2(t)$ sur $\bdR$.
    
    \boxans{
        $2y' - 3y = \sin^2(x) \iff y' - \dfrac{3}{2}y = \dfrac{\sin^2(x)}{2}$. On résout $\mathbf{(H)}: y' - \dfrac{3}{2}y = 0$ sur $\bdR$. De manière évidente :
   
        $S_H = \left\{ x \mapsto \lambda e^{\sfrac{3x}{2}}, \lambda \in \bdR\right\}$. On cherche une solution particulière à $\mathbf{(E)}: 2y' -3y = \sin^2(x) = \dfrac{1-\cos(2x)}{2}$.
        Par superposition, on cherche une solution $f_{p,1}$ à $\mathbf{(E_1)}: 2y' - 3y = \dfrac{1}{2}$ et $f_{p,2}$ à $\mathbf{(E_2)}: 2y' - 3y = -\dfrac{\cos(2x)}{2}$.
        Pour $\mathbf{(E_1)}$, on la cherche sous la forme d'une constante $f_{p,1}(x) = k$ : $-3k = \dfrac{1}{2} \iff k = -\dfrac{1}{6}$. 
        
        Pour $\mathbf{(E_2)}$, on passe dans le monde complexe en trouvant une solution $f_{p,2,\bdC}$ à $\mathbf{({E_2}^\bdC)} : 6y - 4y' = e^{2ix}$.
        
        On a alors $f_{p,2} = \Re{f_{p,2,\bdC}}$. On pose $f_{p,2,\bdC}(x) = \lambda e^{2ix}$. Alors:
        \[6f_{p,2,\bdC}(x) - 4{f_{p,2,\bdC}}'(x) = 6\lambda e^{2ix} - 8i\lambda e^{2ix} = \lambda e^{2ix} \left(6-8i\right) = e^{2ix} \iff \lambda = \dfrac{1}{6-8i} = \dfrac{6+8i}{100}\]
        Donc $f_{p,2} = \Re{f_{p,2,\bdC}} = \Re{\dfrac{6+8i}{100}\times e^{2ix}} = \dfrac{6\cos(2x) - 8\sin(2x)}{100} = \dfrac{3\cos(2x)-4\sin(2x)}{50}$
        
        Finalement, \boxsol{$S_E = \left\{ x \mapsto \lambda e^{\sfrac{3x}{2}} - \dfrac{1}{6}  + \dfrac{3\cos(2x)}{50}- \dfrac{2\sin(2x)}{25}, \lambda \in \bdR \right\}$}
    }
    
    \item Résoudre sur $\bdR$ l'équation $(1-x)y' - y = x$, avec recollement.
    
    \boxans{
    Soit l'équation différentielle $\mathbf{(E)}: (1-x)y' -y = x$ sur $\bdR$. On normalise l'équation $\mathbf{(E)}$ en l'équation $\mathbf{(E_n)}: y' - \dfrac{1}{1-x}y = \dfrac{x}{1-x}$ sur $I = \left]1;+\infty\right[$ ou $I = \left]-\infty;1\right[$. La résolution de l'équation sous forme homogène $\mathbf{(H)}: y' - \dfrac{1}{1-x}y = 0$ donne $S_H = \left\{ x \mapsto \dfrac{\lambda}{x-1}, \lambda \in \bdR \right\}$. 
    
    soit $h_0 \in S_H$ tel que $\forall x \in I$, $h_0(x) = \dfrac{1}{x-1}$ et $\lambda \in \cc{1}{I}{\bdR}$ tel que $f_p(x) = \lambda(x)h_0(x)$ est une solution de l'équation $\mathbf{(E_n)}$ sur $I$. Donc $f_p(x) - \dfrac{1}{1-x}f_p(x) = \lambda'(x)h_0(x) = \dfrac{x}{1-x} \iff \lambda'(x) = -x$
    
    Donc $\exists \ c \in \bdR$, $\lambda(x) = -\dfrac{x^2}{2} + c$ d'où $f_p = \dfrac{x^2+c}{2(1-x)}$. Donc $S_E = \left\{x \mapsto \dfrac{\lambda - \frac{x^2}{2}}{x-1}, \lambda \in \bdR\right\}$.
    Soit $f$ une solution de $\mathbf{(E)}$. $\exists \ (\lambda, \mu) \in \bdR^2$ tels que $f(x) = \left\lbrace\begin{array}{ll}
        \dfrac{\lambda - \frac{x^2}{2}}{x-1} & \text{si} \ x > 0 \\
        \dfrac{\mu - \frac{x^2}{2}}{x-1} & \text{si} \ x < 0
    \end{array}\right.$. Or $\mathbf{(E)}$ impose la continuité et la dérivabilité de $f$ en $1$. La limite de $f$ en $1$ à droite existe si et seulement si $\lambda - \dfrac{x^2}{2}$ annule $x-1$, soit si $\lambda = \dfrac{1}{2}$. De même à gauche, il faut que $\mu = \lambda = \dfrac{1}{2}$.
    En simplifiant, on a  $f(x) = \dfrac{1}{2}(x+1)$ sur $\bdR$, qui est bien dérivable sur $\bdR$.
    
    Finalement, \boxsol{l'équation  $\mathbf{(E)}$ admet une unique solution sur $\bdR$: $x \mapsto  \dfrac{1}{2}(x+1)$}
    }
    
    \item  Énoncer la forme des solutions de l’équation  $\mathbf{(H)}: y'' + ay' + by = 0$, avec $a$, $b$ deux constantes.
    
    Solutions dans $\bdK = \bdC$, et solutions dans $\bdK = \bdR$.
    
    \begin{lemma*}{}{}
        Soit $\varphi_r: x \mapsto e^{rx}$ défini et deux fois dérivable sur $I$.
            \[ \varphi_r \in S_H \iff r \ \text{est solution de} \ \mathbf{(EC)}\]
    \end{lemma*}
    
    \demo{
        Soit $\varphi_r: x \mapsto e^{rx}$ deux fois     dérivable sur $I$. On a ${\varphi_r}' = re^{rx}$ et ${\varphi_r}'' = r^2e^{rx}$. Donc :
        
        $ \varphi_r \in S_H \iff r^2e^{rx} + are^{rx} + be^{rx} = 0 \iff (r^2 + ar +b)e^{rx} = 0 \iff r^2 + ar + b = 0$
    }
    
    \boxans{
        On note l'équation caractéristique $\mathbf{(EC)}: r^2 + ar + b = 0$ sur $\bdK$ et $I$ un intervalle non vide de $\bdR$.
        
        \underline{Premier cas : $\bdK = \bdC$}. Soit $r$ une racine de $\mathbf{(EC)}$ et $f$ une solution de $\mathbf{(H)}$. On pose $\alpha: \begin{array}{ll}
            I &\to \bdK  \\
            x &\mapsto e^{-rx}f(x) 
        \end{array}$.
        
        Soit $x \in I$, $\left\lbrace\begin{array}{ll}
           f(x) &= e^{rx}\alpha(x) \\
            f'(x) &= e^{rx}\left[\alpha'(x) + r\alpha(x)\right]\\
            f''(x) &= e^{rx}\left[\alpha''(x) + 2r\alpha'(x) + r^2\alpha(x)\right]
        \end{array}\right.$. Or $f \in S_H$ donc $f''(x) + af'(x)+bf(x)=0$.
        
        Donc $e^{rx}\left[\alpha''(x) + 2r\alpha'(x) +r^2\alpha(x) + a(\alpha'(x) +r\alpha(x)) + b\alpha(x)\right] = 0$
        
        Donc $\alpha''(x) + (2r+a)\alpha'(x) + (r^2+ar+b)\alpha(x) = 0$
    
    Donc $\alpha''(x) + (2r+a)\alpha'(x) = 0$, donc $\alpha'$ est solution de $y' + (2r+a)y = 0$. 
        
    On distingue alors deux cas selon le discriminant $\Delta = a^2 - 4b$ de l'équation caractéristique. Si $\Delta = 0$, alors $r$ est une racine double telle que $r = -\dfrac{a}{2} \iff 2r+a = 0$. Donc $\alpha''(x) = 0$ donc $\exists (\lambda,\mu) \in \bdK^2$, $\alpha(x) = \lambda x + \mu$.
    
    Donc $f(x) = \left(\lambda x + \mu\right)e^{rx}$. Si $\Delta \neq 0$ alors $2r + a \neq 0$. On a alors deux racines distinctes $r = r_1$ et $r_2$, avec $r_1 + r_2 = -a \iff r_1 + a = -r_2$. L'équation différentielle sur $\alpha'$ donne:
    
    \[ \exists \ (\gamma,\mu) \in \bdK \in \bdK, \quad \alpha'(x) = \gamma e^{-(2r_1 +a)x} \ \text{donc} \ \alpha = \dfrac{\gamma}{-(2r_1 + \alpha)}e^{-(2r_1 +a)x} + \mu \]
    On pose $\lambda = \dfrac{\gamma}{-(2r_1 + \alpha)}$, on a alors $f(x) = \alpha(x)e^{r_1x} = \lambda e^{-(r_1 +a)x}+ \mu e^{r_1x} = \lambda e^{r_2x} + \mu e^{r_1x}$.
    
    Donc \boxsol{$S_H = \left\lbrace\begin{array}{ll}
        \left\{ x \mapsto (\lambda x + \mu)e^{rx}, (\lambda,\mu) \in \bdC^2\right\} & \text{si} \ \Delta = 0\\[5pt]
        \left\{ x \mapsto \lambda e^{r_1x} + \mu e^{r_2x}, (\lambda,\mu) \in \bdC^2\right\} & \text{si} \ \Delta \neq 0
    \end{array}\right.$}
    
    \text{}\\\text{}\\
    \underline{Deuxième cas : $\bdK = \bdR$}. Il suffit d'étudier le cas où les racines sont complexes, soit le cas où $\Delta < 0$.
    
    On garde donc $S_H = \left\lbrace\begin{array}{ll}
        \left\{ x \mapsto (\lambda x + \mu)e^{rx}, (\lambda,\mu) \in \bdR^2\right\} & \text{si} \ \Delta = 0\\[5pt]
        \left\{ x \mapsto \lambda e^{r_1x} + \mu e^{r_2x}, (\lambda,\mu) \in \bdR^2\right\} & \text{si} \ \Delta > 0
    \end{array}\right.$. Si $\Delta < 0$, alors on a $f_\bdC$ une solution dans $\bdC$ telle que $\exists \  (A, B) \in \bdC^2$ $f_\bdC(x) = Ae^{r_1x} + Be^{r_2x}$ et $f$ la solution dans $R$ est telle que $f = \Re{f_\bdC}$.
    
    On a $\Delta = a^2 - 4b = -(4b - a^2) = \left(i\sqrt{4b - a^2}\right)$ donc $r_1 = \dfrac{-a+i\sqrt{4b - a^2}}{2}$ et $r_2 = \dfrac{-a-i\sqrt{4b - a^2}}{2}$.
    
    On pose alors $\alpha = -\dfrac{a}{2}$ et $\beta = \dfrac{\sqrt{4b - a^2}}{2}$ de sorte que $r_1 = \alpha + i\beta$ et $r_2 = \alpha - i\beta$.
    
    On a $f(x) = \Re{f_\bdC(x)} = \dfrac{f_\bdC(x) + \overline{f_\bdC(x)}}{2} = \dfrac{1}{2}\left(Ae^{(\alpha+i\beta)x}+Be^{(\alpha-i\beta)x}+\overline Ae^{(\alpha+i\beta)x}+\overline Be^{(\alpha-i\beta)x}\right)$ donc:
    \[ f(x) = \dfrac{e^{\alpha x}}{2}\left[e^{i\beta x}(A+\overline{B}) + e^{-i\beta x}(B + \overline A)\right] =\dfrac{e^{\alpha x}}{2}\left[\cos{\beta x}(A + \overline{A}) + \sin{\beta x}(B + \overline B)\right] \]
    On pose $\lambda = \Re{A}$ et $\mu = \Re{\beta}$. Alors :
    \[ f(x) = \dfrac{e^{\alpha x}}{2}\left[\cos{\beta x}\times 2\lambda + \sin{\beta x}\times 2\mu \right] = \lambda e^{\alpha x}\cos{\beta x} + \mu e^{\alpha x}\sin{\beta x}\]
    
    Donc \boxsol{$S_H = \left\lbrace\begin{array}{ll}
        \left\{ x \mapsto (\lambda x + \mu)e^{rx}, (\lambda,\mu) \in \bdR^2\right\} & \text{si} \ \Delta = 0\\[5pt]
        \left\{ x \mapsto \lambda e^{r_1x} + \mu e^{r_2x}, (\lambda,\mu) \in \bdR^2\right\} & \text{si} \ \Delta > 0\\[5pt]
        \left\{ x \mapsto \lambda e^{\alpha x}\cos{\beta x} + \mu e^{\alpha x}\sin{\beta x}, (\lambda,\mu) \in \bdR^2\right\} & \text{si} \ \Delta < 0\\
    \end{array}\right.$}
     }
    
\end{enumerate}
\end{document}