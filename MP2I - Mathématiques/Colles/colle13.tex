\documentclass[a4paper,french,bookmarks]{article}
\usepackage{./Structure/4PE18TEXTB}

\renewcommand{\thesubsection}{\Roman{subsection}.}
\begin{document}

\stylizeDoc{Mathématiques}{Programme de khôlle 13}{Énoncés et résolutions}

\section*{Structures algébriques}

\subsection{Lois de composition interne}

\begin{enumerate}
    \ithand Associativité, commutativité. Exemples. Élément neutre, inversibilité. Distributivité.
    
    \ithand Partie stable pour une loi.
\end{enumerate}

\subsection{Structure de groupe}

\begin{enumerate}
    \ithand Définition. Exemples. Itéré d’un élément.

    \ithand Sous-groupe, caractérisation. Sous-groupes de $(\bdZ,+)$. Intersection de sous-groupes.

    \ithand Morphisme de groupes. Image et image réciproque d’un sous-groupe par un morphisme.

    \ithand Image et noyau d’un morphisme. Injectivité. Isomorphisme.
\end{enumerate}

\subsection{Structure d’anneau, de corps}

\begin{enumerate}
    \ithand Définition. Exemples. Calculs dans un anneau. Formule du binôme et $a^n - b^n$ si les éléments $a$ et $b$ commutent.
    
    \ithand Intégrité. Groupe des inversibles d’un anneau. Sous-anneau. Corps, sous-corps.
\end{enumerate}

\section*{Polynômes - Première partie}

\subsection{Structure des polynômes}

\begin{enumerate}
    \ithand Définition formelle par les suites presque nulles. Opérations $+$, $\cdot$,$\times$.
    
    \ithand Structure de $\bdK[X]$ : anneau intègre commutatif.
    
    \ithand Notation définitive : $X = (0, 1, 0, \dots)$.
    
    \ithand Propriété sur les degrés : somme, produit.
    
    \ithand Composition.
\end{enumerate}

\subsection{Diviseurs et division euclidienne}

\begin{enumerate}
    \ithand Multiples, diviseurs. Relation de divisibilité. Polynômes associés.
    
    \ithand Division euclidienne dans $\bdK[X]$ : si $B \neq 0,$ il existe un unique couple $(Q,R)$ vérifiant A = BQ + R et deg(R) < deg(B).
\end{enumerate}

\section*{Questions / Exercices de cours / Savoir faire}

\begin{enumerate}
    \item TODO
\end{enumerate}

\end{document}