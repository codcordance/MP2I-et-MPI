\documentclass[a4paper,french,bookmarks]{article}
\usepackage{./Structure/4PE18TEXTB}

\moretikzsetup

\renewcommand{\thesubsection}{\Roman{subsection}.}
\begin{document}

\stylizeDoc{Mathématiques}{Programme de khôlle 14}{Énoncés et résolutions}

\section*{Matrices - Calcul matriciel}

\begin{enumerate}
    \ithand Définition d’une matrice à $n$ lignes et $p$ colonnes à coefficients dans $\bdK$. Représentation.
    
    \ithand Matrices lignes, colonnes. Matrices carrées, matrice identité, matrices diagonales, matrices triangulaires supérieures ou inférieures.
    
    \ithand Matrices élémentaires $E_{k, \ell}$ de terme général $(\delta_{i, k}\delta_{j, l})_{(i, j) \in \llbracket 1, n\rrbracket \times \llbracket 1, p\rrbracket}$ pour $(k, \ell) \in \llbracket 1, n\rrbracket \times \llbracket 1, p\rrbracket$.
    
    \ithand Combinaisons linéaires de matrices en général. Combinaison linéaire des matrices élémentaires pour générer toutes les matrices, les matrices diagonales, les matrices triangulaires.
    
    \ithand Produit matriciel. Définition. Produit d’une matrice par une colonne. Ecriture matricielle d’un système linéaire. Produit de matrices élémentaires.
    
    \ithand  Propriétés du produit : associativité, élément neutre, distributivité à gauche et à droite par rapport à l’addition, multiplication par un scalaire.
    
    \ithand Structure d’anneau (non commutatif, non intègre) de $(\bcM_n(\bdK), +, \times)$ :
    
    \begin{enumerate}
        \itstar Puissances de matrices carrées. Formule du binôme, $(AB)^m$ et $A^m - B^m$ pour deux matrices qui commutent.
        
        \itstar Matrices inversibles. Groupe linéaire $GL_n(\bdK)$. Inverse d’un produit de matrice carrées inversibles. Caractérisation des matrices inversibles : équivalence entre inverse, inverse à droite et inverse à gauche (admis provisoirement).
        
        \itstar Calcul de l’inverse d’une matrice par résolution du système linéaire $AX = Y$.
        
        \itstar Sous-anneau des matrices diagonales. Critère d’inversibilité.
        
        \itstar Sous-anneau des matrices triangulaires supérieures. Critère d’inversibilité.
    \end{enumerate}
    
    \ithand Calcul de l’inverse d’une matrice. Algorithme du pivot de Gauss. Correction de l’algorithme via les matrices codant les opérations élémentaires $L_i \longleftrightarrow L_j$, $L_i \longleftarrow \lambda L_i$ (avec $\lambda \neq 0$), $L_i \longleftarrow L_i + \mu L_j$.
    
    \ithand Transposition. Linéarité de $A \mapsto {}^t A$, bijection de $\bcM_{n,p}(\bdK)$ sur $\\bcM_{p,n}(\bdK)$.
    
    Matrices symétriques et antisymétriques. Toute matrice de $\bcM_n(\bdK)$ s’écrit de façon unique comme la somme d’une matrice symétrique et d’une matrice antisymétrique. Transposée d’un produit. Transposée et matrice inverse.
    
    \ithand Trace d’une matrice carrée et propriétés.

\end{enumerate}

\section*{Questions de cours}

\begin{enumerate}
    \item Polynômes de Tchebychev : en admettant l’existence d’une famille $\suite{T_n}$ de polynômes tel que $T_n$ est de degré $n$, de coefficient dominant $2^{n-1}$, à coefficients entiers et vérifiant $\forall \theta \in \bdR,\qquad \cos \theta = \cos{n\theta}$. Résoudre $T_n(x) = 0$ sur $[-1, 1]$ et en déduire les racines et une factorisation de $Tn$.
    
    \boxans{
        Soit $n \in \bdR$. Cherchons les racines de $T_n$. On a $T_n$ de degré $n$ donc $T_n$ a au plus $n$ racines.
        
        Cherchons alors $n$ racines distinctes dans $]-1, 1[$. Soit $\theta \in ]0; \pi[$, on a $\cos(n\theta) \in ]-1; 1[$.
        
        Cherchons les $\theta$ tels que $T_n(\cos \theta) = 0$, c'est-à-dire $\cos(n\theta) = 0$. On a alors :
        
        \[n\theta \equiv \dfrac{\pi}{2} \ [\pi] \qquad\text{donc}\qquad \exists k \in \bdN \qquad\text{donc}\qquad n\theta =  k\pi + \dfrac{\pi}{2} \qquad\text{donc}\qquad \theta = \dfrac{(2k+1)\pi}{2n}\]
        
        La fonction $\cos$ est bijective de $]0; \pi[$ sur $]-1; 1[$, et l'on a les équivalences suivantes :
        
        \[ \theta \in ]0; \pi[ \iff 0 < \dfrac{(2k+1)\pi}{2n} < \pi \iff -1 < k < n-1 \iff 0 \leq k \leq n-1\]
        
        Donc \boxsol{$\forall k \in \llbracket0, n-1\rrbracket, \qquad \cos{\dfrac{(2k+1)\pi}{2n}} \in ]-1;1[$ est racine de $T_n$}. $T_n$ est alors scindé sur $\bdR$, ces $n$ racines étant toutes distinctes et $T_n$ étant de degré $n$. En vertu de cela, on obtient finalement la factorisation :
        
        \[ \boxsol{$T_n = 2^{n-1}\displaystyle\prod_{k=0}^{n-1}\left(X - \cos{\dfrac{(2k+1)\pi}{2n}}\right)$} \]
    }
    
    \newpage
    
    \item Factoriser $P = 1 + X^2 + X^4 + \dots + X^{2n - 2}$ dans $\bdC[X]$ et dans $\bdR[X]$. 
    
    \boxans{
        Pour $X \neq 1$, on a par somme géométrique $P = 1 + X^2 + X^4 + \dots + X^{2n - 2} = \sum_{k=0}^{n-1} \displaystyle \left(X^2\right)^{k} = \dfrac{X^{2n}-1}{X^2-1}$.
        \[ \dfrac{X^{2n}-1}{X^2-1} = 0 \iff X^{2n} = 1 \land X^2 \neq 1 \iff X \in \bdU_{2n}\backslash\{1, -1\} \iff \exists k \in \llbracket 1, 2n-1\rrbracket\backslash\{\sfrac{n}{2}\}, \ X = e^{\sfrac{2ik \pi}{2n}}\]
        On a $\deg(P) = 2n-1$ racines distinctes et $P$ unitaire, d'où \boxsol{$\displaystyle P = \prod_{1\leq k \leq 2n-1, k \neq \sfrac{n}{2}} \left(X - e^{\sfrac{ik\pi}{n}}\right)$}.
        
        On a la factorisation avec les racines conjuguées $P = \prod_{k=1}^{n-1}\left(X-e^{\sfrac{ik\pi}{n}}\right)\left(X-\overline{e^{\sfrac{ik\pi}{n}}}\right)$, et donc en développant ce produit on obtient la factorisation dans $\bdR$ : \boxsol{$P = \displaystyle\prod_{k=1}^{n-1} \left(X^2 - 2\cos{\sfrac{k\pi}{n}} + 1\right)$}.
    }
    
    \item Montrer que pour $A \in \bcM_{n,p}(\bdK)$, $B \in \bcM_{p,q}(\bdK)$ et $C \in \bcM_{q,r}(\bdK)$, on a $(AB)C$ = $A(BC)$.
    
    \boxans{
        On a $AB \in \bcM_{n, q}(\bdK)$ d'où $(AB)C \in \bcM_{n, r}(\bdK)$. De plus $BC \in \bcM_{p, r}(\bdK)$, d'où $A(BC) \in \bcM_{n, r}(\bdK)$. Alors : 
        \begin{align*}
            \forall (i, j) \in \llbracket 1 , n\rrbracket \times \llbracket 1, r\rrbracket, \qquad [(AB)C]_{i, j} &= \sum_{k=1}^{q} [AB]_{i, k}[C]_{k, j} = \sum_{k=1}^{q}\left(\sum_{\ell=1}^{p}[A]_{i, \ell}[B]_{\ell, k}\right)[C]_{k, j}\\
            &= \sum_{\ell=1}^{p}[A]_{i, \ell}\left(\sum_{k=1}^q [B]_{\ell, k}[C]_{k, j}\right) = \sum_{\ell = 1}^{n} [A]_{i, \ell}[BC]_{\ell, j} = [A(BC)]_{i, j}\\
            &\text{On obtient donc bien \boxsol{$(AB)C = A(BC)$}.}
        \end{align*}
    }
    
    \item Montrer que le produit de deux matrices triangulaires supérieures de $\bcM_n(\bdK)$ est une matrice triangulaire supérieure.
    
    \boxans{
        Soient deux matrices triangulaires supérieurs $(A, B) \in \bcT_{n}^+(\bdK)^2$. Montrons que $AB \in \bcT_{n}^+(\bdK)$. On a :
        \[\forall (i, j) \in \llbracket 1, n\rrbracket^2, \qquad i < j \implies [A]_{i, j} = [B]_{i, j} = 0\]
        On a $(A, B) \in \bcM_n(\bdK)^2$ d'om $AB \in \bcM_n(\bdK)$. Montrons que $\forall (i, j) \in \llbracket 1, n\rrbracket^2, \qquad i < j \implies [AB]_{i, j} = 0$.
        \[ \forall (i, j) \in \llbracket 1, n\rrbracket^2, \qquad [AB]_{i, j} = \sum_{k=1}^{n} [A]_{i, k}[B]_{k, j} = \sum_{k=1}^{j-1}[A]_{i, k}\underbrace{[B]_{k, j}}_{=0} + \sum_{k=j}^{n}[A]_{i, k}[B]_{k, j} = \sum_{k=j}^{n}[A]_{i, k}[B]_{k, j}\]
        Donc $\forall (i, j) \in \llbracket 1, n\rrbracket^2, \qquad i < j \implies \left(\forall k \in \llbracket j, n\rrbracket, \ k \geq j \implies i < k \implies [A]_{i, k} = 0\right) \implies [AB]_{i, j} = 0$.
        
        Donc finalement \boxsol{$AB \in \bcT_{n}^+(\bdK)$}.
    }
    
    \item Définition de la transposée d’une matrice de $\bcM_{n,p}(\bdK)$. Montrer que $^t(AB) = ^tB \times {}^tA$.
    
    \begin{definition*}{Transposée d'une matrice}{}
        On appelle \bf{transposée de $A \in \bcM_{n, p}(\bdK)$} la matrice $\hg{^tA \in \bcM_{p, n}(\bdK)}$ telle que \[\hg{\forall (i, j) \in \llbracket 1, n\rrbracket \times \llbracket 1, p\rrbracket,\ [A]_{i, j} = [^tA]_{j, i}}\]
        
        On peut voir la transposée d'une matrice comme son symétrique par rapport à l'axe diagonal :

\[\begin{array}{c}
        {} \\
        {} \\
        {} \\
        {}
        \end{array}^t\begin{pNiceMatrix}
\CodeBefore [create-cell-nodes]
    \tikz {
        \draw[-, dashed, line width=0.35mm, line cap=round, main1!65] ([xshift=-6pt,yshift=3pt]1-1.north west) -- ([xshift=6pt,yshift=-3pt]4-5.south east);
        \draw[gradientarrow={0.35pt}{main1}{main5}] ([xshift=0pt,yshift=-5pt]4-5.south east) to[bend left=90,looseness=2] ([xshift=6pt,yshift=0pt]4-5.south east); 
    }
\Body
    a_{1,1} & \Cdots & \Cdots & a_{1,j} & a_{1,p}  \\
        \Vdots &  & & \vdots  & \Vdots \\
        a_{i, 1} & \cdots & \cdots & a_{i, j} & a_{i, p}\\
        a_{n, 1} & \Cdots & \Cdots &  a_{n, j} & a_{n, p}
\end{pNiceMatrix} = \begin{pNiceMatrix}
a_{1,1} & \Cdots & a_{i, 1} & a_{n, 1}\\
\Vdots & & \vdots & \Vdots\\
\Vdots & & \vdots & \Vdots\\
a_{1, j} & \cdots & a_{i, j} & a_{n, j}\\
a_{1,p} & \Cdots & a_{i, p} & a_{n, p}\\
\end{pNiceMatrix}\]
    \end{definition*}
    
    \begin{property*}{Transposée d'un produit}{}
        \[ \hg{\forall A \in \bcM_{n, p}(\bdK), \qquad \forall B \in \bcM_{p, q}(\bdK),\qquad {}^t(AB) = ^tB \times {}^tA}\]
    \end{property*}
    
    \demo{
        Soit $A \in \bcM_{n, p}(\bdK)$ et $B \in \bcM_{p, q}(\bdK)$. On a $^t B \times {}^t A \in \bcM_{q, n}(\bdK)$ et $^t(AB) \in \bcM_{q, n}(\bdK)$ et :
        
        \[\forall (i, j) \in \llbracket 1, q\rrbracket\times\llbracket1, n\rrbracket,\qquad \left[{}^t(AB)\right]_{i, j} = [AB]_{j, i} = \sum_{k=1}^{p} [A]_{j, k}[B]_{k, i} = \sum_{k=1}^{p} \left[{}^tB\right]_{i, k}\left[{}^tA\right]_{k, j} = \left[{}^tB \times {}^tA\right]_{i, j}\]
        
        On a bien ${}^t(AB) = ^tB \times {}^tA$.
    }
    \item  Définition de la trace d’une matrice de $\bcM_n(\bdK)$. Montrer que $\Tr(AB) = \Tr(BA)$.
    
    \begin{definition*}{Trace d'une matrice}{}
        La \bf{trace d'une matrice $A \in \bcM_{n}(\bdK)$}, notée $\hg{\Tr(A)}$, est la somme de ses coefficients diagonaux.
        
        \[ \hg{Tr(A) = \sum_{i=0}^n [A]_{i, i}}\]
    \end{definition*}
    
    \begin{property*}{Trace d'un produit}{}
        \[ \hg{\forall (A,B) \in \bcM_n(\bdK)^2, \qquad \Tr(AB) = \Tr(BA)}\]
    \end{property*}
    
    \demo{
        Soit $(A, B) \in \bcM_n(\bdK)^2$. On a :
        
        \[ \Tr(AB) = \sum_{i=0}^n [AB]_{i, i} = \sum_{i=0}^n \sum_{j=0}^n [A]_{i, j}[B]_{j, i} = \sum_{j=0}^n \sum_{i=0}^n [B]_{j, i}[A]_{i, j} \sum_{j=0}^n [BA]_{j, j} = \Tr(BA)\]
        
        On a bien $\Tr(AB) = \Tr(BA)$.
    }
    
    \item Montrer que toute matrice de $\bcM_n(\bdK)$ s’écrit de façon unique comme la somme d’une matrice symétrique et d’une matrice antisymétrique.
    
    \boxans{
        \textit{Analyse.} Soit $M \in \bcM_{n}(\bdK)$. On cherche $S \in \bcS_n(\bdK)$ et $A \in \bcA_n(\bdK)$ telles que $S + A = M$.
        
        On a alors $^tM = {}^t(S + A) = {}^tS + {}^tA = S - A$. Donc $M + {}^tM = 2S$ et $M - {}^tM = 2A$.
        
        On a donc : $S = \dfrac{M + {}^tM}{2}$ et $A = \dfrac{M - {}^tM}{2}$.
        
        \textit{Synthèse.} Soit $M \in \bcM_{n}(\bdK)$. On pose $S = \dfrac{M + {}^tM}{2}$ et $A = \dfrac{M - {}^tM}{2}$.
        
        \begin{enumerate}
            \ithand Montrons que $S \in \bcS_n(\bdK)$. On a :
            
            \[[S]_{i, j} = \left[\dfrac{M+{}^tM}{2}\right]_{i, j} = \left[\dfrac{M}{2}\right]_{i, j} + \left[\dfrac{{}^tM}{2}\right]_{i, j} = \left[\dfrac{{}^tM}{2}\right]_{j, i} + \left[\dfrac{M}{2}\right]_{j, i} = \left[\dfrac{{}^tM + M}{2}\right]_{j, i} = [S]_{j, i}\]
            
            On a bien $S \in \bcS_n(\bdK)$.
            
            \ithand Montrons que $A \in \bcA_n(\bdK)$. On a :
            
            \[[S]_{i, j} = \left[\dfrac{M-{}^tM}{2}\right]_{i, j} = \left[\dfrac{M}{2}\right]_{i, j} - \left[\dfrac{{}^tM}{2}\right]_{i, j} = \left[\dfrac{{}^tM}{2}\right]_{j, i} - \left[\dfrac{M}{2}\right]_{j, i} = -\left[\dfrac{M - {}^tM}{2}\right]_{j, i} = -[S]_{j, i}\]
            
            On a bien $A \in \bcA_n(\bdK)$.
            
            \ithand Montrons enfin que $M = A + S$. On a :
            
            \[ S + A = \dfrac{M + {}^tM}{2} + \dfrac{M - {}^tM}{2} = \dfrac{M + {}^tM + M - {}^tM}{2} = \dfrac{2M}{2} = M\]
            
            On a bien finalement $M = A + S$.
        \end{enumerate}
    }

    \item Calculer (par binôme, ou polynôme annulateur, au choix) les puissances de la matrice $H_p \in \bcM_p(\bdR)$ définie par $h_{i,i} = 0$ et $h_{i,j} = 1$ pour $i \neq j$.
    
    \boxans{
        On a en fait :
        
        \[ H_p = \begin{pNiceMatrix}
            0       &   1    & \Cdots & \Cdots & 1          \\
            1       & \Ddots & \Ddots &        & \Vdots     \\
            \Vdots  & \Ddots & 0 & \Ddots & \Vdots     \\
            \Vdots  &        & \Ddots & \Ddots & 1          \\
            1       & \Cdots & \Cdots & 1      & 0
        \end{pNiceMatrix} = \begin{pNiceMatrix}
            1       & \Cdots & 1      & \Cdots & 1          \\
            \Vdots  & \Ddots & \Vdots &        & \Vdots     \\
            1       & \Cdots & 1      & \Cdots & 1     \\
            \Vdots  &        & \Vdots & \Ddots & \Vdots          \\
            1       & \Cdots & 1      & \Cdots & 1
        \end{pNiceMatrix} - \begin{pNiceMatrix}
            1       &   0    & \Cdots & \Cdots & 0          \\
            0       & \Ddots & \Ddots &        & \Vdots     \\
            \Vdots  & \Ddots & 1 & \Ddots & \Vdots     \\
            \Vdots  &        & \Ddots & \Ddots & 0          \\
            0       & \Cdots & \Cdots & 0      & 1
        \end{pNiceMatrix}\]
        
        On pose alors $K_p$ tel que $H_p = K_p - I_p$, d'où $H_p^n = (K_p - I_p)^n$. Or $K_p$ et $I_p$ commutent donc :
        
        \[ H_p^n = \binom{n}{k}\sum_{k=0}^n K_p^n (-1)^{n-k}I_p^{n-k}\]
        
        On remarque que pour $k \neq 0$, $K_p^k = p^{k-1}K_p$, donc :
        
        \[ \boxsol{$\displaystyle H_p^n = (-1)^I_p + K_p\sum_{k=1}^n \binom{n}{k} p^{k-1}(-1)^{n-k}$}\]
    }
    

\end{enumerate}

\end{document}