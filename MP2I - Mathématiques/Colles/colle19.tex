\documentclass[a4paper,french,bookmarks]{article}
\usepackage{./Structure/4PE18TEXTB}

\newboxans

\begin{document}

\stylizeDoc{Mathématiques}{Programme de khôlle 19}{Énoncés et
résolutions}

\section*{Algèbre Linéaire (dimension finie)}

\subsection*{Familles libres, familles génératrices, bases}

\begin{enumerate}
    \ithand Familles génératrices. Familles libres, familles liées. Des exemples : dans $\bdR^n$, $\bdR^\bdN$, $\bcF\left(\bdR,
    \bdR\right)$, les familles de polynômes de degré échelonnés sont libres dans $\bdR\left[X\right]$. Bases. Coordonnées d'un vecteur dans une base.
    
    \ithand Familles de vecteur et somme. Soit $F$ et $G$ deux sous-espaces vectoriels de $E$. La concaténation d'une famille génératrice de $F$ et de $G$ donne une famille génératrice de $F + G$.
    
    Si $F$ et $G$ sont en somme directe, la concaténation d'une famille libre de $F$ avec une famille libre de $G$ donne une famille libre de $F \oplus G$. Si $E = F \oplus G$, base de $E$ par concaténation d'une base de $F$ et de $G$.
    
    \ithand Familles de vecteurs et applications linéaires : 
    
    Si $\bsF$ famille génératrice de $E$ et si $f ∈ \bcL\left(E, F\right)$, on a $\Imm f = \Vect\left(f\left(\bsF\right)\right)$.
    
    L'image d'une famille génératrice engendre l'image. L'image d'une famille génératrice par une application linéaire surjective est génératrice. L'image d'une famille libre par une application linéaire injective est libre.
    
    Si $\bcB$ est une base de $E$, $f$ est un isomorphisme de $E$ dans $F$ si et seulement si l'image par $f$ des vecteurs de $\bcB$ forme une base de $f$. Si $\bcB$ est une base, la donnée des images des vecteurs de $\bcB$ définit une unique application linéaire.
\end{enumerate}

\subsection*{Théorie de la dimension}

\begin{enumerate}
    \ithand Theorème : si $E$ est de dimension finie, $E$ est isomorphe à $F$ ssi $F$ est de dimension finie égale à celle de $E$.
    
    Isomorphisme naturel entre un ev de dimension $n$ et $\bdK^n$ si une base de $E$ est fixée.
    
    \ithand Théorème du rang : Si $E$ est un ev de dimension finie et $f \in \bcL\left(E, F\right)$. La restriction de $f$ à un supplémentaire $G$ du noyau réalise un isomorphisme de $G$ sur $\Imm f$. $\dim E = \dim ' f + \rg f$.
    
    \ithand Injectivité. Caractérisation par $\rg f = \dim E$ ou $\dim ' f = 0$. Surjectivité : caractérisation avec $\rg f = \dim F$. Comparaison des dimensions de $E$ et $F$ si f injective, surjective ou bijective.
    
    \ithand Caractérisation des isomorphismes en dimension finie. Théorème : si $\dim E = \dim F$ (finie), une application linéaire $f \in \bcL\left(E, F\right)$ est injective ssi elle est bijective ssi elle est surjective. En particulier : pour un endomorphisme en dimension finie : injectivité équivaut à surjectivité.
    
    \ithand  Dimension de $\bcL\left(E, F\right)$. Invariance du rang par opérations élémentaires ($C_i \leftrightarrow C_j$, $C_i \leftarrow \lambda C_i$, avec $\lambda \neq 0$, et $C_i \leftarrow C_i + \mu C_j$). Calcul pratique du rang : matrice d'une famille de vecteurs et pivot de Gauss. Les vecteurs colonnes non nuls d'une matrice échelonnée forment une famille libre.
    
    \ithand Hyperplans et formes linéaires en dimension finie. Résultats en dimension finie (notons $n = \dim E$ et fixons $\bcB$ une base de $E$) : un hyperplan est un sev de $E$ de dimension $n - 1$. Éq. cartésienne d'un hyperplan relativement à $\bcB$. Formes linéaires coordonnées. Espace $\bcL\left(E,K\right)$ et base des formes linéaires coordonnées. Dimension d'une intersection d'hyperplans. Tout sev en dimension finie $p$ est une intersection de $n − p$ hyperplans.
\end{enumerate}

\questionsdecours

\textbf{\sffamily N.B.} La colle comportera l'étude d'une suite récurrente linéaire d'ordre 2 donnée par l'examinateur.

\begin{enumerate}
    \item Montrer qu'une famille de polynômes de degrés échelonnés est libre dans
    $\bdK\left[X\right]$.
    
    Soit $\alpha \in \bdK$. Montrer que la famille de polynômes $\left(\left(X −
    \alpha\right)^k\right)_{k \in \left\llbracket 0, n\right\rrbracket}$ est une base de
    $\bdK_n\left[X\right]$.
    
    \boxans{
        \begin{enumerate}
            \itt Soit $n \in \bdN$ et $\left(P_k\right)_{k \in \left\llbracket 1,
            n\right\rrbracket} \in \bdK\left[X\right]^n$ une famille de $n$ polynômes
            qu'on prend par hypothèse échelonnés en degrés, en l'ordonnant de telle sorte
            que :
            %
            \[ 0 \leq \deg P_1 < \deg P_2 < \dots < \deg P_{n-1} < P_n\]
            %
            Soient $\left(\lambda_k\right)_{k \in \left\llbracket 1, n\right\rrbracket} \in \bdK^n$ une famille de scalaires telle que $\sum_{k = 1}^n \lambda_k P_k = 0_{\bdK\left[X\right]}$. On suppose par l'absurde que les $\lambda_k$ ne sont pas tous nuls, et on pose $r = \max\left\{k \in \left\llbracket 1,n\right\rrbracket \middle\vert \lambda_k \neq 0 \right\}$. On a alors $P_r = -\sfrac{1}{\lambda_r}\sum_{k=1}^{r-1} \lambda_k P_k$. Or le degré du polynôme $P_r$ est supérieur à celui de tous les polynômes $P_k$ de la somme, donc il y a contradiction. La famille est donc bien libre.
            
            \itt La famille de polynôme $\left(\left(X − \alpha\right)^k\right)_{k \in \left\llbracket 0, n\right\rrbracket}$ est échelonnée en degré donc libre. Par ailleurs elle contient $n+1$ vecteurs et $\dim \bdK_n\left[X\right] = n + 1$ donc c'est une base de $\bdK_n\left[X\right]$.
        \end{enumerate}
    }
    
    \item Caractérisation des supplémentaires en dimension finie.
    
    \noafter
    %
    \boxans{
        \begin{theorem}{Caractérisation des supplémentaires en dimension finie}{}
            Soient $E$ un $\bdK$-espace vectoriel de dimension finie et $F$ et $G$ deux $\bdK$-sous-espaces vectoriels de $E$. Les propositions suivantes sont équivalentes :
        
            \begin{psse}
                \item \hg{$E = F \oplus G$}
                \item \hg{$F \cap G = \left\{0_E\right\}$ et $\dim E = \dim F + \dim G$}
                \item \hg{$E = F + G$ et $\dim E = \dim F + \dim G$}
            \end{psse}
        \end{theorem}
    }
    %
    \yesafter\nobefore
    %
    \begin{nproof}
        Soient $E$ un $\bdK$-espace vectoriel de dimension finie et $F$ et $G$ deux $\bdK$-sous-espaces vectoriels de $E$.
        %
        \begin{enumerate}
            \itt $\boxed{\text{\EBGaramond\itshape(i)} \implies \text{\EBGaramond\itshape(ii)}}$ Supposons que $E = F \oplus G$. On a bien $F \cap G = \left\{0_E\right\}$ par somme directe et :
            %
            \[\dim E = \dim \left( F + G \right) = \dim F + \dim G - \dim \left(F \cap G\right) = \dim F + \dim G + \dim \left\{0_E\right\} = \dim F + \dim G\]
            
            \itt $\boxed{\text{\EBGaramond\itshape(ii)} \implies \text{\EBGaramond\itshape(iii)}}$ Supposons que $F \cap G = \left\{0_E\right\}$ et que $\dim E = \dim F + \dim G$. On a :
            %
            \[ \dim \left(F + G\right) = \dim F + \dim G - \dim \left(F \cap G\right) = \dim F + \dim G - \dim \left\{0_E\right\} = \dim F + \dim G = \dim E\]
            %
            Or $F \subset E$ et $G \subset E$, donc $F + G \subset E$. On a inclusion et égalité de dimension donc $E = F + G$.
            
            \itt $\boxed{\text{\EBGaramond\itshape(iii)} \implies \text{\EBGaramond\itshape(i)}}$ Supposons que $E = F + G$ et que $\dim E = \dim F + \dim G$. On a :
            %
            \[ \dim E = \dim \left(F + G\right) = \dim F + \dim G - \dim \left(F \cap G\right) \qquad\et\qquad \dim E = \dim F + \dim G\]
            %
            On a donc $\dim \left(F \cap G\right) = 0$, donc $F \cap G = \left\{0_E\right\}$, donc $E = F \oplus G$.
        \end{enumerate}
    \end{nproof}
    %
    \yesbefore
    
    \item Montrer le théorème du rang.
    
    \noafter
    %
    \boxans{
        \begin{lemma}{Isomorphisme induit par restriction}{}
            Soient $E$ et $F$ deux $\bdK$-espaces vectoriels, $f \in \bcL\left(E,
            F\right)$ une application linéaire entre ces deux espaces, et $G$ un
            supplémentaire de $\Ker f$ dans $E$.
            %
            \[ \hg{f{}_{\vert G}^{\vert \Imm f} : \begin{array}[t]{rcl}
                G &\to& \Imm f  \\
                x &\mapsto& f(x) 
            \end{array}\ \text{est un isomorphisme de} \ G \
            \text{sur} \ \Imm f} \]
        \end{lemma}
    }
    %
    \nobefore
    %
    \begin{nproof}
        Soient $E$ et $F$ deux $\bdK$-espaces vectoriels, $f \in \bcL\left(E, F\right)$ une application linéaire entre ces deux espaces, et $G$ un supplémentaire de $\Ker f$ dans $E$.
        %
        \begin{enumerate}
            \itt Soit $x \in \Ker f{}_{\vert G}^{\vert \Imm f}$, donc $x \in \Ker f \cap G$. Or $G$ est un supplémentaire de $\Ker f$ dans $E$ donc $x = 0_E$. Ainsi $f{}_{\vert G}^{\vert \Imm f}$ est injective.
            
            \itt $\Imm f{}_{\vert G}^{\vert \Imm f} = \Imm f \cap \Imm f = \Imm f$ donc $f$ est surjective.
        \end{enumerate}
        
        $f{}_{\vert G}^{\vert \Imm f}$ étant injective et surjective, elle est donc bijective. C'est bien un isomorphisme.
    \end{nproof}
    %
    \boxans{
        \begin{theorem}{Théorème du rang}{}
            Soient $E$ et $F$ deux $\bdK$-espaces vectoriels avec $E$ de dimension finie, et $f \in \bcL\left(E, F\right)$ une application linéaire.
            %
            \[ \hg{\dim E = \dim \Ker f + \rg f}\]
        \end{theorem}
    }
    %
    \yesafter
    %
    \begin{nproof}
        Soient $E$ et $F$ deux $\bdK$-espaces vectoriels avec $E$ de dimension finie, et $f \in \bcL\left(E, F\right)$ une application linéaire. Soit $G$ un complémentaire de $\Ker f$ dans $E$. On a donc :
        %
        \[E = \Ker f \oplus G \qquad\text{donc}\qquad \dim E = \dim \Ker f + \dim G - \dim \left(\Ker\left(f\right) \cap G\right) = \dim \Ker f + \dim G\]
        %
        Or $f$ induit un isomorphisme avec $f{}_{\mid G}^{\Imm f}$ entre $G$ et $\Imm f$, ces deux espaces sont donc de même dimension : $\dim G = \dim \Imm f = \rg f$. On a donc bien $\dim E = \dim \Ker f + \rg f$.
    \end{nproof}
    %
    \yesbefore
    
    \newpage
    
    \item Équivalence entre injectivité, bijectivité et surjectivité pour $f \in \bcL\left(E, F\right)$, avec $E$ et $F$ deux ev de même dimension finie.
    
    \noafter
    %
    \boxans{
        \begin{theorem}{Caractérisation des isomorphismes en dimension finie}
            Soient $E$ et $F$ deux $\bdK$-espaces vectoriels de même dimension finie, et $f \in \bcL\left(E, F\right)$ une application linéaire entre ces deux espaces. Les propositions suivantes sont équivalentes :
            %
            \begin{psse}
                \item \hg{$f$ est injective}
                \item \hg{$\rg f = \dim E = \dim F$}
                \item \hg{$f$ est surjective}
                \item \hg{$f$ est un isomorphisme}
            \end{psse}
        \end{theorem}
    }
    %
    \yesafter\nobefore
    %
    \begin{nproof}
        Soient $E$ et $F$ deux $\bdK$-espaces vectoriels de même dimension finie, et $f \in \bcL\left(E, F\right)$ une application linéaire entre ces deux espaces.
        
        \begin{enumerate}
            \itt $\boxed{\text{\EBGaramond\itshape(i)} \implies \text{\EBGaramond\itshape(ii)}}$ Supposons $f$ injective, donc $\Ker f = \left\{0_E\right\}$, donc $\dim \Ker f = 0$ donc par \textit{théorème du rang} $\rg f = \dim E = \dim F$.
            
            \itt $\boxed{\text{\EBGaramond\itshape(ii)} \implies \text{\EBGaramond\itshape(iii)}}$ Si $\rg f = \dim F$, alors $\dim \Imm f = \dim F$. Or $\Imm f \subset F$ donc par inclusion et égalité de dimension, $\Imm f = F$, donc $f$ surjective.
            
            \itt $\boxed{\text{\EBGaramond\itshape(iii)} \implies \text{\EBGaramond\itshape(i)}}$ Si $f$ est surjective, $\Imm f = F$ donc $\rg f = \dim F = \dim E$ donc par \textit{théorème du rang} $\dim \Ker f = 0$, donc $f$ est injective.
            
            \itt $\boxed{\text{\EBGaramond\itshape(i)} \iff \text{\EBGaramond\itshape(iv)}}$. Si $f$ est un isomorphisme alors $f$ est injective. Si $f$ est injective, on a $\text{\EBGaramond\itshape(i)} \iff \text{\EBGaramond\itshape(iii)}$ donc $f$ est surjective, donc $f$ est un isomorphisme.
        \end{enumerate}
    \end{nproof}
    %
    \yesbefore
    
    \item Soit $E$ un ev de dimension finie et $f \in \bcL\left(E\right)$ un endomorphisme de $E$. Soit $g \in \bcL\left(E\right)$ un endomorphisme de $E$ vérifiant $g \circ f = \Id_E$. Montrer que $f$ est un isomorphisme avec $f^{-1} = g$.
    
    \boxans{
        Si $g \circ f = \id_E$, alors $g \circ f$ est bijective (puisque $\id_E$ est bijective) donc $f$ est injective. Par \textit{caractérisation des isomorphismes en dimension finie}, $f$ est un isomorphisme, donc $f^{-1}$ existe. 
        
        Or $g \circ f = \id_E$ donc $g \circ f \circ f^{-1} = \id_E \circ f^{-1}$ soit finalement $f^{-1} = g$.
    }
    
    \item Montrer que $\dim \bcL\left(E, F\right) = \dim E \times \dim F$
    
    \noafter
    %
    \boxans{
        \begin{property}{Dimension de l'espaces des applications linéaires}{}
            Soient $E$ et $F$ deux $\bdK$-espaces vectoriels de dimension finie. \hg{$\dim \bcL\left(E, F\right) = \dim E \times \dim F$}.
        \end{property}
    }
    %
    \nobefore\yesafter
    %
    \begin{nproof}
        Soient $E$ et $F$ deux $\bdK$-espaces vectoriels de dimension finie. On note $p = \dim E \in \bdN$ et $\bcB_E = \left(e_1, e_2, \dots, e_p\right)$ une base de $E$. On introduit :
        %
        \[ \varphi : \begin{array}[t]{rcl}
            \bcL\left(E, F\right) &\to& F^p  \\
            f &\mapsto& \left(f\left(e_1\right), f\left(e_2\right), \dots, f\left(e_p\right)\right) = f\left(\bcB\right)
        \end{array}\]
        %
        $\varphi$ est entièrement caractérisée par les images par $f$ des vecteurs de la base $\bcB$. Donnons-nous $\left(f_1, f_2, \dots, f_n\right) \in F^p$. Il existe une unique application linéaire $f \in \bcL\left(E, F\right)$ telle que $f\left(e_1\right) = f_1$, $f\left(e_2\right) = f_2$, $\dots$, $f\left(e_n\right) = e_n$.
        
        On obtient ainsi la bijectivité de $\varphi$.
        
        La linéarité de $\varphi$ s'obtient directement par la linéarité de $F^p$ et des éléments de $\bcL\left(E, F\right)$. Donc $\varphi$ est un isomorphisme. Par \textit{théorème du rang}, on obtient :
        %
        \[ \dim \bcL\left(E, F\right) = \dim \varphi = \dim \Ker \varphi + \rg \varphi = 0 + \dim F^p = p \dim F = \dim E \times \dim F\]
    \end{nproof}
    %
    \yesbefore
    
    \item Soient $H_1, H_2, \dots, H_p$ des hyperplans de $E$ de dimension finie $n$. Montrer que $n - p \leq \displaystyle \bigcap_{î=1}^p H_i$.
    
    \boxans{
        Pour chaque hyperplan $H_i$, on considère une forme linéaire $\varphi_i \in \bcL\left(E, \bdK\right)$ telle que $H_i = \Ker \varphi_i$.
        
        On se donne $\psi : \begin{array}[t]{rcl}
            E &\to&\bdK^p  \\
            x &\mapsto& \left(\varphi_1\left(x\right), \varphi_2\left(x\right), \dots, \varphi_p\left(x\right)\right)
        \end{array}$. $\psi$ est linéaire et par \textit{théorème du rang}, $\dim E = \dim \Ker \psi + \rg \psi$. Or $\rg \psi \leq \dim \bdK^p = p$. Soit $x \in \Ker \psi$. On a les équivalences :
        %
        \[ \varphi\left(x\right) = 0_{\bdK_p} \iff \left(\varphi_1\left(x\right), \varphi_2\left(x\right), \dots, \varphi_p\left(x\right)\right) = \left(0, 0, \dots, 0\right) \iff \varphi_1\left(x\right) = 0 \et \varphi_2\left(x\right) = 0 \et \dots \et \varphi_p\left(x\right) = 0\]
        %
        Donc $\Ker \psi = \bigcap_{i=1}^p \Ker \varphi_I = \bigcap_{i=1}^p H_i$. Donc $n \leq \dim \Ker \psi + p$, soit $n - p \leq \Ker \psi$. Donc $n - p \leq \displaystyle \bigcap_{î=1}^p H_i$.
    }
    
    \item Soit $E$ un $\bdK$-espace vectoriel de dimension finie $n$ et soit $F$ un sev de $E$ de dimension $p$. Montrer qu’il existe $n − p$ hyperplans $H_{n−p+1}, \dots, H_n$ tels que $F = \bigcap_{p=n-p+1}^n H_i$.
    
    \boxans{
        Soit $\bcB_F = \left(e_1, \dots, e_p\right)$ une base de $F$. On la complète en une base de $E$ : $\bcB_E = \left(e_1, \dots, e_p, e_{p+1}, \dots, e_n\right)$. Prenons alors $\bcB^\star = \left({e_1}^\star, {e_2}^\star, \dots, {e_n}^\star\right)$. Soit $x \in E$. $\exists ! \left(\lambda_1, \lambda_2, \dots, \lambda_n\right) \in \bdK^n$ tels que $x = \displaystyle \sum_{i=1}^n \lambda_ie_i$. On a les équivalences :
        %
        \[ x \in F \iff \sum_{i=1}^p \lambda_ie_i + sum_{i=p+1}^n \lambda_ie_i \in F \iff \forall i \in \left\llbracket p+1, n \right\rrbracket,\qquad \lambda_i = 0 \iff \forall i \in \left\llbracket p+1, n \right\rrbracket,\qquad x \in \Ker\left({e_i}^\star\right)\]
        %
        Donc si et seulement si $x \in \displaystyle\bigcap_{i=p+1}^n \Ker {e_i}^\star$, ssi $x \in \displaystyle\bigcap_{i=p+1}^n H_i$ où $H_i = \Ker {e_i}^\star$ est un hyperplan.
    }

\end{enumerate}

\end{document}