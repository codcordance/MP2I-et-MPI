\documentclass[a4paper,french,bookmarks]{article}
\usepackage{./Structure/4PE18TEXTB}

\begin{document}
    \stylizeDoc{Mathématiques}{Exercices pour Lily}{DL, Continuité, Dérivabilité}
    
    \initcours{}
    
    \section{Développements polynomiaux (développement limité et asymptotique)}
    
    \begin{notation}
        Par \hg{$\mathsf{DL}_n\p{\alpha}$} on désigne un développement limité à l'ordre \hg{$n$} en \hg{$\alpha$}. 
    \end{notation}
    
    \subsection{Calcul de DL}
    
    \begin{exercise}{Quelques DL en 0}{}
        \medskip Déterminer les développements limités en suivants :\medskip
        
        \begin{enumerate}
            \begin{minipage}{0.48\linewidth}
                \itt Un $\mathsf{DL}_3\p{0}$ de $x \mapsto xe^x\vphantom{\dfrac{1}{1}}$
            
                \itt Un $\mathsf{DL}_3\p{0}$ de $x \mapsto \sin{x}\sqrt{1 + x}\vphantom{\dfrac{1}{1}}$
            
                \itt Un $\mathsf{DL}_4\p{0}$ de $x \mapsto e^{\cos x}\vphantom{\dfrac{1}{1}}$
                
                \itt Un $\mathsf{DL}_5\p{0}$ de $x \mapsto \dfrac{1}{\cos x}$
            \end{minipage}
            %
            \hfill
            %
            \begin{minipage}{0.48\linewidth}
                \itt Un $\mathsf{DL}_3\p{0}$ de $x \mapsto \dfrac{x}{\sin x}$
                                
                \itt Un $\mathsf{DL}_2\p{0}$ de $x \mapsto \dfrac{e^x - e^{-x}}{\sqrt{1 + x}}$
            
                \itt Un $\mathsf{DL}_2\p{0}$ de $x \mapsto \dfrac{\ln{1+x^2}}{\sin^2 x}$
                
                \itt Un $\mathsf{DL}_2\p{0}$ de $x \mapsto e^{-x}\dfrac{x}{1 + x}$
            \end{minipage}
        \end{enumerate}
    \end{exercise}
    
    \bigskip
    
    \begin{exercise}{Quelques DL en un point fini non nul}{}
        \medskip \hg{\itshape En vérité, ce genre de questions a un but purement pratique. Les développements limités sont presque toujours en $0$ ou dans un point remarquable pour la fonction. $\ln$ par exemple a son $\mathsf{DL}$ en $1$, et même dans ce cas, beaucoup préfèrent considérer le $\mathsf{DL}$ \guill{usuel} comme étant celui de $x \mapsto \ln{1 + x}$ en $0$}.\medskip
        
       Déterminer les développements limités en suivants :\medskip
        
        \begin{enumerate}
            \begin{minipage}{0.48\linewidth}
                \itt Un $\mathsf{DL}_2\p{\dfrac{\pi}{3}}$ de $x \mapsto \cos x$
                
                \itt Un $\mathsf{DL}_3\p{\dfrac{\pi}{4}}$ de $x \mapsto \cos x$
            
                \itt \bf{Pour tout $n \in \bdN$}, un $\mathsf{DL}_{\hg{n}}\p{-1}$ de $x \mapsto \dfrac{1}{x}$
            \end{minipage}
            %
            \hfill
            %
            \begin{minipage}{0.48\linewidth}
                \itt Un $\mathsf{DL}_4\p{2}$ de $x \mapsto \sqrt{x}$
                
                \itt Un $\mathsf{DL}_3\p{2}$ de $x \mapsto \ln x$
                
                \itt Un $\mathsf{DL}_5\p{\dfrac{\pi}{2}}$ de $x \mapsto \ln{\dfrac{1 + \cos x}{\sin x}}$
            \end{minipage}
        \end{enumerate}
    \end{exercise}
    
    \bigskip
    
        \begin{exercise}{Toujours plus ...}{}
        \medskip Déterminer les développements limités en suivants :\medskip
        
        \begin{enumerate}
            \begin{minipage}{0.48\linewidth}
                \itt Un $\mathsf{DL}_4\p{0}$ de $\displaystyle x \mapsto \int_x^{x^2} \sqrt{1 + t^2}\dif t$
            \end{minipage}
            %
            \hfill
            %
            \begin{minipage}{0.48\linewidth}
                \itt \bf{Pour tout $n \in \bdN$}, un $\mathsf{DL}_{\hg{n+1}}\p{0}$ de 
                
                $x \mapsto \ln{1 + x + \dfrac{x^2}{2!} + \dots + \dfrac{x^n}{n!}}$
            \end{minipage}
        \end{enumerate}
    \end{exercise}
    
    \newpage
    
    \subsection{Calcul de développement asymptotique et d'équivalent}
    
    
    
\end{document}