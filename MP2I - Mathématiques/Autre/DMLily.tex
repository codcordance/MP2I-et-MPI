\documentclass[a4paper,french,bookmarks]{article}
\usepackage{./Structure/4PE18TEXTB}

\begin{document}
\stylizeDoc{Mathématiques}{DM Lily}{Possibilité de réponse}

%\begin{exercise}{}{}
%\hg{On a la fonction $f : x \mapsto \dfrac{1-x^2}{1+x^2}$ définie sur $\bdR$. On a montré que $f$ était strictement décroissante sur $\left[0, +\infty\right[$.
%
%On a également montré qu'elle admettait un unique point fixe $l \in [0; 1]$ tel %que $f(l) = l$.
%
%On peut montrer les résultats suivants :}\newline
%\begin{enumerate}
%    \ithand Soit $\epsilon \in \left]0; 1-l\right[$. On a $1 > l + \epsilon > l > 0$. Par décroissance de $f$ sur $\bdR_+$, on en déduit :
%    \[0 < f(l + \epsilon) < f(l) < 1\]
%    Or $f(l) = l$ et $l < l + \epsilon < 1$ donc on arrive finalement à :
%\begin{equation}\label{eq1}
%     \boxedcol{\forall \epsilon \in \left]0; 1-l\right[, \qquad 0 < f(l+\epsilon) < l < l + \epsilon < 1} 
%\end{equation}
%
%\ithand Soit $\epsilon \in \left]0; l\right[$. On a $0 < l - \epsilon < l < 1$. %Par décroissance de $f$ sur $\bdR_+$, on en déduit :
%\[ 1 > f(l - \epsilon) > l > 0\]
%Or $f(l) = l$ et $l > l - \epsilon > 0$ donc on arrive finalement à :
%\begin{equation}\label{eq2}
%     \boxedcol{\forall \epsilon \in \left]0; l\right[, \qquad 1 > f(l-\epsilon) %> l > l - \epsilon > 0}
%\end{equation}
%\end{enumerate}

%\tcblower 
%Soit $n \in \bdN$, et la suite $\suite{u_n}$, telle que $\forall n \in \bdN$, $u_{n+1} = f(u_n)$ et $u_0 \in \left]0; l\right[$. On raisonne ainsi :

%\begin{enumerate}
%    \ithand Si $0 < u_n < l$, on peut poser $\epsilon \in \left]0; l\right[$ tel %que $u_n = l - \epsilon$. L'équation \eqref{eq2} nous livre alors :
%    \[ 1 > f(l - \epsilon) > l > l - \epsilon > 0 \qquad \text{soit} \qquad 1 > %u_{n+1} > l > u_n > 0\]
%    Puisque $1 > u_{n+1} > l$, on peut poser $\epsilon' \in \left]0; 1-l\right[$ %tel que $u_{n+1} = l + \epsilon'$. ²
%%\end{enumerate}
%%\end{exercise}

%\newpage

\begin{exercise}{}{}
\hg{On a la fonction $f : x \mapsto \dfrac{1-x^2}{1+x^2}$ définie sur $\bdR$. On a montré que $f$ était strictement décroissante sur $\left[0, +\infty\right[$.

On a également montré qu'elle admettait un unique point fixe $l \in [0; 1]$ tel que $f(l) = l$.}

Ensuite, on a montré que $\forall x \in \bdR$, $f \circ f (x) = \dfrac{2x^2}{1+x^4}$. On s'intéresse alors à la fonction $g$ définie pour $x$ réel selon $g(x) = \dfrac{f \circ f(x)}{x} = \dfrac{2x}{1+x^4}$.Par opérations usuelles, la fonction $g$ est dérivable sur $\bdR$ et telle que :
\[ \forall x \in \bdR,\qquad g'(x) = \dfrac{2(1+x^4) - 2x\times4x^3}{(1+x^4)^2} = \dfrac{2(1-3x^4)}{(1+x^4)^2}\]
On a donc $g'(x) > 0 \iff 1-3x^4 > 0 \iff \mod{x} < \sqrt[4]{\dfrac{1}{3}} \iff -\sqrt[4]{\dfrac{1}{3}} < x < \sqrt[4]{\dfrac{1}{3}}$. 

\begin{center}
    \bf{Donc $g$ est strictement croissante sur $\left[0; \sqrt[4]{\dfrac{1}{3}}\right[$.}
\end{center}

Par ailleurs, $l$ est solution de l'équation $x^3 + x^2 + x - 1 = 0$. En posant la fonction $h$ sur $\bdR$ avec $h(x) = x^3 + x^2 + x - 1$, on a donc $h(l) = 0$. Par opérations, $h$ est dérivable sur $\bdR$ et telle que :
\[ \forall x \in \bdR, \qquad h'(x) = 3x^2 + 2x + 1 \ = \ 2x^2 + x^2 + 2x + 1 \ = \  2x^2 + (x+1)^2 > 0\]
Donc la fonction $h$ est strictement croissante sur $\bdR$. On remarquera alors que $h\left(\sqrt[4]{\dfrac{1}{3}}\right) \approx 0.75 > 0$.

Donc $h\left(\sqrt[4]{\dfrac{1}{3}}\right) > h(l)$ d'où (par croissance de $h$) $\sqrt[4]{\dfrac{1}{3}} > l$. Donc $g$ est strictement croissante sur $[0; l]$.

Soit $x \in \bdR$, tel que $0 < x < l$. On a donc $g(x) < g(l)$.
Or $g(l) = \dfrac{f(f(l))}{l}= \dfrac{f(l)}{l} = \dfrac{l}{l} = 1$ et $g(x) = \dfrac{f(f(x))}{x}$, donc :
\[ \forall x \in ]0; l[, \quad \dfrac{f(f(x))}{x} < 1 \qquad \text{donc} \qquad \boxedcol{\forall x \in ]0; l[, \quad  f \circ f (x) < x}\]
\end{exercise}


\end{document}