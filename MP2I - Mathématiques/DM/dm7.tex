\documentclass[a4paper,french,bookmarks]{article}
\usepackage{./Structure/4PE18TEXTB}

\begin{document}
    \def\classevar{MP2I}
    \def\anneevar{2021-2022}
    \newboxans{}

    \renewcommand{\thesection}{\Roman{section}}
    \setlist[enumerate]{font=\color{white5!60!black}\bfseries\sffamily}
    \renewcommand{\thesection}{Partie \Roman{section}}
    \renewcommand{\labelenumi}{\Roman{section}.\arabic{enumi}.}
    \renewcommand*{\labelenumii}{\alph{enumii}.}

    \stylizeDocSpe{Maths}{Devoir maison n° 7}
    {Les intégrales de Wallis et de Gauss}{Novembre 2021}

    \section*{Problème : les intégrales de Wallis et de Gauss}

    L'objectif de ce problème est le calcul de l'intégrale de \textsc{Gauss} $\displaystyle \dfrac{1}{\sqrt{2\pi}} \int_0^{+\infty} e^{-\frac{1}{2}t^2} \dif t$.

    \emph{Notations :}
    \begin{enumerate}
        \itt On dit que $u_n \asymp{n \to +\infty} v_n$ (la suite $u$ est équivalente à la suite $v$) lorsque $\lim\limits_{n \to +\infty} \dfrac{u_n}{v_n} = 1$.
    
        \itt La notation $\displaystyle \int_0^{+\infty} f\p{t}\dif t$ désigne la limite (si elle existe) de $\displaystyle \int_0^X f\p{t}\dif t$ lorsque $X$ tend vers $+\infty$.
    \end{enumerate}

    \section{Intégrales de Wallis}

    Pour $n \in \bdN$, on pose $\displaystyle I_n = \int_0^{\sfrac{\pi}{2}} \cos^n\p{t} \dif t$.

    \begin{enumerate}
        \item Montrer que $\displaystyle I_n = \int_0^{\sfrac{\pi}{2}} \sin^n\p{t}\dif t$.
    
        \indication{on fera un changement de variable.}
        
        \noafter
        %
        \boxans{
            On a $\displaystyle I_n = \int_0^{\sfrac{\pi}{2}} \cos^n\p{t} \dif t = -\int_{\sfrac{\pi}{2}}^0 \cos^n\p{t} \dif t$. En posant $u = t - \dfrac{\pi}{2}$ (donc $\dif u = \dif t$), on a :
            %
            \[ I_n = -\int_0^{\sfrac{\pi}{2}} \cos^n\p{u+\dfrac{\pi}{2}} \dif t = -\int_0^{\sfrac{\pi}{2}} -\sin^n(u) \dif t \]
        }
        %
        \nobefore\yesafter
        %
        \boxansconc{
            Donc $\displaystyle I_n = \int_0^{\sfrac{\pi}{2}} \sin^n\p{t}\dif t$.
        }
        %
        \yesbefore
    
        \item À l'aide d'une intégration par parties, montrer que $I_{n+2} = \dfrac{n+1}{n+2}I_n$.
        
        \indication{On pourra écrire $\cos^{n+2}\p{t} = \cos^{n+1}\p{t}\cos\p{t}$}
        
        \noafter
        %
        \boxans{
            Soit $n \in \bdN$. On intègre par partie en dérivant $\cos^{n+1}$ et en intégrant $\sin$:
            %
            \begin{align*}
                I_{n+2} &= \int_0^{\sfrac{\pi}{2}} \cos^{n+2}\p{t}\dif t = \int_0^{\sfrac{\pi}{2}} \cos\p{t}\cos^{n+1}\p{t}\dif t\\
                &= \intc{\sin\p{t}\cos^{n+1}\p{t}}_0^{\sfrac{\pi}{2}}  - \int_0^{\sfrac{\pi}{2}} \sin\p{t}\times\p{-\sin\p{t}}\cos^n\p{t}\dif t\\
                &= 1\times0^{n+1} - 0\times 1^{n+1} +\int_0^{\sfrac{\pi}{2}} \p{n+1}\sin^2\p{t}\cos^n\p{t} \dif t\\
                &= \p{n+1}\int_0^{\sfrac{\pi}{2}} \p{1-\cos^2\p{t}}\cos^n\p{t} \dif t = \p{n+1}\p{\int_0^{\sfrac{\pi}{2}} \cos^n\p{t} \dif t - \int_0^{\sfrac{\pi}{2}} \cos^{n+2}\p{t} \dif t}\\
                &= \p{n+1}I_n - \p{n+1}I_{n+2}
            \end{align*}
        }
        %
        \nobefore\yesafter
        %
        \boxansconc{
            Donc $I_{n+2}(1+n+1) = \p{n+1}I_n$. Ainsi, pour tout $n \in \bdN$, on a $I_{n+2} = \dfrac{n+1}{n+2}I_n$.
        }
        %
        \yesbefore
    
        \item Calculer $I_0$ et $I_1$.
        
        \boxansconc{
            On a $\displaystyle I_0 = \int_0^{\sfrac{\pi}{2}} \cos^0\p{t}\dif t = \int_0^{\sfrac{\pi}{2}} \dif t = \intc{t}_0^{\sfrac{\pi}{2}} = \dfrac{\pi}{2} - 0$ donc $I_0 = \dfrac{\pi}{2}$.
            
            On a $\displaystyle I_1 = \int_0^{\sfrac{\pi}{2}} \cos{t}\dif t = \int_0^{\sfrac{\pi}{2}} \cos\p{t} \dif t = \intc{\sin\p{t}}_0^{\sfrac{\pi}{2}} = 1 - 0$ donc $I_1 = 1$.
        }
        
        \item Soit $n \in \bdN$. Exprimer le produit $\displaystyle 1 \times 3 \times \dots \times \p{2n-1} = \prod_{k=1}^{n} \p{2k-1}$ selon $n$, en utilisant les factorielles.
        
        \boxansconc{
            \[ \prod_{k=1}^{n} \p{2k-1} = \dfrac{\displaystyle \prod_{k=1}^{n} \p{2k-1} \times \prod_{k=1}^{n} 2k}{\displaystyle \prod_{k=1}^{n} 2k} = \dfrac{\displaystyle \prod_{\substack{k=1\\k\equiv1 [2]} }^{2n} k \times \prod_{\substack{k=1\\k\equiv0 [2]} }^{2n} k}{\displaystyle 2^n\prod_{k=1}^{n}k} = \dfrac{\displaystyle \prod_{k=1}^{2n}k}{2^n n!} = \dfrac{\p{2n}!}{2^n n!} \]
        }
        
        \item Déduire des questions précédentes l'expression de $I_{2n}$ et $I_{2n+1}$ en fonction de $n$.
        
        \noafter
        %
        \boxans{
            On procède par récurrence. Pour tout $n \in \bdN$, on pose le prédicat $H\p{n}$ suivant :
            %
            \[ H\p{n}:\qquad I_{2n} = \dfrac{1\times3\times\dots\times\p{2n-1}}{ 2\times4\times\dots\times \p{2n}}I_0 \qquad\et\qquad
                I_{2n+1}  = \dfrac{2\times4\times\dots\times\p{2n}}{ 1\times3\times\dots\times (2n+1)}I_1
            \]
            %
            \begin{enumerate}
                \itt \underline{Initialisation :} Pour $n = 0$, les produit devant $I_0$ et $I_1$ dans les formules ci-dessus sont vides, donc par convention égaux à $1$. Donc $H\p{0}$ est vrai.
                
                \itt\underline{Hérédité :} Soit $n \in \bdN$ tel que $H\p{n}$ est vrai. On a :
                %
                \[ I_{2\p{n+1}} = I_{2n+2} = \dfrac{2n+1}{2n+2}I_{2n} \qquad\et\qquad I_{2\p{n+1}+1} = I_{2n+3} = \dfrac{2n+2}{2n+3}I_{2n+1} \]
                %
                Par hypothèse de récurrence, on obtient :
                %
                \[ I_{2\p{n+1}} = \dfrac{1\times3\times\dots\times\p{2n-1}\times(2n+1)}{ 2\times4\times\dots\times \p{2n}\times(2n+2)}I_0 \qquad\et\qquad I_{2\p{n+1}+1} = \dfrac{2\times4\times\dots\times\p{2n}\times(2n+2)}{ 1\times3\times\dots\times (2n+1)\times(2n+3)}I_1\]
                %
                Donc $H\p{n+1}$ est vrai.
                
                \itt\underline{Conclusion :} Par \emph{principe de récurrence}, $P\p{n}$ est vrai pour tout $n \in \bdN$.
            \end{enumerate}
            %
        }
        %
        \nobefore\yesafter
        %
        \boxansconc{
            \[ I_{2n} = \dfrac{\displaystyle \prod_{k=1}^n (2k+1)}{\displaystyle \prod_{k=1}^n 2k}\times I_0  = \dfrac{\pi\p{2n}!}{2\p{2^n\times n!}^2} \qquad\et\qquad I_{2n+1} = \dfrac{\displaystyle \prod_{k=1}^n 2k}{\displaystyle \prod_{k=1}^{n+1} 2k}\times I_1 = \dfrac{I_0}{(2n+1)I_{2n}} = \dfrac{\p{2^n\times n!}^2}{\p{2n+1}\p{2n}!} = \dfrac{\p{2^n\times n!}^2}{\p{2n+1}!}\]
        }
        
        \item Montrer que pour tout entier $n$ on a les inégalités $I_{n+2} \leq I_{n+1} \leq I_n$.
        
        En déduire que $I_{n+1} \asymp{n \to +\infty} I_n$.\qquad \textit{(on encadrera pour cela $\dfrac{I_{n+1}}{I_n}$)}.
        
        \boxans{
            Soit $n \in \bdN$. Soit $x \in \intc{0;\dfrac{\pi}{2}}$, $0 \leq \cos x \leq x$. Par croissance de la fonction $x \mapsto x^n$ sur $\bdR_+$, on a $0 \leq \cos^n$. En multipliant, on a donc $0 \leq cos^{n+1} x \leq cos^n x$. Par conservation de l'ordre de l'intégrale, on a donc $\displaystyle\int_0^{\sfrac{\pi}{2}} \cos^{n+1} \dif t \leq \int_0^{\sfrac{\pi}{2}} \cos^{n} \dif t$ soit $I_{n+1} \leq I_{n}$. Il s'en déduit directement \boxsol{$\forall n \in \bdN$, $I_{n+2} \leq I_{n+1} \leq I_n$}.
            
            On a donc $\dfrac{I_{n+2}}{I_n} \leq \dfrac{I_{n+1}}{I_n} \leq \dfrac{I_{n}}{I_n}$. Or $I_{n+2} = \dfrac{n+1}{n+2}I_n$ donc $\dfrac{n+2}{n+1} \leq \dfrac{I_{n+1}}{I_n} \leq 1$. Or $\dfrac{n+1}{n+2} = \dfrac{1+\frac{1}{n}}{1+\frac{2}{n}}\xrightarrow{n \to +\infty} 1$.
            
            Par théorème d'encadrement, on a donc $\dfrac{I_{n+1}}{I_n} \xrightarrow{n \to +\infty} 1$, soit \boxsol{$I_{n+1} \asymp{n \to +\infty} I_n$}
        }
    
        \item Montrer que la suite $(\p{n+1}I_{n+1}I_n)_{n \in \bdN}$ est constante.
        
        En déduire : \quad $I_n \asymp{n \to +\infty} \sqrt{\dfrac{\pi}{2n}}$. \quad On calculera pour cela la limite de $n(I_n)^2$ par encadrement.
            
        \boxans{
            Soit $n \in \bdN$. Si $n$ est pair, $I_{n+1}=I_{2p+1}=\dfrac{I_0}{(2p+1)I_{2p}}=\dfrac{I_0}{\p{n+1}I_n}$ donc $\p{n+1}I_{n+1}I_n = \dfrac{I_0\p{n+1}I_n}{\p{n+1}I_n} = I_0$
            
            
            Si $n$ est impair, $I_{n} = I_{2p+1} = \dfrac{I_0}{(2p+1)I_{2p}} = \dfrac{I_0}{nI_{n-1}}$ et $I_{n+1} = I_{2p+2} =\dfrac{2p+1}{2p+2}I_{2p} = \dfrac{n}{n+1}I_{n-1}$.
            
            Donc $\p{n+1}I_{n+1}I_n = \dfrac{I_0n\p{n+1}I_{n-1}}{n\p{n+1}I_{n-1}} = I_0$. Donc \boxsol{$\forall n \in \bdN$, $\p{n+1}I_{n+1}I_n = I_0 = \dfrac{\pi}{2}$}
        }
        \boxans{
        Donc \boxsol{la suite $(\p{n+1}I_{n+1}I_n)_{n \in \bdN}$ est constante}.
        
        On a $I_{n+1} \asymp{n \to +\infty} I_n$, donc les suites $(\p{n+1}I_{n+1}I_n)_{n \in \bdN}$ et $(n(I_n)^2)_{n\in\bdN}$ sont asymptotiques lorsque $n$ tend vers $+\infty$. Ainsi $n(I_n)^2 \xrightarrow{n \to +\infty} \dfrac{\pi}{2}$ et donc \boxsol{$I_n \asymp{n \to +\infty} \sqrt{\dfrac{\pi}{2n}}$}.
        }
    \end{enumerate}
    
    
\subsection*{Partie 2. Application au calcul de $\displaystyle \int_0^{+\infty} e^{-\sfrac{x^2}{2}}${\normalfont{d}$x$}}

\begin{enumerate}
    \item Justifier l'inégalité $\forall x > -1$, $\ln(1+x)\leq x$.
    
    En déduire, pour $n$ entier $\geq 1$, les inégalités :
    \[ \forall x \in \intc{-\sqrt{n}; \sqrt{n}}, \p{1-\dfrac{x^2}{n}}^n \leq e^{-x^2} \qquad \text{et} \qquad e^{-x^2} \leq \p{1+\dfrac{x^2}{n}}^{-n}.\]
    
    \boxans{
        La tangente à $\ln$ en $1$ est $y = x - 1$. $\ln$ étant concave, on a $\forall x > 0$, $\ln(x) \leq x + 1$. 
        
        Donc \boxsol{$\forall x > -1$, $\ln(1+x)\leq x$}.
        Soit $n \in \bdN$, $n \geq 1$. On applique l'inégalité avec $-\dfrac{x^2}{n}$.
        
        On a alors $\forall x \in \intc{-\sqrt{n}; \sqrt{n}}$, $\ln\p{1-\dfrac{x^2}{n}} \leq -\dfrac{x^2}{n}$. En multipliant par $n$, on a $\ln\p{\p{1-\dfrac{x^2}{n}}^n} \leq -x^2$ donc par croissance de la fonction exponentielle, on a finalement \boxsol{$\forall x \in \intc{-\sqrt{n}; \sqrt{n}}$, $\p{1-\dfrac{x^2}{n}}^n \leq e^{-x^2}$}.
        
        On applique de manière similaire l'inégalité avec $\dfrac{x^2}{n}$. En multipliant par $-n$ et par croissance de la fonction exponentielle sur $\bdR$ on trouve \boxsol{$\forall x \in \intc{-\sqrt{n}; \sqrt{n}}$, $e^{-x^2} \leq \p{1+\dfrac{x^2}{n}}^{-n}$}.
        
    }
    \item On pose pour tout entier $n \geq 1 $ pour $X > 0$ :
    \[B_n(X) = \int_0^X \dfrac{\dif x}{\p{1+\frac{x^2}{n}}^n} \qquad \text{et} \qquad C_n = \int_0^{\sqrt n} \p{1-\dfrac{x^2}{n}}^n \dif x.\]
    \begin{enumerate}
        \item Soit $X \in \bdR_+^*$. En effectuant le changement de variable $x = \sqrt{n}\tan\varphi$ exprimer d'une autre manière l'intégrale $B_n(X)$, puis montrer que $\lim\limits_{X \mapsto +\infty} B_n(X)$ existe et exprimer cette limite à l'aide de $n$ et de $I_{2n-2}$.
        
        \boxans{
            On pose $x = \sqrt{n}\tan\varphi$ avec $\varphi \in \intc{0;\dfrac{\pi}{2}\right[$. La fonction $\tan$ est bien dérivable sur $\intc{0;\dfrac{\pi}{2}\right[$ et telle que $\dif x = \sqrt n (1+\tan^2\varphi) \dif \varphi$.
            
            Lorsque $x = 0$, $\sqrt n \tan \varphi = 0$ donc $\tan\varphi = 0$, donc $\varphi = 0$. Lorsque $x = X$, $\sqrt n \tan \varphi = X$ donc $\tan \varphi = \dfrac{X}{\sqrt n}$ donc $X \in [0;\sqrt{n}[$, donc $\varphi = \arctan{\dfrac{X}{\sqrt n}}$.
            
            Ainsi $\displaystyle B_n(X) = \int_0^{\arctan{\sfrac{X}{\sqrt n}}} \dfrac{\sqrt n\p{1 + \tan^2\varphi }}{\p{1 + \tan^2\varphi }^{n}} \dif \varphi = \sqrt n\int_0^{\arctan{\sfrac{X}{\sqrt n}}} \p{\dfrac{1}{1 + \tan^2\varphi}}^{n-1} \dif \varphi $.
            
            On a $\dfrac{1}{cos^2(x)}=1+\tan^2(x)$ donc \boxsol{$\displaystyle B_n(X) = \sqrt n \int_0^{\arctan{\sfrac{X}{\sqrt n}}} \cos^{2(n-1)}(\varphi) \dif \varphi$}
            
            Les fonctions $\cos$ et $x \mapsto x^{2(n-1)}$ sont continues sur $\intc{0;\dfrac{\pi}{2}\right[$ donc par composition et par intégrale $B_n(X)$ est continue sur $\bdR_+^*$. Ainsi \boxsol{$B_n(X)$ admet une limite quand $X$ tend vers $+\infty$}.
            
            De plus $\arctan{\dfrac{X}{\sqrt n}} \xrightarrow{X \to +\infty} \dfrac{\pi}{2}$, donc \boxsol{$B_n(X) \xrightarrow{X \to +\infty} \sqrt n I_{2n-2}$}.
        }
        \item Transformer $C_n$ grâce au changement de $x = \sqrt{n}\sin\varphi$.
        
        \boxans{
        On a $0 \leq x \leq \sqrt{n}$, ce qui nous permet de poser $x = \sqrt n \sin \varphi$ avec $\varphi \in \intc{0;\dfrac{\pi}{2}}$. La fonction $\sin$ est bien dérivable sur $\intc{0;\dfrac{\pi}{2}}$ et telle que $\dif x = \sqrt n \cos \varphi \dif \varphi$.
        }
        
        \boxans{
        Lorsque $x = 0$, on a $\sqrt n \sin \varphi = 0$ donc $\varphi = 0$. Lorsque $x = \sqrt n$, $\sqrt n \sin \varphi = \sqrt n$ donc $\varphi = \dfrac{\pi}{2}$.
        
        Ainsi $\displaystyle C_n = \sqrt n \int_0^{\sfrac{\pi}{2}} \p{1-\sin^2\varphi}^n\cos\varphi \dif \varphi = \sqrt n \int_0^{\sfrac{\pi}{2}} \p{\cos^2 \varphi }^n\cos \varphi \dif \varphi = \sqrt n \int_0^{\sfrac{\pi}{2}} \cos^{2n+1} \varphi \dif \varphi$.
        Donc \boxsol{$C_n = \sqrt n I_{2n+1}$}
        }
    \end{enumerate}
    \item En déduire l'existence et la valeur de $\displaystyle \int_0^{+\infty} e^{-x^2}\dif x$.
    \boxans{
    Soit $n \in \bdN$ et $x \in \bdR_+$. On a $\p{1-\dfrac{x^2}{n}}^n \leq e^{-x^2} \leq \p{1+\dfrac{x^2}{n}}^{-n}$. Par conservation de l'ordre par l'intégrale, on a donc : 
    \[ \forall n \in \bdN, \quad \forall X \in \bdR_+, \qquad \int_0^{X} \p{1-\dfrac{x^2}{n}}^n \dif x \leq \int_0^{X} e^{-x^2} \dif x \leq \int_0^{X} \p{1+\dfrac{x^2}{n}}^{-n} \dif x\]
    
    On remarque facilement que $x \mapsto 1+\dfrac{x^2}{n}$ est croissant, donc par composition avec la fonction $x \mapsto \dfrac{1}{x^n}$, la fonction $x \mapsto \p{1+\dfrac{x^2}{n}}^{-n}$ est décroissante. En prenant $X \geq {\sqrt n}$, on a donc :
    \[ \forall n \in \bdN, \quad \forall X \in \bdR_+, \quad X > \sqrt n, \qquad B_n(X) \leq \int_0^{X} e^{-x^2} \dif x \leq C_n\]
    Or $B_n(X) \xrightarrow{X \to +\infty} \sqrt n I_{2n-2}$, $C_n = \sqrt n I_{2n+1}$ et $I_n \asymp{n \to +\infty} \sqrt{\dfrac{\pi}{2}}\times\dfrac{1}{\sqrt n}$.
    On considère donc $B_n(X)$ et $C_n$ lorsque $X$ et $n$ tendent vers $+\infty$. On a $B_n(X) \xrightarrow[X \to +\infty]{n\to+\infty} \sqrt{\dfrac{\pi}{2}}\times\dfrac{\sqrt{n}}{\sqrt{2n-2)}} \xrightarrow[X \to +\infty]{n\to+\infty} \dfrac{\sqrt 2}{2}$, et $C_n \xrightarrow{n\to+\infty} \sqrt{\dfrac{\pi}{2}}\times\dfrac{\sqrt n}{\sqrt{2n+1}} \xrightarrow{n\to+\infty} \dfrac{\sqrt \pi}{2}$.
    Par théorème d'encadrement, \boxsol{$\displaystyle \int_0^{+\infty} e^{-x^2}\dif x$ existe et vaut $\displaystyle \int_0^{+\infty} e^{-x^2}\dif x = \dfrac{\sqrt \pi}{2}$}
    
    }
    \item Donner enfin la valeur de $\displaystyle \int_0^{+\infty} e^{-\sfrac{x^2}{2}} \dif x$.
    
    \boxans{
        On applique un simple changement d'indice sur l'intégrale précédente avec $x^2 = \dfrac{X^2}{2}$, avec $X \in \bdR_+$. Donc $x = \dfrac{X}{\sqrt2}$, ainsi $\dif x = \dfrac{1}{\sqrt 2} \dif X$.
        Les bornes étant $0$ et $+\infty$, elles restent inchangées par la multiplication par $\sqrt 2$. Ainsi :
        $\displaystyle \int_0^{+\infty} e^{-x^2}\dif x = \sqrt 2 \int_0^{+\infty} e^{-\sfrac{x^2}{2}} \dif x$. Finalement, \boxsol{$\int_0^{+\infty} e^{-\sfrac{x^2}{2}} \dif x = \sqrt{\dfrac{\pi}{2}}$}.
    }
\end{enumerate}
\end{document}