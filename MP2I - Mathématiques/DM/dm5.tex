\documentclass[a4paper,french,bookmarks]{article}
\usepackage{./Structure/4PE18TEXTB}

\begin{document}
\stylize{Mathématiques}{Devoir Maison 5}

\section*{Problème : Fonction exponentielle}

Au lycée, vous avez admis qu'il existe une fonction vérifiant $f' = f$ et $f(0) = 1$; vous avez pu prouver l'unicité d'une telle fonction; vous avez ainsi introduit la fonction exponentielle réelle et vous avez dû établir les propriétés usuelles (relation fonctionnelle, limites, etc).

Ce problème se propose de démontrer l'existence de cette fonction en partie B.2, son unicité en partie A, de donner des approximations en partie C.

Dans les parties A et B, \underline{les fonctions exponentielles et logarithmes sont supposées ne pas être connues}.

Les parties A, B.1, B.2 et C peuvent être traitées indépendamment même si elles ont un lien entre elles.

\subsection*{Partie A - Unicité, sous réserve d'existence, de la fonction exponentielle}

Dans cette partie; on suppose qu'il existe une fonction $f$ dérivable sur $\bdR$ solution de l'équation différentielle

$(E) : y' = y$ sur $\bdR$ et telle que $f(0) = 1$.

\begin{enumerate}
    \item Démontrer que : $\forall x \in \bdR$, $f(x)f(-x) = 1$.
        \textit{(On pourra dériver $x \mapsto f(x)(f-x)$)}
        
        En déduire que $f$ ne s'annule pas sur $\bdR$.
    \boxans{
        Soit $x \in \bdR$. On pose $u : x \mapsto f(x)f(-x)$. Par composée, $x \mapsto f(-x)$ se dérive en $x \mapsto -f'(-x)$.
        
        $u$ est un produit de fonctions dérivables sur $\bdR$ donc dérivable sur $\bdR$.
        On a donc :
        
        \[ u'(x) = f'(x)f(-x) - f(x)f'(-x) = f(x)f(-x) - f(x)f(-x) = 0 \]
        
        La dérivée de $u$ est la fonction nulle donc $u$ est une fonction constante :
        $\exists k \in \bdR$, $u(x) = k$.
        
        $u(0) = f(0)f(-0) = 1 \times 1 = 1$ donc $k = 1$, donc \boxsol{$\forall x \in \bdR$, $f(x)f(-x) = 1$}.
        
        On a $f(x)f(-x) \neq 0$ donc $f(x) \neq 0$ et $f(-x) \neq 0$. Donc \boxsol{$f$ ne s'annule pas sur $\bdR$}.
    }
    
    \item Démontrer que $f$ est unique. Pour cela, on montrera que si $g$ est une fonction             dérivable sur $\bdR$ solution de $(E)$ sur $\bdR$ telle que $g(0) = 1$ alors $g = f$.
    
        \textit{(On pourra considérer la fonction $\phi = \frac{g}{f}$ et calculer sa dérivée.)}
    \boxans{
        Soit $g$ une fonction dérivable sur $\bdR$ solution de $(E)$ sur $\bdR$ telle que $g(0) = 1$.
        
        Soit $x \in \bdR$. On pose $\phi : x \mapsto \frac{g(x)}{f(x)}$.  
        $f$ ne s'annule pas sur $\bdR$ donc $\phi$ est un quotient toujours défini de fonctions dérivable sur $\bdR$ donc dérivable sur $\bdR$. On a donc :
        
        \[ \phi'(x) = \frac{g'(x)f(x) - g(x)f'(x)}{f(x)^2} = \frac{g(x)f(x) - g(x)f(x)}{f(x)^2}= 0 \]
        
        La dérivée de $\phi$ est la fonction nulle donc $\phi$ est une fonction constante :
        $\exists k \in \bdR$, $\phi(x) = k$.
        
        $\phi(0) = \frac{g(0)}{f(0)} = \frac{1}{1} = 1$ donc $k = 1$, donc $\frac{g(x)}{f(x)}=1$ donc $f(x)=g(x)$. Finalement, \boxsol{$g = f$}.
    }
    
    \item \textit{Relation fonctionnelle}
    \begin{enumerate}
        \item Soit $y \in \bdR$ fixé. Dériver la fonction $\varphi : x \mapsto \frac{f(x+y)}{f(x)}$     de la variable $x$.
        \boxans{
            Soit $x \in \bdR$. $\varphi$ est un quotient de fonctions dérivables sur $\bdR$ et ne s'annulant pas sur $\bdR$, donc $\varphi$ est dérivable sur $\bdR$. Avec $y$ une constante, par composée, $x \mapsto f(x+y)$ se dérive en $x \mapsto f'(x+y)$.
            %\[ \varphi'(x) = \frac{f'(x+y)f(x) - f(x+y)f'(x)}{f(x)^2} = \frac{f(x+y)f(x) - f(x+y)f(x)}{f(x)^2} = 0 \ \text{donc} \ \boxsol{$\varphi' = 0$} \]
        }
        
        \item En déduire : $\forall (x,y) \in \bdR^2$, $f(x+y)=f(x)f(y)$.
        \boxans{
            Soit $y \in \bdR$. Soit $\varphi$ la fonction telle que $\forall x \in \bdR$, $\varphi(x) = \frac{f(x+y)}{f(x)}$.
            La dérivée de $\varphi$ est la fonction nulle donc $\varphi$ est une fonction constante : $\exists k \in \bdR$, $\varphi(x) = k$.
            
            $\varphi(0) = \frac{f(0+y)}{f(0)} = \frac{f(y)}{1} = f(y)$ donc $k = f(y)$. Donc $\varphi(x) = f(y)$ donc $\frac{f(x+y)}{f(x)}=f(y)$.
            
            Donc \boxsol{$\forall (x,y) \in \bdR^2$, $f(x+y)=f(x)f(y)$}.
        }
    \end{enumerate}
    
    \item Déduire des résultats précédents que $f$ est strictement positive sur $\mathbb{R}$.
    
    \boxans{
        Soit $x \in \bdR$. On pose $y = \frac{x}{2}$. On a $f(x)=f(2y)=f(y+y)=f(y)f(y)=f(y)^2$.
        
        Or $f$ est solution de $(E)$ sur $\bdR$ donc $f$ est à valeur dans $\bdR$, donc $f$ est positive sur $\bdR$.
        
        De plus $f$ ne s'annule pas sur $\bdR$ donc \boxsol{$f$ est strictement positive sur $\bdR$}.
    }
\end{enumerate}

\subsection*{Partie B.1 - Méthode d'Euler}

La méthode d'Euler est un algorithme donnant une solution approchée d'une équation différentielle. On construit une solution approchée affine par morceaux par approximations successives. On décrit ci-dessous le principe de l'algorithme pour le cas particulier de l'équation différentielle : $y ' = y$ et $y(0)=1$ en notant $f$ la solution exacte (en supposant qu'elle existe). Pour obtenir une valeur approchée de $f(x)$ où $x \geq 0$, on commence par choisir un entier $n \geq 1$ assez grand et on partage le segment $\left[0,x\right]$ en $n$ intervalles réguliers d'extrémités $x_k = \frac{kx}{n}$ où $k \in \llbracket0,n\rrbracket$.\vspace{2mm}

On va construire ne fonction affine par morceaux notée $\varphi_n$ sur $\left[0,x\right]$ qui approche $f$.\vspace{1mm}


La valeur de $f$ en $0$ est connue : $f(0) = 1$, on pose donc $\varphi_n(0) = 1$.

Pour estimer la valeur en $x_1$, on fait une approximation affine de $f$ grâce à sa tangente en $0$ d'équation $y = f(0) + (x - 0)f'(0)$.

On rappelle que $f' = f$ et $f(0) = 1$ donc $f'(0) = f(0) = 1$ et la tangente de $f$ en $0$ a pour équation $y = 1 + x$.

On pose donc $\varphi_n(x_1) = 1 + x_1 = 1 + \frac{x}{n}$. Puis, pour estimer la valeur en $x_2$ on refait une approximation affine grâce à l'équation de la tangente en $x_1$ : $f(x_2) \approx f(x_1)+f'(x_1)(x_2-x_1) = f(x_1) + f(x_1)(x_2-x_1)$ (toujours car $f' = f$).

Cette fois, comme la valeur exacte $f(x_1)$ n'est pas connue, on la remplace par la valeur approchée de $\varphi_n(x_1)$.

On pose finalement : $\varphi(x_2) = \varphi(x_1) + \varphi(x_1)(x_2-x_1)$ ; et on continue le procédé jusqu'à obtenir la valeur de $\varphi(x_n)$ c'est-à-dire $\varphi(x)$ qui sera considérée comme une valeur approchée de $f(x)$.\vspace{2mm}

Pour tracer la fonction affine par morceaux $\varphi_n$ sur $\left[0, x\right]$, il n'y a plus qu'à relier les points $\left(x_{k-1}, \varphi_n(x_{k-1})\right)$ et $\left(x_k, \varphi_n(x_k)\right)$ pour $k \in \llbracket1,n\rrbracket$.


\begin{enumerate}
    \item Faites un dessin pour illustrer le procédé.
    
    \boxans{
        \pgfplotsset{width=12.5cm}
        \center \begin{tikzpicture}
            \begin{axis}[
                axis lines          = left,
                xmin                = 0,
                xmax                = 5,
                ymin                = 0,
                ymax                = 7.268,
                xtick               = {0, 1, 2, 3, 4, 5},
                xticklabels         = {$x_0$, $x_1$, $x_2$, $x_3$, $x_k$, $x_n$},
                ytick               = {1, 1.487, 2.210, 3.287, 4.888,7.268},
                yticklabels         = {$\varphi_n(x_0)$, $\varphi_n(x_1)$, $\varphi_n(x_2)$, $\varphi_n(x_3)$, $\varphi_n(x_k)$, $\varphi_n(x_n)$ },
                grid                = both,
                grid style          = {line width = .1pt, draw = gray!30},
                major grid style    = {line width=.2pt,draw=gray!50},
                minor tick num      = 5,
                legend pos          = north west,
            ]
            
            \addplot[color=gypl5, style=dashed, line width=0.3mm] { 1     + 0.487*x};
            \addlegendentry{Tangente en $x_0$}
            \addplot[color=gypl4, style=dashed, line width=0.3mm] { 0.764 + 0.723*x};
            \addlegendentry{Tangente en $x_1$}
            \addplot[color=gypl3, style=dashed, line width=0.3mm] { 0.056 + 1.077*x};
            \addlegendentry{Tangente en $x_2$}
            \addplot[color=gypl2, style=dashed, line width=0.3mm] {-1.516 + 1.601*x};
            \addlegendentry{Tangente en $x_3$}
            \addplot[color=gypl1, style=dashed, line width=0.3mm] {-4.632 + 2.38*x};
            \addlegendentry{Tangente en $x_4$}
            
            \addplot[
                color=blue,
                mark=x,
                mark size=4pt,
                point meta=explicit symbolic,
                nodes near coords,
                every node near coord/.append style={anchor=west, color=blue},
            ]
            coordinates {
                (0, 1.000) [$\left(x_0, \varphi(x_0)\right)$]
                (1, 1.487) [$\left(x_1, \varphi(x_1)\right)$]
                (2, 2.210) [$\left(x_2, \varphi(x_2)\right)$]
                (3, 3.287) [$\left(x_3, \varphi(x_3)\right)$]
                (4, 4.888) [$\left(x_k, \varphi(x_k)\right)$]
                (5, 7.268) [$\left(x_5, \varphi(x_5)\right)$]
            };
            \addlegendentry{Tracé par morceaux de $\varphi_n$}
            \end{axis}
        \end{tikzpicture}
    }
    
    \item Pour $n \in \bdN$ et $x \geq 0$, expliciter le réel $\varphi_n(x)$ en fonction de $x$     et de $n$.
    
        \textit{(On calculera $\varphi_n(x_k)$ en fonction de $\varphi_n(x_{k-1})$)}
    
    \boxans{
        Pour $k \in \llbracket0,n-1\rrbracket$, on a $\varphi_n(x_{k+1})=\varphi_n(x_k)+\varphi_n(x_k)(x_{k+1}-x_k)=\varphi_n(x_k)(1+\frac{x}{n})$. Il s'agit d'une suite géométrique de raison $(1+\frac{x}{n})$ et de premier terme $\varphi_n(0)=1$. Or $\varphi_n(x)=\varphi(x_n)$.
        
        Donc $\varphi_n(x)=\varphi(0)\left(1+\frac{x}{n}\right)^n$ donc \boxsol{$\varphi_n(x)=\left(1+\frac{x}{n}\right)^n$}
    }
    \textbf{N.B.} \textit{Rien ne prouve à ce stade que l'approximation est convenable. Dans la suite, on va démontrer que pour tout $x \in \bdR$, $f(x) = \lim\limits_{n \to +\infty}\varphi_n(x)$.}
    
\end{enumerate}

\subsection*{Partie B.2 - Existence de la fonction exponentielle réelle}

On souhaite dans cette partie, établir l'existence d'une fonction $f$ définie et dérivable sur $\bdR$, solution de $(E)$ telle que $f(0) = 1$.

Pour tout réel $x$ et $n$ un entier tel que $n > \mod x$, on pose :

\[ u_n(x) = \left(1 + \dfrac{x}{n}\right)^n\]

Soit $x$ un réel que l'on considérera fixé jusqu'à la question 7.

\begin{enumerate}
    \item Justifier que $\forall n > \mod x$, \quad $u_n(x) > 0$.
    
    \boxans{
        Soit $n$ un entier tel que $n > \mod x$. Alors $\frac{x}{n} > -1$ donc $1 +\frac{x}{n} > 0$. Donc \boxsol{$\forall n > \mod x$, \quad $u_n(x) > 0$}
    }
    
    \item Établir l'inégalité de Bernoulli :
        $\forall a \in \left]-1;+\infty\right[, \forall n \in \bdN^*, \quad (1+a)^n \geq 1 + na$
        
    \boxans{
        Soit $a \in \left]-1;+\infty\right[$ et $n \in \bdN^*$. Soit $u$ la fonction définie et dérivable sur $\bdR$ telle que $u(x)=(1+x)^n$. On a :
        $u'(x) = n(1+x)^{n-1}$.La tangente $T_0$ à $u$ en $0$ est alors $y = u(0) + (x-0)u'(0) = 1^n + xn\times1^{n-1} = 1 + xn$.
        
        Pour $x \in \left]-1;+\infty\right[$, $1 + x > 0$, donc par stricte croissance des fonctions puissances sur $x \in \left]0;+\infty\right[$, $u'$ est strictement croissante sur $\left]-1;+\infty\right[$, donc $u$ est convexe sur $\left]-1;+\infty\right[$ et donc au dessus de ses tangentes.
        En évaluant $T_0$ en $a$, on obtient \boxsol{$\forall a \in \left]-1;+\infty\right[, \forall n \in \bdN^*, \quad (1+a)^n \geq 1 + na$}
    }
    
    \item Soit $n$ un entier tel que $n > \mod x$.
    \begin{enumerate}
        \item Démontrer que : 
            $u_{n+1}(x)=u_n(x)\left(1+\dfrac{x}{n}\right)\left(\dfrac{1+\frac{x}{n+1}}{1+\frac{x}{n    }}\right)^{n+1}$
        
        \boxans{
            \boxsoll{$u_n(x)\left(1+\dfrac{x}{n}\right)\left(\dfrac{1+\frac{x}{n+1}}{1+\frac{x}{n}}\right)^{n+1}$} $= \left(\left(1+\frac{x}{n}\right)\dfrac{1+\frac{x}{n+1}}{1+\frac{x}{n}}\right)^{n+1} = \left(1 + \dfrac{x}{n+1}\right)^{n+1}$\boxsolr{$=u_{n+1}(x)$}
        }
        
        \item En utilisant l'inégalité de Bernoulli, montrer que :                              $\left(\dfrac{1+\frac{x}{n+1}}{1+\frac{x}{n}}\right)^{n+1} \geq                     \dfrac{1}{1+\frac{x}{n}}$
        
        \boxans{
            On a $\left(\dfrac{1+\frac{x}{n+1}}{1+\frac{x}{n}}\right)^{n+1} = \dfrac{\left(1+\frac{x}{n+1}\right)^{n+1}}{\left(1+\frac{x}{n}\right)^{n+1}} = \dfrac{\left(1+\frac{x}{n+1}\right)^{n+1}}{\left(1+\frac{x}{n}\right)^n\left(1+\frac{x}{n}\right)}$. Donc d'après l'inégalité de Bernoulli :
        
            $\left(\dfrac{1+\frac{x}{n+1}}{1+\frac{x}{n}}\right)^{n+1} \geq \dfrac{1+(n+1)\frac{x}{n+1}}{\left(1+(n)\frac{1}{n}\right)\left(1+\frac{x}{n}\right)} = \dfrac{1+x}{\left(1+x\right)\left(1+\frac{x}{n}\right)}$.
            Donc \boxsol{$\left(\dfrac{1+\frac{x}{n+1}}{1+\frac{x}{n}}\right)^{n+1} \geq \dfrac{1}{1+\frac{x}{n}}$}
        }
        
        \item En déduire que la suite $(u_n(x))_{n>\mod x}$ est croissante.
        
        \boxans{
            On a $u_n(x)\left(1+\dfrac{x}{n}\right)\left(\dfrac{1}{1+\frac{x}{n}}\right) \leq u_n(x)\left(1+\dfrac{x}{n}\right)\left(\dfrac{1+\frac{x}{n+1}}{1+\frac{x}{n}}\right)^{n+1}$, donc \boxsol{$u_n \leq u_{n+1}$}
        }
    \end{enumerate}
    
    \small{On sait (Partie A) que la fonction $f$ solution du problème, si elle existe, vérifie $f(x)f(-x)=1$, pour cela on introduit aussi la suite $u_n(x-)$ dans les questions suivantes.}
    
    \item Soit $n$ un entier tel que $n \geq \mod x$.  Montrer que $1 - \dfrac{x^2}{n} \leq     u_n(x)u_n(-x) \leq 1$.
    
    \boxans{
        $u_n(x)u_n(-x) = \left(\left(1+\dfrac{x}{n}\right)\left(1-\dfrac{x}{n}\right)\right)^n = \left(1-\left(\dfrac{x}{n}\right)^2\right)^n$, $x \leq n$ donc $\left(\dfrac{x}{n}\right)^2 \leq 1$ donc $1-\left(\dfrac{x}{n}\right)^2 \leq 1$.
        D'après l'inégalité de Bernoulli, $\left(1+\dfrac{-x^2}{n^2}\right)^n \geq 1+n\dfrac{-x^2}{n^2}$ donc \boxsol{$1 - \dfrac{x^2}{n} \leq u_n(x)u_n(-x) \leq 1$}
    }
    
    \item Vérifier que la suite $(1/u_n(-x))_{n > \mod x}$ est monotone et en déduire qu'elle est convergente.
    
    \boxans{
        La suite $(u_n(-x))_{n>\mod{-x}}$ est croissante donc par composition et décroissance de la fonction inverse la suite $(1/u_n(-x))_{n > \mod x}$ est décroissante. Or $\frac{1}{u_n(-x)} = \left(\dfrac{1}{1-\frac{x}{n}}\right)^{n}$ et $x < n$ donc $1 - \frac{x}{n} > 0$ donc la suite $(u_n(-x))_{n>\mod x}$ est minorée par 0.
        D'après le théorème de convergence monotone, \boxsol{ la suite $(1/u_n(-x))_{n > \mod x}$ converge}.
    }
    
    \item En déduire que la suite $(u_n(-x))_{n > \mod x}$ converge vers la même limite.
    
    On notera $f(x)$ la limite commune des suites de terme général $u_n(x)$ et $1/u_n(-x)$.
    
    \boxans{
        On pose $L = \lim\limits_{n\to+\infty}\left(\dfrac{1}{u_n(-x)}\right)$. On a $u_n(x)u_n(-x) \leq 1$ donc $u_n(x) \leq \dfrac{1}{u_n(-x)}$ donc $u_n(x) \leq L$.
        
        Donc la suite $(u_n(-x))_{n > \mod x}$ est majorée. Or la suite $(u_n(-x))_{n > \mod x}$ est croissante, donc d'après le théorème de convergence monotone, \boxsol{ la suite $(u_n(x))_{n > \mod x}$ converge}. On pose donc $L' =\lim\limits_{n\to+\infty}\left(u_n(x)\right)$. On a $L' \leq L$.
        
        De plus $1 - \dfrac{x^2}{n} \leq u_n(x)u_n(-x)$ donc $\lim\limits_{n\to+\infty} \left(\dfrac{1 - \frac{x^2}{n}}{u_n(-x)}\right) \leq L'$ donc $\lim\limits_{n\to+\infty} \left(\dfrac{1}{u_n(-x)}\right) \leq L'$ donc $L \leq L'$.
        
        Donc d'après le théorème d'encadrement, $L=L'$, on note donc \boxsol{$f(x) = \lim\limits_{n\to+\infty}\left(u_n(x)\right) = \lim\limits_{n\to+\infty}\left(\dfrac{1}{u_n(-x)}\right)$}
    }
    
    \item Justifier les encadrements suivants : pour tout $x \in \bdR$ et $n$ entier tel que $n > \mod x$ :
    
    \[ 0 < u_n(x)  \leq f(x) \leq \dfrac{1}{u_n(-x)} \quad \text{et} \quad 0 \leq f(x) -u_n(x) \leq \dfrac{x^2}{n}\dfrac{1}{u_n(-x)}\]
    
    \boxans{
        La suite $(1/u_n(-x))_{n > \mod x}$ est décroissante et de limite $f(x)$, donc $f(x) \leq \dfrac{1}{u_n(-x)}$. 
        La suite $(u_n(-x))_{n > \mod x}$ est strictement positive, croissante et de limite $f(x)$ donc $0 < u_n(x)  \leq f(x)$, donc \boxsol{$0 < u_n(x)  \leq f(x) \leq \dfrac{1}{u_n(-x)}$}.
        
        On a $u_n(x)  \leq f(x) \leq \dfrac{1}{u_n(-x)}$ donc $0  \leq f(x) - u_n(x) \leq \dfrac{1}{u_n(-x)}-u_n(x)$. Donc $0  \leq f(x) - u_n(x)$ et
        
        $f(x) - u_n(x)\leq \dfrac{1-u_n(x)u_n(-x)}{u_n(-x)}$ donc $f(x) - u_n(x)\leq \dfrac{1-(1-\frac{x^2}{n})}{u_n(-x)}$, donc \boxsol{$0 \leq f(x) -u_n(x) \leq \dfrac{x^2}{n}\dfrac{1}{u_n(-x)}$}. 
    }
    
        On désigne par $f$ la fonction qui à tout réel $x$ associe $f(x)$.
    \textit{On va démontrer que $f$ est une fonction dérivable solution de $(E)$ avec $f(0) = 1$.}
    
    \item Vérifier que $f(0) = 1$ et que : \quad $\forall x \in \bdR$, $f(x)f(-x) = 1$
    
    \boxans{
        $f(0) = \lim\limits_{n\to+\infty}\left(1+\dfrac{0}{n}\right)^n=\lim\limits_{n\to+\infty}1^n$ donc \boxsol{$f(0)=1$}. Soit $x \in \bdR$. On a $1 - \dfrac{x^2}{n} \leq u_n(x)u_n(-x) \leq 1$, or $\lim\limits_{n\to+\infty}\left(1 - \dfrac{x^2}{n}\right)=1$ donc d'après le théorème d'encadrement, \boxsol{ $\forall x \in \bdR$, $f(x)f(-x) = 1$}
    }
    
    \item Soit $x$ et $h$ deux réels fixés.
    \begin{enumerate}
        \item Soit $n$ un entier tel que $n > \mod x + \mod h$. \quad Montrer que $\dfrac{u_n(x + h)}{u_n(x)} \geq 1 + \dfrac{1}{1 + \frac{x}{n}}$
        
        \boxans{
            $\dfrac{u_n(x + h)}{u_n(x)} = \dfrac{\left(1+\frac{x+h}{n}\right)^n}{\left(1+\frac{x}{n}\right)^n} = \left(\dfrac{1+\frac{x}{n}+\frac{h}{n}}{1+\frac{x}{n}}\right)^n = \left(1+\dfrac{h}{n+x}\right)^n$. Donc d'après l'inégalité de Bernoulli :
            
            $\dfrac{u_n(x + h)}{u_n(x)} \geq 1+n\dfrac{h}{n+x}$ donc \boxsol{$\dfrac{u_n(x + h)}{u_n(x)} \geq 1 + \dfrac{h}{1 + \frac{x}{n}}$}.
        }
        
        \item En déduire $f(x+h) - f(x) \geq hf(x)$
        
        \boxans{
            En passant à la limite, on obtient $\dfrac{f(x+h)}{f(x)}\geq 1 + h$, donc $\dfrac{f(x+h)}{f(x)}\geq h$, donc \boxsol{$f(x+h) - f(x) \geq hf(x)$}.
        }
    \end{enumerate}
    
    \item En utilisant l'inégalité précédente, montrer que
    
    \[ \forall x \in \bdR, \forall h \in \bdR, \quad f(x-h) \geq (1-h)f(x) \quad \text{puis} \quad \forall x \in \bdR, \forall h \in \left]-1;1\right[, \quad f(x+h) \leq \dfrac{1}{1-h}f(x)\]
    
    \boxans{
        Soit $x \in \bdR$ et $h \in \bdR$. En utilisant l'inégalité précédente avec $-h$, on obtient $f(x+(-h))-f(x) \geq (-h)f(x)$ donc $f(x-h) \geq f(x)-hf(x)$, donc \boxsol{$\forall x \in \bdR, \forall h \in \bdR, \quad f(x-h) \geq (1-h)f(x)$}.
        
        Soit $x \in \bdR$ et $h \in \left]-1;1\right[$. On pose $x' = x+h$. On a $f(x'-h) \geq (1-h)f(x')$.
        Or $h < 1$ donc $1-h > 0$ donc $\dfrac{1}{1-h}f(x+h-h)\geq f(x+h)$. Donc \boxsol{$\forall x \in \bdR, \forall h \in \left]-1;1\right[, \quad f(x+h) \leq \dfrac{1}{1-h}f(x)$}
    }
    
    \item Soit $x_0$ un réel fixé. Établir la dérivabilité de $f$ à droite et à gauche en $x_0$ et vérifier que $f$ est dérivable en $x_0$ de dérivée $f(x_0)$.
    
    \boxans{
        Soit $h \in \left]0;1\right[$. D'une part $f(x_0 + h) \leq \dfrac{1}{1-h}f(x_0)$ donc $f(x_0 + h) - f(x_0) \leq \dfrac{1-(1-h)}{1-h}f(x_0)$ donc $\dfrac{f(x_0 + h) - f(x_0)}{h} \leq \dfrac{1}{1-h}f(x_0)$.
        D'autre part $hf(x_0) \leq f(x_0 + h) - f(x)$ donc $f(x_0) \leq \dfrac{f(x_0 + h) - f(x_0)}{h}$. Donc
        \( f(x_0) \leq \dfrac{f(x_0 + h) - f(x_0)}{h} \leq \dfrac{1}{1-h}f(x_0) \).
        Or $\lim \limits_{h\to 0}\left(\dfrac{1}{1-h}\right)=1$ donc d'après le théorème d'encadrement, \boxsol{$\lim \limits_{\substack{h\to 0\\h > 0}}\left(\dfrac{f(x_0 + h) - f(x_0)}{h}\right)=f(x_0)$}.
        
        Soit $h \in \left]-1;0\right[$. Similairement, \( f(x_0) \geq \dfrac{f(x_0 + h) - f(x_0)}{h} \geq \dfrac{1}{1-h}f(x_0) \), donc d'après le théorème d'encadrement, \boxsol{$\lim \limits_{\substack{h\to 0\\h < 0}}\left(\dfrac{f(x_0 + h) - f(x_0)}{h}\right)=f(x_0)$}. Donc \boxsol{$f$ est dérivable en $x_0$ de dérivée $f(x_0)$}.
    }
    
    \item Conclure.
    
    \boxans{
        La fonction $f$ définie sur $\bdR$ par $f(x) = \lim\limits_{n\to+\infty}\left(1+\dfrac{x}{n}\right)^n$ est solution de l'équation différentielle $y' = y$ sur $\bdR$ avec $f(0) = 1$. Ainsi définie, \boxsol{on a donc montré l'existence (et donc l'unicité) de la fonction exponentielle réelle}.
    }
    
    \begin{itshape}
        On a ainsi établi l'existence et l'unicité d'une solution de $y' = y$ et $y(0) = 1$.
        
        On appelle \underline{exponentielle réelle}, notée $\exp$ cette unique solution.
        
        On note $e = \exp(1)$ et pour $x$ réel $e^x = \exp(x)$ car la relation fonctionnelle $\forall (x,y) \in \bdR^2$, $\exp(x+y)=\exp(x)\exp(y)$ (cf partie A) coïncide avec les propriétés des exposants permet d'écrire sans confusion : \quad $\forall (x,y) \in \bdR^2$, $e^{x+y}=e^x e^y$
    \end{itshape}
    
    \item Écrire un programme (langage ou pseudo-langage au choix) qui permet d'obtenir une valeur approchée de $e = exp(1)$ à $10^{-3}$ près. On utilisera le deuxième encadrement de la question 7.
    
    \boxans{
        D'après le deuxième encadrement de la question 7, on a $e - u_n(1) \leq \dfrac{1}{nu_n(-1)}$. Dans un premier temps, on cherche donc $n > 1$ tel que $\dfrac{1}{nu_n(-1)} = \dfrac{1}{n\left(1-\frac{1}{n}\right)^n} < 10^{-3}$, puis on calcule $u_n(1)$.
        
    }
    \text{}\\[-12mm]
    \begin{code}{Python}{approx.py}
precision = 3
        
# On calcule la valeur de n
n, u_n, p = 1, 1, 10**(-precision)
while u_n > p:
    n += 1
    u_n = 1 / (n * (1 - 1/n)**n)

# On calcule et on affiche ensuite l'estimation de e
print(f'e ~ {(1 + 1/n)**n/p//1*p}')
    \end{code}
    \text{}\\[-12mm] \boxans{
    Le programme ci-dessus renvoie l'approximation \boxsol{$e \approx 2.718$}
    }
    
    \begin{itshape}
        Remarque : on sera capable, avec le cours de MP2I, de montrer que l'erreur dans cette approximation est majorée par $e/(2n)$ ce qui explique que la convergence est lente.
    \end{itshape}

\end{enumerate}

\subsection*{Partie C - Une autre limite pour $e^x$}

Pour $x \in \bdR$ et $n \in \bdN$, on note $\displaystyle f_n(x)=\sum_{k=0}^n \dfrac{x^k}{k!}$.

\begin{enumerate}
    \item Montrer, par étude de fonction, l'inégalité : \quad $\forall x \in \bdR$, $e^x \geq 1     + x$
    
        En déduire $\lim \limits_{x\to+\infty} e^x$ puis déterminer $\lim \limits_{x\to-\infty} e^x$
    
    \boxans{
        la fonction $\exp$ est convexe sur $\bdR$, donc au dessus de toutes ses tangentes.
        Sa tangente en $0$ est donnée par $y = \exp(0) + (x-0)\exp'(0) = 1 + x$. Donc \boxsol{$\forall x \in \bdR$, $e^x \geq 1 + x$}. Or $\lim \limits_{x\to+\infty} 1 + x = +\infty$.
        Donc d'après le théorème de minoration, \boxsol{$\lim \limits_{x\to+\infty} e^x = +\infty$}. $\lim \limits_{x\to-\infty} e^x = \lim \limits_{x\to+\infty} \dfrac{1}{e^x}$ donc \boxsol{$\lim \limits_{x\to-\infty} e^x = 0$.}
    }
    
    \item
    \begin{enumerate}
        \item Montrer que $P(x) = f_n(x) - (1+\frac{x}{n})^n$ est un polynôme en $x$ à coefficients positifs ou nuls (on pourra développer par la formule du binôme de Newton).
        
        \boxans{
            Soit $x \in \bdR$. $\displaystyle P(x)=\sum_{k=0}^n \dfrac{x^k}{k!}-\sum_{k=0}^n \left(\dfrac{x}{n}\right)^k=\sum_{k=0}^n x^k\dfrac{x^k}{k!}-\dfrac{x^k}{n^k}$. Or $\forall k \in \llbracket0;n\rrbracket$, $k \leq n$, donc $k! \leq n^k$.
            
            Donc \boxsol{$\displaystyle P(x)=\sum_{k=0}^n x^k\lambda_k$, avec $\forall k \in \llbracket0;n\rrbracket$, \quad $\lambda_k = \dfrac{n^k-k!}{n^kk!} \geq 0$}
        }
        
        \item En déduire que : \quad $\forall x \geq 0$, $\left(1 + \dfrac{x}{n}\right)^n \leq f_n(x)$.
        
        \boxans{
            $P$ est un polynôme à coefficients positifs ou nuls, donc pour tout réel $x$ positif, $P(x)$ est positif, donc $f_n(x) - (1+\frac{x}{n})^n > 0$. Donc \boxsol{$\forall x \geq 0$, $\left(1 + \dfrac{x}{n}\right)^n \leq f_n(x)$}
        }
    \end{enumerate}
    
    \item
    \begin{enumerate}
        \item Établir que : \quad $\forall n \in \bdN$, $\forall x \geq 0$, $f_n(x) \leq e^x$. On pourra étudier les variations de $\exp - f_n$ sur $\bdR^+$.
        
        \boxans{
            $\forall n \in \bdN$, \qquad on pose $P(n) : \forall x \geq 0$, \quad $e^x - f_n(x) \geq 0$. Pour $n_0 = 0$, $P(0)$ est immédiat.
            
            Soit $n \in \bdN$ tel que $P(n)$ est vrai. Sur $\bdR^+$, on pose $u=\exp - f_{n+1}$. La fonction $\exp$ est dérivable sur $\bdR^+$ et $f_{n+1}$ est un polynôme réel donc dérivable sur $\bdR^+$, donc $u$ est dérivable sur $\bdR^+$. Soit $x \in \bdR^+$, on a:
            
            \[  u'(x) = e^x - \sum_{k=1}^{n+1} \dfrac{kx^{k+1}}{k!} = = e^x - \sum_{k=0}^{n} \dfrac{(k+1)x^{k}}{(k+1)!} = e^x - \sum_{k=0}^n \dfrac{x^k}{k!} = e^x - f_n(x) \]
            
            Par hypothèse de récurrence, $u'(x) \geq 0$ donc $u$ est croissante sur $\bdR^+$, donc $u(x) \geq u(0)$.
            
            Or $\displaystyle u(0) = e^0 - \sum_{k=0}^{n+1} \dfrac{0^k}{k!} = 1 - 1 = 0$. Donc $e^x-f_{n+1} \geq 0$. donc $P(n+1)$ est vrai.
            
            $P(0)$ est vrai et $\forall n \in \bdN$, $\displaystyle P(n) \implies P(n+1)$ donc par récurrence, \boxsol{$\forall n \in \bdN$, $\forall x \geq 0$, $f_n(x) \leq e^x$}
        }
        
        \item En déduire pour tout entier $p \in \bdN$, la limite $\lim\limits_{x\to+\infty} \dfrac{e^x}{x^p}$.
        
        \boxans{
            Soit $p \in \bdN$ et $x \in \bdR^+$. $e^x \geq f_{p+1}(x)$ donc $\displaystyle \dfrac{e^x}{x^p} \geq \dfrac{\sum_{k=0}^{p+1} \dfrac{x^k}{k!}}{x^p}$ donc $\displaystyle \dfrac{e^x}{x^p} \geq \dfrac{x}{(p+1)!} + \sum_{k=0}^{p} \dfrac{x^{k-p}}{k!}$.
            
            Or $\displaystyle \lim\limits_{x\to+\infty}\left(\dfrac{x}{(p+1)!} + \sum_{k=0}^{p} \dfrac{x^{k-p}}{k!}\right) = \lim\limits_{x\to+\infty} \dfrac{x}{(p+1)!} + \lim\limits_{x\to+\infty} \sum_{k=0}^{p} \dfrac{x^{k-p}}{k!} = +\infty + 0 = +\infty$.
            
            D'après le théorème de minoration, on a donc \boxsol{$\lim\limits_{x\to+\infty} \dfrac{e^x}{x^p} = +\infty$}
        }
        
        \item Déduire des deux encadrements précédents pour tout $x \geq 0$, on a $e^x = \lim\limits_{n\to+\infty} f_n(x)$
        
        \boxans{
            Soit $x \in \bdR^+$. On a $\forall n\in \bdN$, \quad $\left(1 + \dfrac{x}{n}\right)^n \leq f_n(x) \leq e^x$. De plus $\lim\limits_{n\to+\infty} \left(1 + \dfrac{x}{n}\right)^n = e^x$.
            
            D'après le théorème d'encadrement, \boxsol{la limite de $f_n$ existe et est telle que $\lim\limits_{n\to+\infty} f_n(x) = e^x$}.
        }
    
    \end{enumerate}
    \item
    \begin{enumerate}
        \item Justifier que si $P$ est un polynôme à coefficients positifs, on a $\forall x \in \bdR$, $\mod{P(x)} \leq P(\mod x)$
        
        \boxans{
            Soit $P \in \bdR[X]$, $n = \deg P$ et $\forall k \in \llbracket0;n\rrbracket$, $\lambda_k \in \bdR^+$ tel que $\displaystyle P = \lambda_nX^n + \dots + \lambda_1X +\lambda_0 = \sum_{k=0}^n \lambda_kX^k$.
            
            Soit $x \in \bdR$. $\displaystyle \mod{P(x)} = \mod{\sum_{k=0}^n \lambda_kx^k} \leq \sum_{k=0}^n \mod{\lambda_kx^k} = \sum_{k=0}^n \lambda_k\mod{x^k}$ donc \boxsol{$\forall x \in \bdR$, $\mod{P(x)} \leq P(\mod x)$}.
            
            
        }
        \item En déduire $\forall x \in \bdR$, $\forall n \in \bdN$, $\mod{f_n(x) - \left(1 + \dfrac{x}{n}\right)^n} \leq f_n(\mod x) - \left(1 + \dfrac{\mod x}{n}\right)^n$.
        
        \boxans{
            $P(x) = f_n(x) - (1+\frac{x}{n})^n$ est un polynôme à coefficients positifs donc en appliquant le résultat précédent, on a de manière immédiate \boxsol{$\forall x \in \bdR$, $\forall n \in \bdN$, $\mod{f_n(x) - \left(1 + \dfrac{x}{n}\right)^n} \leq f_n(\mod x) - \left(1 + \dfrac{\mod x}{n}\right)^n$}.
        }
        
        \item En déduire que $\forall x \in \bdR$, $\displaystyle e^x = \lim\limits_{n\to+\infty} \sum_{k=0}^n \dfrac{x^k}{k!}$ 
        
        
        \boxans{
            Le résultat a été montré plus haut pour $x \in \bdR^+$. Soit $x \in \bdR^-$ et $n \in \bdN$. On a :
            
            $\mod{f_n(x) - \left(1 + \dfrac{x}{n}\right)^n} \leq f_n(\mod x) - \left(1 + \dfrac{\mod x}{n}\right)^n$ donc $0 \leq \mod{f_n(x) - \left(1 + \dfrac{x}{n}\right)^n} \leq f_n(-x) - \left(1 + \dfrac{-x}{n}\right)^n$.
            Or $\lim\limits_{n\to+\infty} \left( f_n(-x) - \left(1 + \dfrac{-x}{n}\right)^n\right)=e^{-x} - e^{-x}=0$.
            Donc d'après le théorème d'encadrement, $\lim\limits_{n\to+\infty} \mod{f_n(x) - \left(1 + \dfrac{x}{n}\right)^n} = 0$ donc $\lim\limits_{n\to+\infty} \left(f_n(x) - \left(1 + \dfrac{x}{n}\right)^n\right) = 0$.
            
            Donc $\forall x \in \bdR^-, \lim\limits_{n\to+\infty} f_n(x) = e^x$, donc \boxsol{ $\forall x \in \bdR$, $\displaystyle e^x = \lim\limits_{n\to+\infty} \sum_{k=0}^n \dfrac{x^k}{k!}$ }.
        }
    \end{enumerate}
    
    \item Montrer que pour $x \geq 0$, on a $0 \leq e^x - f_n(x) \leq \dfrac{x^{n+1}}{(n+1)!}e^x$ (on pourra faire une étude de fonction).
    
    \boxans{
        Le résultat $e^x - f_n(x) \geq 0$ a été montré. $\forall n \in \bdN$, on pose $P(n) : \forall x \geq 0$, $e^x - f_n(x) \leq \dfrac{x^{n+1}}{(n+1)!}e^x$.
        
        Pour $n_0 = 0$, $e^x-f_0(x)=e^x-1$ et $\dfrac{x^{0+1}}{(0+1)!}e^x = xe^x$. On a $e^x-1 \leq xe^x$ donc $P(0)$ est vrai. 
        
        Soit $n \in \bdN$ tel que $P(n)$ est vrai. Sur $\bdR^+$, on pose $u(x)=\dfrac{x^{n+2}}{(n+2)!}e^x -e^x + f_{n+1}$. La fonction $\exp$ est dérivable sur $\bdR^+$, $f_{n+1}$ est un polynôme réel donc dérivable sur $\bdR^+$, et la fonction $x \mapsto \dfrac{x^{n+2}}{(n+2)!}$ est dérivable sur $\bdR^+$ donc $u$ est dérivable sur $\bdR^+$. On a montré plus haut que $f_{n+1}' = f_n$. Soit $x \in \bdR^+$, on a :
        
        \( u'(x) = \dfrac{(n+2)x^{n+1}}{(n+2)!}e^x + \dfrac{x^{n+2}}{(n+2)!}e^x -e^x + f_n(x) \geq  \dfrac{x^{n+1}}{(n+1)!}e^x -e^x + f_n(x) \).
        Par hypothèse de récurrence, $\dfrac{x^{n+1}}{(n+1)!}e^x \geq e^x - f_n(x)$ donc $\dfrac{x^{n+1}}{(n+1)!}e^x -e^x + f_n(x) \geq 0$ donc $u'(x) \geq 0$ donc $u(x)$ est croissante sur $\bdR^+$. Donc $u(x) \geq u(0)$. Or $u(0) = \dfrac{0^{n+2}}{(n+2)!}e^0 - e^0 - f_{n+1}(0) = 0 - 1 + 1 = 0$. Donc $\dfrac{x^{n+2}}{(n+2)!}e^x -e^x + f_{n+1} \geq 0$ donc $P(n+1)$ est vrai.
        
        $P(0)$ et $P(n)\implies P(n+1)$, donc par récurrence, \boxsol{$\forall x \in \bdR^+, \forall n \in \bdN, 0 \leq e^x - f_n(x) \leq \dfrac{x^{n+1}}{(n+1)!}e^x$}.
    }
    
    \item En utilisant la question précédente, écrire un programme (dans le langage de votre choix) renvoyant une valeur approchée de $e$ à $10^{-3}$ près et commenter l'amélioration par rapport au programme écrit dans la partie précédente (B.13.)
    
    \boxans{
        D'après la question précédente, on a $e-f_n(1)\leq \dfrac{1}{(n+1)!}e$. Comme $e$ figure dans les deux membres, et qu'on cherche à trouver $n > 0$ tel que $e-f_n(1) < 10^{-3}$, on majore $\dfrac{1}{(n+1)!}e$ par $\dfrac{4}{(n+1)!}$.
    
       Dans un premier temps, on cherche donc $n$ tel que $\dfrac{4}{(n+1)!} < 10^{-3}$, puis on calcule $f_n(1)$.\\
       
    }
    \text{}\\[-13mm]
    \begin{code}{Python}{approx2.py}
precision = 3

# On calcule récursivement la valeur de n
n, r, p = 0, 4, 10**(-precision)
while r > p:
    n += 1
    r /= (n+1)

# On calcule récursivement l'approximation e
e, f = 1, 1
for k in range(1,n+1):
    f *= k
    e += 1/f
    
# On affiche l'estimation de e
print(f'e ~ {e/p//1*p}')
    \end{code}
    \text{}\\[-13mm] \boxans{
    Le programme ci-dessus renvoie l'approximation \boxsol{$e \approx 2.718$}.\\
    
    On constate une nette amélioration par rapport au programme écrit dans la partie précédente, celui-ci ayant nettement moins de calculs à faire : 
    
    D'une part la bonne valeur de $n$ est atteint bien plus rapidement ($n = 7$ ici, pour $n = 2719$ avec le programme précédent).
    
    D'autre part les calculs sont ici faits récursivement : à chaque itération de la boucle correspond toujours le même nombre d'opération (contrairement au programme précédent).
    }
\end{enumerate}
\end{document}