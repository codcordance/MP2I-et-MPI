\documentclass[a4paper,french,bookmarks]{article}

\usepackage{./Structure/4PE18TEXTB}
\newboxans

\begin{document}
\stylizeDoc{Mathématiques}{Devoir Maison 15}{Algèbre Linéaire}

\section*{Problème}

Soit $f : \bdR^3 \to \bdR^3$ définie par :
%
\begin{equation}
     f\left(x, y, z\right) = \left(2y - 2z, x+ y - 2z, x - y\right)
\end{equation}
%
On note $\id = \id_{\bdR^3}$ et on rappelle que $f^0 = \id$ et si 
$n \in \bdN^*$,\quad $f^n = f \circ f \circ \dots \circ f$, $n$ fois.

On note $\widetilde 0$ l'endomorphisme nul de $\bdR^3$.

\subsection*{Préliminaires}
\addcontentsline{toc}{subsection}{Préliminaires}

\begin{enumerate}
    \item \begin{enumerate}
        \item Montrer que $f \in \bcL\left(\bdR^3\right)$.
        
        \boxans{
            On a $f \in \bcF(\bdR^3, \bdR^3)$, et $\bdR^3$ est un
            $\bdR$-espace vectoriel. Soit $u = \left(x, y, z\right) 
            \in \bdR^3$, $v = \left(x', y', z'\right) \in \bdR^3$ et
            $\lambda \in \bdR$.
            %
            \begin{align*}
                f\left(\lambda u + v\right) &= f\left(\lambda\left(x, y,
                z\right) + \left(x', y', z'\right)\right) =
                f\left(\lambda x + x', \lambda y + y', \lambda z +
                z'\right)\\
                &= \left(2\lambda y + 2y' - 2\lambda x - 2x', \lambda x'
                + x + \lambda y + y' - 2\lambda z - z', \lambda x + x' -
                \lambda y' - y\right)\\
                &= \lambda\left(2y - 2z, x + y - 2z, x - y\right) +
                \left(2y' - 2z', x' + y' - 2z', x' - y'\right)\\
                &= f\left(x, y, z\right) + f\left(x', y', z'\right) =
                \lambda f(u) + f(v)
            \end{align*}
            
            Donc on a bien $f \in \bcL(E)$.
        }
        
        \item Déterminer $\Ker f$ et $\Imm f$, et donner pour chaque
        sous-espace vectoriel une famille génératrice.
        
        \boxans{
            Soit $u = (x, y, z) \in \bdR^3$. On a :
            %
            \[ u \in \Ker f\iff f\left(u\right) = 0_{\bdR^3} \iff
            \left\lbrace\begin{array}{rcl}
                2y- 2z &=& 0  \\
                x + y - 2z &=& 0 \\
                x - y &=& 0\\
            \end{array}\right. \iff \left\lbrace\begin{array}{rcl}
                y &=& z  \\
                0 &=& 0 \\
                x  &=& y\\
            \end{array}\right. \iff u \in \Vect\left(\left(1, 1,
            1\right)\right)\]
            %
            Donc $\Ker f = \Vect\left(\left(1, 1, 1\right)\right)$. On a
            $\dim\left(\Ker f\right) = 1$ donc par \textit{théorème du
            rang}, $\dim\left(\Imm f\right) = 2$. On a :
            %
            \[ f\left(1, 0, 0\right) = \left(0, 1, 1\right) \in \Imm f
            \qquad\et\qquad f\left(0, 1, 0\right) = \left(2, 1,
            -1\right) \in \Imm f\]
            %
            Or $\left(0, 1, 1\right)$ et $\left(2, 1, -1\right)$ sont
            linéairement indépendants, donc $\Imm f =
            \Vect\left(\left(0, 1, 1\right), \left(2, 1,
            -1\right)\right)$
        }
        
        \item Vérifier que $\Imm f = \Vect\left(v, w\right)$ avec $v =
        \left(0, 1, 1\right)$ et $w = \left(1, 1, 0\right)$.
        
        \boxans{
            On a :
            %
            \[ f\left(1, 0, 0\right) = \left(0, 1, 1\right) = v \in \Imm
            f \qquad\et\qquad f\left(1, 1, \sfrac{1}{2}\right) =
            \left(1, 1, 0\right) = w \in \Imm f\]
            %
            Or $v$ et $w$ sont linéairement indépendants, donc $\Imm f =
            \Vect\left(v, w\right)$
        }
        
        \item $f$ est-elle injective ? surjective ? bijective ?
        
        \boxans{
            $\Ker f \neq \left\{0,_{\bdR^3}\right\}$ et $\Imm f \neq
            \bdR^3$, donc $f$ n'est ni injective, ni surjective, ni
            bijective.
        }
        
        \item On note $e_1 = \left(1, 0, 0\right)$, $e_2 = \left(0, 1,
        0\right)$ et $e_3 = \left(0, 0, 1\right)$, et on pose $u =
        \left(1, 1, 1\right)$.
        
        Montrer que $e_1$, $e_2$ et $e_3$ sont des combinaisons
        linéaires de $u$, $v$ et $w$.
        
        \boxans{
            \[\left\lbrace\begin{array}{ccccccc}
                u - v &=& \left(1, 1, 1\right) - \left(0, 1, 1\right)
                &=& \left(1, 0, 0\right) &=& e_1 \\
                v + w - u &=& \left(1, 1, 0\right) + \left(0, 1,
                1\right) - \left(1, 1, 1\right) &=& \left(0, 1, 0\right)
                &=& e_2\\
                u - w &=& \left(1, 1, 1\right) - \left(1, 1, 0\right)
                &=& \left(0, 0, 1\right) &=& e_3
            \end{array}\right.\]
            %
            Donc $e_1$, $e_2$ et $e_3$ sont des combinaisons linéaires
            de $u$, $v$ et $w$.
        }
        
        \item Soit $\left(x, y, z\right) \in \bdR^3$. Écrire ce vecteur
        comme une combinaison linéaire de $u$, $v$ et $w$.
        
        \boxans{
            $\left(x, y, z\right) = xe_1 + ye_2 + ze_3 = x\left(u -
            v\right) + y\left(v + w - u\right) + z\left(u - w\right) =
            \left(x - y + z\right)u + \left(y -x\right)v + \left(y -
            z\right)w$
        }
        
        \item En déduire que $\bdR^3 = \ker f \oplus \Imm f$.
        
        \boxans{
            Pour tout vecteur $u$ de $\bdR^3$, $\exists ! (x, y, z) \in
            \bdR^3$, tels que $u = (x, y, z)$. D'après la question
            précédente, il existe donc une unique écriture selon une
            combinaison linéaire $u$, $v$ et $w$. Or $\Ker f = \Vect u$
            et $\Imm f = \Vect\left(v, w\right)$, donc $\bdR^3 = \ker f
            \oplus \Imm f$.
        }
        
        \item\label{question:1h} On introduit $p$ la projection sur
        $\Ker f$ parallèlement à $\Imm f$.
        
        Calculer $p\left(u\right)$, $p\left(v\right)$ et
        $p\left(w\right)$ et en déduire une expression de $p\left(x, y,
        z\right)$ pour tout $\left(x, y, z\right) \in \bdR^3$.
        
        \boxans{
            On a $u \in \Ker f$ donc $p(u) = u$. De plus $v \in \Imm f$
            et $w \in \Imm f$ donc $p(v) = p(w) = 0$. On a alors :
            %
            \[ p\left(x, y, z\right) = p\left(\left(x - y + z\right)u +
            \left(y -x\right)v + \left(y - z\right)w\right)
            \eq{\text{linéarité}} \left(x - y + z\right)u\]
        }
    \end{enumerate}
    
    \item \begin{enumerate}
        \item Soit $\left(x, y, z\right) \in \bdR^3$. Calculer
        $f\left(e_i\right)$, $f^2\left(e_i\right)$ et
        $f^3\left(e_i\right)$ pour tout $i \in \{1, 2, 3\}$.
        
        En déduire une expression de $f^2\left(x, y, z\right)$ et de
        $f^3\left(x, y, z\right)$.
        
        \boxans{
            \begin{minipage}{0.45\linewidth}
                \centering
                \begin{NiceTabular}{>{\color{white}}cccc}[corners,hvlines,rules/color=[gray]{0.8}]
                    \CodeBefore
                        \rectanglecolor{white}{2-2}{4-4}
                        \columncolor{main1!70}{1}
                    \Body
                    \RowStyle[nb-rows=1,rowcolor=main1!70,color=white]{}
                        & $e_1$ & $e_2$ & $e_3$ \\
                        $\vphantom{\displaystyle\sum} f$ & $\left(0, 1,
                        1\right)$ & $\left(2, 1, -1\right)$ & $\left(-2,
                        -2, 0\right)$ \\
                        $\vphantom{\displaystyle\sum}f^2$ & $\left(0,
                        -1, -1\right)$ & $\left(4, 5, 1\right)$ &
                        $\left(-4, -4, 0\right)$ \\
                        $\vphantom{\displaystyle\sum}f^3$ & $\left(0, 1,
                        1\right)$ & $\left(8, 7, 1\right)$ & $\left(-8,
                        -8, 0\right)$
                \end{NiceTabular}
            \end{minipage}
        %
        \hfill
        %
        \begin{minipage}{0.50\linewidth}
            On a ci-contre $f\left(e_i\right)$, $f^2\left(e_i\right)$ et
            $f^3\left(e_i\right)$ pour tout $i$ dans $\{1, 2, 3\}$. On
            en déduit alors :
            %
            \[ f^2\left(x, y, z\right) = \left(4y -4z, -x + 5y - 4z, -x
            + y\right)\]
            %
            Ainsi que :
            %
            \[ f^3\left(x, y, z\right) = \left(8y - 8z, x + 7y - 8z, x +
            y\right)\]
        \end{minipage}
        }
        
        \item En déduire que $f^3 - f^2 - 2f = \widetilde 0$. On note
        désormais $P = X^3 - X^2 - 2X$, et on note $P\left(f\right)$
        l'endomorphisme obtenu en substituant $f$ à l’indéterminée $X$.
        Ainsi, $P\left(f\right) = f^3 - f^2 - 2f = \widetilde 0$.
        
        \boxans{
            On a $\left(f^3 - f^2 - 2f\right) \left(\begin{array}{c}
                x \\
                y\\
                z
            \end{array}\right) = \left(\begin{array}{c}
                8y - 8z - \left(4y - 4z\right) -2\left(2y - 2z\right) \\
                x + 7y - 8z - \left(-x + 5y - 4z\right) - 2\left(x + y -
                2z\right)\\
                x + y - \left(-x + y\right) - 2\left(x - y\right)
            \end{array}\right) = \left(\begin{array}{c}
                0 \\
                0\\
                0
            \end{array}\right) = 0_{\bdR^3}$.
        }
        
        \item Montrer qu'il existe $\left(a, b, c\right) \in \bdR^3$
        tels que
        %
        \begin{equation}
            \dfrac{1}{X^3 - X^2 - 2X} = \dfrac{a}{X} + \dfrac{b}{X-2} +
            \dfrac{c}{X+1}
        \end{equation}
        %
        puis que 
        %
        \begin{equation}
             1 = \sfrac{-1}{2}\left(X + 1\right)\left(X - 2\right) +
             \sfrac{1}{6}X\left(X+1\right) -
             \sfrac{1}{3}X\left(X-2\right)
        \end{equation}
        
        \boxans{
            Soient $\left(a, b, c\right) \in \bdR^3$. On a :
            
            \[ \dfrac{a}{X} + \dfrac{b}{X-2} + \dfrac{c}{X+1} =
            \dfrac{a\left(X - 2\right)\left(X+1\right) +
            bX\left(X+1\right) +
            cX\left(X-2\right)}{X\left(X-2\right)\left(X+1\right)} =
            \dfrac{\left(a + b + c\right)X^2 + \left(b - a - 2c\right)X
            - 2a}{X^3 - X^2 - 2X}\]
            %
            On veut que le numérateur soit égal à $1$, ainsi par
            identification on a $\left\lbrace\begin{array}{rcl}
                a + b + c &=& 0 \\
                - a + b - 2c &=& 0 \\
                -2a &=& 1
            \end{array}\right.$, ce qui donne $a = \sfrac{-1}{2}$, $b =
            \sfrac{1}{6}$ et $c = \sfrac{1}{3}$. L'égalité des
            numérateurs donne alors :
            %
            \[ 1 = \sfrac{-1}{2}\left(X + 1\right)\left(X - 2\right) +
            \sfrac{1}{6}X\left(X+1\right) -
            \sfrac{1}{3}X\left(X-2\right) \]
        }
        
        \item En déduire que $\id = \sfrac{-1}{2}\left(f +
        \id\right)\circ\left(f - 2\id\right) + \sfrac{1}{6}f\circ
        \left(f + \id\right) + \sfrac{1}{3} f \circ \left(f -
        2\id\right)$.
        
        \boxans{
            On spécialise l'égalité polynomiale ci-dessus en $f$,
            appartenant aux endomorphismes de $\bdR^3$ de neutre $\Id$ :
            \[ \Id = \sfrac{-1}{2}\left(f + \Id\right)\circ\left(f -
            2\id\right) + \sfrac{1}{6}f\circ \left(f + \Id\right) +
            \sfrac{1}{3} f \circ \left(f - 2\Id\right) \]
        }
        On pose trois endomorphismes $q$, $r$ et $s$ définis par :
        %
        \begin{equation}
            \left\lbrace\begin{array}{rcl}
                q &=& \sfrac{-1}{2}\left(f + \id\right)\circ\left(f -
                2\id\right)= \sfrac{-1}{2}\left(f^2 - f - 2\id\right)\\
                r &=& \sfrac{1}{6}f\circ \left(f +
                \id\right)=\sfrac{1}{6}\left(f^2 + f\right)\\
                s &=& \sfrac{1}{3} f \circ \left(f - 2\id\right) =
                \sfrac{1}{3}\left(f^2 - 2f\right)
            \end{array}\right.
        \end{equation}
        %
        On a donc $\boxed{q + r + s = \id}$.
    \end{enumerate}
    
\end{enumerate}

\subsection*{Calcul de $\mathsf{f^n}$}
\addcontentsline{toc}{subsection}{Calcul de fⁿ}

\begin{enumerate}[resume]
    \item \begin{enumerate}
        \item\label{question:3a} Montrer que $q$ est un projecteur.
        
        \boxans{
            On a $q \in \bcL\left(\bdR^3\right)$, $q =
            \sfrac{-1}{2}\left(f^2 - f -
            2\Id\right)$, donc
            $q^2 = \sfrac{-1}{2}\left(f^2 - f - 2\Id\right)\circ
            \sfrac{-1}{2}\left(f^2 - f - 2\Id\right)$ soit :
            %
            \[ q^2 = \sfrac{1}{4}\left(f^4 - f^3 - 2f^2 - f^3 +
            f^2 + 2f - 2f^2 + 2f + 4\Id\right) = \sfrac{-1}{4}\left(
            f^4 - 2f^3 - 3f^2 + 4f + 4\Id\right)\]
            
            Or $f^3 - f^2 - 2f = \widetilde 0$ donc $f^4 - f^3 - 2f^2 =
            \widetilde 0$ soit $q^2 = \sfrac{1}{4}\left(- f^3 - f^2 + 4f
            + 4\Id\right) = \sfrac{1}{4}\left(-2f^2 + 2f + 4\Id\right)$.
            %
            On obtient donc $q^2 = \sfrac{-1}{2}\left(f^2 - f -
            2\Id\right) = q$, donc $q$ est un projecteur.
        }
        
        \item Montrer que $\Ker q \subset \Imm f$ et que $q \circ f =
        \widetilde 0$. En déduire que $\Ker q = \Imm f$.
        
        \boxans{
            Soit $x \in \Ker q$. Alors $q\left(x\right) = 0_{\bdR^3}$,
            soit
            $\sfrac{-1}{2}\left(f^2 - f - 2x\right)\left(x\right) = 0$,
            et donc $\sfrac{1}{2}\left(f^2\left(x\right) -
            f\left(x\right)\right) = x$.
            
            On a donc $f\left(\sfrac{1}{2}\left(f\left(x\right) -
            x\right)\right) =
            x$, donc $x \in \Imm f$. De plus :
            %
            \[ q \circ f = \sfrac{-1}{2}\left(f^2 - f - 2\Id\right)\circ
            f = \sfrac{-1}{2}\left(f^3 - f^2 - 2f\right) = \widetilde
            0\]
            %
            Soit $y \in \Imm f$, donc $\exists x \in \bdR^3$ tel que $y
            = f(x)$. On a $q\left(y\right) = q \circ f \left(x\right) =
            0_{\bdR^3}$ donc $y \in \Ker q$.
            
            On conclut ainsi que $\Ker q = \Imm f$.
        }
        
        \item Montrer que $f \circ q = \widetilde 0$ et que $\Ker f
        \subset \Ker\left(q - \id\right)$. En déduire que $\Ker \left(q
        - \id\right) = \Ker f$.
        
        \boxans{
            Soit $x \in \Ker f$, donc $f(x) = 0_{\bdR^3}$. On a :
            %
            \[\left(q - \Id\right)\left(x\right) = q\left(x\right) - x =
            \sfrac{-1}{2}\left(f^2\left(x\right) - f\left(x\right) -
            2x\right) - x = \sfrac{-1}{2}f\left(f\left(x\right) -
            x\right) = \sfrac{-1}{2}f\left(x\right) = 0_{\bdR^3}\]
            %
            Donc $x \in \Ker\left(q - \Id\right)$. Par ailleurs $f$
            commute avec ses itérés par $\circ$ et avec $\Id$, donc $f
            \circ q = q \circ f = \widetilde 0$. Soit $x \in \bdR^3$, on
            a donc :
            %
            \[x \in \Ker\left(q - \Id\right) \implies q\left(x\right) =
            x \implies f \circ q \left(x\right) = f\left(x\right)
            \implies f(x) = 0_{\bdR^3} \implies x \in \Ker f\]
            %
            On a donc $\Ker \left(q - \id\right) = \Ker f$.
        }
        
        \item Décrire les espaces caractéristiques du projecteur $q$ et
        les comparer à ceux de $p$ (défini en \enumref{question:1h})
        
         Donner alors une relation simple entre $p$ et $q$.
        
        \boxans{
            $q$ est la projection géométrique sur $\Imm q = \Ker \left(q
            - \Id\right)$ parallèlement à $\Ker q$, donc la projection
            sur $\Ker f$ parallèlement à $\Imm f$. Il s'agit des mêmes
            espaces que pour la projection $p$, donc par caractérisation
            on a $p = q$.
        }
        
        \item Montrer que $r$ et $s$ sont deux projecteurs.
        
        \boxans{
            On a $r \in \bcL\left(\bdR^3\right)$ et $s \in
            \bcL\left(\bdR^3\right)$. Par ailleurs :
            %
            \begin{align*}
                 r^2 &= \sfrac{1}{36}\left(f^4 + 2f^3 + f^2\right) =
                 \sfrac{1}{36}\left(f^4 - f^3 + 3f^3 - 2f^2 +
                 3f^2\right) = \sfrac{1}{36}\left(3f^3 + 3f^2\right)\\
                 &= \sfrac{1}{36}\left(3f^3 - 3f^2 + 6f^2 - 6f +
                 6f\right) = \sfrac{1}{36}\left(6f^2 + 6f\right) =
                 \sfrac{1}{6}\left(f^2 +f\right) = r
            \end{align*}
            %
            On procède de même pour $s$, et on trouve donc que $r$ et
            $s$ sont bien des projecteurs.
        }
        
        \item Vérifier que $r \circ s = s \circ r = 0_{\bdR^3}$ et que
        $f \circ r  = 2r$ et $f \circ s = -s$.
        
        \boxans{
            $f$ commute avec ses itérés par $\circ$ donc $f$, $r$ et $s$
            commutent. On a alors :
            %
            \[\left\lbrace\begin{array}{c}
                s \circ r = r \circ s = \sfrac{1}{6}\left(f^2 +
                f\right)\circ \sfrac{1}{3}\left(f^2 - 2f\right) =
                \sfrac{1}{18}\left(f^4 - f^3 - 2f^2\right) = 
                0_{\bdR^3} \\[5pt]
                f \circ r = f \circ \sfrac{1}{6}\left(f^2 + f\right) =
                \frac{1}{6}\left(f^3 + f^2\right) =
                \sfrac{1}{6}\left(f^3 - f^2 + 2f^2 - 2f + 2f\right) =
                \sfrac{1}{6}\left(2f^2 + 2f\right) = 2r\\[5pt]
                f \circ s = f \circ \sfrac{1}{3}\left(f^2 - 2f\right) =
                \sfrac{1}{3}\left(f^3 - 2f^2\right) =
                \sfrac{1}{3}\left(f^3 - 2f^2 - 2f + 2f\right) =
                \sfrac{1}{3}\left(-f^2 + 2f\right) = -s
            \end{array}\right.\]
        }
        
        \item Montrer (sans longs calculs) que $f = 2r - s$.
        
        \boxans{
            On a $2r - s = f \circ r + f \circ s = f \circ \left(r +
            s\right) = f \circ \left(1 - q \right) = f - f \circ q = f$.
        }
        
        \item En déduire que pour tout entier $n \in \bdN^*$,\qquad $f^n
        = 2^nr + \left(-1\right)^n s$.
        
        \boxans{
            On a $f^n = \left(2r -s\right)^n$ donc on applique le binôme
            de Newton, sachant que $r$ et $s$ commutent :
            %
            \[ f^n = \sum_{k=0}^n \binom{n}{k}
            \left(2r\right)^k\left(-s\right)^{n-k} = \binom{n}{n}2^nr^n
            + \binom{n}{0}\left(-1\right)^ns^n + \sum_{k=1}^n
            \binom{n}{k}2^k\left(-1\right)r^ks^{n-k}\]
            %
            Or $p$ et $s$ sont des projecteurs donc l'expression se
            simplifie en :
            %
            \[ f^n = 2^nr + \left(-1\right)^ns + \sum_{k=1}^{n-1}
            \binom{n}{k}2^k\left(-1\right)^{n-k}sr = 2^nr +
            \left(-1\right)^ns\]
        }
        
        \item Calculer enfin $f^n\left(x, y, z\right)$ pour tout
        $\left(x, y, z\right) \in \bdR^3$ et $n \in \bdN^*$.
        
        \boxans{
            On a donc $f^n \sfrac{2^n}{6}\left(f^2 +
            f\right) + \sfrac{\left(-1\right)^n}{3}\left(f^2 - 2f\right)
            = \underbrace{\dfrac{2^n+2\left(-1\right)^n}{6}}_{\lambda_n
            }f^2 + \underbrace{\dfrac{2^n -
            4\left(-1\right)^n}{6}}_{\mu_n}f = \lambda_nf^2 + \mu_nf$.
            
            Or $f^2\left(\begin{array}{c}
                x\\
                y\\
                z
            \end{array}\right) = \left(\begin{array}{c}
                4y - 4z  \\
                -x + 5y - 4z \\
                - x + y
            \end{array}\right)$ d'où $f^n\left(\begin{array}{c}
                x \\
                y \\
                z
            \end{array}\right) =
            \left(\begin{array}{c}
                \left(4\lambda_n + 2\mu_n\right)\left(y - z\right) \\
                \left(\mu_n - \lambda_n\right)x + \left(5\lambda_n +
                \mu_n\right)y - \left(4\lambda_n + 2\mu_n\right)z \\
                \left(\mu_n - \lambda_n\right)\left(x - y\right)
            \end{array}\right)$
            %
            \[\text{De plus} \quad \left\lbrace\begin{array}{c}
            4\lambda_n + 2\mu_n = \dfrac{1}{6}\left(4\left(2^n +
            2\left(-1\right)^n\right) + 2\left(2^n -
            4\left(-1\right)^n\right)\right) =
            \dfrac{1}{6}\left(6\times2^n\right) = 2^n\\[5pt]
            \mu_n - \lambda_n = \dfrac{1}{6}\left(2^n -
            4\left(-1\right)^n - 2^n - 2\left(-1\right)^n\right) =
            \dfrac{-6}{6}\left(-1\right)^n =
            \left(-1\right)^{n+1}\\[5pt]
            5\lambda_n + \mu_n = \dfrac{1}{6}\left(5\left(2^n +
            2\left(-1\right)^n\right) + 2^n - 4\left(-1\right)^n\right)
            = \dfrac{1}{6}\left(6\times^2n - 6\left(-1\right)^n\right) =
            2^n + \left(-1\right)^n\end{array}\right.\]
            %
            On a donc $f^n\left(\begin{array}{c}
                x \\
                y \\
                z
            \end{array}\right) = \left(\begin{array}{c}
                2^n\left(y - z\right)  \\
                \left(-1\right)^{n+1}x + \left(2^n +
                \left(-1\right)^n\right)y - 2^nz\\
                \left(-1\right)^{n+1}\left(x - y\right)
            \end{array}\right) = \left(\begin{array}{c}
                2^n\left(y - z\right)  \\
                \left(-1\right)^n\left(y - x\right) + 2^n\left(y -
                z\right)\\
                \left(-1\right)^n\left(y - x\right)
            \end{array}\right)$
        }
    \end{enumerate}
    
\end{enumerate}

\subsection*{Autre méthode pour calculer $\mathsf{f^n}$}
\addcontentsline{toc}{subsection}{Autre méthode pour calculer fⁿ}
    
\begin{enumerate}[resume]
    \item \begin{enumerate}
        \item[] On pose $P = X^3 - X^2 - 2X$.
    
        \item Déterminer le reste de la division euclidienne de $X^n$
        par $P$.
        
        \boxans{
            Soit $n \in \bdN^*$, on pose la division euclidienne de
            $X^n$ par $P$ :
            %
            \[ \exists ! Q_n \in \bdR_{n-3}\left[X\right], \left(A_n,
            B_n,
            C_n\right) \in \bdR^3,\qquad X^n = Q_nP + A_nX^2 + B_nX +
            C_n\]
            %
            On évalue alors l'équation polynomiale en les racines de $P$
            (à savoir
            $0$, $-1$ et $2$) :
            %
            \[ \left\lbrace\begin{array}{rcl}
                0^n &=& Q_n\times 0 + A_n\times0^2 + B_n\times 0 + C_n
                \\
                \left(-1\right)^n &=& Q_n \times 0 +
                A_n\left(-1\right)^2 + B_n\left(-1\right) + C_n \\
                2^n &=& Q_n \times 0 + A_n\times2^2 + B_n\times 2
                + C_n
            \end{array}\right. \qquad\text{donc}\qquad
            \left\lbrace\begin{array}{rcl}
                C_n &=& 0 \\
                A_n - B_n &=& \left(-1\right)^n \\
                4A_n + 2B_n &=& 2^n
            \end{array}\right.\]
            %
            On a donc $A_n = \dfrac{1}{6}\left(2^n +
            2\left(-1\right)^n\right) =
            \lambda_n$ et $B_n = \dfrac{1}{6}\left(2^n -
            4\left(-1\right)^n\right)
            = \mu_n$ et $C_n = 0$.
        }
        
        \item On rappelle que $P\left(f\right) = \widetilde 0$. En
        déduire l'expression de $f^n$ en fonction de $f$ et $f^2$.
        
        \boxans{
            \[\forall n \in \bdN^*,\qquad X^n = Q_n(X)P(X) + A_nX^2 +
            B_nX + C_n =
            Q_n(X)P(X) + \dfrac{2^n + 2\left(-1\right)^n}{6}X^2 +
            \dfrac{2^n -
            4\left(-1\right)^n}{6}\]
            %
            Or $P\left(f\right) = \widetilde 0$ donc $f^n = A_n f^2 +
            B_n f =
            \lambda_n f^2 + \mu_n f$.
        }
        
        \item Retrouver l'expression de $f^n\left(x, y, z\right)$ pour
        tout $\left(x, y, z\right) \in \bdR^3$ et $n \in \bdN^*$.
        
        \boxans{
            On retrouve avec exactement le même développement
            $f^n\left(\begin{array}{c}
                x \\
                y \\
                z
            \end{array}\right)  \left(\begin{array}{c}
                2^n\left(y - z\right)  \\
                \left(-1\right)^n\left(y - x\right) + 2^n\left(y -
                z\right)\\
                \left(-1\right)^n\left(y - x\right)
            \end{array}\right)$.
        }
    \end{enumerate}
\end{enumerate}
    
\subsection*{Application 1 : Résolution d'un système de suites
récurrentes}
\addcontentsline{toc}{subsection}{Application 1 : Résolution d'un
système de suites récurrentes}

\begin{enumerate}[resume]
    \item \begin{enumerate}
        \item[] Soient $\suite{x_n}$, $\suite{y_n}$ et $\suite{z_n}$
        trois suites de $\bdR^\bdN$ telles que :
        %
        \begin{equation}
            \forall n \in \bdN,\qquad \left\lbrace\begin{array}{rcl}
                x_{n+1} &=& 2y_n - 2z_n  \\
                y_{n+1} &=& x_n + y_n - 2z_n \\
                z_{n+1} &=& x_n - y_n
        \end{array}\right.
        \end{equation}
        %
        \item Montrer que $\left(x_n, y_n, z_n\right) = f^n\left(x_0,
        y_0, z_0\right)$ pour tout entier $n \in \bdN^*$.
        
        \boxans{
            On a pour tout entier $n \in \bdN^*$ non nul
            :$\left(x_{n+1}, y_{n+1}, z_{n+1}\right) = f\left(x_n, y_n,
            z_n\right)$. Par récurrence simple, on montre donc
            $\left(x_n, y_n, z_n\right) = f^n\left(x_0, y_0,
            z_0\right)$.
        }
    
        \item En déduire les expressions explicites de $x_n$, $y_n$ et
        $z_n$ en fonction de $n$ ainsi que de $x_0$, $y_0$ et $z_0$.
        
        \boxans{
            On a donc $\left\lbrace\begin{array}{rcl}
                x_n &=& 2^n\left(y_0 - z_0\right)  \\
                y_n &=& \left(-1\right)^n\left(y_0 - x_0\right) +
                2^n\left(y_0 - z_0\right)\\
                z_n &=& \left(-1\right)^n\left(y_0 - x_0\right)
            \end{array}\right.$
        }
    \end{enumerate}
    
\end{enumerate}

\subsection*{Application 2 : Résolution d'un système d'équations
différentielles}
\addcontentsline{toc}{subsection}{Application 2 : Résolution d'un
système d'équations différentielles}

\begin{enumerate}[resume]
    \item \begin{enumerate}
        \item[] On considère trois fonctions $x$, $y$ et $z$ de $\bdR$
        dans $\bdR$ dérivables qui vérifient le système :
        %
        \begin{equation}
            \forall t \in \bdR,\qquad \left\lbrace\begin{array}{rcl}
                x'\left(t\right) &=& 2y\left(t\right) -
                2z\left(t\right)\\
                y'\left(t\right) &=& x\left(t\right) + y\left(t\right) -
                2z\left(t\right)\\
                z'\left(t\right) &=& x\left(t\right) - y\left(t\right)
            \end{array}\right.
        \end{equation}
        %
        avec la condition initiale $x\left(0\right) = x_0 \in \bdR$,
        $y\left(0\right) = y_0 \in \bdR$, et $z\left(0\right) = z_0 \in
        \bdR$.
        
        On a donc pour tout $t \in \bdR$, $f\left(x\left(t\right),
        y\left(t\right), z\left(t\right)\right) =
        \left(x'\left(t\right), y'\left(t\right),
        z'\left(t\right)\right)$
    
        \textbf{Rappel} : Pour tout $\left(x, y, z\right) \in \bdR^3$,
        on a :
        %
        \begin{equation}
            \forall t \in \bdR,\qquad \left\lbrace\begin{array}{rcl}
                q\left(x, y, z\right) &=& \left(x - y + z, x - y + z, x
                - y + z\right)\\
                r\left(x, y, z\right) &=& \left(y - z, y - z, 0\right)\\
                z'\left(t\right) &=& \left(0, y - x, y - x\right)
            \end{array}\right.
        \end{equation}
        
        \item Justifier (sans calculs) les égalités :
        %
        \begin{equation}
            r\left(x'\left(t\right), y'\left(t\right),
            z'\left(t\right)\right) = 2r\left(x\left(t\right),
            y\left(t\right), z\left(t\right)\right) \quad\text{et}\quad
            s\left(x'\left(t\right), y'\left(t\right),
            z'\left(t\right)\right) = -s\left(x\left(t\right),
            y\left(t\right), z\left(t\right)\right)
        \end{equation}
        
        \boxans{
            On a $r\left(x'\left(t\right), y'\left(t\right),
            z'\left(t\right)\right) = r\circ f\left(x\left(t\right),
            y\left(t\right),
            z\left(t\right)\right) = 2r\left(x\left(t\right),
            y\left(t\right),
            z\left(t\right)\right)$.
            
            De même $s\left(x'\left(t\right), y'\left(t\right),
            z'\left(t\right)\right) = s\circ f\left(x\left(t\right),
            y\left(t\right),
            z\left(t\right)\right) = -s\left(x\left(t\right),
            y\left(t\right),
            z\left(t\right)\right)$.
        }
        
        \item Soient $a, b$ et $c$ trois fonctions de $\bdR$ dans $\bdR$
        telles que $r\left(x\left(t\right), y\left(t\right),
        z\left(t\right)\right) = \left(a\left(t\right), b\left(t\right),
        c\left(t\right)\right)$.
        
        Calculer $a$, $b$ et $c$, vérifier qu'elles sont dérivables et
        que :
        %
        \begin{equation}
            r\left(x'\left(t\right), y'\left(t\right),
            z'\left(t\right)\right) = \left(a'\left(t\right),
            b'\left(t\right), c'\left(t\right)\right)
        \end{equation}
        
        \boxans{
            On a $r\left(x, y, z\right) = \left(y - z, y - z, 0\right)$
            donc pour
            tout $t \in \bdR$, $\left\lbrace\begin{array}{rcl}
                a(t) &=& b(t) - a(t) \\
                b(t) &=& b(t) - a(t) \\
                c(t) &=& 0
            \end{array}\right.$ Par ailleurs :
            %
            \[ \forall t \in \bdR,\qquad r\left(x'\left(t\right),
            y'\left(t\right),
            z'\left(t\right)\right) = \left(y'\left(t\right) -
            z'\left(t\right),
            y'\left(t\right) - z'\left(t\right), 0\right) =
            \left(a'\left(t\right),
            b'\left(t\right), c'\left(t\right)\right)\]
        }
        
        \item En déduire que $r\left(x\left(t\right), y\left(t\right),
        z\left(t\right)\right) = e^{2t}\times r\left(x_0, y_0,
        z_0\right)$.
        
        \boxans{
            On a $\left(a', b', c'\right) = r\left(x', y', z'\right) =
            2r\left(x,
            y, z\right) = 2\left(a, b, c\right)$, donc pour tout réel $t
            \in \bdR$,
            on obtient :
            %
            \[\left(a\left(t\right), b\left(t\right),
            c\left(t\right)\right) = \left(\lambda e^{2t}, \mu  e^{2t},
            \gamma 
            e^{2t}\right) \qquad\text{Aini on a :}\]
            %
            \[r\left(x\left(t\right), y\left(t\right),
            z\left(t\right)\right) = 
            e^{2t}\left(\lambda, \gamma, \mu\right) \qquad\text{et en
            évaluant en
            0}\qquad r(x_0, y_0, z_0) = \left(\lambda, \gamma,
            \mu\right)\]
            %
            On a donc bien pour tout $t \in \bdR$ que
            $r\left(x\left(t\right), y\left(t\right),
            z\left(t\right)\right) = e^{2t}\times r\left(x_0, y_0,
            z_0\right)$.
        }
        
        \item De même, montrer que $s\left(x\left(t\right),
        y\left(t\right), z\left(t\right)\right) = e^{-t}\times
        s\left(x_0, y_0, z_0\right)$.
        
        \boxans{
            On se donne $a, b, c$ des fonctions telles que $s\left(x, y,
            z\right) =
            \left(a, b, c\right)$. On obtient avec le même raisonnement
            qu'aux deux
            questions précédentes que $s\left(x\left(t\right),
            y\left(t\right),
            z\left(t\right)\right) = e^{-t}\times s\left(x_0, y_0,
            z_0\right)$.
        }
        
        \item Enfin, vérifier que $q\left(x'\left(t\right),
        y'\left(t\right), z'\left(t\right)\right) = 0_{\bdR^3}$ et en
        déduire une relation entre $x\left(t\right)$, $y\left(t\right)$
        et $z\left(t\right)$.
        
        \boxans{
            On a $q\left(x'\left(t\right), y'\left(t\right),
            z'\left(t\right)\right) = q\circ f\left(x\left(t\right),
            y\left(t\right),
            z\left(t\right)\right) = 0_{\bdR^3}$. Or :
            %
            \[ q\left(x', y',
            z'\right) = \left(x' - y' + z', x' - y' + z', x' - y' +
            z'\right) =
            \left(0, 0, 0\right) \qquad\text{donc}\qquad
            x'\left(t\right) -
            y'\left(t\right) + z'\left(t\right) = 0\]
            %
            En intégrant on a donc $x\left(t\right) - y\left(t\right) +
            z\left(t\right) = x_0 - y_0 + z_0$.
        }
        
        \item En déduire les expressions de $x\left(t\right)$,
        $y\left(t\right)$ et $z\left(t\right)$.
        
        \boxans{
            Les trois équations obtenues précédemment donnent :
            %
            \[ \left\lbrace\begin{array}{rcl}
                y\left(t\right) - z\left(t\right) &=& 
                e^{2t}\left(y_0 - z_0\right)  \\
                y\left(t\right) - x\left(t\right) &=& 
                e^{-t}\left(y_0 - x_0\right)\\
                x\left(t\right) - y\left(t\right) + z\left(t\right) &=&
                x_0 - y_0 + z_0
            \end{array}\right.\]
            %
            On obitent alors :
            %
            \[ \left\lbrace\begin{array}{rcl}
                x\left(t\right) &=& e^{2t}\left(y_0 - z_0\right) + x_0 -
                y_0 + z_0\\
                y\left(t\right) &=& e^{-t}\left(y_0 - x_0\right) + x_0 -
                y_0 + z_0\\
                z\left(t\right) &=& e^{2t}\left(y_0 - z_0\right) +
                e^{-t}\left(y_0 - x_0\right) + x_0 - y_0 + z_0
            \end{array}\right.\]
        }
    \end{enumerate}
\end{enumerate}

\end{document}