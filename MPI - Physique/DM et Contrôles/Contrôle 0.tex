\documentclass[a4paper,french,bookmarks]{article}

\usepackage{./Structure/4PE18TEXTB}

\newboxans

\begin{document}

    \stylizeDocSpe{Physique}{Contrôle 0}{Réponses}
    
    \section{Electrocinétique}
    
    \subsection{Circuits électriques}
    
    \begin{enumerate}
        \item Que signifie l'approximation des régimes quasi stationnaires ?
        
        \boxans{
            ARQS : $L \ll \lambda$ / ARQS $\iff \dfrac{\dif \rho}{\dif t} = 0$.
        }
        
        \item Enoncer la loi des noeuds. L'exprimer en fonction du potentiel du noeud considéré et de ceux de l'extrémité des branches qui lui sont liées.
        
        \boxans{
            La loi des noeuds donne $i_1 = i_2 + i_3$. Pour les potentiels, on a le théorème de Millman $V_A = \dfrac{\sum \frac{V_k}{Z_k}}{\sum \frac{1}{Z_k}}$
        }
        
        \item Pont diviseur
        
        \boxans{
            \[ U_BC = \dfrac{R_2}{R_1 + R_2}E \]
        }
        
        \item Mode de fonctionnement
        
        \item Equation de charge du condensateur
        
        \boxans{
            Retour ur $i = C\dfrac{\dif u_C}{\dif t}$ qui vient de $u_C = \dfrac{q}{C}$ qui vient de $V_A - V_B = \dfrac{Q_A}{C}$.
        }
    \end{enumerate}
    
\end{document}