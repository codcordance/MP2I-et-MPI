\documentclass[a4paper,french,bookmarks]{article}

\usepackage{./Structure/4PE18TEXTB}

\newboxans
\usepackage{booktabs}

\begin{document}

    \renewcommand{\thesection}{\Roman{section}} 
    \renewcommand{\thesubsection}{\thesection.\Alph{subsection}}
    \setlist[enumerate]{font=\color{white5!60!black}\bfseries\sffamily}
    \renewcommand{\labelenumi}{\thesection.\arabic{enumi}.}
    \renewcommand*{\labelenumii}{\thesection.\arabic{enumi}.\arabic{enumii}.}
    \renewcommand*{\labelenumiii}{\alph{enumiii}.}
    
    \stylizeDocSpe{Physique}{Devoir maison n° 4}{}{Pour le lundi 10 octobre 2022}
    
    \section{Tir d'un obus vers le zénith}
    
    \emph{Au \textsc{XVII}\ieme~siècle, le Père \textsc{Mersenne}, ami et
    correspondant de \textsc{Descartes}, se livra à un tir d'obus, le canon
    étant pointé vers le zénith. Le résultat ne fut pas du tout celui escompté.
    On se propose d'étudier l'influence de différents facteurs physiques sur
    la trajectoire de l'obus.}\medskip
        
    En un lieu $A$ de latitude $\lambda = \qty{48}{\degree N}$, un canon
    tire un obus à la vitesse $v_0 = \qty{100}{\metre\per\second}$ suivant la
    verticale ascendante $\left\vert A z\right)$. On désigne par $\p{A, x, y,
    z}$ un repère orthonormé lié à la Terre, $\left\vert A x \right)$ étant
    dirigé vers le sud. On assimile la Terre à une sphère homogène, tournant
    autour de l'axe des pôles à la vitesse angulaire $\omega_0 =
    \qty{7.5e-5}{\radian\per\second}$. On note $\vec{g_0}$ le champ de
    pesanteur terrestre, de module $g_0$ supposé constant égal à
    $\qty{10}{\metre\per\second\squared}$.

    \begin{enumerate}
        \item On considère le référentiel lié à la Terre.
        
        \begin{enumerate}
            \item On néglige la résistance de l'air.

            \begin{enumerate}
                \item\label{qu:I.1.1.a} Donner l'expression de la vitesse
                $\vec v$ de l'obus à un instant quelconque.
                
                \item Exprimer l'énergie mécanique de l'obus. Varie-t-elle au
                cours du temps ? Calculer l'altitude maximale de atteinte par
                l'obus.
                
                \item En quel point et au bout de combien de temps l'obus retombe-t-il ?
            \end{enumerate}

            \item La résistance de l'air sur l'obus, de forme sphérique de
            rayon $r_0 = \qty{5}{\centi\metre}$ et animé d'une vitesse $v$, se
            traduit par une force de module $k\pi {r_0}^2 v^2$. Au voisinage
            des conditions normales, $k = \qty{0.25}{USI}$. L'obus est en plomb
            de masse volumique $\rho = \qty{11.3}{\g\per\centi\metre\cubed}$.

            \begin{enumerate}
                \item Préciser l'unité de $k$.
                
                \item Comparer la force de frottement au poids. Que penser ?
                
                \item On prend en compte cette force de frottement fluide. On
                pose $u=v^2$. Montrer que, dans la phase ascendante, $u$
                vérifie l'équation :
                %
                \[
                    \dfrac{\dif u}{\dif t} = -2g_0 -2\dfrac{k\pi}{m}{r_0}^2u
                \]
                %
                Expliciter la fonction $z\p{u}$. En déduire l'altitude
                maximale atteinte par l'obus. On posera $d = \dfrac{m}{2k\pi
                r_0^2}$.
            \end{enumerate}
        \end{enumerate}
    \end{enumerate}
    %
    Dans la suite du problème, on ne prend pas en compte les frottements de l'air sur l'obus.
    %
    \begin{enumerate}[resume]
        \item On considère que le référentiel lié à la Terre $\p{A, x, y
        , z}$ est non galiléen.

        \begin{enumerate}
            \item Écrire l'équation du mouvement de l'obus. Pourquoi la force
            d'inertie d'entraînement n'intervient-elle pas explicitement dans
            l'équation du mouvement ?
            
            \item On évalue la force d'inertie de \textsc{Coriolis} en utilisant la loi de vitesse obtenue à la question \quref{I.1.1.a}.
            Justifier cette méthode de calcul.

            Soit \(\left(\vec i, \vec j, \vec k\right)\)la base orthonormée associée au repère \(\left(A,x,y,z\right)\). \(\bcM\) repérant la position de l'obus, on pose \(\vec{A\bcM}=x\vec i + y\vec j + z\vec k\). Montrer que \(\dot{\dot{y}} \approx -2\omega_0\left(v_0 - g-0t\right)\cos(\lambda)\)

            En déduire une expression approchée de l'ordonnée $y$ de l'obus. Évaluer y au moment où l'obus tombe sur le sol. La déviation se fait-elle vers l'Ouest ou vers l'Est ? Le résultat dépend-il de l'hémisphère dans lequel on effectue le tir ?

            \item L'expression de la force de \textsc{Coriolis} utilisée ne permet pas de mettre en évidence une déviation dans l'axe Nord-Sud. Or cette déviation existe. Expliquer cette apparente contradiction. Que penser de cette déviation comparée à celle calculée à la question précédente ?
            
        \end{enumerate}

        \item Analyse du mouvement de l'obus dans le référentiel géocentrique

        \begin{enumerate}
            \item 
            \begin{enumerate}
                \item Rappeler la définition du référentiel géocentrique. Pourquoi peut-il être supposé galiléen pour ce type d'expérience ? 

                \item Déterminer la vitesse initiale de l'obus dans ce référentiel.
            \end{enumerate}
            \item Si on considère le champ de gravitation uniforme sur la trajectoire de l'obus, montrer alors que celui-ci retombe en $A$. Comment 'observateur géocentrique peut-il justifier la déviation évaluée au I-2-2
        \end{enumerate}
    \end{enumerate}
    
\end{document}