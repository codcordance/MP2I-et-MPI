\documentclass[a4paper,french,bookmarks]{article}

\usepackage{./Structure/4PE18TEXTB}

\newboxans
\usepackage{booktabs}

\begin{document}

    \renewcommand{\thesection}{\Roman{section}}
    \setlist[enumerate]{font=\color{white5!60!black}\bfseries\sffamily}
    \renewcommand{\labelenumi}{\thesection.\arabic{enumi}.}
    \renewcommand*{\labelenumii}{\thesection.\arabic{enumi}.\arabic{enumii}.}
    
    \stylizeDocSpe{Physique}{Travaux pratiques n° 8}{Séquence frottement solide}{Le mercredi 16 novembre 2022}
    
    \section{Loi de Coulomb et coefficient de frottement statique}
    
    L'objectif est de vérifier la validité des lois de \textsc{Coulomb} du frottement solide et de savoir mesurer un coefficient de frottement statique.
    
    \noafter
    %
    \boxans{
        \begin{experience}{Mesure du coefficient de frottement statique}{}
            \begin{enumerate}
                \ithand Mesurer les coefficients de frottement statique métal/plastique, métal feutre et métal/mousse.
            \end{enumerate}
        \end{experience}
    }
    %
    \nobefore\yesafter
    %
    \begin{expcom}
        On a le dispositif suivant :
        %
        \begin{minipage}{0.45\linewidth}
            \begin{center}
                \begin{tikzpicture}
                    \draw (0.3, 0) -- (5, 2.7) -- (5, 0) -- (0.3, 0);
                
                    \draw[fill = main3!30] (2.5, 1.28) --++(-0.5, 0.8660) --++(0.8660, 0.5) --++(0.5, -0.8660) --++(-0.8660, -0.5);
                
                    \draw[main1, very thick, ->] (2.683, 1.963) --++(0, -1) node[label={[font=\footnotesize]right:$m\vec g$}] {};
                
                    \fill (2.683, 1.963) circle[radius=0.05];
                
                    \draw[main2, very thick, ->] (2.933, 1.53)  --++(0.8660, 0.5) node[label={[font=\footnotesize]below:$\vec T$}] {};
                
                    \draw[main2, very thick, ->] (2.933, 1.53)  --++(-0.5, 0.8660) node[label={[font=\footnotesize]above:$\vec N$}] {};
                
                    \fill (2.683, 1.963) circle[radius=0.05];
                    \fill (2.933, 1.53) circle[radius=0.05];
                \end{tikzpicture}
            \end{center}
        \end{minipage}
        %
        \hfill
        %
        \begin{minipage}{0.5\linewidth}
            Soit $G$ le centre d'inertie de la masse $m$ (ci-contre en violet).
        \end{minipage}
    \end{expcom}
    %
    \yesbefore
    
\end{document}