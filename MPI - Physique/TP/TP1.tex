\documentclass[a4paper,french,bookmarks]{article}

\usepackage{./Structure/4PE18TEXTB}

\newboxans
\usepackage{booktabs}

\begin{document}

    \renewcommand{\thesection}{\Roman{section}}
    \setlist[enumerate]{font=\color{white5!60!black}\bfseries\sffamily}
    \renewcommand{\labelenumi}{\thesection.\arabic{enumi}.}
    \renewcommand*{\labelenumii}{\thesection.\arabic{enumi}.\arabic{enumii}.}
    
    \stylizeDocSpe{Physique}{Travaux pratiques n° 1}{Oscilloscope et mesure d'impédance}{Le mercredi 14 septembre 2022}
    
    \section{Observation du signal d'un Générateur Basse Fréquence}
    
    \noafter
    %
    \boxans{
        
        \begin{experience}{Entraînement}{}
            \begin{enumerate}
                \ithand Régler le menu \texttt{déclenchement} (\texttt{trigger}), notamment les paramètres \texttt{voix} (\texttt{source}) et \texttt{niveau} (\texttt{level}).
                
                \ithand Choisir entre les mode \textsf{\hg{CHOP}} et \textsf{\hg{ALTERN}} selon la fréquence (pour les oscillateurs analogiques).
            \end{enumerate}
        \end{experience}
    }
    %
    \nobefore\yesafter
    %
    \begin{expcom}
        On prend un signal sinusoïdal de fréquence $f = \SI{1.00}{\kilo\hertz}$ et d'amplitude $A = \SI{20}{\volt}$.
    \end{expcom}
    %
    \yesbefore
    
    \begin{enumerate}
        \item Ne pas oublier que pour augmenter la précision des mesures les courbes doivent occuper un espace maximum sur l'écran de l'oscilloscope. Expliquer pourquoi ?
        
        \boxans{
            On veut être sûr que la courbe qu'on observe est bien celle qu'on cherche à observer, ainsi maximiser la taille permet de s'assurer que l'on n'est pas simplement face à du bruit. Par ailleurs, cela permet de régler le menu \texttt{déclenchement} avec plus de précision.
        }
    \end{enumerate}
    
    Dorénavant choisir les calibres temporels et en tension qui permettent les meilleures mesures possibles.
    
    $\bdC \ \bcC \ \bsC \ \bfC \ \bbC \ C$
    
    \noafter
    %
    \boxans{
        \begin{experience}{}{}
            
        \end{experience}
    }
    
\end{document}