\documentclass[a4paper,french,bookmarks]{book}

\usepackage{booktabs}
\usepackage{minitoc}
\usepackage{Structure/4PE18TEXTB}
\usepackage{pdfpages}

\newboxans
\renewcommand{\questionsdecours}{\section*{\centering\EBGaramond\Large Questions~ de~ cours}}
\renewcommand{\thechapter}{\Roman{chapter}}
\renewcommand{\thesubsection}{Exercice \arabic{subsection}}
\setcounter{secnumdepth}{5}
\setcounter{tocdepth}{0}
\mtcsettitle{minitoc}{}
%\renewcommand{\thesection}{\hspace{-11pt}}
%\renewcommand{\thesubsection}{}

\newcommand{\chaptertoc}[0]{
    \setcounter{minitocdepth}{4}
    \begin{tcolorbox}[
        enhanced,
        frame hidden,
        sharp corners,
        detach title,
        spread outwards     = 5pt,
        halign              = center,
        valign              = center,
        borderline west     = {3pt}{0pt}{main20!50!main2!95!gray!90},
        coltitle            = main20!50!main2!95!gray!90, 
        interior style      = {
            left color      = main1white2!65!gray!11,
            middle color    = main1white2!50!gray!10,
            right color     = main1white2!35!gray!9
        },
        arc                 = 0 cm,
        title               = SOMMAIRE,
        fonttitle           = \bfseries\sffamily,
        overlay             = {
            \node[rotate=90, minimum width=1cm, anchor=south,yshift=-0.8cm] at (frame.west) {\tcbtitle};
        }
    ]
        \begin{minipage}{0.83\linewidth}
            \sffamily
            \minitoc
        \end{minipage}
    \end{tcolorbox}
}

\begin{document}
    
    %==============================
    % METADONNEES
    %==============================
    
    \title{TD de Physique de MPI/MPI* (2022-2023)}
    \author{SIAHAAN--GENSOLLEN Rémy}
    \date{\today}
    \hypersetup{
        pdftitle={TD de Physique de MPI/MPI* (2022-2023)},
        pdfauthor={SIAHAAN--GENSOLLEN Rémy},
        pdflang={fr-FR},
        pdfsubject={MPI/MPI*, TD de Physique},
        pdfkeywords={MPI/MPI*, TD de Physique, 2022-2023}
        pdfstartview=
    }
    
    %==============================
    % MISE EN PAGE
    %==============================
    
    \titleformat{\chapter}[display]{\normalfont\huge\bfseries}{}{0pt}{
        \begin{tcolorbox}[
            enhanced,
            frame hidden,
            sharp corners,
            spread sidewards    = 5pt,
            halign              = center,
            valign              = center,
            interior style      = {color=main10!20},
            arc                 = 0 cm,
            fontupper           = \color{black}\sffamily\bfseries\huge,
            fonttitle           = \normalfont\color{white}\sffamily\small,
            top                 = 1cm, 
            bottom              = 0.7cm,
            title               = Chapitre \thechapter,
            attach boxed title to bottom center = {
                yshift=\tcboxedtitleheight/2,
            },
            boxed title style = {
                frame code={
                \path[left color=main21!95!gray!90,right color=main21!95!gray!90] 
                    ([xshift=-10mm]frame.north west) -- 
                    ([xshift=10mm]frame.north east) -- 
                    ([xshift=10mm]frame.south east) -- 
                    ([xshift=-10mm]frame.south west) -- 
                    cycle;
                },
                interior engine=empty
            }
        ]
            #1
        \end{tcolorbox}%
    }
    \titlespacing*{\chapter}{0pt}{-120pt}{-15pt}
    \titleformat{name=\chapter,numberless}[display]{\normalfont\huge\bfseries}{}{0pt}{
        \begin{tcolorbox}[
            enhanced,
            frame hidden,
            sharp corners,
            spread sidewards    = 5pt,
            halign              = center,
            valign              = center,
            interior style      = {color=main10!20},
            arc                 = 0 cm,
            outer arc           = 0pt,
            leftrule            = 0pt,
            rightrule           = 0pt,
            fontupper           = \color{black}\sffamily\bfseries\huge,
            enlarge left by     = -1in-\hoffset-\oddsidemargin, 
            enlarge right by    = -\paperwidth+1in+\hoffset+\oddsidemargin+\textwidth,
            width               = \paperwidth, 
            left                = 1in+\hoffset+\oddsidemargin, 
            right               = \paperwidth-1in-\hoffset-\oddsidemargin-\textwidth,
            top                 = 1cm, 
            bottom              = 1cm
        ]
            #1
        \end{tcolorbox}%
    }
    \titlespacing*{name=\chapter,numberless}{0pt}{-115pt}{0pt}
    
    %==============================
    % PREMIERE DE COUVERTURE
    %==============================

    %\includepdf[pages={1},scale=1.15,offset=0mm -18mm]{LDCCover.pdf}
    
    %==============================
    % PAGE VIDE
    %==============================
    
    %\pagestyle{empty}
    
    %==============================
    % PAGE DE COUVERTURE INTERNE
    %==============================
    
    \begin{titlepage}
	    \begin{center}
	        {\scshape SIAHAAN--GENSOLLEN Rémy\par}
	        \vspace{2cm}
	        {\huge\sffamily TD de\par}
	        \vspace{0.5cm}
	        {\Huge\bfseries\sffamily PHYSIQUE\par}
	        \vspace{1cm}
	        {\Large\textit{donné pendant mon année de \textsf{MPI/MPI*} à
	        Janson-de-Sailly}\\[5pt]\texttt{(2022-2023)}\par}
	        \vfill
	        {\large\EBGaramond Dernière compilation le \today\par}
        \end{center}
    \end{titlepage}
    
    %==============================
    % PAGE VIDE
    %==============================
    
    \pagestyle{empty}\text{}\newpage
    
    %==============================
    % STYLE DES EN-TÊTES ET PIEDS DE PAGES
    %==============================
    
    \renewcommand\chaptermark[1]{\markboth{#1}{}}
    
    \fancypagestyle{intro}{
        \fancyhf{}
        \renewcommand{\headrulewidth}{0pt}
        \renewcommand{\footrulewidth}{0pt}\fancyfoot[RO,LE]{\GillSansMTMedium\color{white5}\thepage\;/\;\pageref{LastPage}}
        \fancyhead[LE]{\GillSansMTMedium\color{white5}\bfseries TD DE PHYSIQUE}
        \fancyhead[RE]{\GillSansMTMedium\color{white5}Avant-propos}
        \fancyhead[LO]{\GillSansMTMedium\color{white5}\rightmark}
        \fancyhead[RO]{\GillSansMTMedium\color{white5}\textbf{MPI/MPI*} 2022-2023 \quad Janson-de-Sailly}
    }
    
    \fancypagestyle{toc}{
        \fancyhf{}
        \renewcommand{\headrulewidth}{0pt}
        \renewcommand{\footrulewidth}{0pt}\fancyfoot[RO,LE]{\GillSansMTMedium\color{white5}\thepage\;/\;\pageref{LastPage}}
        \fancyhead[LE]{\GillSansMTMedium\color{white5}\bfseries TD DE PHYSIQUE}
        \fancyhead[RE]{\GillSansMTMedium\color{white5}Table des matières}
        \fancyhead[LO]{\GillSansMTMedium\color{white5}\rightmark}
        \fancyhead[RO]{\GillSansMTMedium\color{white5}\textbf{MPI/MPI*} 2022-2023 \quad Janson-de-Sailly}
    }
    
    \fancypagestyle{plain}{
        \fancyhf{}
        \renewcommand{\headrulewidth}{0pt}
        \renewcommand{\footrulewidth}{0pt}\fancyfoot[RO,LE]{\GillSansMTMedium\color{white5}\thepage\;/\;\pageref{LastPage}}
        \fancyhead[LE]{\GillSansMTMedium\color{white5}\bfseries TD DE PHYSIQUE}
        \fancyhead[RE]{\GillSansMTMedium\color{white5}Chapitre \thechapter : \nouppercase{\leftmark}}
        \fancyhead[LO]{\GillSansMTMedium\color{white5}\nouppercase{\rightmark}}
        \fancyhead[RO]{\GillSansMTMedium\color{white5}\textbf{MPI/MPI*} 2022-2023 \quad Janson-de-Sailly}
    }
    
    %==============================
    % PREFACE 
    %==============================
    
    \chapter*{Avant-propos}
    \thispagestyle{intro}
    \addcontentsline{toc}{chapter}{Avant-propos}
    
    \text{\Large\EBGaramond\itshape À tout lecteur potentiel, quelques mots...}\newline\newline\newline
    
    \begin{center}
        \begin{minipage}{0.85\linewidth}
            \large \qquad Ce livre contient la résolution des exercices de TD donnés pendant les cours de physique de mon année de MPI/MPI*. Il vient en complément du livre de cours associé.\newline\newline\newline\text{}
        \end{minipage}
    \end{center}
    
    \hfill{\large\textsc{Siahaan--Gensollen Rémy}}
    
    \pagestyle{intro}
    
    %==============================
    % TABLE DES MATIERES
    %==============================
    
    \newpage
    \dominitoc\nomtcrule 
    {\sffamily\tableofcontents}\mtcaddchapter\pagestyle{toc}
    
    \cleardoublepage
    
    %==============================
    % COURS
    %==============================
    
    \pagestyle{plain}
    
    \chapter{Quelques notions indispensables}
    
    \subsection{}
    
    Calculer pour un gaz parfait la quantité :
    %
    \[ \dep{\dfrac{\partial V}{\partial T}}_P \dep{\dfrac{\partial T}{\partial P}}_V \dep{\dfrac{\partial P}{\partial V}}_T\]
    
    \noafter
    %
    \boxans{
        L'équation des gaz parfaits donne $PV = nRT$. Pour $n$ donné et puisque $R$ est une constante, on a donc :
        %
        \[ V\p{P, T} = \dfrac{nRT}{P} \qquad T\p{P, V} = \dfrac{PV}{nR} \qquad P\p{V, T} = \dfrac{nRT}{V}\]
        %
        Ainsi on a :
        %
        \[ \dep{\dfrac{\partial V}{\partial T}}_P = \dfrac{nR}{P}\qquad \dep{\dfrac{\partial T}{\partial P}}_V = \dfrac{V}{nR} \qquad \dep{\dfrac{\partial P}{\partial V}}_T = -\dfrac{nRT}{V^2}\]
        %
        On multiplie donc pour obtenir :
    }
    %
    \nobefore\yesafter
    %
    \boxansconc{
        \[ \dep{\dfrac{\partial V}{\partial T}}_P \dep{\dfrac{\partial T}{\partial P}}_V \dep{\dfrac{\partial P}{\partial V}}_T = \dfrac{nR}{P}\dfrac{V}{nR}\p{-\dfrac{nRT}{V^2}} = -\dfrac{nRT}{PV} = -1 \]
    }
    %
    \yesbefore
    
    \subsection{}
    
    La force $\vec{F}$ s'exprime dans le repère $\p{\vec{e_x}, \vec{e_y}, \vec{e_z}}$ de la façon suivante : \quad $\vec{F} = \left(\begin{array}{c}
        x  \\
        z^2 y \\
        y^2 z + C
    \end{array}\right)$.
    %
    \begin{enumerate}
        \item Est-ce une force conservative ?
        
        \noafter
        %
        \boxans{
            Soit $\delta W_{\vec{F}} = \vec{F} \cdot \dif \vec{OM}$. On cherche à savoir s'il existe une fonction $E_\text p$ telle que $\delta W_{\vec{F}} = -\dif E_\text p$.
            
            Posons les composantes $F_x = \vec{F} \cdot \vec{e_x}$, $F_y = \vec{F} \cdot \vec{e_y}$ et $F_z = \vec{F} \cdot \vec{e_z}$. L'énoncé amène :
            %
            \[ F_x\p{x, y, z} = x \qquad F_y\p{x, y, z} = z^2y \qquad F_z\p{x, y, z} = y^2z + C \]
            %
            On a donc $\delta W_{\vec{F}} = F_x\p{x, y, z}\dif x + F_y\p{x, y, z}\dif y + F_z\p{x, y, z}\dif z$.
            
            \begin{enumerate}
                \begin{minipage}{0.48\linewidth}
                    \itt $\dep{\dfrac{\partial F_x}{\partial y}}_{x, z} = 0$
                    
                    \itt $\dep{\dfrac{\partial F_x}{\partial z}}_{x, y} = 0$
                    
                    \itt $\dep{\dfrac{\partial F_y}{\partial z}}_{x, y} = 2yz$
                \end{minipage}
                %
                \hfill
                %
                \begin{minipage}{0.48\linewidth}
                    \itt $\dep{\dfrac{\partial F_y}{\partial x}}_{y, z} = 0$
                    
                    \itt $\dep{\dfrac{\partial F_z}{\partial x}}_{y, z} = 0$
                    
                    \itt $\dep{\dfrac{\partial F_z}{\partial y}}_{x, z} = 2yz$
                \end{minipage}
            \end{enumerate}
            %
            On remarque que $\dep{\dfrac{\partial F_x}{\partial y}}_{x, z} = \dep{\dfrac{\partial F_y}{\partial x}}_{y, z}$, que $\dep{\dfrac{\partial F_x}{\partial z}}_{x, y} = \dep{\dfrac{\partial F_z}{\partial x}}_{y, z}$, et que $\dep{\dfrac{\partial F_y}{\partial z}}_{x, y} = \dep{\dfrac{\partial F_z}{\partial y}}_{x, z}$. Il existe donc bien une telle fonction $E_\text p$, qui est l'énergie potentielle associée à $\vec{F}$.
        }
        %
        \nobefore\yesafter
        %
        \boxansconc{
            Puisqu'il existe une énergie potentielle associée à la force $\vec{F}$, celle-ci est bien conservative.
        }
        %
        \yesbefore
        
        \item Si oui, quelle est l'énergie potentielle associée ?
        
        \noafter
        %
        \boxans{
            On a $\delta W_{\vec{F}} = -\dif E_\text p$, il s'agit donc d'intégrer $\delta W_{\vec{F}}$. D'après la question précédente, on a :
            %
            \[ \dif E_\text p = -\delta W_{\vec{F}} = -F_x \dif x - F_y \dif y - F_z \dif z \qquad\text{où}\qquad F_x = x, \qquad F_y = z^2y, \qquad F_Z = y^2z + C \]
            %
            \underline{\EBGaramond\itshape Première méthode :} On intègre de $O = \p{0, 0, 0}$ à $M = \p{X, Y, Z}$, on a en décomposant sur chaque axe (la force étant conservative on peut parfaitement choisir le chemin qu'on veut) :
            %
            \[ E_\text p\p{M} = E_\text p\p{M} - E_\text p\p{O} = \int_{0, 0, 0}^{X, Y, Z} \dif E_p = -\int_{x = 0}^{X} F_x\p{x, 0, 0}\dif x - \int_{y = 0}^{Y} F_y\p{X, y, 0}\dif y - \int_{z = 0}^{Z}  F_Z\p{X, Y, z}\dif z\]
            %
            Or on a :
            %
            \[ \left\lbrace\begin{array}{lll}
                \displaystyle\int_{x = 0}^{X} F_x\p{x, 0, 0}\dif x &= \displaystyle\int_0^X x\dif x &= \dfrac{X^2}{2}  \\[10pt]
                \displaystyle\int_{y = 0}^Y F_y\p{X, y, 0}\dif y &= \displaystyle\int_{0}^Y 0\times y^2\dif y &= 0 \\[10pt]
                \displaystyle\int_{z = 0}^Z F_z\p{X, Y, z}\dif z &= \displaystyle\int_0^Z \p{Y^2z + C}\dif z &= \dfrac{Y^2Z^2}{2} + CZ
            \end{array}\right.\]
            %
            En développant puis en simplifiant, on obtient :
        }
        %
        \nobefore
        %
        \boxansconc{
            \[ E_\text p\p{X, Y, Z} = - CZ - \dfrac{X^2 + Y^2Z^2}{2} + E_\text p\p{0, 0, 0}\]
        }
        %
        \boxans{
            \underline{\EBGaramond\itshape Deuxième méthode :} En dérivant  $E_\text p$ par rapport à $x$, on obtient : \quad $\dfrac{\partial E_\text p}{\partial x} = -F_x = -x$. En intégrant, on a donc :
            %
            \[ E_\text p \p{x, y, z} = -\dfrac{x^2}{2} + D\p{x, y} \qquad\text{avec} \ D \ \text{une fonction à déterminer}\]
            %
            Pour déterminer $D$, on dérive $E_\text p$ par raport à $y$ : \quad $\dfrac{\partial E_\text p}{\partial y} = -F_y = -z^2y$. Or :
            %
            \[ \dfrac{\partial E_p}{\partial y} = 0 + \dfrac{\partial D}{\partial y}\qquad\text{donc}\qquad \dfrac{\partial D}{\partial y} = -z^2y \]
            %
            En intégrant, on a donc $D\p{x, y} = -\dfrac{z^2y^2}{2} + D'\p{z}$, avec $D'$ une fonction à déterminer. On a donc :
            %
            \[ E_\text p\p{x, y, z} = -\dfrac{x^2}{2} - \dfrac{z^2y^2}{2} + D'\p{z}\]
            %
            Pour déterminer $D'$, on réitère le procédé. On a $\dfrac{\partial E_\text p}{\partial z} = -F_z = -y^2z - C$ et :
            %
            \[ \dfrac{\partial E_\text p}{\partial z} = -y^2z + \dfrac{\partial D'}{\partial y} \qquad\text{donc}\qquad \dfrac{\partial D'}{\partial y} = -C \qquad\text{donc}\qquad D'\p{z} = -Cz + cte\]
            %
            On a donc finalement :
            %
        }
        \yesafter
        %
        \boxansconc{
            \[ E_\text p\p{x, y, z} = - cZ - \dfrac{x^2 + y^2z^2}{2} + cte\]
        }
        %
        \yesbefore
    \end{enumerate}

    \subsection{}
    
    La fréquence de vibration d'une goutte de liquide dépend de plusieurs paramètres : le rayon de la goutte $R$, la masse volumique du liquide $\rho$, la tension superficielle due à l'interface liquide-extérieur $A$ homogène à une énergie surfacique. Donner l'expression de la fréquence $f$ sous la forme
    %
    \[ f = kR^\alpha \rho^\beta A^\gamma \qquad\text{(où} \ k \ \text{est un nombre sans dimension que l'on ne cherche pas à déterminer)}\]
    %
    en déterminant les valeurs de $\alpha$, $\beta$ et $\gamma$.
    
    \noafter
    %
    \boxans{
        On procède par analyse dimensionnelle. On a $\intc R = \textsf{L}$, $\intc \rho = \textsf{M} \cdot \textsf{L}^{-3}$, $\intc A = \textsf{M} \cdot \textsf{T}^{-2}$ et $\intc f = \textsf{T}^{-1}$.
        
        Avec $f = kR^\alpha \rho^\beta A^\gamma$, on a  donc :
        %
        \[ \textsf{T}^{-1} = \textsf{L}^\alpha \cdot \p{\textsf{M} \cdot \textsf{L}^{-3}}^{\beta} \cdot \p{\textsf{M} \cdot \textsf{T}^{-2}}^{\gamma} = \textsf{T}^{-2c} \cdot \textsf{L}^{\alpha - 3b} \cdot \textsf{M}^{b + c} \]
        %
        On a donc le système $\left\lbrace\begin{array}{ccc}
            -1 &=& -2c \\
            0 &=& a - 3b \\
            0 &=& b + c
        \end{array}\right.$ qui se resout en $\left\lbrace\begin{array}{ccc}
            c &=& \sfrac{1}{2} \\
            b &=& -\sfrac{1}{2} \\
            a &=& -\sfrac{3}{2}
        \end{array}\right.$.
        
        On a donc :
        %
        \[ f = kR^{-\sfrac{3}{2}}\rho^{-\sfrac{1}{2}}A^{\sfrac{1}{2}} \]
        %
        Finalement, on obtient :
        %
    }
    %
    \nobefore\yesafter
    %
    \boxansconc{
        \[ f = k\sqrt{\dfrac{A}{\rho R^3}}\]
    }
    %
    \yesbefore
    
    \subsection{}
    
    Pour calculer la viscosité $\eta$ d'un fluide, on utilise l'expression de la force de frottement exercée sur une sphère de rayon $r$, de masse $m$ et de vitesse $\vec{v}$ établie par \textsc{Stokes} : \quad $\vec{f} = -6\pi\eta r \vec{v}$.
    
    On fait l'expérience suivante : on mesure la vitesse limite acquise par une bille en chute libre dans le fluide en mesurant le temps de parcours sur une distance $h$ connue. 
    
    \begin{enumerate}
        \item En négligeant la poussée d'\textsc{Archimède} montrer que $\eta = \dfrac{mg\tau}{6\pi rh}$.
        
        \noafter
        %
        \boxans{
            \begin{minipage}{0.40\linewidth}
                On se place dans la base orthnormée directe $\p{\vec{u_x}, \vec{u_y}, \vec{u_z}}$ décrite dans le schéma ci-contre. Le bilan des forces donne :
                %
                \begin{enumerate}
                    \itt Poids $\vec{P} = m\vec{g} = mg\vec{u_y}$
                    
                    \itt Frottements $\vec{f} = -6\pi\eta r \vec{v}$
                    
                    \itt Poussée d'\textsc{Archimède} $\vec{\pi}$ négligeable.
                \end{enumerate}
            \end{minipage}
            %
            \hfill
            %
            \begin{minipage}{0.6\linewidth}
                \begin{center} 
                     \includegraphics{Figures/sphere-in-liquid.pdf}
                \end{center}
            \end{minipage}
            
            Le mouvement est rectiligne, selon l'axe porté par $\vec{u_y}$. On pose $v_y = \vec{v} \cdot \vec{u_y}$ et on applique le \textit{principe fondamental de la dynamique} sur cet axe :
            %
            \[ m\dfrac{\dif v_y}{\dif t} = mg - 6\pi\eta r v_y \qquad\text{soit}\qquad \dfrac{\dif v_y}{\dif t} + \dfrac{6\pi\eta r}{m}v_y = g\]
            %
            Cette équation différentielle du premier ordre a pour solution :
            %
            \[ v_y\p{t} = A\exp{-\dfrac{6\pi\eta r}{m}t} + \dfrac{m}{6\pi\eta r}g \qquad\text{avec} \ A \ \text{une constante à déterminer}\]
            %
            La condition initiale $v_y\p{0} = 0$ livre $A = -\dfrac{mg}{6\pi\eta r}$ d'où $v_y\p{t} = \dfrac{mg}{6\pi\eta r}\p{1 - \exp{-\dfrac{6\pi\eta r}{m}t}}$. On intègre :
            %
            \[ y\p{t} = \dfrac{mg}{6\pi\eta r}\intc{t + \dfrac{m}{6\pi\eta r}\exp{-\dfrac{6\pi\eta r}{m}t}} + B \qquad\text{avec} \ B \ \text{une constante à déterminer}\]
            %
            Pour de grandes valeurs de $\tau$, on a alors $y\p{\tau} \approx \dfrac{mg}{6\pi\eta r}\tau$, donc pour $y\p{\tau} = h$ on obtient $h = \dfrac{mg\tau}{6\pi\eta r}$, d'où :
            %
        }
        %
        \nobefore\yesafter
        %
        \boxansconc{
            \[ \eta = \dfrac{mg\tau}{6\pi rh}\]
        }
        %
        \yesbefore
        
        \item On mesure $m = \SI{1.00}{\g}$ avec $u\p{m} = \SI{0.01}{\g}$, $m\tau= \SI{1.00}{\s}$ avec $u\p{\tau} = \SI{0.05}{\s}$, $r = \SI{1.0}{\mm}$ avec $u\p{r} = \SI{0.1}{\mm}$. On prendra $g = \SI{9.81}{\m \cdot \s^{-2}}$. Donner et encadrer la valeur de $\eta$.
    \end{enumerate}
    
    \subsection{}
    
    La mise en équation de la tension de sortie de l'oscillateur à pont de \textsc{Wien} donne le résultat suivant :
    %
    \[ \dfrac{\dif^2 V_\text s}{\dif t^2} + 2\Omega_0 \alpha + \Omega_0^2V_\text s = 0 \qquad\text{où} \ \Omega \ \text{et un réel positif et} \ \alpha \ \text{est un réel}\]
    %
    Discuter la stabilité du système et les différentes solutions possibles en fonction de $\alpha$.
    
    \boxans{
        Il s'agit d'un système [décrit pas une équation différentielle] du deuxième ordre. On considère donc 
    }
    
    \subsection{}
    
    Un liquide placé dans un récipient cylindrique de section $S$, occupe au repos un volume caractérisé par une hauteur $h_0$. On fait tourner le récipient autour de son axe à vitesse angulaire constante. Une fois le régime permanent établi, la surface libre du liquide prend une forme parabolique d'équation $z = ar^2 + h$.
    
    \begin{enumerate}
        \item Exprimer $h$ en fonction de $h_0$ et des autres données de l'énoncé.
        
        \boxansconc{
            On suppose que l'eau est une phase incompressible et indilatable (PCII). 
        }
        
        \item Une jarre sphérique de rayon $R$ contient une hauteur $h_0$ d'eau. Le volume qui s'évapore par unité de temps est proportionnel à la surface eau-air. Déterminer le temps pour que la jarre soit vide en fonction du coefficient de proportionnalité.
    \end{enumerate}

    \chapter{Électronique, analyse harmonique d'un signal périodique, numérisation}
    
    \subsection{}
    
    \begin{minipage}{0.48\linewidth}
         Soit le circuit suivant :\medskip
    
    A l'instant initial, le condensateur de capacité $C$ est chargé avec la charge $Q_0$, celui de capacité $C'$ est déchargé.
    \end{minipage}
    %
    \hfill
    \begin{minipage}{0.48\linewidth}
            \begin{center}
        \begin{circuitikz}
            \draw (0, 0) --++(0, 2) to[C, l=$C$, v=$u_C\p{t}$] ++(3, 0) to[C, l=$C'$, v=$u_{C'}\p{t}$] ++(3, 0) to[short, i=$i\p{t}$] ++(0, -2) to[R, l=$R$, v=$u_R\p{t}$] ++(-6, 0);
        \end{circuitikz}
    \end{center}
    \end{minipage}

    \begin{enumerate}
        \item Établir l'expression de $i\p{t}$, le courant qui parcourt le circuit au cours du temps.
        
        \boxans{
        
        }
        
        \item Quelle est l'énergie dissipée par le circuit ? Comparer avec les énergies initiales et finales ?
        
        \boxans{
        
        }
    \end{enumerate}
    
    \subsection{}
    
    \subsection{}
    
    La tension $u\p{t}$ a la forme ci-dessous :
    
    DESSIN 
    
    On a $T < RC$ et $\alpha < 1$. Déterminer les allures de $u_1$ et $u_2$.
    
    \boxans{
    
    }
    \subsection{}

    \subsection{}

    \subsection{}
    % je sais pas dessiner le circuit (xD)
    \begin{enumerate}
        \item \(e(t) = e_0\cos(\omega t)\) On veut que la composante sinusoïdale en \(A\) soit atténué de 40 dB. En déduire une condition sur R, C et \(\omega\), on supposera cette condition vérifiée par la suite.

        \boxans{
            Sur une sinusoïde : \[V_B = \frac{Z_C}{R + Z_C}V_A \text{donc} H = \frac{1}{1 + jRC\omega}\]
            Donc sur l'entrée souhaitée \(V_A = ke_0^2\cos^2(\omega t) = \dfrac{ke_0^2}{2} + \dfrac{ke_0^2\cos(2\omega t}{2}\) ainsi la sortie de la composante sinusoïdale est \[V_B = \dfrac{1}{\sqrt{1+ 4(Rc\omega)^2}}\dfrac{ke_0^2}{2}\cos(2 \omega t - \arctan(2RC\omega))\]
            On a enfin par définition \(G_{dB} = 20\log\left|\dfrac{V_B}{V_A}\right|\) donc 
            \(G_{dB} = -40 \Leftrightarrow \omega \geq 100 \omega_0\)
        }

        \item Déterminer \(s(t)\)

        \boxans{
            On a trivialement \(s^2(t) = k^{-1}V_B\) or \(\omega_0 << \omega\) donc 
            \(V_B \equiv V_B(0) = k\dfrac{e_0^2}{2}\) donc \(s(t) = \pm \dfrac{e_0}{2}\)
        }

        \item Le dipôle \(D\) est une diode telle que \(s(t) = \lnot i(t)\)

        \boxans{
            La diode force la tension positive, donc \(s(t) = \dfrac{\left|e_0\right|}{\sqrt{2}}\)
        }

        \item Montrer que ce montage fonctionne comme un voltmètre. Est-ce une mesure AC ou DC ? Le fonctionnement serait-il modifié si le signal d'entrée n'était pas sinusoïdal ?

        \boxans{
            On observe que \(s(t)\) est bel et bien la valeur RMS de \(e(t)\) donc ce montage se comporte bien comme un voltmètre, on est en DC car la valeur moyenne a de l'influence sur le résultat final.
        }
    \end{enumerate} 


    \subsection{}
    % je sais pas dessiner le circuit
    \begin{enumerate}
        \item Déterminer et caractériser la fonction de transfert \(H(j\omega\) du circuit.

        \boxans{
            En utilisant le théorème de \textsc{Millman} on trouve \(H = \dfrac{1 - jRC\omega}{1 + jRC\omega}\) ainsi \(G(\omega) = 1\) et \(\varphi = 2 tan^{-1}(RC\omega)\) le filtre est donc déphaseur.
        }


        \item La tension d'entrée \(v_e(t)\) est la fonction créneau définie par \(f(t) = \dfrac{4E}{\pi} \sum_{i=1}^{+\infty} \dfrac{1}{2i+1} \sin \left( 2 \pi (2i+1) \dfrac{t}{T} \right)\) avec \(T >> \tau = RC\) quelle est la tension de sortie ?

        \boxans{
            On a par définition la sortie \(s(t) = e_0\left|H_0\right| + \sum_{n=1}^{\infty} e_n \left|H(jn\omega)\right|\cos\left(n\omega t + \varphi_n + \arg(H(jn\omega))\right)\) \\
            Dans notre cas on a donc \[s(t) = \dfrac{4E}{\pi}\sum_{n=1}^{\infty} \dfrac{1}{2n+1} \sin\left((2n+1)\omega(t - 2\tau)\right)\]

            La tension de sortie est donc simplement celle d'entrée déphasée d'environ -180° comme \(T>>\tau\) on est à l'asymptote.
        }
    \end{enumerate}
    
    
\end{document}
