\documentclass[a4paper,french,bookmarks]{article}

\usepackage[
    top         = 1in,
    bottom      = 1in,
    inner       = 1.5in,
    outer       = 1in,
    headheight  = 16pt,
    headsep     = 0.4in,
    footskip    = 0.4in,
    includeheadfoot,
    heightrounded,
    twoside,
    %showframe,
    ]{geometry}
\usepackage{./Structure/4PE18TEXTBnogeom}

\newboxans
\usepackage{booktabs}

\begin{document}
    \renewcommand{\thesection}{\Roman{section}}
    \setlist[enumerate]{font=\color{white5!60!black}\bfseries\sffamily}
    \renewcommand{\labelenumi}{\arabic{enumi}.}
    \renewcommand*{\labelenumii}{\thesection.\arabic{enumi}.\arabic{enumii}.}
    
    \stylizeDocSpe{Physique}{Oral au CCMP}{}{Thermodynamique}

    \bigskip

    \section{Exercice}

    On considère un fil de cuivre $\ \p{\rho, c, \lambda, \gamma}$ cylindrique de rayon $a$, traversé par une intensité $I$ constante. Les parois en $x = 0$ et $x = \ell$ sont calorifugés, et on considère les transferts conducto-convectifs avec l'air.

    \begin{center}
        \begin{tikzpicture}
            \draw[->] (-1, 0) -- (9, 0) node[right] {$x$};

            \pattern[pattern=north east lines] (-1,1.5) rectangle (0,-1.5);

            \fill[gray!50,opacity=0.2] (0, 0.75) rectangle (7, -0.75);

            \draw[] (0, 1.5) -- (0, -1.55) node[below] {$x = 0$};

            \pattern[pattern=north east lines] (7,1.5) rectangle (8,-1.5);

            \draw[] (7, 1.5) -- (7, -1.55) node[below] {$x = \ell$};

            \draw[very thick] (0, 0.75) -- (7, 0.75);

            \draw[very thick] (0, -0.75) -- (7, -0.75);

            \draw[main3, ->, thick] (2, 0.5) --++ (0, 0.5) node[above] {$h_\text{cc}$};

            \draw[main3, ->, thick] (2, -0.5) --++ (0, -0.5) node[below] {$h_\text{cc}$};

            \node[main3] at (3.5, 1.25) {$\text{air } \p{T_0}$};

            \draw[->] (5.5, 0) -- (5.5, 0.75) node[midway, right] {$a$};

            \node[main3] at (3.5, -1.25) {$\text{air} \p{T_0}$};

            \node[] at (3.5, 0.2) {$T\p{x, t}$};
        \end{tikzpicture}
    \end{center}

    \begin{enumerate}
        \item Déterminer l'équation aux dérivées partielles vérifiée par $T\p{x, t}$.\medskip

        \item On suppose maintenant que la conductivité électrique $\gamma$ dépend de $T$ et vérifie la relation
        %
        \[ \dfrac{1}{\gamma}\p{T} = \dfrac{1}{\gamma_0}\p{1 + \alpha\p{T - T_0}}\]
        %
        En déduire une équation différentielle sur $T$ avec $\gamma_0$.\medskip

        En quoi et comment la conductivités thermique et électrique dépendent-t-elles de la température ? Explication \guill{macroscopique}.\medskip

        \item On pose 
        %
        \[ \theta\p{t} = \int_0^\ell T\p{x, t}\dif x\]
        %
        Quelle est la signification de $\theta\p{t}$ ? Donner l'équation différentielle vérifie par $\theta$.\medskip

        \item \emph{(donné à l'oral) En résolvant l'équation, on fait apparaître un terme devant $\p{\theta - T_0}$, dont le signe dépend de $I$.}
        
        Déterminer une expression d'une valeur $I_\text{c}$ critique, expliquer ce qui se passe dans le cas $I < I_\text c$ et dans le cas $I > I_\text c$.\medskip

        

        \item \emph{(donné à l'oral)} Application numérique pour $I_\text c$.

        Données : $h_\text{cc} = \qty{1}{\watt\per\meter}$, $\alpha = \qty{5e-3}{\per\kelvin}$, $\gamma_0 = \qty{e7}{\siemens\per\meter}$
    \end{enumerate}

    \section{Question de cours}

    \begin{center}
        \emph{Propagation d'une onde électromagnétique dans un plasma}
    \end{center}

\end{document}