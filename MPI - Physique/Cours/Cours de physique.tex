\documentclass[a4paper,french,bookmarks]{book}

\usepackage{booktabs}
\usepackage{minitoc}
\usepackage{./Structure/4PE18TEXTB}
\usepackage{proof}
\usepackage{pdfpages}
\usepackage[version=4]{mhchem}
\usepackage{xcolor}
\usepackage{tikz-3dplot}


\makeatletter
\renewcommand*\l@section{\@dottedtocline{1}{1.8em}{3.5em}}
\renewcommand*\l@subsection{\@dottedtocline{2}{5.3em}{3.5em}}
\makeatother

\newboxans
\renewcommand{\thechapter}{\Roman{chapter}}
\renewcommand{\thesubsection}{\thesection.\Alph{subsection}}
\mtcsettitle{minitoc}{}

\DeclareDocumentCommand\NO{g}{\funlv{N.O.}{#1}}



\newcommand{\chaptertoc}[0]{
    \setcounter{tocdepth}{2}
    \begin{tcolorbox}[
        enhanced,
        frame hidden,
        sharp corners,
        detach title,
        spread outwards     = 5pt,
        halign              = center,
        valign              = center,
        borderline west     = {3pt}{0pt}{main20!50!main2!95!gray!90},
        coltitle            = main20!50!main2!95!gray!90, 
        interior style      = {
            left color      = main1white2!65!gray!11,
            middle color    = main1white2!50!gray!10,
            right color     = main1white2!35!gray!9
        },
        arc                 = 0 cm,
        title               = SOMMAIRE,
        boxrule             = 0pt,
        fonttitle           = \bfseries\sffamily,
        overlay             = {
            \node[rotate=90, minimum width=1cm, anchor=south,yshift=-0.8cm]
            at (frame.west) {\tcbtitle};
        }
    ]
        \begin{minipage}{0.83\linewidth}
            \sffamily
            \minitoc
        \end{minipage}
    \end{tcolorbox}
}

\begin{document}
    
    %==============================
    % METADONNEES
    %==============================
    
    \title{Cours de Physique de MPI/MPI* (2022-2023)}
    \author{SIAHAAN--GENSOLLEN Rémy, Rayan Drissi}
    \date{\today}
    \hypersetup{
        pdftitle={Cours de Physique de MPI/MPI* (2022-2023)},
        pdfauthor={SIAHAAN--GENSOLLEN Rémy, Rayan Drssi},
        pdflang={fr-FR},
        pdfsubject={MPI/MPI*, Cours de Physique},
        pdfkeywords={MPI/MPI*, Cours de Physique, 2022-2023}
        pdfstartview=
    }
    
    %==============================
    % MISE EN PAGE
    %==============================
    
    \titleformat{\chapter}[display]{\normalfont\huge\bfseries}{}{0pt}{
        \begin{tcolorbox}[
            enhanced,
            frame hidden,
            sharp corners,
            spread sidewards    = 5pt,
            halign              = center,
            valign              = center,
            interior style      = {color=main1!20},
            arc                 = 0 cm,
            fontupper           = \color{black}\sffamily\bfseries\huge,
            fonttitle           = \normalfont\color{white}\sffamily\small,
            top                 = 1cm, 
            bottom              = 0.7cm,
            title               = Chapitre \thechapter,
            attach boxed title to bottom center = {
                yshift=\tcboxedtitleheight/2,
            },
            boxed title style = {
                frame code={
                \path[left color=main2!95!gray!90,
                right color=main1!95!gray!90] 
                    ([xshift=-10mm]frame.north west) -- 
                    ([xshift=10mm]frame.north east) -- 
                    ([xshift=10mm]frame.south east) -- 
                    ([xshift=-10mm]frame.south west) -- 
                    cycle;
                },
                interior engine=empty
            }
        ]
            #1
        \end{tcolorbox}%
    }
    \titlespacing*{\chapter}{0pt}{-120pt}{-15pt}
    \titleformat{name=\chapter,numberless}[display]{\normalfont\huge\bfseries}
    {}{0pt}{
        \begin{tcolorbox}[
            enhanced,
            frame hidden,
            sharp corners,
            spread sidewards    = 5pt,
            halign              = center,
            valign              = center,
            interior style      = {color=main1!20},
            arc                 = 0 cm,
            outer arc           = 0pt,
            leftrule            = 0pt,
            rightrule           = 0pt,
            fontupper           = \color{black}\sffamily\bfseries\huge,
            enlarge left by     = -1in-\hoffset-\oddsidemargin, 
            enlarge right by    = -\paperwidth+1in+\hoffset +
            \oddsidemargin+\textwidth,
            width               = \paperwidth, 
            left                = 1in+\hoffset+\oddsidemargin, 
            right               = \paperwidth-1in-\hoffset -
            \oddsidemargin-\textwidth,
            top                 = 1cm, 
            bottom              = 1cm
        ]
            #1
        \end{tcolorbox}%
    }
    \titlespacing*{name=\chapter,numberless}{0pt}{-115pt}{0pt}
    
    %==============================
    % PREMIERE DE COUVERTURE
    %==============================

    \includepdf[pages={1},scale=1.15,offset=0mm -18mm]{CPCover.pdf}
    
    %==============================
    % PAGE VIDE
    %==============================
    
    \pagestyle{empty}
    
    %==============================
    % PAGE DE COUVERTURE INTERNE
    %==============================
    
    \begin{titlepage}
	    \begin{center}
	        {\scshape SIAHAAN--GENSOLLEN Rémy, Rayan Drssi\par}
	        \vspace{2cm}
	        {\huge\sffamily Cours de\par}
	        \vspace{0.5cm}
	        {\Huge\bfseries\sffamily PHYSIQUE\par}
	        \vspace{1cm}
	        {\Large\textit{donné pendant mon année de \textsf{MPI/MPI*} à
	        Janson-de-Sailly}\\[5pt]\texttt{(2022-2023)}\par}
	        \vfill
	        {\large\EBGaramond Dernière compilation le \today\par}
        \end{center}
    \end{titlepage}
    
    %==============================
    % PAGE VIDE
    %==============================
    
    \pagestyle{empty}\text{}\newpage
    
    %==============================
    % STYLE DES EN-TÊTES ET PIEDS DE PAGES
    %==============================
    
    \renewcommand\chaptermark[1]{\markboth{#1}{}}
    
    \fancypagestyle{intro}{
        \fancyhf{}
        \renewcommand{\headrulewidth}{0pt}
        \renewcommand{\footrulewidth}{0pt}\fancyfoot[RO,LE]{\GillSansMTMedium\color{white5}\thepage\;/\;\pageref{LastPage}}
        \fancyhead[LE]{\GillSansMTMedium\color{white5}\bfseries COURS DE PHYSIQUE}
        \fancyhead[RE]{\GillSansMTMedium\color{white5}Avant-propos}
        \fancyhead[LO]{\GillSansMTMedium\color{white5}\rightmark}
        \fancyhead[RO]{\GillSansMTMedium\color{white5}\textbf{MPI/MPI*} 2022-2023 \quad Janson-de-Sailly}
    }
    
    \fancypagestyle{toc}{
        \fancyhf{}
        \renewcommand{\headrulewidth}{0pt}
        \renewcommand{\footrulewidth}{0pt}\fancyfoot[RO,LE]{\GillSansMTMedium\color{white5}\thepage\;/\;\pageref{LastPage}}
        \fancyhead[LE]{\GillSansMTMedium\color{white5}\bfseries COURS DE PHYSIQUE}
        \fancyhead[RE]{\GillSansMTMedium\color{white5}Table des matières}
        \fancyhead[LO]{\GillSansMTMedium\color{white5}\rightmark}
        \fancyhead[RO]{\GillSansMTMedium\color{white5}\textbf{MPI/MPI*} 2022-2023 \quad Janson-de-Sailly}
    }
    
    \fancypagestyle{plain}{
        \fancyhf{}
        \renewcommand{\headrulewidth}{0pt}
        \renewcommand{\footrulewidth}{0pt}\fancyfoot[RO,LE]{\GillSansMTMedium\color{white5}\thepage\;/\;\pageref{LastPage}}
        \fancyhead[LE]{\GillSansMTMedium\color{white5}\bfseries COURS DE PHYSIQUE}
        \fancyhead[RE]{\GillSansMTMedium\color{white5}Chapitre \thechapter : \nouppercase{\leftmark}}
        \fancyhead[LO]{\GillSansMTMedium\color{white5}\nouppercase{\rightmark}}
        \fancyhead[RO]{\GillSansMTMedium\color{white5}\textbf{MPI/MPI*} 2022-2023 \quad Janson-de-Sailly}
    }
    
    %==============================
    % PREFACE 
    %==============================
    
    \chapter*{Avant-propos}
    \thispagestyle{intro}
    \addcontentsline{toc}{chapter}{Avant-propos}
    
    \text{\Large\EBGaramond\itshape À tout lecteur potentiel, quelques mots...}\newline\newline\newline
    
    \begin{center}
        \begin{minipage}{0.85\linewidth}
            \large \qquad Comme son nom l'indique, l'objectif de cet ouvrage est de fournir un cours de physique en accord avec le programme des classes préparatoires \textsf{MPI/MPI*}. Il contiendra principalement des notes de cours, dont je serai dispensé mon année de \guill{spé'} (année 2022/2023) à \textit{Janson-de-Sailly}, par M. \textsc{Marc Antoine Blain}. J'essaierai par ailleurs de détailler et d'enrichir le plus possible son contenu au fil de l'année, à l'aide de mes cours de première année, d'autres ouvrages et de recherches en général. La rédaction de ce cours constitue un important projet, d'autant plus que j'en mène un similaire pour les enseignements de mathématiques et de physique cette année. C'est un travail qui peut s'avérer extrêmement chronophage, aussi risque-t-il d'être rarement mené jusqu'au bout.\newline
    
            \qquad Je ne prétends à aucun moment être enseignant, et ce livre reste avant tout destiné à mon usage personnel, aussi j'aviserai tout lecteur potentiel à faire preuve de prudence lors du parcours de ce texte, à ne pas hésiter à en vérifier le contenu par lui même. Il est très probable que de multiples erreurs (en tout genre) se soient glissés durant la rédaction, que je n'aurait su repérer, ou que le manque de temps empêche la correction. N'hésitez d'ailleurs pas à me le signaler, ou à me faire part de vos remarques en général.\newline
    
            \qquad J'espère enfin, et malgré les points exprimés précédemment, que ce cours pourra avoir une quelconque utilité à ceux qui s'y aventureraient, que sa lecture et son style en seront agréable (la mise en page et la composition graphique en général sont de ma conception personnelle, enrichie par les retours de mes camarades, et le fruit de plusieurs mois d'apprentissage de \LaTeX) et enrichissante.\newline\newline\newline\text{}
        \end{minipage}
    \end{center}
    
    \hfill{\large\textsc{Siahaan--Gensollen Rémy, , Rayan Drssi}}
    
    \pagestyle{intro}
    
    %==============================
    % TABLE DES MATIERES
    %==============================
    
    \newpage
    \dominitoc\nomtcrule 
    {\sffamily\tableofcontents}\mtcaddchapter\pagestyle{toc}
    
    \cleardoublepage
    
    %==============================
    % COURS
    %==============================
    
    \pagestyle{plain}
    
    \setcounter{chapter}{9}
    \chapter{Équations de Maxwell}
    
    Ce chapitre a pour objectif d'étudier les équations de \textsc{Maxwell}, et de les appliquer à des billes d'énergies élémentaire, ainsi qu'aux ondes élémentaires.
    
    \chaptertoc{}
    
    \section{Équations de Maxwell}
    
    \subsection{Rappels sur les équations locales}
    
    Rappelons tout d'abord les équations locales en électrostatique
    %
    \begin{form}{Loi électrostatiques locales}{}
        \[ \begin{array}{c}
            \displaystyle\hg{\oiint_S \vec E \cdot \dif \vec S = \dfrac{Q_\text{int}}{\epsilon_0}} \qquad\text{donc par théorème d'\textsc{Ostrogradski},}\qquad \hg{\vec \nabla \cdot \vec E = \dfrac{\rho}{\epsilon_0}}\\
            \displaystyle \hg{\oint_\bcC \vec E \cdot \dif \vec{\ell} = 0} \qquad\text{donc par théorème de \textsc{Stokes},}\qquad \hg{\vec \nabla \wedge \vec E = \vec 0}
        \end{array} \]
    \end{form}
    %
    De même pour les équations en magnétostatique :
    %
    \begin{form}{Loi électrostatiques locales}{}
        \[ \begin{array}{c}
            \displaystyle \hg{\oiint_S \vec B \cdot \dif \vec S = 0} \qquad\text{donc par théorème d'\textsc{Ostrogradski}}\qquad \hg{\vec \nabla \cdot \vec B = 0}\\
            \displaystyle \hg{\oint_\bcC \vec B \cdot \dif \vec{\ell} = 0} \qquad\text{donc par théorème de \textsc{Stokes}}\qquad \hg{\vec \nabla \wedge \vec B = \mu_0 \vec \imath }
        \end{array} \]
    \end{form}
    %    
    En régime variable, $I_\text e$ n'est pas définit de manière uniforme, et le phénomène d'induction impose une relation entre $\vec E$ et $\vec B$, donnée la loi de \textsc{Faraday} :
    %
    \begin{form}{loi de Faraday}{}
        \[ \hg{e = -\dfrac{\dif Q}{\dif t}\phi_{\vec B}\p{\vec{S_\bsC}}}\]
        %
        où $e$ est la \hg{force électromotrice}, et $\hg{\phi_{\vec B}\p{\vec{S_\bsC}} = \displaystyle\niint_{S_\bsC} \vec B \cdot \dif \vec S}$.
    \end{form}
    
    
    
    
    23/01 11h30 ça me soule d'ecrire a la main j'ecrit en latex 
    
    
    \begin{definition}{$\bsE_{elm_{int}}$}{}
    on introduit $\bsE_{elm}$ la densité volumique d'energie electromagnetique
    
    \[ \bsE_{elm_{int}}(v) = \iiint e_{elm}d\bsV  \]
    \end{definition}
    
    \begin{definition}{}{}
        On introduig le vecteur densité de courant d'energie electromagnetique $\vec{j}_{\bsE_{elm}}$ tel que:
        
        \[ \boxed{ \vec{j_{\bsE_{elm}}} \cdot \vec{dS} = \dfrac{ \delta \bsE_{elm}(\vec{dS}) }{dt}} \]
        
    on nomme et on note ce vecteur: vecteur de \underline{Poynting} 
    \[ \vec{j_{elm}} = \vec{R} \quad ou \quad \vec{\Pi} \]
    \end{definition}
    
    
    \begin{definition}{$\dfrac{ \delta \bsE_{elm_{crée}}(v)}{dt}$}{}
        
        \[\dfrac{ \delta \bsE_{elm_{crée}}(v)}{dt} \]
        
        $ \delta \bsE_{elm_{cree}}(v) =  \delta \bsE_{elm_{obstrue}}$ 
        les energies blablablablablablablbabl
        
        plein de calcul chiant pour arriver à:
        
        $$\boxed{\bcp_{elm_{absorbé}} = \dfrac{\delta \bsE_{elm_{abs}}}{dt d\tau} = \vec{j}\cdot \vec{E} }$$
        
    \end{definition}
    
    \begin{form}{}{}
        cherchons l'equation locale de conservation de l'$\bsE_{elm} $
        
        $\dfrac{d}{dt}(\iiint_\bcV e_{elm}d\tau) = - \varoiint_{S_G} \vec{R} \vec{dS}  - \iiint_\bcV \vec{j}E d\tau $
        
        blablabla 
        
        \[\textcolor{red}{\boxed{ \dfrac{\delta e_{elm}}{\delta t} = div\vec{R} + \vec{j E} = 0 }}\]
        
        Equation de Poynting
        Equation locale de conservation de l'$\bcE_{elm}$
        
        blablablabla 
        
        \[\textcolor{red}{\boxed{ \vec{R} = \dfrac{\vec{E} \wedge \vec{B}}{\mu_0} } = \text{Vecteur de poynting}} \]
            
        remarque importante: 
        
        \begin{enumerate}
            \itt $e_{elm}$, $\vec{R}$, $p_{v_{disparition}}$ sont trois grandeur energitique \underline{\textbf{quadratique}} en $\vec{E} \wedge \vec{B}$
            
            \itt Certains calcul ou  raisonmeent sont plus simple en modelisant l'onde elm 
            par un flux de photon 
            
        \end{enumerate}
        
    \end{form}
    
    \begin{example}{le conducteur ohmique}{}
        \begin{enumerate}
            \itt \hg{Préliminaire : modélisation de charge d'un conducteur électrique.} Dans un conducteur électrique, il y a des charges dites \guill{libres} (des électrons), lesquelles peuvent se déplacer macroscopiquement. On modélise l'interaction entre les charges et le réseau d'ions positifs par une force frottement visqueux 
            %
            \[ \hg{\vec f = \dfrac{-m}{\tau} \vec v} \]
            %
            Une étude mécanique du mouvement des charges permet d'établir l'équation d'état d'un conducteur ohmique (\cf TD).
            
            
            On donne la \hg{loi d'\textsc{Ohm} locale} : \qquad $\hg{\boxed{\vec \jmath = \gamma \vec E}}$ où $\gamma$ est la \emph{conductivité du matériaux}. Pour un conduteur ohmique $\gamma \in \bdR$ est une constant positive, avec généralement 
            %
            \[ \qty{1e6}{\per\ohm\per\meter} \leq \gamma \leq \qty{1e8}{\per \ohm \per \meter}\]
            
            \begin{minipage}{0.5\linewidth}
                \itt \hg{Bilan d'énergie élémentaire :} pour un conducteur ohmique cylindrique d'un axe $\p{Oz}$, de rayon $R$, de hauteur $h$ et parcouru par un courant $I$ en régime stationnaire.\medskip
                
                Le courant $I$ est uniforme sur la section $\bcS = \pi R^2$. Par ailleurs, $\vec \jmath  = \dfrac{I}{\pi r^2} \vec {e_z}$ 
            \end{minipage}
            %
            \begin{minipage}{0.5\linewidth}
                \begin{center}
                    \tdplotsetmaincoords{70}{20}
                    \pgfplotsset{compat=newest}
                    
                    \begin{tikzpicture}[tdplot_main_coords, scale = 1.8]
                        \draw[-stealth] (0, 0, 0) --++ (1, 0, 0) node[above] {$x$};
                        \draw[-stealth] (0, 0, 0) --++ (0, 1, 0) node[above] {$y$};
                        \draw[-stealth] (0, 0, 0) --++ (0, 0, 1) node[above] {$z$};
                        
                        \draw[dashed, gray] (0, 0, 0) --++ (-1, 0, 0);
                        \draw[dashed, gray] (0, 0, 0) --++ (0, -1, 0);
                        \draw[dashed, gray] (0, 0, 0) --++ (0, 0, -1);
                        
                        \draw plot[variable=\x,domain=0:360,samples=180] ({cos(\x)},{sin(\x)},-0.5);
                        \draw plot[variable=\x,domain=-45:135,samples=180] ({cos(\x)},{sin(\x)},0.5);

                        \foreach \x in {135,-45}
                            {\draw ({cos(\x)},{sin(\x)},-0.5) -- ({cos(\x)},{sin(\x)},0.5);}
                    \end{tikzpicture}
                \end{center}
            \end{minipage}
            %helpppppppppp
            
            
            $\forall r<R$ $\vec{E} = \dfrac{\vec j}{\gamma} = \dfrac{I}{\pi \gamma R^2}\vec u_z$
            
            Si $r>R$ $\gamma = 0 $
            
            Symetrie invarience: $\Pi_s(M,\vec u_y,\vec u_r)$ \Rightarrow{} $\vec B =  B(r)\vec{e_\theta}$
            
            Cercle de rayon $r < R$ orienté suivant $\vec{e_\theta}$
            
            \[ \oint \vec{B} \cdot \vec{dl} = \mu_0 (I_e + I_o)\]
            \[ 2 \Pi_r B(r) = \mu_0 \dfrac{I}{\pi r^2} * \pi r^2 \Rightarrow{} \Vec{B(r)} = \dfrac{\mu_0 I_r}{2 \pi r^2} \vec{e_\theta}  \]
            
            Bilan d'energie sur le condensateur :
            
            \begin{enumerate}
                \item $\displaystyle\dfrac{\dif \bsE_\text{element int}}{\dif t} = \dfrac{\dif}{\dif t} \iiint e_\text{elem}\dif \tau = \dfrac{\dif}{\dif t}\oiint \p{\dfrac{1}{2}\epsilon_0 E^2 + \dfrac{B^2}{2\mu}}\dif \tau$
                
                \item $\oiint_{\bsS_G} \vec R \cdot \dif \vec S = 0 + 0 + \p{-\dfrac{RI^2}{2\pi^2R^4\gamma}}2\pi Rh$.
            \end{enumerate}
            
            
        \end{enumerate}
    \end{example}
    
    \chapter{Induction}
    
    Bien que l'\emph{induction} soit un phénomène étudié en première année, on se propose ici d'en rappeler les principes dans un chapitre de format différent, et de les approfondir légèrement, à l'aide d'exercices. L'idée importante de l'induction est le lien entre le mécanique et l'électromagnétique : les phénomènes à l'\oe{}uvre permettent de transformer l'énergie mécanique et énergie électromagnétique, et inversement. Les exercices qui en découlent posent généralement une certaine difficulté aux élèves, car remplis de problèmes de signe (orientation des volumes, des surfaces, orientation des vecteurs dans les formules, \dots). On présentera donc les résultats avec une grande rigueur en ce qui concerne ce point.
    
    \chaptertoc{}
    
    \section{Méthode générale}
    
    Comme on l'a dit précédemment, les problèmes de signes sont nombreux en induction. On donne ci-dessous une méthode, à suivre rigoureusement pour ne pas se tromper.
    
    \begin{form}{Méthode}{}
        \begin{enumerate}
            \itt \colorbox{colform!20}{\textnormal{\color{colform}\sffamily\bfseries \,étape 1\,}} on choisit arbitrairement un sens pour l'intensité $i$ dans le circuit.
            
            \itt \colorbox{colform!20}{\textnormal{\color{colform}\sffamily\bfseries \,étape 2\,}} $\text C$ est orienté par $\vec \imath$ donc $\bcS_\text C$ aussi.
            
            \itt \colorbox{colform!20}{\textnormal{\color{colform}\sffamily\bfseries \,étape 3\,}} On calcule les forces de \textsc{Laplace} avec $\vec{\dif \ell}$ orienté par $\vec \imath$.
        \end{enumerate}
    \end{form}
    
    \section{Exercices}
    
    \subsection{Couplage de deux barres}
    
    On considère deux rails parallèles à la direction $\vec{Ox}$, parfaitement conducteurs, séparés par la distance $d$ et contenus dans le plan $\p{xOy}$. On pose dessus deux barres, de résistance $R$ et de masse $m$, qui peuvent glisser sans frottement. Le tout est plongé dans le champ magnétique stationnaire et uniforme $\vec B = B_0\vec{u_z}$.
    
    \begin{enumerate}
        \item On impose une vitesse constante à la première barre $\vec v_1 = v_0\vec{u_x}$. Quelle est l'évolution de la vitesse de la deuxième barre, sachant qu'elle est initialement immobile ?
        
        
    \end{enumerate}
    
    `\chapter{Milieux dispersifs : Définition et exemples}
    
    Une onde sinusoïdale a une pulsation $\omega$ fixée par la source. L'expression de son vecteur d'onde $k$ dépend du milieu de propagation : l'équation de dispersion $k = k\p{\omega}$ permet de connaître les caractéristiques de la propagation dans le milieu (elle s'obtient en injectant l'onde dans l'équation de propagation).
    
    \chaptertoc
    
    \section{Phénoème de dispersion}
    
    \subsection{Ondes réelles (ou ondes physiques)}
    
    \subsubsection{Rappel : modèle de l'OPPM}
    
    \begin{form}{Modèle de l'OPPM}
        Une \hg{OPPM} vérifie l'équation complexe suivante :
        %
        \[ \underline{\vec E}\p{M, t} = \vec{E_0}e^{\jj\p{\omega t - \vec k \vec r}}\]
        %
        où $\omega$ est la pulsation de l'onde, $\vec k = k\vec{u}$ est son vecteur d'onde avec $\vec u$ le vecteur directeur de la direction de propagation.
    \end{form}
    
    Comme on l'a vu, une OPPM a des surfaces d'onde infinies, donc nécessite une énergie infinie : elle n'existe donc pas concrètement, et il s'agit uniquement d'un modèle. Cependant ce modèle est applicable dans beaucoup de situations :
    %
    \begin{center}
        \emph{les ondes réelles présentent localement la structure d'une OPPM}
    \end{center}
    %
    Donnons quelques exemples :
    %
    \begin{example}{}{}
        \begin{center}
            \begin{tikzpicture}
                \node[main1] at (-0.3, 0.2) {S};
                
                \draw[dashed, gray, ->] (-1, 0) -- (5.5, 0);
                \draw[main1] (0, 0) -- (5, 0);
                \draw[main1] (0, 0) -- (1, 1) -- (5, 1);
                \draw[main1] (0, 0) -- (1, -1) -- (5, -1);
                
                \draw[thick, <->] (1, -1.4) -- (1, 1.4);
                
                \draw[main10] (3, -0.5) --++ (0, 1);
                \draw[main10] (3.2, -0.5) --++ (0, 1);
                \draw[main10] (3.4, -0.5) --++ (0, 1);
                \draw[main10] (3.6, -0.5) --++ (0, 1);
                \draw[main10] (3.8, -0.5) --++ (0, 1);
                \draw[main10] (4, -0.5) --++ (0, 1);
                
                \draw (2.9, -0.4) rectangle (4.1, 0.4);
                 \draw[thick, main3 ,<->] (4.3, -0.4) --node[midway, right] {$\ell$} (4.3, 0.4);
                
                
                \draw[thick, main3 ,<->] (5.7, -1.4) --node[midway, right] {$\phi$} (5.7, 1.4);
            \end{tikzpicture}
        \end{center}
        %
        Pour $\ell \ll \phi$, l'onde se comporte comme une OPPM.\bigskip
        
        
        \begin{center}
            \begin{tikzpicture}
            
                \node[main1] at (-0.3, 0) {S};
                
                \draw[dashed, gray, ->] (-1, 0) -- (5.5, 0);
                
                \draw (0,-1) arc (-90:90:1);
            \end{tikzpicture}
        \end{center}
    \end{example}
    
    On va en fait montrer que les OPPM forment une base des ondes réelles.
    
    \subsubsection*{Somation d'OPPM : Transformée de \textsc{Fourier}}
    
    Soit $\underline f\p{t}$ une fonction bornée $\bcC^1$ telle que $f\p{\pm \infty} = 0$ (on posera $\tau$ la durée sur laquelle $f\p{t}$ a des valeurs notables). On a :
    %
    \[ f\p{t} = \int_0^{+\infty} g\p{\omega}\cos{\omega t}\dif \omega\]
    %
    ainsi que $k \dfrac{2\pi}{\lambda}$ et $\Delta k = 2\pi \dfrac{\delta \lambda}{\lambda^2}$.
    
    \begin{enumerate}
        \itt $g\p{\omega}$ apparaît comme l'amplitude de la composante sinusoïdale de pulsation $\omega$ de $f\p{t}$.
        
        \itt $g\p{\omega}$ est le spectre de $f\p{t}$
    \end{enumerate}
    
    \begin{property}{Propriété fondamentale}{}
        \[ \hg{\Delta \omega \times \tau \approx 1}\]
    \end{property}
    
    On notera que $g\p{\omega}$ est liée à la transformée de \textsc{Fourier} de $f$.
    
    \subsubsection*{Généralisation}
    
    Soit une onde $f\p{t, z}$ de volume fini, de durée $\tau$ et de longueur $L$. On a $L = v\tau$.
    
    On admet que, avec l'équation de dispersion $k = k\p{\omega}$, on a 
    %
    \[ f\p{t, z} = \int_0^{+\infty} g\p{\omega}\cos{\omega t - k\p{\omega}z}\dif \omega \qquad\ou\qquad \int_0^{+\infty} h\p{k}\cos{\omega\p{k}t - kz}\dif k\]
    %
    Ici $q\p{\omega}$ est le spectre en pulsation, $h\p{k}$ est le spectre en vecteur d'onde (et $h_2\p{\lambda}$ est le spectre en longueur d'onde)
    
    \begin{form}{A retenir}
        Toute onde réelle peut s'écrire comme une somme d'OPPM, avec
        %
        \[ \hg{\Delta \omega \tau \approx 1} \qquad\et\qquad \hg{\Delta k L \approx 1}\]
        %
        Conclusion : les OPPM forment une base des ondes réelles.
    \end{form}
    
    \begin{example}{à connaître}{}
        \begin{center}
            \emph{Onde quasi-sinusoïdale}, aussi appelée \emph{paquet d'onde}.
            
            \begin{tikzpicture}
                \begin{axis}[
                    axis lines          =   middle,
                    axis line style     =   {-stealth,shorten >=-3mm},
                    trig format plots   =   rad,
                    trig format         =   rad,
                    domain              =   0:2,
                    xmin                =   0,
                    xmax                =   2,
                    ymin                =   -1.2,
                    ymax                =   1.2,
                    xlabel              =   $t$,
                    ylabel              =   $f\p{t, x}$,
                    grid                =   none,
                    width               =   14cm,
                    height              =   9cm,
                    xtick               =   {},
                    ytick               =   {},
                    yticklabels         =   {}
                ]
                    \addplot[color=main1, samples=300, smooth, thick] {0.8*sin(15*pi*x)*exp(-15*(x-1)^4)};
                        
                    \draw [<->, main3] (0.9, -0.85) --node[midway, below] {$\tau_0$} (1.0333, -0.85);
                \end{axis}
            \end{tikzpicture}
        \end{center}
    \end{example}
    
    \subsection{Propagation d'une onde réelle}
    
    \subsubsection*{Positionnement du problème : milieu dispersif}
    
    Pour une OPPM on a $v_\varphi = \dfrac{\omega}{k}$ la vitesse de phase.
    
    \begin{nproof}
        \[ \varphi\p{t, z} = \omega t - kz = -k\p{z - \dfrac{\omega}{k}t} = f\p{z - \dfrac{\omega}{k}t}\]
        %
        La phase progresse donc à la vitesse $v_\varphi = \dfrac{\omega}{k}$.
    \end{nproof}
    
    Or une OPPM ne dépend que de sa phase donc elle se propage aussi à la vitesse $v_\varphi$.
    
    \begin{warning}{}{}
        \begin{center}
            \emph{Une OPPM n'existe pas donc \hg{on peut parfaitement avoir $v_\varphi > c$} !}
        \end{center}
    \end{warning}
    
    Qualitativement, une onde réelle est une superposition d'OPPM chacune allant à la vitesse $v_\varphi = \frac{\omega}{k}$. Or $k = k\omega$ donc $v_\varphi = \dfrac{\omega}{k\p{\omega}}$.
    
    \begin{enumerate}
        \itt Si $v_\varphi$ est une indépendante de $\omega$, alors chaque composante de l'onde réelle se déplace à la même vitesse. Ainsi l'onde réelle se déplace aussi à la vitesse $v_\varphi$ sans se déformer.
        
        \itt En revanche si $v_\varphi$ est une fonction de $\omega$, chaque composante sinusoïdale de l'onde réelle a une vitesse différente, et elles se décalent les unes par rapport aux autres : l'onde réelle se propage à une vitesse à définir, et \emph{\EBGaramond a priori} en se déformant.
    \end{enumerate}
    
    \begin{definition}{Milieu dispersif}{}
        On appelle \hg{milieu dispersif} un milieu dans lequel \hg{$v_\varphi$ est une fonction de $\omega$}.
    \end{definition}
    
    Remarquons que $v_\varphi = \dfrac{\omega}{k\p{\omega}}$, donc dans un milieu dispersif $k\p{\omega} \neq \omega \times\; \p{\text{facteur indépendant de $\omega$}}$. En conséquence :
    %
    \[ \emph{un milieu dispersif est un milieu dans lequel l'équation de dispersion n'est pas linéaire}\]
    %
    \begin{definition}{Indice de réfraction}{}
        L'\hg{indice de réfraction} d'un milieu matériel est défini par \hg{$n = \dfrac{c}{v_\varphi}$}.
    \end{definition}
    %
    Par exemple, dans le vide $k = \dfrac{\omega}{c}$ donc $v_\varphi = c$, ainsi $n = 1$. Il s'agit là d'un milieu non dispersif.
    
    \begin{enumerate}
        \itt Si $\Delta \omega$ du spectre d'une onde réelle est suffisamment petit on pourra toujours prendre
        %
        \[ v_\varphi = v_\varphi\p{\omega_0} + \underbrace{\dfrac{\dif v_\varphi}{\dif \omega}\int_{\omega_0}\dif \omega}_{\text{négligeable}}\]
    \end{enumerate}
    
    \subsubsection*{Cas d'une onde quasisinusoïdale}
    
    \newpage
    
    \section{Exemple de milieu dispersif : le plasma}
    
    \subsection{Modélisation}
    
    \subsubsection*{Présentation générale}
    
    \begin{definition}{Plasma}{}
        Un \hg{plasma} est un mélange d'ions et d'électrons (c'est un gaz ionisé).
    \end{definition}
    
    Dans toute la suite de cette étude, on considéra valides les hypothèses suivantes :
    %
    \begin{center}
        \begin{minipage}{0.7\linewidth}
            \emph{Les plasma étudiés sont monovalents (constitués d'ions monovalents), on aura donc autant d'$e^-$ que d'ions. On ne modélisera par ailleurs les interactions intérieures au plasma que par des chocs.}
        \end{minipage}
    \end{center}
    %
    On étudiera donc un mélange de deux gaz parfaits : celui des ions et celui des électrons, ce qui est généralement possible quand le plasma est peu dense.
    
    
    
    \begin{notation}
        On pose tout d'abord quelques notations :
        %
        \begin{enumerate}
            \itt $M$ désigne la masse des ions
            
            \itt $m$ désigne la masse des $e^-$
            
            \itt $n_+$ désigne la densité volumique d'ions et $n_-$ la densité volumique d'$e^-$.
            
            \itt On aura $N_-$ et $n_+$ tels que $n_- = \dfrac{\dif N_-}{\dif V}$ et $n_+ = \dfrac{\dif N_+}{\dif V}$.
            
            \itt Les quantités $v_+$ et $v_-$ désignent les vitesses \guill{moyenne} respectivement des ions et des $e^-$.
        \end{enumerate}
    \end{notation}
    
    Initialement à l'équilibre, le plasma est neutre d'où l'on a $n_+\p{0}  = n_-\p{0} = n_0$. En présence d'une onde, on aura
    %
    \[ n_+ = n_0 + \delta n_+ \qquad\et\qquad n_- = n_0 + \delta n_-\]
    %
    En l'absence d'onde, on aura aussi $v_+ = v_- = 0$ (plasma neutre à l'équilibre).\bigskip
    
    \underline{Remarque :} on étudie la propagation d'une OPPM d'amplitude \guill{pas trop grande} dans le plasma afin de pouvoir supposer que les perturbations due à l'onde sont faibles, et pouvoir faire des DL à l'ordre 1. Ainsi $v_+, v_-, \delta n_+, \delta n_-$ sont des quantités infinitésimales.
    
    \begin{property}{Distribution de charge et densité de courant d'un plasma monovalent}{}
        Pour un \hg{plasma monovalent $\bfP$} (supposé uniforme), on a la \hg{distribution de charge}
        %
        \[ \hg{\rho = e\p{\delta n_+ + \delta n_-}} \]
        %
        et la \hg{densité de courant}
        %
        \[ \hg{\vec \jmath = n_0e\p{\vec{v_+} + \vec{v_-}}} \]
    \end{property}
    %
    \begin{nproof}
        \begin{align*}
            \forall M \in \bfP,&& \rho\p{M} &= n_+e n_-\p{-e} = \p{n_0 + \delta n_+}e - \p{n_0 + \delta n_-}e = e\p{\delta n_+ + \delta n_-}\\
            \forall M \in \bfP,&& \vec \jmath \p{M} &= n_+e\vec{v_+} + n_-\p{-e}\vec{v_-} = n_0e\underbrace{\p{\vec{v_+} + \vec {v_-}}}_{\text{ordre } 1} + e\underbrace{\p{\delta n_+ \vec{v_+} - \delta n_-\vec{v_-}}}_{\text{ordre } 2} \approx n_0e\p{\vec{v_+} + \vec{v_-}}
        \end{align*}
    \end{nproof}
    
    \subsubsection*{Loi constitutive}
    
    On cherche $\vec j = f\p{\vec E}$. Pour cela, on applique la RFD à un $e^-$ :
    %
    \[ m \dfrac{\dif \vec{v_-}}{\dif t} = m\vec g - e\vec E - e\vec{v_-} \wedge \vec B\]
    %
    \begin{enumerate}
        \itt On peut déjà négliger le poids $\vec{P} = m\vec g$ devant la force électrique $\vec{F_e} = e\vec E$. En effet :
        %
        \[ \dfrac{mg}{eE} \approx \dfrac{10^{-30} \times 10}{10^{-19} \times E} = \dfrac{10^{-10}}{E} \ll 1 \]
        %
        dès que $E \gg \qty{10e-10}{\volt \per \meter}$, ce qu'on considérera comme étant toujours vrai.
        
        \itt Par ailleurs, on voudrait également pouvoir négliger la force magnétique $\vec{F_m} = e\vec{v_-}\wedge B$. On a :
        %
        \[ \dfrac{F_m}{F_e} \approx \dfrac{ev_-B}{eE} = \dfrac{v_-B}{E} \underset{\text{(OPPM)}}{\approx} \dfrac{v_-}{v_\varphi} \]
        %
        Or $v_\varphi = \dfrac{c}{n} = \approx c$ donc tant que les électrons ne sont pas relativistes on a $F_m \ll F_e$.
    \end{enumerate}
    %
    En conclusion, $m\dfrac{\dif \vec{v_-}}{\dif t} = -e\vec E$. Or $\vec{\underline E} = \vec{\underline{E_0}} e^{\jj \p{\omega t - \vec k \cdot \vec r}}$, d'où
    %
    \[ \]
    %
    On relève un premier problème : l'équation n'est pas linéaire en $r\p{t}$.
    


    \newcommand{\deriv}[1]{\dv{#1}{t}}
    \newcommand{\derivv}[1]{\dfrac{\mathrm{d}^2 #1}{\mathrm{d}t^2}}

    %mais arrête de créer 3000 commandes.... copie le cours plutôt mdr
    \begin{form}{Première méthode}{}
        On cherche une solution sous la forme 
        
        $\vec v_- = \vec{v_{0_-}} e^{\jj(\omega t-\vec{k}\dot \vec r}$
        \newline 
        On a donc $\deriv{\vec{ v_-}} = \vec{ v_{0-}} j(w- \vec k \deriv{\vec r})e^{j(wt-\vec k \cdot \vec r)}$. Or
        %
        \[ \dfrac{\mod{\vec k \cdot \dfrac{\dif \vec r}{\dif t}}}{\omega} \approx \dfrac{\dfrac{2\pi}{\lambda} v_-}{\omega} = 2\pi \dfrac{v_-}{\omega \lambda} = \dfrac{v_-}{v_\varphi} \l 1\]
    \end{form}
    
    On présente ci-dessous une seconde méthode

    \begin{form}{Deuxième méthode}{}
        TODO 
        
        \[ \vec E = \vec E_0 e^{\jj\p{\omega t - \vec k \vec r_0}}e^{-\jj \vec k \cdot \delta \vec r}\]
        %
        Or $\vec k \cdot \delta \vec r \ll \pi$ d'où $e^{-\jj \vec k \cdot \delta \vec r} \approx 1$. Ainsi
        %
        \[ \vec E \pprox \vec{e_0}e^{\jj\p{\omega t - \vec k \vec{r_0}}} \qquad\text{d'où}\qquad m\dfrac{\dif\vec{v_-}}{\dif t} = \p{-e\vec{E_0}e^{-\jj\vec k \cdot \vec{r_0}}}e^{\jj \omega t}\]
        %
        On intègre : \qquad $m\vec v_- = \p{-e\vec{E_0}e^{-\jj \vec k \vec{r_0}}}\dfrac{e^{\jj \omega t}}{\jj \omega}$
    \end{form}
    
    En conclusion, on a 
    %
    \begin{property}{}{}
        Puisque $v_- \ll c$ (soit $a \ll \lambda$) entraîne
        %
        \[ \hg{\dfrac{\dif \vec{v_-}}{\dif t} \approx \dfrac{\partial \vec{v_-}}{\partial t}} \]
    \end{property}
    
    On a donc :
    %
    \begin{align*}
        && m\dfrac{\dif v}{\dif t} &= -e\vec{E_0} e^{\jj \p{\omega t - \vec k \cdot \vec r}}\\
        \implies && m\dfrac{\partial v}{\partial t} &= -e\vec{E_0} e^{\jj \p{\omega t - \vec k \cdot \vec r}}\\
        \implies && m\vec{v_-} &= -e\vec{E_0}\dfrac{e^{\jj\p{\omega t - \vec k\vec r}}}{\jj \omega} + \underbrace{C}_{= 0 \ \text{car on étudie la réponse à ?}}
    \end{align*}
    %
    Soit finalement
    %
    \[ \vec{v_-} = \dfrac{-e\vec E}{\jj m \omega}\]
    %
    ...
    
    On a donc $M \dfrac{\dif \vec{v_+}}{\dif t} = e\vec E$ d'où $\vec{v_+} = \dfrac{e\vec E}{\jj M \omega} + \underbrace{\vec{cte}}_{= 0}$. P
    
    \begin{form}{Remarque}{}
        \begin{enumerate}
            
            \itast Consequence energitique de $\gamma \in i \bcR$
            
            \itast TODO
            
            
        \end{enumerate}
    \end{form}
    
    \subsection{Propagation d'une OPPM  dans un plasma }
        \subsubsection{Equation de Maxwell et Consequence}
        
        
        $\vec E = \vec E_0 exp{j(\omega t - \vec k \cdot r)}$
        
        \itast Maxwell Faraday TODOOOOOO
        
        \begin{form}{}{}
            Equation de dispertion 
            
            (-j\vec k)^2xM TODO
            
        \end{form}
    
    
    
\end{document}  