\documentclass[a4paper,french,bookmarks]{article}

\usepackage{./Structure/4PE18TEXTB}

\newboxans
\usepackage{booktabs}

\begin{document}

    \renewcommand{\thesection}{\Roman{section}} 
    \renewcommand{\thesubsection}{\thesection.\Alph{subsection}}
    \setlist[enumerate]{font=\color{white5!60!black}\bfseries\sffamily}
    \renewcommand{\labelenumi}{\thesection.\arabic{enumi}.}
    \renewcommand*{\labelenumii}{\thesection.\arabic{enumi}.\arabic{enumii}.}
    \renewcommand*{\labelenumiii}{\alph{enumiii}.}
    
    \stylizeDocSpe{Physique}{Devoir maison n° 4}{}{Pour le lundi 10 octobre 2022}
    
    \section{Tir d'un obus vers le zénith}
    
    \emph{Au \textsc{XVII}\ieme~siècle, le Père {\scshape Mersenne}, ami et
    correspondant de {\scshape Descartes}, se livra à un tir d'obus, le canon
    étant pointé vers le zénith. Le résultat ne fut pas du tout celui escompté.
    On se propose d'étudier l'influence de différents facteurs physiques sur
    la trajectoire de l'obus.}\medskip
        
    En un lieu $A$ de latitude $\lambda = \qty{48}{\degree N}$, un canon
    tire un obus à la vitesse $v_0 = \qty{100}{\metre\per\second}$ suivant la
    verticale ascendante $\vec{A_z}$. On désigne par $\p{A, x, y,
    z}$ un repère orthonormé lié à la Terre, $\vec{A_x}$ étant
    dirigé vers le sud. On assimile la Terre à une sphère homogène, tournant
    autour de l'axe des pôles à la vitesse angulaire $\omega_0 =
    \qty{7.5e-5}{\radian\per\second}$. On note $\vec{g_0}$ le champ de
    pesanteur terrestre, de module $g_0$ supposé constant égal à
    $\qty{10}{\metre\per\second\squared}$.

    \begin{enumerate}
        \item On considère le référentiel lié à la Terre.
        
        \begin{enumerate}
            \item On néglige la résistance de l'air.

            \begin{enumerate}
                \item\label{qu:I.1.1.a} Donner l'expression de la vitesse
                $\vec v$ de l'obus à un instant quelconque.
                
                \noafter
                %
                \boxans{
                    Soit $m$ la masse de l'obus.Puisque l'on néglige la résistance de l'air, la seule force
                    qui s'applique sur l'obus est son poids $\vec P = m\vec{g_0} = -mg_0\vec{A_z}$. En vertu du 
                    \textit{principe fondamental de la dynamique}, on a donc :
                    %
                    \[ m\dfrac{\dif \vec v}{\dif t} = -mg_0\vec{A_z} \qquad\text{d'où}\qquad \dfrac{\dif \vec v}{\dif t} = -g_0\vec{A_z}\]
                    %
                    On intègre entre $0$ et $t$, ainsi $\vec v\p{t} = \vec v\p{0} - g_0t\vec{A_z}$.
                }
                %
                \nobefore\yesafter
                %
                \boxansconc{
                    Puisque $\vec v\p{0} = v_0\vec{A_z}$, on a finalement $\vec v\p{t} = \p{v_0 - g_0t}\vec{A_z} = v_z\p{t}\vec{A_z}$.
                }
                %
                \yesbefore
                
                \item Exprimer l'énergie mécanique de l'obus. Varie-t-elle au
                cours du temps ? Calculer l'altitude maximale de atteinte par
                l'obus.
                
                \noafter
                %
                \boxans{
                    Soit $E_\text m$ l'énergie mécanique de l'obus et $E_\text c$ son énergie cinétique. On note $z\p{t}$ la position de l'obus sur l'axe orienté porté par $\vec{A_z}$ au temps $t$. On note alors $E_\text{pp}$ son énergie potentielle de pesanteur, nulle au niveau du sol ($z\p{t} = 0$), ainsi $E_\text pp{t} = mgz\p{t}$. 
                    %
                    %On a $\dfrac{\dif z}{\dif t} = v_z$ d'où $z\p{t} = v_0t - \dfrac{g_0t^2}{2} + z\p{0}$. On considère que l'obus est au sol à l'instant $t = 0$, d'où $z\p{0} = 0$.
                    %
                     On a :
                    %
                    \[ E_\text m\p{t} = E_\text c\p{t} + E_\text{pp}\p{t} = \dfrac{1}{2}m\vec{v}\p{t}^2 + mgz\p{t} = m\p{\dfrac{1}{2}v_z\p{t}^2 + gz\p{t}} \]
                    %
                    Puisque la seule force s'exerçant sur le système est conservative (ici le poids), le système est conservatif d'où $E_\text m$ est constant. Dès lors, $E_\text m\p{t} = E_\text m\p{0} = m\p{\dfrac{1}{2}{v_0}^2 + g_0z\p{0}}$.
                }
                %
                \nobefore
                %
                \boxansconc{
                    On considère que l'obus est au sol à l'instant $t = 0$ ($z\p{0} = 0$), donc $E_m\p{t} = \dfrac{1}{2}m{v_0}^2$.
                }
                %
                \boxans{
                    L'altitude maximale de l'obus, atteinte à l'instant $t_\text{max}$, est donnée par $z\p{t_\text{max}} = z_\text{max}$, et se caractérise par un changement de signe de $\dfrac{\dif z}{\dif t}$, \ie $v_z\p{t_\text{max}} = 0$. A cet instant, l'énergie mécanique est donc $E_\text m\p{t} = mg_0z\p{t_\text{max}} = mgz_\text{max}$.
                }
                %
                \yesafter
                %
                \boxansconc{
                    On a donc $z_\text{max} = \dfrac{E_\text m}{mg_0} = \dfrac{{v_0}^2}{2g_0}$. L'application numérique livre $z_\text{max} = \qty{0.50}{\kilo\metre}$.
                }
                \yesbefore
                
                \item\label{qu:I.1.1.c} En quel point et au bout de combien de temps l'obus retombe-t-il ?
                
                \noafter
                %
                \boxans{
                    Le mouvement se faisant uniquement sur l'axe porté par $A_\text z$, les coordonées $x$ et $y$ de l'obus sont constantes au cours du mouvement. L'obus retombe donc exactement au point d'où il a été tiré (\ie sur le canon). D'ailleurs, prendre en compte la résistance de l'air, qui s'oppose au mouvement, ne changerait pas ce résultat. On a $\dfrac{\dif z}{\dif t} = v_z$ donc en intégrant entre $t'= 0$ et $t' = t$ selon $t'$, on obtient $z\p{t} = v_0t - \dfrac{g_0t^2}{2} + z\p{0} =  v_0t - \dfrac{g_0t^2}{2}$. L'obus retombe à l'instant $t_1 > 0$ tel que $z\p{t_1} = 0$, d'où l'on a 
                }
                %
                \nobefore\yesafter
                %
                \boxansconc{
                 $v_0 - \dfrac{g_0t_1}{2} = 0$, ce qui donne $t_1 = 2\dfrac{v_0}{g_0}$. L'application numérique livre $t_1 = \qty{20}{\second}$.
                }
                %
                \yesbefore
            \end{enumerate}

            \item La résistance de l'air sur l'obus, de forme sphérique de
            rayon $r_0 = \qty{5}{\centi\metre}$ et animé d'une vitesse $v$, se
            traduit par une force de module $k\pi {r_0}^2 v^2$. Au voisinage
            des conditions normales, $k = \qty{0.25}{USI}$. L'obus est en plomb
            de masse volumique $\rho = \qty{11.3}{\g\per\centi\metre\cubed}$.

            \begin{enumerate}
                \item Préciser l'unité de $k$.
                
                \noafter
                %
                \boxans{
                    La dimension d'une force est celle d'une masse multiplié par une accélération d'où :
                    %
                    \[ \intc{k\pi{r_0}^2v^2} \equiv \mathsf{M} \cdot \mathsf{L} \cdot \mathsf{T}^{-2} \qquad \text{donc}\qquad \intc{k} \cdot \mathsf{L}^2 \cdot \p{\mathsf{L} \cdot \mathsf{T}^{-1}}^2 \equiv \mathsf{M} \cdot \mathsf{L} \cdot \mathsf{T}^{-2} \qquad\text{donc}\qquad \intc{k} \equiv \mathsf{M} \cdot \mathsf{L}^{-3}\]
                    %
                    
                }
                %
                \nobefore\yesafter
                %
                \boxansconc{
                    Donc $k$ s'exprime dans le système international en $\unit{\kg\per\metre\cubed}$.
                }
                %
                \yesbefore
                
                \item Comparer la force de frottement au poids. Que penser ?
                
                \boxansconc{
                    Le module de la force de frottement est maximal quand la vitesse est maximale, soit avec $v_0$, et l'application numérique livre $\approx \qty{20}{\newton}$. On considère $m \approx \qty{20}{\kg}$, le module du poids est donc $\qty{200}{\newton}$. Lorsque la vitesse est maximale, la force de frottement vaut un dixième du poids, ce qui n'est pas tout à fait négligeable. On devrait donc la prendre en compte.
                }
                
                \item On prend en compte cette force de frottement fluide. On
                pose $u=v^2$. Montrer que, dans la phase ascendante, $u$
                vérifie l'équation :
                %
                \[
                    \dfrac{\dif u}{\dif z} = -2g_0 -2\dfrac{k\pi}{m}{r_0}^2u
                \]
                %
                Expliciter la fonction $z\p{u}$. En déduire l'altitude
                maximale atteinte par l'obus.
                
                On posera $d = \dfrac{m}{2k\pi{r_0}^2}$.
                
                %
                \noafter
                %
                \boxans{
                    On a $u = v^2 = \vec{v}^2 = {v_z}^2$. Or :
                    %
                    \[ \dfrac{\dif u}{\dif z} = \dfrac{\dif u}{\dif v_z}\dfrac{\dif v_z}{\dif z} = 2v_z\dfrac{\dif v_z}{\dif t}\dfrac{\dif t}{\dif z} = 2v_z\dfrac{\dif v_z}{\dif t}\dfrac{1}{v_z} = 2\dfrac{\dif v_z}{\dif t} \]
                    %
                    Le \textit{principe fondamental} amène $m\dfrac{\dif v_z}{\dif t} = -mg_0 - k\pi{r_0}^2 {v_z}^2$, donc $\dfrac{\dif v_z}{\dif t} = -g_0 - \dfrac{k\pi}{m}{r_0}^2u$.
                }
                %
                \nobefore
                %
                \boxansconc{
                    On obtient bien $\dfrac{\dif u}{\dif z} = -2g_0 -2\dfrac{k\pi}{m}{r_0}^2u$.
                }
                %
                \boxans{
                    On sépare alors les variables : \qquad $\dif z = -\dfrac{m}{2}\dfrac{\dif u}{mg_0 + k\pi {r_0}^2 u}$, puis on intègre suivant $u$ :
                    %
                    \[ z\p{u} = -\dfrac{m}{2k\pi{r_0}^2}\ln{mg_0 + k\pi{r_0}^2 u} + \mu = -d\ln{mg_0 + k\pi{r_0}^2u} + \mu \]
                    %
                    Or $z\p{{v_0}^2} = 0 = -d\ln{mg_0 + k\pi{r_0}^2 {v_0}^2} + \mu$, d'où $\mu = d\ln{mg_0 + k\pi{r_0}^2 {v_0}^2}$.
                }
                %
                \yesafter
                %
                \boxansconc{
                    Finalement, $z\p{u} = d\ln{\dfrac{mg_0 + k\pi{r_0}^2 {v_0}^2}{mg_0 + k\pi {r_0}^2 u}}$. On a $z_\text{max} = z\p{0} = d\ln{1 + \dfrac{{v_0}^2}{2dg_0}}$. Pour $d$ grand :
                    %
                    \[ z_\text{max} \approx d\p{\dfrac{{v_0}^2}{2dg_0} - \dfrac{{v_0}^4}{8d^2{g_0}^2}} = \dfrac{{v_0}^2}{2g_0}\p{1 - \dfrac{k\pi{r_0}^2{v_0}^2}{2mg_0}}\]
                    %
                    Lorsque $m$ est très grand, on retrouve $z_\text{max} = \dfrac{{v_0}^2}{2g_0} = \qty{0,50}{\kilo\metre}$. Pour $m \approx \qty{20}{\kg}$, $z_\text{max} = \qty{0,48}{\kilo\metre}$.
                }
                %
                \yesbefore
            \end{enumerate}
        \end{enumerate}
    \end{enumerate}
    %
    Dans la suite du problème, on ne prend pas en compte les frottements de l'air sur l'obus.
    %
    \begin{enumerate}[resume]
        \item On considère que le référentiel lié à la Terre $\p{A, x, y
        , z}$ est non galiléen.

        \begin{enumerate}
            \item Écrire l'équation du mouvement de l'obus. Pourquoi la force
            d'inertie d'entraînement n'intervient-elle pas explicitement dans
            l'équation du mouvement ?
            
            \noafter
            %
            \boxans{
                \begin{minipage}{0.5\linewidth}
                    \centering
                    \begin{tikzpicture}
                        
                    \end{tikzpicture}
                \end{minipage}
                %
                \begin{minipage}{0.5\linewidth}
                    On note $\bcR_2 = \p{A, x, y, z}$ le référentiel lié à la Terre, et $\bcR_1$ le référentiel géocentrique, supposé galiléen. $\bcR_2$ est en rotation uniforme autour de $\bcR_1$ à la vitesse angulaire $\omega_0$. On note $\vec u$ le vecteur unitaire portant l'axe orienté du pôle sud au pôle nord de la terre, ainsi :
                    %
                    \[ \vec u =  \sin{\lambda} \vec{A_z} - \cos{\lambda} \vec{A_x}\]
                \end{minipage}
                
                On a $\vec{\Omega}_{\bcR_2/\bcR_1} = \omega_0 \vec{u}$. On note $\vec v$ la vitesse dans le référentiel $\bcR_2$, $\vec{a_\text c}$ et $\vec{a_\text e}$ les accélérations de \textsc{Coriolis} et d'entraînement de l'obus pour $\bcR_2$ par rapport à $\bcR_1$. On a :
                %
                \[ \vec{a_\text c} = 2\vec{\Omega}_{\bcR_2/\bcR_1} \wedge \vec v \qquad \vec{a_\text e} =  \vec{\Omega}_{\bcR_2/\bcR_1} \wedge \p{\vec{\Omega}_{\bcR_2/\bcR_1} \wedge \vec{AM}} \]
                %
            }
            %
            \nobefore
            %
            \boxansconc{
                Puisque $\vec{\Omega}_{\bcR_2/\bcR_1}$ est constant, $\vec{a_\text e}$ est nulle donc la force d'inertie d'entraînement n'intervient pas.
            }
            %
            \boxans{
                On pose $\vec{F_\text{iC}} = -m\vec{a_\text c}$ la force de \textsc{Coriolis}, ainsi le \textit{principe fondamental de la dynamique} en référentiel non galiléen livre :
                %
                \[ m\dfrac{\dif \vec v}{\dif t} = \vec{P} + \vec{F_\text{iC}} = -mg_0\vec{A_z} - m\vec{a_\text c} \qquad\text{donc}\qquad   \dfrac{\dif \vec v}{\dif t} = - g_0\vec{A_z} - 2\vec{\Omega}_{\bcR_2/\bcR_1} \wedge \vec v \]
                %
                Avec $\vec v = \begin{pNiceMatrix}v_x \\ v_y \\ v_z\end{pNiceMatrix}$, on a $\vec{\Omega}_{\bcR_2/\bcR_1} \wedge \vec v = \begin{pNiceMatrix}-\omega_0\cos \lambda\\0\\ \omega_0\sin \lambda\end{pNiceMatrix} \wedge \begin{pNiceMatrix}v_x \\ v_y \\ v_z\end{pNiceMatrix} = \begin{pNiceMatrix}-\omega_0v_y\sin \lambda\\ \omega_0\p{v_x\sin \lambda  + v_z\cos \lambda} \\ -\omega_0 v_y\cos \lambda \end{pNiceMatrix}$. Finalement :
            }
            %
            \yesafter
            %
            \boxansconc{
                \[ \dfrac{\dif}{\dif t}\begin{pNiceMatrix}v_x \\ v_y \\ v_z\end{pNiceMatrix} = \begin{pNiceMatrix}-2\omega_0v_y\sin \lambda\\ 2\omega_0\p{v_x\sin \lambda  + v_z\cos \lambda} \\ - g_0 - 2\omega_0 v_y\cos \lambda \end{pNiceMatrix} \]
            }
            %
            \yesbefore
            
            \item On évalue la force d'inertie de \textsc{Coriolis} en utilisant la loi de vitesse obtenue à la question \quref{I.1.1.a}.
            Justifier cette méthode de calcul.
            
            \noafter
            %
            \boxans{
                %Le référentiel $\bcR_2$ correspond au référentiel $\p{A, x, y, z}$ utilisé à la question \quref{I.1.1.a}, à ceci près qu'on le considère comme non galiléen
                %
                On note $\vec{v_\text g}$ la vitesse déterminée à la question \quref{I.1.1.a}, on a $\dfrac{\dif \vec{v_\text g}}{\dif t} = -g_0\vec{A_z}$. Ainsi $\dfrac{\dif \vec{v}}{\dif t} = \dfrac{\dif \vec{v_\text g}}{\dif t} -2\vec\Omega_{\bcR_2 / \bcR_1} \wedge \vec v$. En intégrant, on obtient \emph{à une constante près} $\vec v = \vec{v_\text g} + \displaystyle\int \Omega_{\bcR_2 / \bcR_1} \wedge \vec v \dif t$. Pour calculer la force d'inertie de \textsc{Coriolis}, on a donc :
                %
                \begin{align*}
                    \vec\Omega_{\bcR_2 / \bcR_1} \wedge \vec v &= \vec\Omega_{\bcR_2 / \bcR_1} \wedge \p{\vec{v_\text g} - \int \vec\Omega_{\bcR_2 / \bcR_1} \wedge \vec v \dif t} + \vec{cte}\\
                    &= \vec\Omega_{\bcR_2 / \bcR_1} \wedge \vec{v_\text g} - \vec\Omega_{\bcR_2 / \bcR_1} \wedge \p{\vec\Omega_{\bcR_2 / \bcR_1} \wedge \int \vec v \dif t}+ \vec{cte}\\
                    &= \vec\Omega_{\bcR_2 / \bcR_1} \wedge \vec{v_\text g} - \underbrace{\vec\Omega_{\bcR_2 / \bcR_1} \wedge \p{\vec\Omega_{\bcR_2 / \bcR_1} \wedge \int \vec{AM}}}_{\vec{a_\text e} = 0}+ \vec{cte}
                \end{align*}
                %
            }
            %
            \nobefore\yesafter
            %
            \boxansconc{
                On peut donc calculer la force de \textsc{Coriolis} \emph{à une constante près} (ici prise nulle) en utilisant la vitesse $\vec{v_\text g}$.
            }
            %
            \yesbefore
            

            \item\label{qu:I.2.3} Soit $\p{\vec i, \vec j, \vec k}$ la base orthonormée associée au repère $\p{A, x, y, z}$. Le point $M$ repérant la position de l'obus, on pose $\vec{AM} = x\vec i + y\vec j + z\vec k$. Montrer que $\ddot y \approx -2\omega_0\p{v_0 - g_0t}\cos \lambda$.

            En déduire une expression approchée de l'ordonnée $y$ de l'obus. Évaluer $y$ au moment où l'obus tombe sur le sol. La déviation se fait-elle vers l'ouest ou vers l'est ? Le résultat dépend-il de l'hémisphère dans lequel on effectue le tir ?
            
            \noafter
            %
            \boxans{
                Avec l'expression précédente, le calcul de $\vec{v}$ se fait plus facilement :
                %
                \[ \dfrac{\dif}{\dif t} \begin{pNiceMatrix}
                    v_x\\
                    v_y\\
                    v_z
                \end{pNiceMatrix} \approx -g_0\vec{A_z} - 2\vec{\Omega}_{\bcR_2 / \bcR_1} \wedge \vec{v_\text g} = \begin{pNiceMatrix}0\\0\\-g_0\end{pNiceMatrix} - 2\begin{pNiceMatrix}-\omega_0\cos \lambda\\0\\\omega_0\sin \lambda\end{pNiceMatrix} \wedge \begin{pNiceMatrix}0\\0\\v_0 - g_0 t\end{pNiceMatrix} = \begin{pNiceMatrix}0\\-2\omega_0\p{v_0 - g_0t}\cos \lambda\\-g_0\end{pNiceMatrix}\]
                %
            }
            %
            \nobefore
            %
            \boxansconc{
                Puisque $\dot y = v_y$, on a bien $\ddot y = \dfrac{\dif v_y}{\dif t} \approx -2\omega_0\p{v_0 - g_0t}\cos \lambda$.
            }
            %
            \boxans{
                En intégrant deux fois selon $t$, et avec les conditions initiales $v_y\p{0} = 0$ et $y\p{0} = 0$, on obtient que $y\p{t} \approx -\omega_0\p{v_0 - \dfrac{g_0t}{3}}t^2\cos \lambda$. L'expression de $z$ n'est pas changé par rapport à la question \quref{I.1.1.c} donc $z\p{t_1} = 0$ avec $t_1 = \dfrac{2v_0}{g_0}$. Soit $\Delta_y$ la déviation que l'on cherche à calculer, orientée dans le sens de $\vec{A_y} = \vec{j}$, \ie vers l'est (puisque $\vec{A_x}$ est vers le sud et $\vec{A_z}$ vers le \guill{haut}). On a $\Delta_y = y\p{t_1} = -\omega_0v_0\p{1 - \dfrac{2}{3}}\p{\dfrac{2v_0}{g_0}}^2\cos \lambda$.
            }
            %
            \yesafter
            %
            \boxansconc{
                On a donc une déviation $\Delta_y = - \dfrac{4\omega_0{v_0}^3}{3{g_0}^2}\cos \lambda$. Puisque $\lambda > 0$, on a $\cos \lambda > 0$ donc $\Delta_y < 0$, ainsi on a une déviation vers l'ouest. Dans l'hémisphère sud, on aurait $\lambda < 0$ donc $\cos \lambda < 0$, ainsi la déviation serait dans l'autre sens (vers l'est).
            }
            %
            \yesbefore

            \item L'expression de la force de \textsc{Coriolis} utilisée ne permet pas de mettre en évidence une déviation dans l'axe nord-sud, or cette déviation existe. Expliquer cette apparente contradiction. Que penser de cette déviation comparée à celle calculée à la question précédente ?
            
            \boxansconc{
                On a fait l'approximation que la force d'inertie de \textsc{Coriolis} était calculable uniquement à partir de $\vec{v_g}$, \ie que la constante était nulle, ce qui n'est pas forcément le cas. Cette déviation cependant est bien plus faible que celle calculée à la question précédente.
            }
            
        \end{enumerate}

        \item Analyse du mouvement de l'obus dans le référentiel géocentrique.

        \begin{enumerate}
            \item 
            \begin{enumerate}
                \item Rappeler la définition du référentiel géocentrique. Pourquoi peut-il être supposé galiléen pour ce type d'expérience ? 
                
                \boxansconc{
                    Le référentiel géocentrique est le référentiel d'origine le centre de la terre et dont les axes pointent vers 3 étoiles lointaines, (les mêmes que dans le référentiel héliocentrique). Sur le temps et la distance caractéristique de l'expérience (quelques secondes, quelques kilomètres), le mouvement de la terre autour du soleil peut être assimilé à un mouvement rectiligne uniforme. Le référentiel héliocentrique étant supposé galiléen, le référentiel géocentrique l'est également ici. 
                }

                \item Déterminer la vitesse initiale de l'obus dans ce référentiel.
                
                \boxansconc{
                    La vitesse initiale $v_\text{géo}$ de l'obus dans sa référentiel est sa vitesse initiale $v_0$ sur l'axe porté par $\vec{A_z}$, additionné à sa vitesse donnée par la rotation de la terre, soit $\omega_0 r_\text t$, où $r_\text t$ est la rotation de la terre, suivant $\vec{A_y}$. Ainsi $v_\text{géo} = v_0\vec{A_z} + \omega_0 r_\text t\vec{A_y}$.
                }
            \end{enumerate}
            
            \item Si on considère le champ de gravitation uniforme sur la trajectoire de l'obus, montrer alors que celui-ci retombe en $A$. Comment l'observateur géocentrique peut-il justifier la déviation évaluée au \quref{I.2.3} ?
            
            
            \boxans{
                ?
            }
        \end{enumerate}
    \end{enumerate}
    
\end{document}