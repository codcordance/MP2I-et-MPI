\documentclass[a4paper,french,bookmarks]{article}

\usepackage{./Structure/4PE18TEXTB}

\usepackage{graphicx}
\usepackage{listings}

% Définir les couleurs pour les mots clés
\definecolor{keywords}{RGB}{255,0,90}

% Définir les mots clés
\lstset{
    language=C,
    keywordstyle=\color{keywords}\bfseries,
    morekeywords={Données,graphe,deja_vu,G,listes,adjacences,sommet,départ,s,a_traiter,implémentée,file,deja_vu,dist,pred,tant,faire,retourner,pour,tout,voisin,tel,que,est}
}


\newboxans
\usepackage{booktabs}

\begin{document}

    \renewcommand{\thesection}{\Roman{section}} 
    \renewcommand{\thesubsection}{\thesection.\Alph{subsection}}
    \setlist[enumerate]{font=\color{white5!60!black}\bfseries\sffamily}
    \renewcommand{\labelenumi}{\thesection.\arabic{enumi}.}
    \renewcommand*{\labelenumii}{\alph{enumii}.}
    \renewcommand*{\labelenumiii}{\alph{enumiii}.}
    
    \def\authorvar{DRISSI Rayan}
    \stylizeDocSpe{Info}{Chapitre 6}{}{Algorithme d'approximation}
    \section{Intro}
    
    Comme on l'a vu avec l'étude de problèmes NP-Complet il existe des problèmes qu'on ne sait a priori pas résoudre ou temps polynomial. 

    On a notamment vu des problèmes d'optimisation dont les problèmes de décision de décision assenée (avec un seuil sont NP-complet)

    \section{Algoritme d'approxiamtion}

    \begin{definition}{$\alpha$-Approxiamation}{}
        On dis qu'un al 
    \end{definition}


    \begin{property}{}{}
        Supposons que chaque clause de $\ph$ contient au moins 
        $k$ littéraux. Alors
        %
        \[\bdE\p{\varphi} \geq c\p{1 - \dfrac{1}{2^k}} \]

        Alors $\bcE(\phi) \geq c(1 - 1/2^k) $
    \end{property}
    
    \begin{nproof}
        Soit $\psi$ une clause de $\varphi$ à $k' \geq k$ littéraux. On a 
        %
        \[ \bdE\p{\psi} = 1 - \dfrac{1}{2^{k'}} \geq 1 - \dfrac{1}{2^k}\]
        %
        Donc $\bdE\p{\varphi} = \displaystyle\sum_{\psi \in \varphi} \bdE\p{\psi} \geq \sum_{\psi \in \varphi} \p{1 - \dfrac{1}{2^k}} = c\p{1 - \dfrac{1}{2^k}}$
    \end{nproof}
    
    \begin{corollary}{}{}
        Avec $k = 1$, on a $\bdE\p{\varphi} \geq \dfrac{c}{2}$.
    \end{corollary}

    \subsection{Derandomisation methode de l'esperance conditionnelel}
    
   \begin{form}{Principe de l'algorithme}
   
    Voici le principe de l'algo

    \begin{enumerate}
        \itt on affecte a tour de role une valeur de verité a chacune des n variables 
        $x_1, ..., x_n$ de $\phi$

        \itt Pour choisir $\v(x_i)$ en connaissa
        
    \end{enumerate}
   
   \end{form} 
    

    
    % ????
         
\end{document}
