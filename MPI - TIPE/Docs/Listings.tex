\documentclass[a4paper,french,bookmarks]{article}

\usepackage[
    top         = 1in,
    bottom      = 1in,
    inner       = 1.5in,
    outer       = 1in,
    headheight  = 16pt,
    headsep     = 0.4in,
    footskip    = 0.4in,
    includeheadfoot,
    heightrounded,
    twoside,
    %showframe,
    ]{geometry}
\usepackage{booktabs}
\usepackage{minitoc}
\usepackage{./Structure/4PE18TEXTBnogeom}
\usepackage{proof}
\usepackage{pdfpages}
\usepackage[skip=10pt plus1pt,indent=40pt]{parskip}
\usepackage{blindtext}
\usepackage{biblatex}
\usepackage{svg}

\addbibresource{refs.bib}

\makeatletter
\renewcommand\tableofcontents{%
    \@starttoc{toc}%
}
%\renewcommand*\l@section{\@dottedtocline{1}{1em}{3em}}
\renewcommand*\l@subsection{\@dottedtocline{2}{2em}{3em}}
\makeatother

\newboxans
\renewcommand{\thesection}{\Roman{section}}
\renewcommand{\thesubsection}{\thesection.\arabic{subsection}}
\mtcsettitle{minitoc}{}

\DeclareDocumentCommand\Sp{g}{\funlv{Sp}{#1}}

\lstnewenvironment{outputlog}{
    \lstset{
        tabsize=2,
        breaklines,
        basicstyle=\footnotesize\ttfamily,
        frame=leftline
    }
}{}

\begin{document}
    
    %==============================
    % METADONNEES
    %==============================
    
    \title{}
    \author{SIAHAAN--GENSOLLEN Rémy}
    \date{\today}
    \hypersetup{
        pdftitle={Programmes informatiques du TIPE : Durcissement des villes modernes face aux rayonnements ionisants},
        pdfauthor={SIAHAAN--GENSOLLEN Rémy},
        pdflang={fr-FR},
        pdfsubject={TIPE, Durcissement des villes modernes face aux rayonnements ionisants},
        pdfkeywords={TIPE, 2022-2023}
        pdfstartview=
    }
    
    %==============================
    % MISE EN PAGE
    %==============================

    %top         = 1.5in,
    %bottom      = 1.5in,
    %inner       = 1.5in,
    %outer       = 1in,
    %headheight  = 16pt,
    %headsep     = 0.3in,
    %footskip    = 0.3in,
    %includeheadfoot,
    %heightrounded,
    %twoside
    
    %==============================
    % STYLE DES EN-TÊTES ET PIEDS DE PAGES
    %==============================
    
    \fancypagestyle{plain}{
        \fancyhf{}
        \renewcommand{\headrulewidth}{0pt}
        \renewcommand{\footrulewidth}{0pt}
        \fancyfoot[RO,LE]{\sffamily\color{white5}\thepage~/~\pageref{LastPage}}
        %\fancyhead[LE]{\sffamily\color{white5}\bfseries SIAHAAN--GENSOLLEN Rémy}
        \fancyhead[LE]{\sffamily\color{white5}Programmes informatiques du TIPE}
        %\fancyhead[LO]{\sffamily\color{white5}\nouppercase{\rightmark}}
        \fancyhead[RO]{\sffamily\color{white5}Durcissement des villes modernes face aux rayonnements ionisants}
    }

    \pagestyle{plain}

    %==============================
    % CONTENU
    %==============================
    
    \begin{tcolorbox}[
            enhanced,
            frame hidden,
            sharp corners,
            spread upwards      = 0.1in,
            halign              = center,
            valign              = center,
            interior style      = {color=main3!20},
            arc                 = 0in,
            outer arc           = 0pt,
            leftrule            = 0pt,
            rightrule           = 0pt,
            fontupper           = \color{black},
            %width               = \paperwidth, 
            top                 = 0.4in, 
            bottom              = 0.3in
        ]
            {\large{\scshape{SIAHAAN--GENSOLLEN}} Rémy\par}
            \vspace{0.2in}
            {Candidat $n^\circ\ 15930$\par}
    	\vspace{0.05in}
            {\Huge\bfseries\sffamily Programmes informatiques du TIPE\par}
    \end{tcolorbox}

    \bigskip

    \begin{tcolorbox}[
        enhanced,
        frame hidden,
        sharp corners,
        detach title,
        spread outwards,
        halign              = center,
            valign              = center,
        borderline west     = {3pt}{0pt}{main3},
        coltitle            = main3, 
        interior style      = {
            left color      = main1white2!65!gray!11,
            middle color    = main1white2!50!gray!10,
            right color     = main1white2!35!gray!9
        },
        arc                 = 0 cm,
        title               = SOMMAIRE,
        boxrule             = 0pt,
        fonttitle           = \bfseries\sffamily,
        overlay             = {
            \node[rotate=90, minimum width=1cm, anchor=south,yshift=-0.8cm]
            at (frame.west) {\tcbtitle};
        },
    ]
        \begin{minipage}{0.83\linewidth}
            \sffamily
            \tableofcontents
        \end{minipage}
    \end{tcolorbox}

    \bigskip
    
    \section{Code de \textsc{Hamming}}

    \subsection{Matrice $\bbG$}

        \begin{C}
const uint8_t g[7] =
        {
                0b1000,
                0b0100,
                0b0010,
                0b0001,
                0b1101,
                0b1011,
                0b0111
        };
    \end{C}

    \subsection{Matrice $\bbH$}
    
    \begin{C}
const uint8_t h[3] =
        {
            0b1101100,
            0b1011010,
            0b0111001
        };
    \end{C}

    \subsection{Sommation des composantes dans $\p{\bdZ/2\bdZ}^8$}
    
    \begin{C}
uint8_t sumBits(uint8_t b)
{
    uint8_t s = 0;
    for (uint8_t m = 1; m < 0b10000000; m <<= 1) {
        if (b & m) s++;
    }
    return s & 0b0000001;
}
    \end{C}

    \subsection{Multiplication par $\bbG$}
    \begin{C}
uint8_t hammingG(uint8_t b)
{
    uint8_t e = 0;
    for (uint8_t i = 0; i < 7; i++) {
        e <<= 1;
        if (sumBits(g[i] & b)) e |= 1;
    }
    return e;
}
    \end{C}

    \subsection{Multiplication par $\bbH$}
    
    \begin{C}
uint8_t hammingH(uint8_t e)
{
    uint8_t s = 0;
    for (uint8_t i = 0; i < 3; i++)
    {
        s <<= 1;
        if (sumBits(h[i] & e)) s |= 1;
    }
    return s;
}
    \end{C}


\newpage     \subsection{Correspondance avec $S$ et décodage avec correction d'une erreur}
    \begin{C}
const uint8_t sCor[8] =
        {
                0 << 0,
                1 << 0,
                1 << 1,
                1 << 4,
                1 << 2,
                1 << 5,
                1 << 6,
                1 << 3
        };

uint8_t hammingFix(uint8_t e) {
    return ((e ^ sCor[hammingH(e)]) & 0b1111000)
    >> 3;
}

    \end{C}


    \subsection{Test de performance}

    \begin{C}
int main() {
    uint8_t code = 0b1011;
    uint8_t pert = 1 << 4;
    int size = 10*1000*1000;
    time_t start = time(NULL);
    for (int i = 0; i < size; i++)
        if (hammingFix(hammingG(code) ^ pert)
        != code) {
            printf("Erreur d'éxécution !\n");
            return 1;
    }
    unsigned long d = difftime(time(NULL), start);
    printf("Exécution réussie en %ld s\n", d);
    return 0;
}
    \end{C}

    \medskip
    
    \begin{outputlog}
Exécution réussie en 1 s
    \end{outputlog}

\newpage 
\section{RAID-6}

    \subsection{Calcul des puissances de $2$}

    \begin{C}
uint8_t tpow2(uint8_t u, int k)
{
    return (k >= 0) ? (u << k) : (u >> -k);
}
    \end{C}

    \subsection{Calcul de $p$ et $q$}

    \begin{C}
uint8_t computeP(const uint8_t *u, int n)
{
    uint8_t p = 0;
    for (int i = 0; i < n; i++) p ^= u[i];
    return p;
}

uint8_t computeQ(const uint8_t *u, int n)
{
    uint8_t q = 0;
    for (int i = 0; i < n; i++)
        q ^= tpow2(u[i], i);
    return q;
}
    \end{C}

    \subsection{Calcul de $u_k$ depuis $p$}

    \begin{C}
uint8_t computeUkFromP(uint8_t *u, int n,
    int k, int p)
{
    uint8_t ptilde = 0;
    for (int i = 0; i < n; i++)
        ptilde ^= (i != k) ? u[i] : 0;
    u[k] = p ^ ptilde;
    return u[k];
}
    \end{C}

 \newpage
     \subsection{Calcul de $u_k$ depuis $q$}

    \begin{C}
uint8_t computeUkFromQ(uint8_t *u, int n,
    int k, int q)
{
    uint8_t qtilde = 0;
    for (int i = 0; i < n; i++)
        qtilde ^= (i != k) ? tpow2(u[i], i) : 0;
    u[k] = tpow2(q ^ qtilde, -k);
    return u[k];
}
    \end{C}

    \subsection{Calcul de $u_k$ et $u_\ell$ depuis $p$ et $q$}

    \begin{C}
void computeUkUl(uint8_t *u, int n, int k,
    int l, int p, int q)
{
    uint8_t ptilde = 0;
    uint8_t qtilde = 0;
    for (int i = 0; i < n; i++) {
        if (i != k && i != l) {
            ptilde ^= u[i];
            qtilde ^= tpow2(u[i], i);
        }
    }
    u[l] = (tpow2(q ^ qtilde, -k) ^
        tpow2(p ^ ptilde, k - l ))
        / (tpow2(1, l - k) ^ 1);
    u[k] = (p ^ ptilde) ^ u[l];
}
    \end{C}

    \newpage

    \subsection{Test de performance}
    
    \begin{C}
int main()
{
    uint8_t* m = malloc(4 * sizeof(uint8_t));
    m[0] = 0b01;
    m[1] = 0b10;
    m[2] = 0b11;
    m[3] = 0b10;
    int size = 10*1000*1000;
    time_t start = time(NULL);
    for (int i = 0; i < size; i++) {
        uint8_t p = computeP(m, 4);
        uint8_t q = computeQ(m, 4);
        if (m[2] != computeUkFromP(m, 4, 2, p) ||
        m[2] != computeUkFromQ(m, 4, 2, q))
            return 1;
        computeUkUl(m, 4, 2, 3, p, q);
        if (m[2] != 0b11 || m[3] != 0b10)
            return 2;
    }
    unsigned long d = difftime(time(NULL), start);
    printf("Exécution réussie en %ld s", d);
    free(m);
    return 0;
}
    \end{C}

    \medskip 
    \begin{outputlog}
        Exécution réussie en 2 s
    \end{outputlog}


    \end{document}