\documentclass[a4paper,french,bookmarks]{article}

\usepackage{../../Structure/4PE18TEXTB}

\newboxans
\usepackage{booktabs}

\begin{document}

    \renewcommand{\thesection}{\Roman{section}}
    \setlist[enumerate]{font=\color{white5!60!black}\bfseries\sffamily}
    \renewcommand{\thesection}{\Roman{section}}
    \renewcommand{\labelenumi}{\Roman{section}.\arabic{enumi}.}
    \renewcommand*{\labelenumii}{\alph{enumii}.}

    \stylizeDocSpe{Maths}{Brouillon}
    {Notes variées}{En cours de maths et ailleurs}
    
    \begin{definition}{Module de continuité}{}
        On appelle \hg{module de continuité} la quantité \hg{$\omega\p{\delta} = \sup\limits_{\norm{x - y} \leq \delta} \norm{f\p{x} - f\p{y}}$}.
    \end{definition}
    %
    On notera que $\omega$ est bien défini si $f$ est borné, que $\omega$ est croissante, et qu'on peut donc obtenir :
    %
    \[ \norm{f\p{x} - f\p{y}} \leq \omega \p{\norm{x - y}}\]
    %
    Enfin, $f$ est uniformément continue si $\omega \lima{\delta \to 0} 0$. Montrons ce dernier point :
    %
    \boxansconc{
        Montrons que $f$ est uniformément continue. Soit $\epsilon > 0$, on peut trouver $\delta$ tel que $\omega\p{\delta} \leq \epsilon$ (puisque $\omega\p{\delta} \lima{\delta \to 0} 0$).
        
        Dès lors :
        %
        \[ \forall \epsilon \in \bdRp,\qquad \exists \delta \in \bdRp,\qquad \norm{x - y}\leq \delta \implies \norm{f\p{x} - f\p{y}} \leq \omega\p{\norm{x - y}} \leq \epsilon\]
    }
    
    Sommes de \textsc{Riemann}
    
    \newpage
    
    \section{Homéomorphisme et difféomorphisme}
    
    \begin{definition}{Homéomorphisme}{}
        $f : A \subset E \rightarrow B \subset F$ est un homéomorphisme lorsque :
        %
        \[ f \in \bcC^0\p{A, B} \qquad f \ \text{bijective} \qquad f^{-1} \in \bcC^0{B, A}\]
    \end{definition}
    
    \begin{definition}{Difféomorphisme}{}
        $f : A \subset E \rightarrow B \subset F$ est un $\bcC^k$-difféomorphisme lorsque :
        %
        \[ f \in \bcC^k\p{A, B} \qquad f \ \text{bijective} \qquad f^{-1} \in \bcC^k{B, A}\]
    \end{definition}
    
    \begin{property}{CNS de difféomorphisme}{}
        $f$ définit un $\bcC^k$-difféomorphisme de $I \rightarrow f\p{I}$ ssi $f \in \bcC^k$ et $f'$ ne s'annule pas
    \end{property}
    
    \newpage
    
    Soit $\p{x_0, y_0}$ l'unique solution de $\p{\bsS_1}$, telle que $x_0\p{t_0} = \widetilde{x_0} > 0$ et $y_0\p{t_0} = \widetilde{y_0} > 0$. Supposons qu'il existe $t_1$ tel que $x\p{t_1} = 0 = \widetilde{x_1}$, et posons $\widetilde{y_1} = y\p{t_1}$. Notons alors $\p{x_1, y_1}$ l'unique couple solution de $\p{\bsS_1}$ avec $t_1$, $\widetilde{x_1}$ et $\widetilde{y_1}$. On a par la question \quref{1.(a)} :
    %
    \[ \forall t \in \bdR,\qquad x_1\p{t} = 0 \qquad\et\qquad y_1\p{t} = \widetilde{y_1}e^{-d\p{t - t_1}}\]
    %
    Or $x_0\p{t_1} = \widetilde{x_1}$ et $y_0\p{t_1} = \widetilde{y_1}$, et de puis $x_0$ et $y_0$ vérifient les équations 
    %
    \[ x'\p{t} = \p{a - by\p{t}}x\p{t} \qquad\et\qquad y'\p{t} = \p{cx\p{t} - d}y\p{t}\]
    %
    Donc $x_0$ et $y_0$ vérifient $\p{\bsS_1}$ avec $t_1$, $\widetilde{x_1}$ et $\widetilde{y_1}$. Par unicité, on a donc $x_0 = x_1$ et $y_0 = y_1$.
    
    On en déduit $\widetilde{x_0} = x_0\p{t_0} = x_1\p{t_0} = 0$ ce qui est absurde.
    
    \newpage
    
    Soit $f$ la fonction définie sur $\intc{0, 1}$ par $f\p{x} = \ln{x + \sqrt{1 + x^2}}$.
    
    \begin{enumerate}
        \itt On vérifie aisément que $f$ est strictement croissante de $\intc{0, 1}$ dans $\intc{0, \ln{1 + \sqrt{2}}}$. Elle est donc bijective, et on note $g : \intc{0, \ln{1 + \sqrt{2}}} \to \intc{0, 1}$ sa réciproque.
        
        \itt En posant le changement de variable $x = f\p{t}$, d'où $\dif x = \dif f\p{t} = f'\p{t}\dif t$, on obtient 
        %
        \[ \boxed{\int_0^{\ln{1 + \sqrt{2}}} g\p{x}\dif x = \int_0^1 tf'\p{t}\dif t} \]
        %
        En intégrant par partie, on obtient :
        %
        \[ \int_0^{\ln{1 + \sqrt{2}}} g\p{x}\dif x = \intc{tf\p{t}}_0^1 - \int_0^1 f\p{t}\dif t \qquad\text{d'où}\qquad \boxed{\int_0^1 f\p{t}\dif t = \ln{1 + \sqrt{2}} - \int_0^{\ln{1 + \sqrt{2}}} g\p{t}\dif t}\]
        
        \itt Exprimons $f'$ sur $\intc{0, 1}$ :
        %
        \[ f'\p{x} = \dfrac{1 + \dfrac{2x}{2\sqrt{1 + x^2}}}{x + \sqrt{1 + x^2}} = \dfrac{\dfrac{\sqrt{1 + x^2}}{\sqrt{1 + x^2}} + \dfrac{x}{\sqrt{1 + x^2}}}{x + \sqrt{1 + x^2}} = \dfrac{\dfrac{x + \sqrt{1 + x^2}}{\sqrt{1 + x^2}}}{x + \sqrt{1 + x^2}} = \dfrac{1}{\sqrt{1 + x^2}}\]
        %
        On en déduit par la formule précédente que 
        %
        \[ \int_0^{\ln{1 + \sqrt{2}}} g\p{x}\dif x = \int_0^1\dfrac{t}{\sqrt{1 + x^2}}\dif t = \intc{\sqrt{x^2 + 1}}_0^1 = \sqrt{2} - 1\]
        %
        Finalement, on obtient donc $\boxed{\displaystyle \int_0^1 f\p{t}\dif t = \ln{1 + \sqrt{2}} + 1 - \sqrt{2}}$.
    \end{enumerate}
    
    
    

\end{document}