\documentclass[a3paper,french,bookmarks]{article}

\usepackage{./Structure/4PE18TEXTB}

\usepackage{graphicx}
\usepackage{listings}

% Définir les couleurs pour les mots clés
\definecolor{keywords}{RGB}{255,0,90}

% Définir les mots clés
\lstset{
    language=C,
    keywordstyle=\color{keywords}\bfseries,
    morekeywords={Données,graphe,deja_vu,G,listes,adjacences,sommet,départ,s,a_traiter,implémentée,file,deja_vu,dist,pred,tant,faire,retourner,pour,tout,voisin,tel,que,est}
}


\newboxans
\usepackage{booktabs}

\begin{document}

    \renewcommand{\thesection}{\Roman{section}} 
    \renewcommand{\thesubsection}{\thesection.\Alph{subsection}}
    \setlist[enumerate]{font=\color{white5!60!black}\bfseries\sffamily}
    \renewcommand{\labelenumi}{\thesection.\arabic{enumi}.}
    \renewcommand*{\labelenumii}{\alph{enumii}.}
    \renewcommand*{\labelenumiii}{\alph{enumiii}.}
    
    \def\authorvar{DRISSI Rayan}
    \stylizeDocSpe{Maths}{Produit Scalaire}{}{Espaces Prehilbertiens}

    
    
    \chaptertoc{}
    \section{Espaces préhilbertiens}
    
    \subsection{Définition}
    
    Soit $E$ un $\bdR$-espace vectoriel de (dimension finie ou non). On appelle \emph{produit scalaire sur $E$} une forme bilinéaire  $\p{\cdot \mid \cdot} : E \times E \to \bdR$ vérifiant :
    %
    \begin{enumerate}
        \itast Pour tout $\p{x, y} \in E^2$, on a $\p{x \mid y} = \p{y \mid x}$ \qquad\emph{(symétrie)}
        
        \itast Pour tout $x \in E$, on a $\p{x \mid x} \geq 0$ \qquad\emph{(positivité)}
    
        \itast Pour tout $x \in E$,\qqua$x\notequal 0 \implies \p{x \mid x} > 0$  
    \end{enumerate}
    
    \subsection{Exemple}
    
    Sur $R^\bcN: X=(x_{j})_j \in \bcR^{\bcN}$

    
    $(X|Y) = \sum_{i} x_i y_i = \mtrans{X}Y$ p.s canonique
    \\
    
    - Bilin \\
    - Sym \\
    - $X \notequal 0 \implies (X|X) > 0 = \sum x_i^2 $
    \\
    Sur E de dim finie. $\qquad \qquad \bcB = (e_i)_i$ base 
    \\
    $x = \sum x_i e_i \qquad y= \sum y_i e_i$
    \\
    $(x|y) = \mtrans{X}Y = \sum x_i y_i \qquad$         ps pour lequel $\bcB $ est ON
    
    
    \vspace{2cm}
    
    
    Sur $E = \ens{f \quad \bcC^0 \text{sur [}0;\infty[ \text{ tq }x \to f^2(x)e^{-x}\quad L^1 \text{ sur } \bcR_+ }$
    
    \begin{enumerate}
        \itast E ev 
        \itast (|)  produit scalaire sur E 
    \end{enumerate}
    
    \textasteriskcentered Intervale $\pi:I \to \bcR_+ \qquad C$ 
    
    \subsection{Norme associé}
    
    
    
    
    
\end{document}
 