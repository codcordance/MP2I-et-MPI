\documentclass[a4paper,french,bookmarks]{article}

\usepackage{../../Structure/4PE18TEXTB}

\newboxans
\usepackage{booktabs}

\begin{document}

    \renewcommand{\thesection}{\Roman{section}}
    \setlist[enumerate]{font=\color{white5!60!black}\bfseries\sffamily}
    \renewcommand{\thesection}{\Roman{section}}
    \renewcommand{\labelenumi}{\Roman{section}.\arabic{enumi}.}
    \renewcommand*{\labelenumii}{\alph{enumii}.}

    \stylizeDocSpe{Maths}{Devoir maison $\star$ n° 3}
    {CMP MP MATH 2 1997}{Pour le vendredi 14 octobre 2022}
    
    \begin{enumerate}
        \itt Dans tout ce problème, $n$ est un entier au moins égal à $1$. On note $\bcM_{n, p}\p{\bdC}$ l'espace vectoriel des matrices à $n$ lignes et $p$ colonnes, à coefficients complexes.
    
        \itt On identifiera une matrice colonne $X$ (un élément de $\bcM_{n, 1}\p{\bdC}$) et le vecteur de $\bdC^n$ dont les composantes dans la base canonique de $\bdC^n$ sont les coefficients de la matrice $X$. Pour $M \in \bcM_{n, n}\p{\bdC}$, on note $\overline M$ l'endomorphisme canoniquement associé de $\bdC^n$ : $\overline M$ est l'endomorphisme de $\bdC^n$ dont $M$ est la matrice dans la base canonique de $\bdC^n$. Par ailleurs, $E_\lambda\p{\overline M}$ est l'espace propre associé à la valeur propre $\lambda$ de l'endomorphisme $\overline M$.
    
        \itt Pour une matrice $M \in \bcM_{n, n}\p{\bdC}$ de coefficients $\p{m_{i, j}}_{i, j \in \iint{1, n}^2}$ et pour $k \in \iint{0, n-1}$, on appelle $k$-ième diagonale supérieure de $M$, notée $D_k\p{M}$, l'ensemble des coefficients $\p{m_{i, i+k}}_{i \in \iint{1, n-k}}$. Une diagonale supérieure $D_k\p{M}$ est dite nulle lorsque tous ses éléments sont nuls.
    
        \itt Si $V$ et $W$ sont deux espaces supplémentaires de $\bdC^n$, on note $p_V$ la projection sur $V$ parallèlement à $W$ : pour $x = x_V +x_W$ avec $x_V \in V$ et $x_W \in W$, on a $p_V\p{x} = x_V$. Pour un endomorphisme $u$ de $\bdC^n$, on note $u_V$ sa restriction à $V$, de sorte que si $i_V$ représente l'injection de $V$ dans $\bdC^n$, $u_V\p{y} = u\p{i_V\p{y}}$.
    \end{enumerate}
    
    \section{Algèbres de Lie}
    
    On appelle crochet de \textsc{Lie} de deux éléments $X$ et $Y$ de $\bcM_{n, n}\p{\bdC}$ la matrice, notée $\intc{X, Y}$, définie par :
    %
    \[ \intc{X, Y} = XY - YX\]
    %
    \text{}\\[-25pt]
    %
    \begin{definition}{}{}
        Soit \hg{$\bcU$} un \hg{sous-espace vectoriel de $\bcM_{n, n}\p{\bdC}$}. On note \hg{$\intc{\bcU}$} l'\hg{espace vectoriel engendré par les crochets de \textsc{Lie} $\intc{X, Y}$} lorsque \hg{$X$ et $Y$ décrivent $\bcU$}. On dit que \hg{$\bcU$} est une \hg{algèbre de \textsc{Lie}} lorsque
        %
        \[ \hg{\intc{\bcU} \subset \bcU} \]
    \end{definition}
    %
    Soient $\bcU$ et $\bcV$ deux algèbres de \textsc{Lie} qui vérifient $\intc{\bcU} \subset \bcV \subset \bcU$. On souhaite prouver le théorème suivant.
    %
    \begin{theorem}{}{th1}
        Si \hg{$X \in \bcM_{n, 1}\p{\bdC}$} est une \hg{colonne propre pour toute matrice $M$ dans $\bcV$} et si \hg{$A$} est une \hg{matrice dans $\bcU$} alors \hg{$AX$} est \hg{soit la matrice colonne nulle}, \hg{soit une matrice colonne propre pour toute matrice $M$ dans $\bcV$}. De plus, si pour \hg{$M \in \bcV$}, \hg{$MX = \lambda X$}, alors \hg{$M\p{AX} = \lambda\p{AX}$}.
    \end{theorem}
    %
    Soit $X \in \bcM_{n, 1}\p{\bdC}$ une matrice colonne propre pour toute matrice $M$ dans $\bcV$, et soit $A$ une matrice de $\bcU$.
    
    \begin{enumerate}
        \item Établir l'existence d'une forme linéaire $\lambda$ sur $\bcV$, à valeurs dans $\bdC$, telle que $\forall M \in \bcV$, $MX = \lambda\p{M}X$.
        %
        \noafter
        %
        \boxans{
            $X$ est un vecteur propre pour toute matrice $M$ dans $\bcV$, ainsi
            %
            \[ \forall M \in \bcV,\qquad \exists \lambda_M \in \bdC,\qquad MX = \lambda_M X \qquad\qquad\p{\lambda_M \ \text{est la valeur propre de} \ M \ \text{associée à} \ X}\]
            %
            On définit alors l'application $\lambda : \begin{array}[t]{rcl}
                \bcV &\to& \bdC  \\
                M &\mapsto& \lambda_M 
            \end{array}$ ainsi pour tout $M \in \bcV$, on a $MX = \lambda\p{M}X$. Vérifions que $\lambda$ est linéaire.
            %
            Soient $\alpha \in \bdC$ et $\p{M, M'} \in \bcV^2$. Puisque $\bcV$ est un $\bdC$-espace vectoriel on a $\alpha M \in \bcV$. et $M + M' \in \bcV$. De plus :
            %
            \[ \p{\alpha M + M'}X = \lambda_{\alpha M + M'}X \qquad\et\qquad \p{\alpha M + M'}X = \alpha\p{MX} + M'X = \alpha\lambda_MX + \lambda{M'}X = \p{\alpha \lambda_M + \lambda_{M'}}X \]
            %
            Donc $\lambda\p{\alpha M + M'} = \alpha\lambda{M} + \lambda{M'}$.
        }
        %
        \nobefore\yesafter
        %
        \boxansconc{
            Donc il existe bien une forme linéaire $\lambda$ sur $\bcV$ vérifiant que pour tout $M \in \bcV$, $MX = \lambda\p{M}X$.
        }
        %
        \yesbefore
        
        \item Montrer que pour tout $M \in \bcV$, $\intc{M, A}$ appartient à $\bcV$.
        
        \boxansconc{
            On a $M \in \bcV$ et $\bcV \subset \bcU$ donc $M \in \bcU$. De plus $A \in \bcU$ donc par définition $\intc{M, A} \in \intc{\bcU}$. Or $\intc{\bcU} \subset \bcV$ donc  $\intc{M, A} \in \bcV$.
        }
    \end{enumerate}
    %
    On considère la suite de matrices colonnes $\p{X_k}_{k \in \bdN}$ définie par
    %
    \[ X_0 = X,\qquad\qquad \forall k \in \bdN,\qquad X_{k+1} = AX_k\]
    %
    Pour $M \in \bcV$, on considère la suite de nombres complexes $\p{\lambda_k\p{M}}_{k \in \bdN}$ définie par
    %
    \[ \lambda_0\p{M} = \lambda\p{M},\qquad\qquad \forall k \in \bdN,\qquad \lambda_{k+1}\p{M} = \lambda_k\p{\intc{M, A}}\]
    
    \begin{enumerate}[resume]
        \item Démontrer, que pour tout entier $i \in \bdN$, et pour tout $M \in \bcV$, les identités suivantes :
        %
        \begin{align}
            MX_i &= \sum_{j=0}^i \binom{i}{j} \lambda_{i -j}\p{M}X_j\label{eq:1}\\
            \intc{M, A}X_i &= \sum_{j=0}^i \binom{i}{j} \lambda_{i -j + 1}\p{M} X_j\label{eq:2}
        \end{align}
        
        \noafter
        %
        \boxans{
            Montrons (\ref{eq:1}) et (\ref{eq:2}) par récurrence sur $i \in \bdN$. Soit $M \in \bcV$. Pour $i = 0$, on a :
            %
            \[ MX_0 = MX = \lambda\p{M}X = \binom{0}{0}\lambda_{0-0}\p{M}X_0 \qquad \intc{M, A}X_0 = \intc{M, A}X = \lambda\p{\intc{M, A}}X = \binom{0}{0}\lambda_1X_0\]
            %
            Donc les identité (\ref{eq:1}) (\ref{eq:2}) sont valides au rang $i = 0$. Soit $i \in \bdN$ tel que les identités soient valides au rang $i$.
            %
            \begin{align*}
                MX_{i+1} &= MAX_i = \p{\intc{M, A} + AM}X_i = \intc{M, A}X_i + AMX_i\\
                &= \sum_{j = 0}^i \binom{i}{j} \lambda_{i - j + 1}\p{M}X_j + A\sum_{j = 0}^i \binom{i}{j} \lambda_{i - j}\p{M}X_j\\
                &=\sum_{j = 0}^i \binom{i}{j} \lambda_{i - j + 1}\p{M}X_j + A\sum_{j = 1}^{i+1} \binom{i}{j-1} \lambda_{i - j+1}\p{M}X_{j-1}\\
                &= \binom{0}{i} \lambda_{i+1}\p{M}X_0 + \sum_{j=1}^i \lambda_{i-j+1}\p{M}\p{\binom{i}{j}X_j + \binom{i}{j-1}AX_{j-1}} + A\binom{i}{i}\lambda_0\p{M}X_i\\
                &=\binom{0}{i+1} \lambda_{i+1-0}\p{M}X_0 + \sum_{j=1}^i \lambda_{i+1-j}\p{M}\p{\binom{i}{j} + \binom{i}{j-1}}X_j + \binom{i+1}{i+1}\lambda_{i+1-\p{i+1}}\p{M}X_{i+1}\\
                &= \sum_{j=0}^{i+1} \binom{i+1}{k} \lambda_{i+1-j}\p{M}X_j
            \end{align*}
            %
            Donc (\ref{eq:1}) est vrai au rang $i+1$. Puisque $\intc{M, A} \in \bcV$ on obtient :
            %
            \[ \intc{M, A}X_{i+1} = \sum_{j=0}^{i+1} \binom{i+1}{k} \lambda_{i+1-j}\p{\intc{M, A}}X_j = \sum_{j=0}^{i+1} \binom{i+1}{k} \lambda_{i+1-j+1}\p{M}X_j\]
        }
        %
        \nobefore\yesafter
        %
        \boxansconc{
            Par \emph{principe de récurrence}, on a bien montré les identités (\ref{eq:1}) et (\ref{eq:2}) pour $i \in \bdN$.
        }
        %
        \yesbefore
        
        \item On identifie dorénavant matrices colonnes et vecteurs de $\bdC^n$. Démontrer qu'il existe un plus grand entier $q$ tel que la famille de vecteurs $\p{X_0, X_1, X_2, \dots, X_q}$ soit libre.
        
        \boxansconc{
            La famille $\p{X_0}$ est évidemment libre. De plus $\bdC^n$ est de dimension $n$, donc toute famille libre $\p{X_0, X_1, \dots, X_q}$ de vecteurs distincts de $\bdC^n$ a au plus $n$ éléments. Ainsi, $q$ existe et est majoré par $n$.
        }
    \end{enumerate}
    
    On note $G$ l'espace vectoriel engendré par la famille $\p{X_0, X_1, X_2, \dots, X_q}$.
    
    \begin{enumerate}
        \setcounter{enumi}{4}
        \item Montrer que $\overline M_G$, $\overline A_G$ et $\overline{\intc{M, A}}_G$ sont des endomorphismes de $G$.
        
        \noafter
        %
        \boxans{
            La base $\bcB = \p{X_0, X_1, \dots, X_q}$ de $G$ détermine entièrement les endomorphismes en question. Soit $i \in \iint{1, q}$.
            %
            \begin{enumerate}
                \itt $\overline{M}_G\p{X_i} = MX_i =$ une combinaison linéaire des vecteurs de $\bcB$ par la relation (\ref{eq:1}), donc $\overline M_G \in \bcL\p{G}$.
                
                \itt De même pour $\overline{\intc{M, A}}_G$ avec la relation (\ref{eq:2}).
                
                \itt $\overline{A}_G\p{X_i} = AX_i = X_{i+1} \in G$ par la question précédente.
            \end{enumerate}
        }
        %
        \nobefore\yesafter
        %
        \boxansconc{
            Donc $\overline M_G$, $\overline A_G$ et $\overline{\intc{M, A}}_G$ sont des endomorphismes de $G$.
        }
        %
        \yesbefore\newpage
        
        \item Calculer la trace de $\overline{\intc{M, A}}_G$.
        
        \boxansconc{
            \begin{align*}
                \Tr{\overline{\intc{M, A}}_G} &= \Tr{\overline{MA - AM}_G} = \Tr{\overline{MA}_G - \overline{AM}_G} = \Tr{\overline M_G \overline A_G} -\Tr{\overline A_G \overline M_G}\\
                &= \Tr{\overline M_G \overline A_G} - \Tr{\overline M_G \overline A_G} = 0 
            \end{align*}
        }
        
        \item Quelle est la matrice de $\overline{\intc{M, A}}_G$ dans la base $\p{X_0, X_1, X_2, \dots, X_q}$ ?
        
        \boxansconc{
            On pose $\lambda_k = \title{0}$ pour $k \in \bdZ_-$. On a $\Mat_\bcB \overline{\intc{M, A}}_G = \p{X_i^\star\p{\vphantom{\dfrac{-}{-}}\overline{\intc{M, A}}_G\p{X_j}}}_{\p{i, j} \in \iint{0, q}^2} = \p{\binom{j}{i}\lambda_{j-i+1}\p{M}}_{\p{i, j} \in \iint{0, q}^2}$ d'où :
            %
            \[ \Mat_\bcB \overline{\intc{M, A}}_G = \begin{pNiceMatrix}
                \lambda_1\p{M} & \lambda_2\p{M} & \Cdots & \lambda_{q+1}\p{M}\\
                0 & \lambda_1\p{M} & \Cdots & q\lambda_q\p{M}\\
                \Vdots & \Vdots & \Ddots & \Vdots\\
                0 & 0 & \Cdots & \lambda_1\p{M}
            \end{pNiceMatrix}_{\intc{q+1}}\]
        }
        
        \item Pour $M \in \bcV$, que vaut $\lambda\p{\intc{M, A}}$ ?
        
        \boxansconc{
            On remarque que $\Tr\p{\overline{\intc{M, A}}_G} = \displaystyle\sum_{i=0}^{q} \intc{\Mat_\bcB \overline{\intc{M, A}}_G}_{i, i} = \sum_{i=0}^{q} \lambda_1\p{M} = \p{q+1}\lambda\p{\intc{M, A}}$.
            
            Or $\Tr\p{\overline{\intc{M, A}}_G} = 0$ donc $\lambda\p{\intc{M, A}} = 0$.
        }
        
        \item Établir le théorème {\sffamily\ref{th:th1}}.
        
        \begin{nproof}
            On a $\intc{M, A}X = \lambda\p{\intc{M, A}}X = 0$. Or $\intc{M, A} = MA - AM$ donc $MAX = AMX$.
            
            De plus $MX = \lambda X$ donc $MAX = A\lambda X = \lambda\p{AX}$. Donc $AX$ est soit la matrice colonne nulle, soit une matrice colonne propre pour $M$, avec $MAX = \lambda\p{AX}$ où $MX = \lambda X$. 
        \end{nproof}
    \end{enumerate}
    
    \section{Algèbres de Lie résolubles}
    
    \begin{definition}{}{}
        Soit \hg{$\bcU$} une \hg{algèbre de \textsc{Lie}} et \hg{$p$} un \hg{entier naturel non nul}. On dit que \hg{$\bcU$ est une algèbre de \textsc{Lie} résoluble de longueur $p$} lorsqu'\hg{il existe des algèbres de \textsc{Lie} $\bcU_0, \bcU_1, \dots, \bcU_p$} telles que :
        %
        \begin{align}
            \hg{\ens{0} = \bcU_p \subset \bcU_{p-1} \subset \dots \subset \bcU_1 \subset \bcU_0 = \bcU}\\
            \hg{\forall i \in \iint{1, p-1},\qquad \intc{\bcU_i} \subset \bcU_{i+1}}
        \end{align}
    \end{definition}
    %
    On se propose de montrer le théorème suivant.
    
    \begin{theorem}{}{th2}
        \hg{$\bcU$} est une \hg{\hg{algèbre de \textsc{Lie}} résoluble} si et seulement s'\hg{il existe une matrice $P$ inversible} telle que, \hg{pour tout $M \in \bcU$, $P^{-1}MP$ est triangulaire supérieure}.
    \end{theorem}
    
    Soit $P$ une matrice inversible de $\bcM_{n, n}\p{\bdC}$ et $\bcT_P$ l'ensemble des matrices $M \in \bcM_{n, n}\p{\bdC}$ telles que $P^{-1}MP$ soit triangulaire supérieure.
    
    \begin{enumerate}
        \setcounter{enumi}{9}
        \item Traduire la propriété \guill{il existe une matrice $P$ inversible telle que pour tout $M \in \bcU$, $P^{-1}MP$ est triangulaire supérieure} en une propriété sur les endomorphismes canoniquement associés aux éléments de $\bcU$.
        
        \boxansconc{
            S'il existe une matrice $P$ inversible telle que pour tout $M \in \bcU$, on ait $P^{-1}MP$ triangulaire supérieure, alors pour tout $M \in \bcU$ il existe une matrice $T_M$ triangulaire supérieure telle que $M = P^{-1}T_MP$, autrement dit \emph{les endomorphismes de $\bcU$ sont trigonalisables dans une même base}.
        }
        
        \item Montrer que $\bcT_P$ est une algèbre de \textsc{Lie} résoluble de longueur $n$.
        
        \indication{On pourra considérer les sous-espaces $\p{\bcN_k}_{k \in \iint{0, n}}$ tels que $\bcN_0 = \bcT_P$ et pour tout entier $k \in \iint{1, n}$, $\bcN_k$ est l'ensemble des matrices $M \in \bcT_P$ telles que les $k$ diagonales supérieures $D_0\p{P^{-1}MP}, D_1\p{P^{-1}MP}, \dots, D_{k-1}\p{P^{-1}MP}$ sont nulles.}
        
        \noafter
        %
        \boxans{
            On a $\bcN_n$ l'ensemble des matrices de $\bcT_p$ telles que toutes les diagonales supérieures sont nulles, donc $\bcN_n = \ens{0}$. On a également $\bcU_k \supset \bcU_{k+1}$ pour $k \in \iint{0, n-1}$. Soit maintenant $k \in \iint{0, n}$ et $\p{A, B} \in \bcN_k$.\medskip
            
            On remarque que $AB \in \bcN_{k+1}$ (la $k+1$\ieme~diagonale supérieure s'annule), donc $\intc{A, B} = AB - BA \in \bcN_{k+1}$, d'où $\intc{\bcN_k} \subset \bcN_{k+1} \subset \bcN_k$. On en déduit que $\bcN_k$ est une algèbre de \textsc{Lie}.
        }
        %
        \nobefore\yesafter
        %
        \boxansconc{
            Par définition, $\bcT_P$ est une algèbre de \textsc{Lie} résoluble de longueur $p$.
        }
        %
        \yesbefore
    \end{enumerate}
    
    Dans les questions \quref{12} à \quref{17}, on suppose que $\bcU$ est une algèbre de \textsc{Lie} résoluble de longueur $p = 1$.
    
    \begin{enumerate}
        \setcounter{enumi}{11}
        \item \label{qu:12} Montrer que pour tout $\p{M, M'} \in \bcU^2$, on a $MM' = M'M$.
        
        \boxans{
            Par définition $\intc{\bcU} = \intc{\bcU_0} \subset \bcU_1 = \emptyset$ donc $\intc{M, M'} = 0$ d'où $MM' - M'M = 0$ donc $MM' = M'M$.
        }
        
        \item Soit $r$ un entier non nul et une famille $\p{M_1, M_2, \dots, M_r}$ d'éléments de $\bcU$. Montrer qu'il existe un vecteur propre commun aux endomorphismes $\overline{M_1}, \overline{M_2}, \dots, \overline{M_r}$.
        
        \boxans{
            Procédons par récurrence sur $r \in \bdN^\star$. Pour une famille $\p{M_1}$ avec un unique vecteur, la propriété est évidente. Soit maintenant $r \in \bdN^\star$ telle que la propriété est vraie au rang $r$. Considérons une famille $\p{M_1, M_2, \dots, M_r, M_{r+1}} \in \bcU^{r+1}$. Puisque $\overline{M_{r+1}}$ est un endomorphisme de $\bdC^n$, son 
        }
        
        \item Montrer qu'il existe au moins un vecteur propre commun à tous les endomorphismes $\ens{\overline M,\ M \in \bcU}$.
        
        \boxansconc{
            On applique la question précédente à une base de $\bcU$ (qui est de dimension finie puisque sous-espace vectoriel de $\bcM_{n, n}\p{\bdC}$).
        }
    \end{enumerate}
    
    On note dorénavant $\overline \bcU = \ens{\overline M,\ M \in \bcU}$.\medskip
    
    Soit $F$ et $H$ deux espaces supplémentaires de $\bdC^n$ et $u$ et $v$ deux endomorphismes de $\bdC^n$. De plus, on suppose, d'une part, que $F$ est stable par $u$ et $v$ et, d'autre part, que $u$ et $v$ commutent.
    
    \begin{enumerate}
        \setcounter{enumi}{14}
        \item Montrer les relations suivantes :
        %
        \[ p_H u = p_H u p_H \qquad\et\qquad p_H v = p_H v p_H\]
        
        \item Montrer que $p_H u p_H$ et $p_H v p_H$ commutent puis que $p_H u_H$ et $p_H v_H$ commutent.
        
        \item\label{qu:17} En procédant par récurrence sur $n$, établir le théorème {\sffamily\ref{th:th2}} dans le cas $p=1$.
    \end{enumerate}
\end{document}