\documentclass[a4paper,french,bookmarks]{article}

\usepackage{../../Structure/4PE18TEXTB}

\newboxans
\usepackage{booktabs}

\begin{document}

    \renewcommand{\thesection}{\Roman{section}}
    \setlist[enumerate]{font=\color{white5!60!black}\bfseries\sffamily}
    \renewcommand{\thesection}{Partie \Roman{section}}
    \renewcommand{\labelenumi}{\Roman{section}.\arabic{enumi}.}
    \renewcommand*{\labelenumii}{\alph{enumii}.}

    \stylizeDocSpe{Maths}{Devoir maison $\star$  n° 2}
    {Polytechnique MP MATH 2 1997}{Pour le vendredi 30 septembre 2022}

    On se propose, dans ce problème, d'établir quelques propriétés des
    polynômes d'interpolation. On fixe un entier $n \in \bdN^*$ et on
    désigne par
    %
    \begin{enumerate}
        \itt $E_n$ l'espace vectoriel des fonctions polynômes à $n$
        variables réelles $x_1, x_2, \dots x_n$ et à coefficients
        réels ;

        \itt $F_n$ le sous-espace vectoriel de $E_n$ formé des fonctions
        polynômes qui s'annulent dès que deux variables sont égales (dans
        le cas où $n \geq 2$) ;

        \itt $\partial_i$ (où $i \in \iint{1, n}$) l'opérateur de
        dérivation partielle $\dfrac{\partial}{\partial x_i}$ ;

        \itt $\bfS_n$ le groupe des permutations de $\iint{1, n}$.

    \end{enumerate}

    On fait agir $\bfS_n$ sur $E_n$ de la façon suivante :
    %
    \[ \forall \sigma \in \bfS_n,\quad \forall P \in E_n,\quad T_\sigma
    \p{P}\p{x_1, \dots, x_n} = P\p{x_{\sigma\p{1}}, \dots,
        x_{\sigma\p{n}}}\]

    \section{}

    \begin{enumerate}
        \item Montrer que pour tout $P$ appartenant à $F_2$, il existe
        un unique $Q_P$ appartenant à $E_2$ tel que :
        %
        \[ P\p{x_1, x_2} = \p{x_1 - x_2}Q_P\p{x_1, x_2}\]

        \noafter
        %
        \boxans{
            Soit une fonction polynomiale $P \in F_2$, qu'on associe à \textit{un}
            polynôme formel $\bcP \in \bdR\intc{X_2, X_1} = \bdR\intc{X_2}\intc{X_1}$.
            
            Il s'agit d'un polynôme d'indéterminée $X_1$ à coefficient
            dans l'anneau commutatif$\ \p{\bdR\intc{X_2}, +, \times}$.
            
            \begin{lemma*}{Division euclidienne par $X_1 - X_2$}{}
                \begin{center}
                    \( \displaystyle \hg{\forall \bcA \in \bdR\intc{X_2, X_1},\qquad
                    \exists ! \p{\bcQ, r} \in \bdR\intc{X_2, X_1} \times \bdR,\qquad
                    \bcA = \p{X_1 - X_2} \bcQ + r} \)
                \end{center}
            \end{lemma*}
        }
        %
        \nobefore
        %
        \begin{nproof}
            Montrons d'abord l'existence, en procédant par récurrence sur le degré. 
            %
            \begin{enumerate}
                \itast \underline{Initialisation :} Soit $\bcA \in \bdR\intc{X_2, X_1}$
                tel que $\deg \bcA < 1$. On peut écrire $\bcA = \p{X_1 - X_2}\times 0 +
                \bcA = r$.
                    
                \itast \underline{Hérédité :} Soit $n \geq p$ tel que la propriété est
                vraie pour tout polynôme de $\bdR_n\intc{X_2, X_1}$. On considère $\bcA
                \in \bdR\intc{X_2, X_1}$ de degré exactement $n+1$, et $\p{A_0, A_1,
                \dots, A_{n+1}} \in \bdR\intc{X_2}^{n+2}$ tels que :
                %
                \[ \bcA = A_{n+1}{X_1}^{n+1} + \dots + A_1{X_1} + A_0 = \sum_{k=0}^{n+1}
                A_k{X_1}^k \qquad\et\qquad A_{n+1} \neq 0\]
                %
                On pose $\bcB = \bcA - A_{n+1}{X_1}^{n}\p{X_1 - X_2}$. On a :
                %
                \[ \bcB = A_{n+1}{X_1}^{n+1} + \sum_{k=0}^n A_k{X_1}^k -
                A_{n+1}{X_1}^{n+1} + A_{n+1}{X_1}^nX_2 = \sum_{k=0}^n A_k{X_1}^k +
                A_{n+1}{X_1}^nX_2
                \]
                %
                Ainsi $\deg \bcB \leq n$, on peut donc appliquer l'hypothèse de récurrence
                : 
                %
                \[ \exists ! \p{\bcQ, r'} \in \bdR\intc{X_2, X_1}\times \bdR,\qquad 
                \bcB =  \bcA - A_{n+1}{X_1}^{n}\p{X_1 - X_2} = \p{X_1 - X_2}\bcQ + r'\]
                %
                Donc $ \bcA = \p{X_1 - X_2}\p{\bcQ + A_{n+1}{X_1}^n} + r$. On a bien
                l'hérédité.
            \end{enumerate}
                
            Pour l'unicité, on considère $\bcA \in \bcR\intc{X_2, X_1}$ et $\p{\bcQ_1,
            r_1, \bcQ_2, r_2} \in \p{\bdR\intc{X_2, X_1} \times \bdR}^2$ tel que
            %
            \[ \bcA = \p{X_1 - X_2}\bcQ_1 + r_1 \qquad\et\qquad \bcA = 
            \p{X_1 - X_2}\bcQ_2 + r_2 \]
            %
            Dès lors $\p{X_1 - X_2}\p{\bcQ_1 - \bcQ_2} = r_2 - r_1$. Supposons que 
            $\bcQ_1 \neq \bcQ_2$ donc $\deg\p{\bcQ_1 - \bcQ_2} \geq 0$. Ainsi :
            %
            \[ 1 = \deg\p{r - r'} = \deg\p{\p{X_1 - X_2}\p{\bcQ_1 - \bcQ_2}} = 
            \deg\p{X_1 - X_2} + \deg\p{\bcQ_1 - \bcQ_2} \geq 2\]
            %
            Ce qui est absurde, donc $\bcQ_1 = \bcQ_2$, et donc $r_2 - r_1 = 0$ d'où 
            $r_1 = r_2$. On a bien l'unicité.
        \end{nproof}
        %
        \boxans{
            Dès lors, il existe un unique $\bcQ_P \in \bdR\intc{X_2, X_1}$ et un unique
            réel $r$ tel que $\bcP = \p{X_1 - X_2}Q_P + r$. On spécialise $\bcQ_P$ en
            une unique fonction polynomiale $Q_P \in E_2$, ainsi on a 
            $P\p{x_1, x_2} = \p{x_1 - x_2}Q_P\p{x_1 - x_2} + r$.
            
            On a $P \in F_2$ donc $P\p{x, x} = 0$. Or $P\p{x, x} = \p{x - x}Q_P\p{x, x}
            + r = 0 + r = r$. Donc $r = 0$.
        }
        %
        \yesafter
        %
        \boxansconc{
            Pour tout $P \in F_2$, il existe bien un unique $Q_P$ tel que $P\p{x_1, x_2}
            =\p{x_1 - x_2}Q_P\p{x_1, x_2}$.
        }
    \end{enumerate}
    %
    On admettra que pour $n > 2$ et pour tout $P$ appartenant à $F_n$, il
    existe un unique $Q_P$ appartenant à $E_n$ tel que
    %
    \[ P\p{x_1, \dots, x_n} = \p{\prod_{1 \leq i < j \leq n} x_i - x_j}Q_P\p{x_1,
    \dots,  x_n}\]
    %
    \begin{enumerate}[resume]
        \item Comparer $Q_{T_\sigma\p{P}}$ et $T_\sigma\p{Q_P}$.
        
        \noafter
        %
        \boxans{
            Soit $n \in \bdN^*$ et $P \in F_n$, il existe un unique $Q_P \in E_n$
            tel que \(\displaystyle P\p{x_1, \dots, x_n} = \p{\prod_{1 \leq i < j \leq n} x_i -
            x_j}Q_P\p{x_1, \dots,  x_n}\).
            
            On considère une permutation $\sigma \in \bfS_n$, qu'on fait agir sur $P$ en $T_\sigma \p{P} \in F_n$ :
            %
            \begin{align*}
                 T_\sigma \p{P}\p{x_1, \dots, x_n} = P\p{x_{\sigma\p{1}}, \dots,
            x_{\sigma\p{n}}} &= \p{\prod_{1 \leq i < j \leq n} x_{\sigma\p{i}} - x_{\sigma\p{j}}} Q_P\p{x_{\sigma\p{1}}, \dots, x_{\sigma\p{n}}}\\
            &= \epsilon\p{\sigma}\p{\prod_{1 \leq i < j \leq n} x_i - x_j}T_\sigma\p{Q_P}\p{x_1, \dots, x_n}
            \end{align*}
            %
            De plus il existe un unique $Q_{T_\sigma\p{P}} \in E_n$ tel que $\displaystyle T_\sigma \p{P}\p{x_1, \dots, x_n} = \p{\prod_{1 \leq i < j \leq n} x_i - x_j}Q_{T_\sigma\p{P}}\p{x_1, \dots, x_n}$.
        }
        %
        \nobefore\yesafter
        %
        \boxansconc{
            Donc $Q_{T_\sigma\p{P}} = \epsilon\p{\sigma} T_\sigma\p{Q_P}$
        }
        %
        \yesbefore
        
        \item En supposant $n = 2$, exprimer $Q_P\p{x, x}$ en fonction de $\p{\partial_1 P}\p{x, x}$, puis de  $\p{\partial_2 P}\p{x, x}$.
        
        \noafter
        %
        \boxans{
            Soit $P \in F_2$, il existe un unique $Q_P \in E_2$ tel que $P\p{x_1, x_2} = \p{x_1 - x_2}Q_P\p{x_1, x_2}$. On a :
            %
            \begin{align*}
                \dfrac{\partial P}{\partial x_1}\p{x_1, x_2} &= Q_P\p{x_1, x_2} - \p{x_1 - x_2}\dfrac{\partial Q_P}{\partial x_1}\p{x_1, x_2} &&\text{d'où}&& \p{\partial_1 P}\p{x, x} = Q_P\p{x, x}\\
                \dfrac{\partial P}{\partial x_2}\p{x_1, x_2} &= -Q_P\p{x_1, x_2} - \p{x_1 - x_2}\dfrac{\partial Q_P}{\partial x_2}\p{x_1, x_2} &&\text{d'où}&& \p{\partial_2 P}\p{x, x} = -Q_P\p{x, x}
            \end{align*}
            %
            Ainsi :
        }
        %
        \nobefore\yesafter
        %
        \boxansconc{
            Donc $Q_P\p{x, x} = \p{\partial_1 P}\p{x, x} = -\p{\partial_2 P}\p{x, x}$
        }
        %
        \yesbefore
        
        \item Supposant $n = 3$, exprimer $Q_P\p{x, x, y}$ en fonction de $\p{\partial_1 P}\p{x, x, y}$ pour $x \neq y$, puis $Q_P\p{x, x, x}$ en fonction de $\p{\partial_1\partial_2^2 P}\p{x, x, x}$.
        
        \noafter
        %
        \boxans{
            Soit $P \in F_3$, il existe un unique $Q_P \in E_3$ tel que $P\p{x_1, x_2, x_3} = \p{x_1 - x_2}\p{x_1 - x_3}\p{x_2 - x_3}Q_P\p{x_1, x_2, x_3}$.
            %
            Donc $P\p{x_1, x_2, x_3} = \p{{x_1}^2 - \p{x_3 + x_2}x_1 + x_2x_3}\p{x_2 - x_3}Q_P\p{x_1, x_2, x_3}$.
            %
            \begin{align*}
                \dfrac{\partial P}{\partial x_1}\p{x_1, x_2, x_3} &= \p{2x_1 - x_2 - x_3}\p{x_2 - x_3}Q\p{x_1, x_2, x_3} + \p{{x_1}^2 - \p{x_3 + x_2} + x_2x_3}\p{x_2 - x_3}\dfrac{\partial Q_P}{\partial x_1}\p{x_1, x_2, x_3}\\
                &= \p{2x_1x_2 -2x_1x_3 - {x_2}^2 + {x_3}^2}
            \end{align*}
            %
            Tout polynôme étant de classe $\bcC^\infty$, le \textit{théorème de \textsc{Scharwz}} livre $\dfrac{\partial^3 P}{\partial x_1\partial {x_2}^2} = \dfrac{\partial^3 P}{\partial {x_2}^2\partial x_1}$. Après beaucoup de calculs, on obtient  $\p{\partial_2^2 \partial_1 P}\p{x, x, y} = \p{\partial_1\partial_2^2 P}\p{x, x, x} = -2Q_P\p{x, x, x}$.
        }
        %
        \nobefore\yesafter
        %
        \boxansconc{
            Ainsi $\p{\partial_1 P}\p{x, x, y} = \p{x - y}^2Q_P\p{x, x, y}$ et $Q_P\p{x, y, z} = -\dfrac{1}{2}\p{\partial_1\partial_2^2 P}\p{x, x, x}$. 
        }
        %
        \yesbefore
        
    \end{enumerate}
    
    \section{}
    
    Dans cette partie on fixe un élément $f$ de $E_1$ et on note $P_1$ l'élément de $F_n$ défini par :
    %
    \[ P_f\p{x_1, \dots, x_n} = \det{\p{a_{i, j}}_{\p{i, j} \in \iint{1, n}^2}}\]
    %
    où
    %
    \[ \forall \p{i, j} \in \iint{1, n},\qquad a_{i, 1} = f\p{x_i} \quad\et\quad \forall j \in \iint{2, n},\qquad a_{i, j} = {x_i}^{n -j} \]
    %
    On écrira $Q_f$ au lieu de $Q_{P_f}$.
    
    \begin{enumerate}[resume]
        \item Vérifier que, si $f\p{x} = x^{n-1}$, on a :
        %
        \[ P_f\p{x_1, \dots, x_n} = \prod_{1 \leq i < j \leq n} \p{x_i - x_j}\]
        %
        \noafter
        %
        \boxans{
             On note $\sigma \in \bfS_n$ la permutation qui échange $i \in \iint{1, n}$ avec $n + 1 - i$ ($\sigma$ échange $1$ avec $n$, $2$ avec $n-1$, \dots). La signature de $\sigma$ est $\epsilon\p{\sigma} = \prod_{1 \leq i < j \leq n} \p{-1}$. On a :
             %
             \[ P_f\p{x_1, \dots, x_n} = \begin{vNiceMatrix}
                f\p{x_1} & {x_1}^{n-2} & {x_1}^{n-3} & \Cdots & 1 \\
                f\p{x_2} & {x_2}^{n-2} & {x_2}^{n-3} & \Cdots & 1 \\
                f\p{x_3} & {x_3}^{n-2} & {x_3}^{n-3} & \Cdots & 1 \\
                \Vdots   & \Vdots      & \Vdots & \Ddots & \Vdots\\
                f\p{x_n} & {x_n}^{n-2} & {x_n}^{n-3} & \Cdots & 1
            \end{vNiceMatrix} = \p{\prod_{1 \leq i < j \leq n} \p{-1}} \begin{vNiceMatrix}
                1      & {x_1}  & {x_1}^2 & \Cdots & f\p{x_1} \\
                1      & {x_2}  & {x_2}^2 & \Cdots & f\p{x_2} \\
                1      & {x_3}  & {x_3}^2 & \Cdots & f\p{x_3} \\
                \Vdots & \Vdots & \Vdots  & \Ddots & \Vdots\\
                1      & {x_n}  & {x_n}^2 & \Cdots &  f\p{x_n}
            \end{vNiceMatrix} \]
            %
            Si $f\p{x} = x^{n-1}$, alors on reconnait une matrice de \textsc{Vandermonde} d'où :
            %
        }
        %
        \nobefore\yesafter
        %
        \boxansconc{
            \[ P_f\p{x_1, \dots, x_n} = \p{\prod_{1 \leq i < j \leq n} \p{-1}}\p{\prod_{1 \leq i < j \leq n} \p{x_j - x_i}} = \prod_{1 \leq i < j \leq n} \p{x_i - x_j}\]
        }
        %
        \yesbefore
        
        \item Trouver des éléments $A_1, \dots, A_n$ de $E_n$ tels que 
        %
        \[ \forall f \in E_1,\qquad Q_f\p{x_1, \dots, x_n} = \sum_{i =1}^n \dfrac{f\p{x_i}}{A_i\p{x_1, \dots, x_n}}\]
        %
        et démontrer leur unicité.
        
        \noafter
        %
        \boxans{
            \[ Q_f\p{x_1, \dots, x_n} = \dfrac{P_f\p{x_1, \dots, x_n}}{\displaystyle\prod_{1 \leq i < j \leq n}\p{x_i - x_j}} = \dfrac{1}{\displaystyle\prod_{i \leq i < j \leq n}\p{x_i - x_j}}\begin{vNiceMatrix}
                f\p{x_1} & {x_1}^{n-2} & {x_1}^{n-3} & \Cdots & 1 \\
                f\p{x_2} & {x_2}^{n-2} & {x_2}^{n-3} & \Cdots & 1 \\
                f\p{x_3} & {x_3}^{n-2} & {x_3}^{n-3} & \Cdots & 1 \\
                \Vdots   & \Vdots      & \Vdots & \Ddots & \Vdots\\
                f\p{x_n} & {x_n}^{n-2} & {x_n}^{n-3} & \Cdots & 1
            \end{vNiceMatrix} \]
            %
            On développe alors selon la première colonne :
            %
            \begin{align*} Q_f\p{x_1, \dots, x_n} &= \dfrac{1}{\displaystyle\prod_{i \leq i < j \leq n}\p{x_i - x_j}}\left(f\p{x_1}\begin{vNiceMatrix}
                {x_2}^{n-2} & {x_2}^{n-3} & \Cdots & 1 \\
                {x_3}^{n-2} & {x_3}^{n-3} & \Cdots & 1 \\
                \Vdots      & \Vdots & \Ddots & \Vdots\\
                {x_n}^{n-2} & {x_n}^{n-3} & \Cdots & 1
            \end{vNiceMatrix} - f\p{x_2}\begin{vNiceMatrix}
                {x_1}^{n-2} & {x_1}^{n-3} & \Cdots & 1 \\
                {x_3}^{n-2} & {x_3}^{n-3} & \Cdots & 1 \\
                \Vdots      & \Vdots & \Ddots & \Vdots\\
                {x_n}^{n-2} & {x_n}^{n-3} & \Cdots & 1
            \end{vNiceMatrix}\right. \\
            &\qquad\left.+ \dots + \p{-1}^{n-1} f\p{x_n}
            \begin{vNiceMatrix}
                {x_1}^{n-2} & {x_1}^{n-3} & \Cdots & 1 \\
                {x_2}^{n-2} & {x_2}^{n-3} & \Cdots & 1 \\
                {x_3}^{n-2} & {x_3}^{n-3} & \Cdots & 1 \\
                \Vdots      & \Vdots & \Ddots & \Vdots\\
                {x_{n-1}}^{n-2} & {x_{n-1}}^{n-3} & \Cdots & 1
            \end{vNiceMatrix}\right) = \sum_{k=1}^n f\p{x_k}\dfrac{\p{-1}^{n-k}V\p{\p{x_i}_{i \neq k}}}{\displaystyle\prod_{i \leq i < j \leq n}\p{x_i - x_j}}\\
            &= \sum_{k=1}^n f\p{x_k}\dfrac{\p{-1}^{n-k}\displaystyle \prod_{\substack{1 \leq i < j \leq n\\ i \neq k \quad j \neq k}} \p{x_i - x_j}}{\displaystyle\prod_{1 \leq i < j \leq n} \p{x_i - x_j}} = \sum_{k=1}^n \dfrac{f\p{x_k}}{\p{-1}^{n-k}\displaystyle\prod_{i = 1}^{k-1} \p{x_k - x_i}\prod_{i=k+1}^n \p{x_i - x_k}}
            \end{align*}
        }
        %
        \nobefore\yesafter
        %
        \boxansconc{
            Donc ils existent d'uniques $A_1, \dots A_n$ tels que décrit avec $A_k = \displaystyle\prod_{i = 1,\ i \neq k}^{n} \p{x_k - x_i}$.
        }
        %
        \yesafter
        
        \item Montrer que $Q_f$ est invariant par permutation des variables.
        
        \noafter
        %
        \boxans{
            Soit $\sigma \in \bfS_n$. On a $T_\sigma\p{Q_{P_f}} = \epsilon\p{\sigma}Q_{T_\sigma\p{P_f}}$. Or $T_\sigma\p{P_f} \in F_n$ d'où :
            %
            \[ T_\sigma\p{Q_f}\p{x_1, \dots, x_n} = \epsilon\p{\sigma}\sum_{i=1}^n \dfrac{f\p{x_i}}{T_\sigma\p{A_i}\p{x_1, \dots, x_n}} = \epsilon\p{\sigma} \sum_{i=1}^n \dfrac{f\p{x_i}}{\epsilon\p{\sigma}A_i\p{x_1, \dots, x_n}} = Q_f\p{x_1, \dots, x_n}\]
        }
        %
        \nobefore\yesafter
        %
        \boxansconc{
            Donc $T_\sigma\p{Q_f} = Q_f$, d'où $Q_f$ est invariant par permutation des variables.
        }
        %
        \yesbefore
        
        \item Déterminer $Q_f$ lorsque $f\p{x} x^k$, avec $k = 0, 1, \dots, n$.
        
        \boxansconc{
            Pour $k = n-1$, on a déjà montré que $Q_f = 1$. Pour $k < n-1$, il y a deux colonnes égales dans le déterminant donc $Q_f = 0$. 
        }
    \end{enumerate}
    
    \section{}
    
    Dans cette partie, on fixe des nombres réels deux à deux distincts $a_1, \dots, a_n$ et une fonction $f$ sur $\bdR$ à valeurs réelles.
    
    \begin{enumerate}[resume]
        \item Trouver un polynôme $g_f$ de degré inférieur à $n-1$ vérifiant $g\p{a_i} = f\p{a_i}$ pour tout $i \in \iint{1, n}$ et démontrer son unicité.
        
        \noafter
        %
        \boxans{
            Soit $i \in \iint{1, n}$, on pose $L_i$ la fonction polynomiale relle que $L_i\p{x} = \displaystyle \prod_{k = 1,\ k \neq i}^n \dfrac{x - a_k}{a_i - a_k}$.
            
            On obtient que $P\p{x_j} = \delta_{i, j}$ et $\deg L_i \leq n-1$
            
            On pose alors $g_f = \displaystyle\sum_{k=1}^n f\p{a_k} L_k$, d'où $g_f\p{a_i} = f\p{a_i}$ et $\deg g_f \leq n -1$. En supposant $h_f$ vérifiant les mêmes propriétés, $h_f$ et $g_f$ ont $n$ points communs distincts et sont de degré $n - 1$ au plus, donc par rigidité, $g_f = h_f$.
        }
        %
        \nobefore\yesafter
        %
        \boxansconc{
            Donc il existe un unique $g_f = \displaystyle\sum_{i=1}^n f\p{a_i}\prod_{k = 1,\ k \neq i}^n \dfrac{x - a_k}{a_i - a_k}$ tel que décrit.
        }
        %
        \yesbefore
    \end{enumerate}
    
    On notera $R_f\p{a_1, \dots, a_n}$ le coefficient $x^{n-1}$ dans $g_f$.
    
    \begin{enumerate}[resume]
        \item Supposant que $f$ est un polynôme, comparer $R_f\p{a_1, \dots, a_n}$ et $Q_f\p{a_1, \dots, a_n}$.
        
        \boxansconc{
            On a directement $R_f\p{a_1, \dots, a_n} = \displaystyle\sum_{k=1}^n f\p{a_k}\dfrac{1}{\displaystyle\prod_{i=1,\ i \neq k}^n \p{a_k - a_i}} = \sum_{k=1}^n \dfrac{f\p{a_k}}{A_k\p{a_1, \dots, a_n}} = Q_f\p{a_1, \dots, a_n}$.
        }
        
        \item Supposant $n = 2$ et $k$ entier supérieur ou égal à $0$, donner une condition suffisante portant sur $f$ pour que $R_f$ se prolonge en une fonction de classe $\bcC^k$ sur $\bdR^2$ (on pourra utiliser la formule de \textsc{Taylor} avec reste intégral à l'ordre $1$).
        
        \noafter
        %
        \boxans{
            Supposons $f$ de classe $\bcC^{k+1}$. On a $R_f\p{x, y} = Q_f\p{x, y} = \dfrac{f\p{x} - f\p{y}}{x - y}$. Sans perte d'information, c'est équivalent à $R_f\p{x+h, x} = \dfrac{f\p{x+h} - f\p{x}}{h}$. Il est évident que $R_f$ est continue sur son intervalle de définition, reste donc le cas où $h$ tends vers $0$ (\ie $x = y$). Par définition, on prolonge alors en $R_f\p{x, x} = f'\p{x}$. Par ailleurs
            %
            \[ \dfrac{\partial}{\partial x}R_f\p{x, y} = \dfrac{f'\p{x}}{x-y} - \dfrac{f\p{x} - f\p{y}}{\p{x - y}^2} \qquad\text{donc on prolonge}\qquad \dfrac{\partial}{\partial x}R_f\p{x, x} = \lim\limits_{h \to 0} \dfrac{f'(x) - f'\p{x+h}}{h} = f''\p{x}\]
            %
            Par récurrence, on montre de même que les seuls problème de définitions pour les $k$ premières dérivées par rapport à $x$ de $R_f$ surviennent lorsque $x = y$, et qu'on prolonge $R_f^{\p{i}}\p{x, x} = f^{\p{i+1}}\p{x}$ pour $i \in \iint{i, k}$. Par symétrie entre $x$ et $y$ (on montre que $R_f\p{x, y} = R_f\p{y, x}$), il en va de même pour les dérivées par rapport à $y$.
        }
        %
        \nobefore\yesafter
        %
        \boxansconc{
            $f$ de classe $\bcC^{k+1}$ est une condition suffisante pour que $\bcR_f$ se prolonge en une fonction de classe $\bcC^k$ sur $\bdR^2$.
        }
        %
    \end{enumerate}
    
    \section{}
    
    Dans cette partie, on fixe des nombres réels deux à deux distincts $a_1, \dots, a_n$ et des entiers positifs ou nuls $r_1, \dots, r_n$ ; on pose :
    %
    \[ r = \p{\sum_{i=1}^n \p{r_i + 1}} - 1\]
    %
    On note $G$ l'espace vectoriel des fonctions polynômes à une variable, à coefficients réels, de degré au plus $r$.
    
    \newpage
    {\color{main3}\textbf{\sffamily N.B. :} \textit{Du à un manque de temps, ne figurent aux questions ci-dessous que des ébauches de réponse.}}
    
    \begin{enumerate}[resume]
        \item On se propose d'étudier les familles de polynômes $P_{i, j}$ avec $i \in \iint{1, n}$ et $j \in \iint{0, r_i}$ vérifiant :
        %
        \[ P_{i, j}^{\p{k}}\p{a_\ell} = \delta_{i, \ell}\delta_{j, k} \qquad\text{pour}\qquad \ell \in \iint{1, n} \qquad\et\qquad k \in \iint{0, r_\ell} \]
        %
        où le $\delta$ désigne le symbole de \textsc{Kronecker}.
        
        \begin{enumerate}
            \item Démontrer l'unicité d'une telle famille $\p{P_{i, j}}$.
            
            \noafter
            %
            \boxans{
                Supposons l'existence de deux $P_{i, j}$ et $Q_{i, j}$ satisfaisant les conditions ci-dessus. Montrons que $P_{i, j} = Q_{i, j}$. Puisque ces deux fonctions sont dans $G$, elles sont au plus de degré $r$. On procède par rigidité, en montrant qu'elles possèdes ont $r+1$ points communs.
                
                Or pour un $\ell \in \iint{1, n}$ donné, pour tout $k \in \iint{0, r_\ell}$, la propriété ci-dessous livre $P_{i, j}^{\p{k}}\p{a_\ell} = Q_{i, j}^{\p{k}}\p{a_\ell}$. On a donc $\displaystyle\sum_{\ell=1}^n \p{r_\ell + 1} = r + 1$ points communs d'où $P_{i, j} = Q_{i, j}$.
            }
            %
            \nobefore\yesafter
            %
            \boxansconc{
                Si deux familles existent et vérifient les propriétés, alors elles sont identiques, il y a donc unicité.
            }
            %
            \yesbefore
            
            \item Montrer qu'une telle famille $\p{P_{i, j}}$, si elle existe, est une base de $G$.
            
            \boxansconc{
                Par rigidité, tout polynôme $P$ de $G$ est déterminé par la donnée des $\p{P^{\p{k}}\p{a_\ell}}_{\ell \times k \in \iint{1, n}\times \iint{0, r_\ell}}$. Or chaque $P_{i, j}$ vaut $1$ pour exactement une de ces valeurs, d'où le caractère générateur des $\p{P_{i, j}}$. Or $\dim G = r$, et $\p{P_{i, j}}$ contient exactement $r$ vecteurs, d'où $\p{P_{i, j}}$ est une base.
            }
            
            \item Démontrer l'existence d'une telle famille $\p{P_{i, j}}$.
            
            \boxansconc{
                \itshape \underline{Algorithme} (\cf application dans la question suivante) : on construit le polynôme $P_{i, k-1}$ à sa dérivée $k$-ième auquel le coefficient $a_i$ est non nul, selon sa factorisation (racines de multiplicité adaptée). On intègre/dérive et on calcule les termes constants pour convenir à la définition. 
            }
            
            \item Déterminer explicitement la famille $\p{P_{i, j}}$ dans le cas où $n = 2$ et $r_1 = r_2 = 1$.
            
            \noafter
            %
            \boxans{
                On a $P_{1, 0}$, $P_{2, 0}$, $P_{1, 1}$ et $P_{2, 1}$ de degré $3$ tels que :
                %
                \begin{enumerate}
                    \begin{minipage}{0.2\linewidth}
                        \itt $P_{1, 0}\p{a_1} = 1$
                        
                        \itt $P_{1, 0}\p{a_2} = 0$
                        
                        \itt $P_{1, 0}'\p{a_1} = 0$
                        
                        \itt $P_{1, 0}'\p{a_2} = 0$
                    \end{minipage}
                    %
                    \hfill
                    %
                    \begin{minipage}{0.2\linewidth}
                        \itt $P_{2, 0}\p{a_1} = 0$
                        
                        \itt $P_{2, 0}\p{a_2} = 1$
                        
                        \itt $P_{2, 0}'\p{a_1} = 0$
                        
                        \itt $P_{2, 0}'\p{a_2} = 0$
                    \end{minipage}
                    %
                    \hfill
                    %
                    \begin{minipage}{0.2\linewidth}
                        \itt $P_{1, 1}\p{a_1} = 0$
                        
                        \itt $P_{1, 1}\p{a_2} = 0$
                        
                        \itt $P_{1, 1}'\p{a_1} = 1$
                        
                        \itt $P_{1, 1}'\p{a_2} = 0$
                    \end{minipage}
                    %
                    \hfill
                    %
                    \begin{minipage}{0.2\linewidth}
                        \itt $P_{2, 1}\p{a_1} = 0$
                        
                        \itt $P_{2, 1}\p{a_2} = 0$
                        
                        \itt $P_{2, 1}'\p{a_1} = 0$
                        
                        \itt $P_{2, 1}'\p{a_2} = 1$
                    \end{minipage}
                \end{enumerate}
                %
                Déterminons $P_{1, 0}$. On a d'abord $P_{1, 0}'\p{x} = \lambda_1\p{x-a_1}\p{x-a_2}$ puisque de degré $2$, soit $P_{1, 0}'\p{x} = \lambda_1\p{x^2 -\p{a_1 + a_2}x + a_1a_2}$. Il en va de même pour $P_{2, 0}$, ainsi :
                %
                \[ P_{1, 0}\p{x} = \lambda_1\p{\dfrac{1}{3}x^3 - \dfrac{a_1 + a_2}{2}x^2 + a_1a_2x + c_1} \qquad\et\qquad P_{2, 0}\p{x} = \lambda_2\p{\dfrac{1}{3}x^3 - \dfrac{a_1 + a_2}{2}x^2 + a_1a_2x + c_2}\]
                %
                Or $P_{1, 0}\p{a_2} = 0$ d'où $c_1 = -\dfrac{1}{3}{a_2}^3 + \dfrac{a_1 + a_2}{2}{a_2}^2 - a_1{a_2}^2$, et $P_{1, 0}\p{a_1} = 1$ ainsi :
                %
            }
            %
            \nobefore
            %
            \boxansconc{
                \[ P_{1, 0}\p{x} = \dfrac{\frac{1}{3}\p{x^3 - {a_2}^3} + \frac{a_1 + a_2}{2}\p{{a_2}^2 - x^2} + a_1a_2\p{x - a_2}}{\frac{1}{3}\p{{a_1}^3 - {a_2}^3} + \frac{a_1 + a_2}{2}\p{{a_2}^2 - {a_1}^2} + a_1a_2\p{a_1 - a_2}}\]
            }
            %
            \boxans{
                De même :
            }
            %
            \boxansconc{
                \[ P_{2, 0}\p{x} = \dfrac{\frac{1}{3}\p{x^3 - {a_1}^3} + \frac{a_1 + a_2}{2}\p{{a_1}^2 - x^2} + a_1a_2\p{x - a_1}}{\frac{1}{3}\p{{a_2}^3 - {a_1}^3} + \frac{a_1 + a_2}{2}\p{{a_1}^2 - {a_2}^2} + a_1a_2\p{a_2 - a_1}}\]
            }
            %
            \boxans{
                Pour $P_{1,1}$ et $P_{2, 1}$, on a :
                %
                \[ P_{1, 1}\p{x} = \lambda_3\p{x - a_1}\p{x - a_2}^2\qquad\et\qquad P_{2, 1}\p{x} = \lambda_4\p{x - a_1}^2\p{x - a_2} \]
                %
                En dérivant :
                %
                \[ P_{1, 1}'\p{x} = \lambda_3\p{x - a_2}\p{-2a_1 - a_2 + 3x} \qquad\et\qquad P_{2, 1}'\p{x} = \lambda_4\p{x - a_1}\p{-2a_2 - a_1 + 3x}\]
                %
                Avec $P_{1, 1}'\p{a_1} = 1$ et $P_{2, 1}'\p{a_1} = 1$ on obtient $\lambda_3 = \dfrac{1}{\p{a_1 - a_2}^2}$ et $\lambda_4 = \dfrac{1}{\p{a_2 - a_1}^2} = \lambda_3$ d'où finalement :
                %
            }
            %
            \yesafter
            %
            \boxansconc{
                \[ P_{1, 1}'\p{x} = \dfrac{\p{x - a_2}\p{-2a_1 - a_2 + 3x}}{\p{a_1 - a_2}^2} \qquad\et\qquad P_{2, 1}'\p{x} = \dfrac{\p{x - a_1}\p{-2a_2 - a_1 + 3x}}{\p{a_2 - a_1}^2}\]
            }
        \end{enumerate}
        
        \item On suppose maintenant $n = 2$ et $r_1 = r_2 = 1$. Pour tout entier $p \geq 0$ et tout réel $x$ on pose $\phi_p\p{x} = x^p$. Pour tout $f \in E_1$ et tous réels $x_1, x_2$ on pose :
        %
        \[ D_f\p{x_1, x_2} = \begin{vNiceMatrix}
            f\p{x_1} & \phi_2\p{x_1} & \phi_1\p{x_1} & \phi_0\p{x_1} \\
            f'\p{x_1} & \phi_2'\p{x_1} & \phi_1'\p{x_1} & \phi_0'\p{x_1} \\
            f\p{x_2} & \phi_2\p{x_2} & \phi_1\p{x_2} & \phi_0\p{x_2} \\
            f'\p{x_2} & \phi_2'\p{x_2} & \phi_1'\p{x_2} & \phi_0'\p{x_2}
        \end{vNiceMatrix}\]
        
        \begin{enumerate}
            \item Vérifier que l'on a $\p{\partial_1^k D_f}\p{x, x} =0$ pour $k \in \iint{0, 3}$.
            
            \item En déduire l'existence d'un polynôme $\Delta_f \in E_2$ tel que :
            %
            \[ D_f\p{x_1, x_2} = \p{x_1 - x_2}^4 \Delta_f\p{x_1, x_2}\]
            %
            \item Comparer $\Delta_f\p{a_1, a_2}$ et le coefficient de $x^3$ dans le polynôme $g_f$.
        \end{enumerate}
    \end{enumerate}
\end{document}