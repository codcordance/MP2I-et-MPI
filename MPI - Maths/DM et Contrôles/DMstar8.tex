\documentclass[a4paper,french,bookmarks]{article}

\usepackage{../../Structure/4PE18TEXTB}

\newboxans
\usepackage{booktabs}

\begin{document}

    \renewcommand{\thesection}{\Roman{section}}
    \setlist[enumerate]{font=\color{white5!60!black}\bfseries\sffamily}
    \renewcommand{\thesubsection}{\Roman{section}.\Alph{subsection}}
    \renewcommand{\labelenumi}{\thesubsection.\arabic{enumi})}
    \renewcommand*{\labelenumii}{\alph{enumii})}

    \stylizeDocSpe{Maths}{Devoir maison $\star$ n° 8}{CCS MP MATH 2 2016}{Pour le vendredi 27 janvier 2023}
    
    Le problème étudie quelques propriétés de variables aléatoires réelles finies de la forme $\sum_{k=1}^n a_kX_k$, où les $a_k$ sont des réels et les $X_k$ sont des variables aléatoires mutuellement indépendantes à valeurs dans $\ens{-1, 1}$.
    
    \begin{enumerate}
        \itt La première partie établit des résultats sur des intégrales, utilisés dans les parties suivantes.
        
        \itt À partir de la deuxième partie, on suppose donnée une suite $\suiteZ{X_n}$ de variables aléatoires mutuellement indépendantes définies sur un espace probabilisé $\p{\Omega, \bcA, \bbP}$ à valeurs dans $\ens{1, -1}$ et vérifiant
        %
        \[ \forall k \in \bdN^\star,\qquad \bbP\p{X_k = 1} = \bbP\p{X_k = -1} = \dfrac{1}{2}\]
    \end{enumerate}
    
    \section{Suites et intégrales}
    
    \subsection{Étude d'une intégrale à paramètre}
    
    Pour $x \in \bdR_+$, on pose
    %
    \[ f\p{x} = \int_0^{+\infty} \dfrac{1 - \cos t}{t^2}e^{-xt}\dif t\]
    %
    \begin{enumerate}
        \item Montrer que $f$ est définie et continue sur $\intor{0, +\infty}$ et de classe $\bcC^2$ sur $\into{0, +\infty}$.
        
        \noafter
        %
        \boxans{
            Soit $t \in \bdR_+$. Par inégalité de \textsc{Taylor-Lagrange}, on obtient
            %
            \[ \mod{\cos t - \cos 0 - \p{t - 0}\cos'\p{0}} \leq \sup_\bdR \mod{\cos ''} \dfrac{\mod{t - 0}^2}{2!} = \dfrac{t^2}{2}\]
            %
            Or $\mod{\cos t - \cos 0 - \p{t - 0}\cos'\p{0}} = \mod{1 - \cos t}$ d'où pour $A \in \bdR_+$ et $x \in \into{0, +\infty}$ on a
            %
            \[ \int_0^A \mod{\dfrac{1 - \cos t}{t^2}e^{-xt}}\dif t \leq \int_0^A  \dfrac{t^2}{2t^2}e^{-xt}\dif t \leq \dfrac{1}{2}\int_0^A e^{-xt}\dif t = \dfrac{1}{2}\intc{\dfrac{-1}{x}e^{-xt}}_0^A = \dfrac{1-e^{-xA}}{2x} \lima{A \to +\infty} \dfrac{1}{2x}\]
            %
            Pour $x = 0$, on prend $A > 1$ et l'on découpe l'intégrale en deux parties
            %
            \[ \int_0^A \mod{\dfrac{1 - \cos t}{t^2}e^{-xt}}\dif t = \int_0^1 \mod{\dfrac{1 - \cos t}{t^2}}\dif t + \int_1^A \mod{\dfrac{1 - \cos t}{t^2}}\dif t \leq \dfrac{1}{2}\int_0^1 \dif t + 2\int_1^A \dfrac{\dif t}{t^2} = \dfrac{1}{2} + 2\intc{-\dfrac{1}{t}}_1^A \lima{A \to +\infty} \dfrac{5}{2}\]
        }
        %
        \nobefore
        %
        \boxansconc{
            Ainsi $f$ est bien définie sur $\intor{0, +\infty}$ (et majorée par $x \mapsto \dfrac{1}{2x}$ sur $\into{0,+\infty}$).
        }
        %
        \boxans{
            Posons $\varphi : \begin{array}[t]{ccc}
                \into{0, +\infty}^2 &\to& \bdR  \\
                \p{x, t} &\mapsto& \dfrac{1-\cos t}{t^2}e^{-xt} 
            \end{array}$ Cette fonction est $\bbL^1$ sur $\into{0, +\infty}$ (ci-dessus), et $\bcC^\infty$ par rapport à ses deux variables. On a 
            %
            \[ \dfrac{\partial \varphi}{\partial x}\p{x, t} = -t\dfrac{1 - \cos t}{t^2}e^{-xt} \qquad\text{donc}\qquad \mod{\dfrac{\partial \varphi}{\partial x}\p{x, t}} \leq \dfrac{t}{2}e^{-xt} = \varphi_1\p{t}\]
            %
            On a $\varphi_1\p{t} \lima{t \to 0} 0$ et $\varphi_1\p{t} \lima{t \to 0} 0$ donc $\varphi_1$ est $\bbL_1$ sur $\into{0, +\infty}$. De plus
            %
            \[ \dfrac{\partial^2 \varphi}{\partial x^2}\p{x, t} = t^2\dfrac{1 - \cos t}{t^2}e^{-xt} \qquad\text{donc}\qquad \mod{\dfrac{\partial^2 \varphi}{\partial x^2}\p{x, t}} \leq \dfrac{t^2}{2}e^{-xt} = \varphi_2\p{t}\]
            %
            On obtient similairement que $\varphi_2$ est $\bbL_2$ sur $\into{0, +\infty}$.
        }
        %
        \noafter
        %
        \boxansconc{
            Par \emph{théorème de dérivabilité des intégrales à paramètres}, $f$ est de classe $\bcC^2$ sur $\into{0, +\infty}$.
        }
        %
        \yesbefore
        
        \item Déterminer les limites de $f$ et $f'$ en $+\infty$.
        
        \boxansconc{
            Pour tout $x \in \into{0, +\infty}$, on a montré que $f\p{x} \leq \dfrac{1}{2x} \lima{x \to +\infty} 0$ et, donc $\lim\limits_{x \to +\infty} f\p{x} = 0$. Par ailleurs, on a
            %
            \[ \forall x \in \into{0, +\infty},\qquad f'\p{x} = \int_0^{+\infty} \dfrac{\partial \varphi}{\partial x}\p{x, t}\dif t \leq \int_0^{+\infty} \dfrac{t}{2}e^{-xt} = \intc{\dfrac{-1}{2x}e^{-xt}}_0^{+\infty} = \dfrac{1}{2x} \lima{x \to +\infty} 0\]
            %
            donc de même $\lim\limits_{x \to +\infty} f'\p{x} = 0$.
        }
        
        \item Exprimer $f''$ sur $\into{0, +\infty}$ à l'aide de fonctions usuelles et en déduire que
        %
        \[ \forall x \in \into{0, +\infty},\qquad f'\p{x} = \ln{x} -\dfrac{1}{2}\ln{x^2 + 1}\]
        
        \noafter
        %
        \boxans{
            On a $f''\p{x} = \displaystyle\int_0^{+\infty} \dfrac{\partial^2}{\partial x^2}\p{x, t}\dif t = \int_0^{+\infty} \p{1 - \cos t}e^{-xt}\dif t = \dfrac{1}{x} - \int_0^{+\infty}\cos{t} e^{-xt}$. On intègre par parties :
            %
            \begin{align*}
                \int_0^{+\infty} \cos{t}e^{-xt} \dif t&= \intc{\cos{t}\p{-\dfrac{e^{-xt}}{x}}}_0^{+\infty} - \int_0^{+\infty} \p{-\sin{t}}\p{-\dfrac{e^{-xt}}{x}}\dif t\\
                &= \dfrac{1}{x} - \dfrac{1}{x}\int_0^{+\infty} \sin{t}e^{-xt}\dif t = \dfrac{1}{x} - \dfrac{1}{x}\p{\intc{\sin{t}\p{-\dfrac{e^{-xt}}{x}}}_0^{+\infty} - \int_0^{+\infty}\cos{t}\p{-\dfrac{e^{-xt}}{x}}\dif t}\\
                &= \dfrac{1}{x} - \dfrac{1}{x^2}\int_0^{+\infty}\cos{t}e^{-xt}\dif t\\
                &= \dfrac{1}{x}\dfrac{1}{1 + \frac{1}{x^2}} = \dfrac{x}{x^2 + 1}
            \end{align*}
        }
        %
        \nobefore\yesafter
        %
        \boxansconc{
        
        }
    \end{enumerate}
    
\end{document}