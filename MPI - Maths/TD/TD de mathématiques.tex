\documentclass[a4paper,french,bookmarks]{book}

\usepackage{booktabs}
\usepackage{minitoc}
\usepackage{./Structure/4PE18TEXTB}
\usepackage{proof}
\usepackage{pdfpages}

\newboxans
\renewcommand{\questionsdecours}{\section*{\centering\EBGaramond\Large Questions~ de~ cours}}
\renewcommand{\thechapter}{\Roman{chapter}}
\setcounter{secnumdepth}{5}
\setcounter{tocdepth}{0}
\mtcsettitle{minitoc}{}

\renewcommand{\thechapter}{\Roman{chapter}}
\renewcommand{\thesubsection}{Exercice \arabic{subsection}}
%\renewcommand{\thesection}{\hspace{-11pt}}
%\renewcommand{\thesubsection}{}

\newcommand{\chaptertoc}[0]{
    \setcounter{minitocdepth}{4}
    \begin{tcolorbox}[
        enhanced,
        frame hidden,
        sharp corners,
        detach title,
        spread outwards     = 5pt,
        halign              = center,
        valign              = center,
        borderline west     = {3pt}{0pt}{main20!50!main2!95!gray!90},
        coltitle            = main20!50!main2!95!gray!90, 
        interior style      = {
            left color      = main1white2!65!gray!11,
            middle color    = main1white2!50!gray!10,
            right color     = main1white2!35!gray!9
        },
        arc                 = 0 cm,
        title               = SOMMAIRE,
        fonttitle           = \bfseries\sffamily,
        overlay             = {
            \node[rotate=90, minimum width=1cm, anchor=south,yshift=-0.8cm] at (frame.west) {\tcbtitle};
        }
    ]
        \begin{minipage}{0.83\linewidth}
            \sffamily
            \minitoc
        \end{minipage}
    \end{tcolorbox}
}

\begin{document}
    
    %==============================
    % METADONNEES
    %==============================
    
    \title{TD de Mathématiques de MPI/MPI* (2022-2023)}
    \author{SIAHAAN--GENSOLLEN Rémy}
    \date{\today}
    \hypersetup{
        pdftitle={TD de Mathématiques de MPI/MPI* (2022-2023)},
        pdfauthor={SIAHAAN--GENSOLLEN Rémy},
        pdflang={fr-FR},
        pdfsubject={MPI/MPI*, TD de Mathématiques},
        pdfkeywords={MPI/MPI*, TD de Mathématiques, 2022-2023}
        pdfstartview=
    }
    
    %==============================
    % MISE EN PAGE
    %==============================
    
    \titleformat{\chapter}[display]{\normalfont\huge\bfseries}{}{0pt}{
        \begin{tcolorbox}[
            enhanced,
            frame hidden,
            sharp corners,
            spread sidewards    = 5pt,
            halign              = center,
            valign              = center,
            interior style      = {color=main10!20},
            arc                 = 0 cm,
            fontupper           = \color{black}\sffamily\bfseries\huge,
            fonttitle           = \normalfont\color{white}\sffamily\small,
            top                 = 1cm, 
            bottom              = 0.7cm,
            title               = \thechapter,
            attach boxed title to bottom center = {
                yshift=\tcboxedtitleheight/2,
            },
            boxed title style = {
                frame code={
                \path[left color=main21!95!gray!90,right color=main21!95!gray!90] 
                    ([xshift=-10mm]frame.north west) -- 
                    ([xshift=10mm]frame.north east) -- 
                    ([xshift=10mm]frame.south east) -- 
                    ([xshift=-10mm]frame.south west) -- 
                    cycle;
                },
                interior engine=empty
            }
        ]
            #1
        \end{tcolorbox}%
    }
    \titlespacing*{\chapter}{0pt}{-120pt}{-15pt}
    \titleformat{name=\chapter,numberless}[display]{\normalfont\huge\bfseries}{}{0pt}{
        \begin{tcolorbox}[
            enhanced,
            frame hidden,
            sharp corners,
            spread sidewards    = 5pt,
            halign              = center,
            valign              = center,
            interior style      = {color=main10!20},
            arc                 = 0 cm,
            outer arc           = 0pt,
            leftrule            = 0pt,
            rightrule           = 0pt,
            fontupper           = \color{black}\sffamily\bfseries\huge,
            enlarge left by     = -1in-\hoffset-\oddsidemargin, 
            enlarge right by    = -\paperwidth+1in+\hoffset+\oddsidemargin+\textwidth,
            width               = \paperwidth, 
            left                = 1in+\hoffset+\oddsidemargin, 
            right               = \paperwidth-1in-\hoffset-\oddsidemargin-\textwidth,
            top                 = 1cm, 
            bottom              = 1cm
        ]
            #1
        \end{tcolorbox}%
    }
    \titlespacing*{name=\chapter,numberless}{0pt}{-115pt}{0pt}
    
    %==============================
    % PREMIERE DE COUVERTURE
    %==============================

    %\includepdf[pages={1},scale=1.15,offset=0mm -18mm]{LDCCover.pdf}
    
    %==============================
    % PAGE VIDE
    %==============================
    
    %\pagestyle{empty}
    
    %==============================
    % PAGE DE COUVERTURE INTERNE
    %==============================
    
    \begin{titlepage}
	    \begin{center}
	        {\scshape SIAHAAN--GENSOLLEN Rémy\par}
	        \vspace{2cm}
	        {\huge\sffamily TD de\par}
	        \vspace{0.5cm}
	        {\Huge\bfseries\sffamily MATHÉMATIQUES\par}
	        \vspace{1cm}
	        {\Large\textit{donné pendant mon année de \textsf{MPI/MPI*} à
	        Janson-de-Sailly}\\[5pt]\texttt{(2022-2023)}\par}
	        \vfill
	        {\large\EBGaramond Dernière compilation le \today\par}
        \end{center}
    \end{titlepage}
    
    %==============================
    % PAGE VIDE
    %==============================
    
    \pagestyle{empty}\text{}\newpage
    
    %==============================
    % STYLE DES EN-TÊTES ET PIEDS DE PAGES
    %==============================
    
    \renewcommand\chaptermark[1]{\markboth{#1}{}}
    
    \fancypagestyle{intro}{
        \fancyhf{}
        \renewcommand{\headrulewidth}{0pt}
        \renewcommand{\footrulewidth}{0pt}\fancyfoot[RO,LE]{\GillSansMTMedium\color{white5}\thepage\;/\;\pageref{LastPage}}
        \fancyhead[LE]{\GillSansMTMedium\color{white5}\bfseries TD DE MATHÉMATIQUES}
        \fancyhead[RE]{\GillSansMTMedium\color{white5}Avant-propos}
        \fancyhead[LO]{\GillSansMTMedium\color{white5}\rightmark}
        \fancyhead[RO]{\GillSansMTMedium\color{white5}\textbf{MPI/MPI*} 2022-2023 \quad Janson-de-Sailly}
    }
    
    \fancypagestyle{toc}{
        \fancyhf{}
        \renewcommand{\headrulewidth}{0pt}
        \renewcommand{\footrulewidth}{0pt}\fancyfoot[RO,LE]{\GillSansMTMedium\color{white5}\thepage\;/\;\pageref{LastPage}}
        \fancyhead[LE]{\GillSansMTMedium\color{white5}\bfseries TD DE MATHÉMATIQUES}
        \fancyhead[RE]{\GillSansMTMedium\color{white5}Table des matières}
        \fancyhead[LO]{\GillSansMTMedium\color{white5}\rightmark}
        \fancyhead[RO]{\GillSansMTMedium\color{white5}\textbf{MPI/MPI*} 2022-2023 \quad Janson-de-Sailly}
    }
    
    \fancypagestyle{plain}{
        \fancyhf{}
        \renewcommand{\headrulewidth}{0pt}
        \renewcommand{\footrulewidth}{0pt}\fancyfoot[RO,LE]{\GillSansMTMedium\color{white5}\thepage\;/\;\pageref{LastPage}}
        \fancyhead[LE]{\GillSansMTMedium\color{white5}\bfseries TD DE MATHÉMATIQUES}
        \fancyhead[RE]{\GillSansMTMedium\color{white5}Chapitre \thechapter : \nouppercase{\leftmark}}
        \fancyhead[LO]{\GillSansMTMedium\color{white5}\nouppercase{\rightmark}}
        \fancyhead[RO]{\GillSansMTMedium\color{white5}\textbf{MPI/MPI*} 2022-2023 \quad Janson-de-Sailly}
    }
    
    %==============================
    % PREFACE 
    %==============================
    
    \chapter*{Avant-propos}
    \thispagestyle{intro}
    \addcontentsline{toc}{chapter}{Avant-propos}
    
    \text{\Large\EBGaramond\itshape À tout lecteur potentiel, quelques mots...}\newline\newline\newline
    
    \begin{center}
        \begin{minipage}{0.85\linewidth}
            \large \qquad Ce livre contient la résolution des exercices de TD donnés pendant les cours de mathématiques de mon année de MPI/MPI*. Il vient en complément du livre de cours associé.\newline\newline\newline\text{}
        \end{minipage}
    \end{center}
    
    \hfill{\large\textsc{Siahaan--Gensollen Rémy}}
    
    \pagestyle{intro}
    
    %==============================
    % TABLE DES MATIERES
    %==============================
    
    \newpage
    \dominitoc\nomtcrule 
    {\sffamily\tableofcontents}\mtcaddchapter\pagestyle{toc}
    
    \cleardoublepage
    
    %==============================
    % COURS
    %==============================
    
    \pagestyle{plain}
    
    \chapter{Algèbre générale}
    
    \chapter{Espaces vectoriels généraux}
    
    \subsection{}
    
    \begin{enumerate}
        \item Soient $\p{A, B} \in \bcM_k\p{\bdR}$, avec $B$ de rang $r$ et $P\p{t} = \det{A + tB}$. En écrivant $B = PJ_rQ$ avec $P$ et $Q$ deux matrices inversibles, montrer que $P$ est un polynôme en $t$ de degré au plus $r$.
        
        \item Exemple : $r = 1$. Montrer que si $B = J_1$ alors $\det{a + tJ_1} = \det{A} + \lambda t$ où $\lambda \in \bdR$.
        
        \item En déduire $\det{A + J_1}\det{A - J_1} \leq \det{A^2}$ et caractériser les cas d'égalité.
        
        \item Montrer que $\det{A + B}\det{A - B} \leq \det{A}^2$ si $\rg{B} = 1$.
    \end{enumerate}
    
    \subsection{}
    
    \begin{enumerate}
        \item Soient $E$ un $\bbK$-espace vectoriel, et $\p{F, G}$ deux sous espaces vectoriels. Montrer que $F \cup G$ est un sous espace vectoriel si et seulement si $F \subset G$ ou $G \subset F$.
    \end{enumerate}
    
    \subsection{}
    
    \subsection{}
    
    Soit une matrice $A \in \bcM_n\p{\bdR}$ telle que $\forall i \in \iint{1, n}$, on a $\mod{a_{i, i}} < \displaystyle\sum_{j = 1,\ j \neq i}^n \mod{a_{i, j}}$.
    
    \begin{enumerate}
        \item Soit $\begin{pNiceMatrix}
            x_1 \\ x_2 \\ \Vdots \\ x_n
        \end{pNiceMatrix} \in \Ker A$. Montrer qu'il existe $i \in \iint{1, n}$ tel que $\mod{x_i} = \sup\limits_J \mod{x_j}$. On note $M$ ce $\sup$.
    \end{enumerate}
    
    \chapter{Diagonalisation}
    
    \subsection{}
    
    Soit $u \in \bcL\p{E}$ et $\varphi : \begin{array}[t]{rcl}
        \bcL\p{e} &\to& \bcL\p{E}  \\
        v &\mapsto& uv 
    \end{array}$. Déterminer les valeurs propres de $\varphi$.
    %
    Résolvons $uv = \lambda v$. On a :
    %
    \[ uv = \lambda v \iff uv - \lambda v = 0 \iff \]
    
    
    \subsection{}
    
    On se place sur $\bdR_{2n}\intc{X}$, soit $u \p{P} = \p{X^2 - 1}P' - 2nXP$.
    
    \subsection{}
    
    \newpage
    
    \chapter{Topologie}
    
    \subsection{Exercice 11}
    
    Soit $E$ un espace vectoriel normé et $C \subset E$ une partie convexe.
    
    \chapter{Séries}
    
    \subsection*{Exercice 21}
    
    Soit $E$ l'espace vectoriel des suites vérifiant $u_{n+2} = u_n + u_{n+1}$. On note $u \in E$ la suite vérifiant $u_0 = 1$, $u_1 = 0$ et $v \in E$ la suite vérifiant $v_0 = 0$, $v_1 = 1$.
    
    \begin{enumerate}
        \item Montrer que $\p{u, v}$ est une base de $E$.
        
        \boxans{
            Une suite $w \in E$, est entièrement déterminée par $w_0$ et $w_1$. On a donc $w = w_1u + w_0v$ (générateur).
            
            Si $\lambda u + \mu v = 0$, alors en $n = 1$ on obtient $\lambda = 0$ et en $n = 0$ on obtient $\mu = 0$ (libre).
        }
        
        \item Montrer que $u_{n+1}v_n - u_nv_{n+1} = \p{-1}^{n+1}$.
        
        \boxans{
            Pour $n = 0$, on a $u_1v_0 - u_0v_1 = 0 - 1 = \p{-1}^1$. Donc l'hypothèse est vérifiée.
            
            Pour $n = 1$, on a $u_2v_1 - u_1v_2 = u_2 = u_1$
            
            
            Supposons l'hypothèse vérifiée jusqu'à un rang $n \geq 2$. On a :
            %
            \[ u_{n+2}v_{n+1} - u_{n+1}v_{n+2} = \p{nu_n + u_{n+1}}v_{n+1} - u_{n+1}\p{nv_n + v_{n+1}} = n\p{u_nv_{n+1} - u_{n+1}v_n} = \p{-1}^{n+2}\]
        }
        
        \item Montrer que $u_n \geq n - 2$ et $v_n \geq n - 1$ puis $n^2 = \O{}{u_n}$ et $n^2 = \O{}{v_n}$.
        
        \boxans{
            Par récurrence. On a $u_0 \geq -1$, $u_1 \geq 0$. Si la propriété est vraie jusqu'au rang $n+1$ avec $n \in \bdN$, alors $u_{n+2} = u_{n+1} + u_n \geq n + n - 1 = 2n - 1 \geq n - 2$, \etc . De même pour $v_n$.
            
            On 
        }
        
        \item Montrer la convergence de la série $\sum \dfrac{u_{n+1}}{v_{n+1}} - \dfrac{u_n}{v_n}$. On note $L$ sa limite, qui est la limite de $\dfrac{u_n}{v_n}$.
        
        \boxans{
            On a $\dfrac{u_{n+1}}{v_{n+1}} - \dfrac{u_n}{v_n} = \dfrac{u_{n+1}v_n - u_nv_{n+1}}{v_{n+1}v_n} = \dfrac{\p{-1}^{n+1}}{v_nv_{n+1}} \leq \dfrac{1}{n\p{n-1}}$ CVA.
        }
        
        \item En utilisant les séries alternées, montrer $v_nL - u_n \lima 0$ donc $L - \dfrac{u_n}{v_n} = \o{\dfrac{1}{v_n}}$.
        
        \boxans{
            $v_nL - u_n$
        }
        
        \item Soit $w = au + bv$. En écrivant $w = v\p{b + a\dfrac{u}{v}}$, établir la convergence et calculer la limite de $w$ (éventuellement infinie).
        
        \boxans{
            
        }
    \end{enumerate}
    
    \subsection*{Produits infinis}
    
    On considère une suite $u_n \in \bdR$ ou $\bdC$. On pose $P_n = \prod\limits_{k=1}^n u_k$.
    
    On parle de produit convergeant si $P_n \to L \neq 0$ et de produit divergeant sinon.
    
    Produit convergeant $\implies u_n \to 1$. On considère $\prod\limits_{k=1}^\infty \p{1 + u_k}$.\bigskip
    
    Cas réel $\displaystyle\prod\limits_{k \geq k_0} \p{1 + u_k}$. Quand est-ce qu'il y a convergence ?
    
    Pour CNS : utilisation du logarithme. Avec $u_k$ positif. 
    On a $\ln{\displaystyle\prod\limits_{k \geq k_0} \p{1 + u_k}} = \sum\limits_{k \geq k_0} \ln{1 + u_k}$, qui converge ssi (par équivalent) $\displaystyle\sum\limits_{k \geq k_0} u_k$ converge.\bigskip
    
    Considérons $\displaystyle\prod\limits_{k \geq 2} \p{1 + \dfrac{\p{-1}^k}{k^\alpha}}$. On a $\alpha > 0$ ou divergence grossière. Donc somme $\displaystyle\sum_{k \geq 2} \ln{1 + \dfrac{\p{-1}^k}{k^\alpha}}$.
    
    Or $\ln{1 + \dfrac{\p{-1}^k}{k^\alpha}} \asymp \dfrac{\p{-1}^k}{k^\alpha}$. donc CVA si $\alpha > 1$ et non CVA si $0 \leq \alpha \leq 1$.
    
    On a $\ln{1 + \dfrac{\p{-1}^k}{k^\alpha}} = \underbrace{\dfrac{\p{-1}^k}{k^\alpha}}_{\text{CVS}} - \underbrace{\dfrac{1}{2k^{2\alpha}}} + \o{}{\dfrac{1}{k^{2\alpha}}}$ 
    
    
    $u_n = \dfrac{1! + \dots + n!}{\p{n+2}!} = \dfrac{1! + \dots + {n-1}!}{\p{n+2}!} + \dfrac{n!}{\p{n+2}!}$
    
    
    \subsection*{Bijection}
    
    Soit $\sigma$ une bijection de $\bdN^\star$. Montrer que $\sum\limits_{n \geq 1} \dfrac{\sigma\p{n}}{n^2}$ diverge.
    
    Soit $S_n = \dfrac{\sigma\p{1}}{1^2} + \dots + \dfrac{\sigma\p{n}}{n^2}$ et $A_n = \sigma\p{1} + \dots + \sigma\p{2}$.
    
    On a $\sigma\p{n} = A_n - A_{n-1}$
    
    \subsection*{Intégrale sinus cardinal}
    
    $\displaystyle\int_0^\infty \dfrac{\sin t}{t}\dif t$ -> IPP
    
    \[ \int_a^A \dfrac{\sin t}{t}\dif t = \left[\vphantom{\dfrac{a}{b}}\right.\underbrace{- \dfrac{\cos t}{t}}_{\lima{t \to \infty} 0}\left.\vphantom{\dfrac{a}{b}}\right]_a^A - \int_a^A \underbrace{\dfrac{\cos t}{t^2}\dif t}_{\substack{L^1 \ \text{sur} \ \intor{a, \infty}\\ \text{car} \ \O{}{\frac{1}{t^2}}}}\]
    %
    Pour $A \to \infty$ : \qquad $\displaystyle \int_a^\infty \dfrac{\sin t}{t}\dif t = \dfrac{\cos a}{a} - \int_a^\infty \dfrac{\cos t}{t^2}\dif t \qquad\et\qquad \int_0^\infty \dfrac{\sin t}{t}\dif t$ SCV.
    
    

    
    \subsection{Etude}
    
    Soit $F\p{x} = \displaystyle\int_0^\infty \dfrac{\sin u}{x^2 + u^2}\dif u$
    
    \begin{enumerate}
        \item Continuité ? Tout d'abord, existence. On pose $f\p{x, uu} = \dfrac{\sin u}{x^2 + u^2}$. On a $u \mapsto f\p{x, u}$ $\bcC^0$ sur $\into{0, +\infty}$ si $x = 0$, $\intor{0, +\infty}$ si $x > 0$.
        
        \begin{enumerate}
            \ithand Si $x = 0$, étude en $0$ : $\dfrac{\sin u}{u^2} \asymp $
        \end{enumerate}
    \end{enumerate}
    
    \newpage
    
    \subsection*{Exercice 9}
    
    Étude de $F\p{x} = \displaystyle\sum_{n \geq 0} \p{\arctan{n + x} - \arctan{n}}$
    
    \begin{enumerate}
        \item Établir la convergence simple, en utilisant par exemple $\arctan\p{x} - \arctan{\dfrac{1}{x}} = \mathrm{sgn}\p{x}\dfrac{\pi}{2}$.
        
        \boxansconc{
            Soit $x \in \bdR_+$, on a $\mathrm{sgn}\p{x} = 1$ et $\arctan{x} = \dfrac{\pi}{2} - \arctan\p{\dfrac{1}{x}}$. Ainsi pour $N \in \bdN$, on a 
            %
            \begin{align*}
                \sum_{n = 0}^N \p{\arctan{n+x} - \arctan{n}} &= \sum_{n=0}^N \p{\dfrac{\pi}{2} - \arctan{\dfrac{1}{n+x}} - \dfrac{\pi}{2} + \arctan{\dfrac{1}{n}}}\\
                &= \sum_{n = 0}^N \p{\arctan{\dfrac{1}{n}} - \arctan{\dfrac{1}{n+x}}}
            \end{align*}
        }
    \end{enumerate}
    
    \chapter{Interversion}
    
    \subsection{}
    
    A l’aide d’une interversion série / intégrale, calculer $\displaystyle\int_0^1 \dfrac{\ln t}{t- 1}\dif t$.
    
    \boxansconc{
        Posons $f_n : \begin{array}[t]{rcl}
            \intol{0, 1} &\to& \bdR   \\
            t &\mapsto & t^n\ln{\frac{1}{t}} 
        \end{array}$. Pour tout $n \in \bdN$, la fonction $f_n$ est intégrable sur $\intol{a, 1}$ avec $a > 0$. 
        
        De plus $f_n = \O{t \to 0}\p{\dfrac{1}{t^2}}$ donc par comparaison à une intégrale de \textsc{Riemann}, $f_n$ est intégrable sur $\intol{0, 1}$. Par ailleurs, $f_n$ est continue sur $\intol{0, 1}$. Remarquons que pour $N \in \bdN$, on a :
        %
        \[ \sum_{n=0}^N f_n = \sum_{n=0}^N t^n\ln{\dfrac{1}{t}} = -\ln{t}\sum_{n=0}^N t^n = -\ln{t}\dfrac{t^{N+1}-1}{t - 1} \lima{N \to +\infty} \dfrac{\ln t}{t - 1}\]
        %
        Pour tout $t \in \intol{0, 1}$, $\ln{\dfrac{1}{t}} = -\ln{t} \geq 0$ d'où $\mod{f_n\p{t}} = f_n\p{t} \geq 0$, d'où
        %
        \[ \int_0^1 \mod{f_n} = -\int_0^1 t^n\ln{t}\dif t = -\intc{\dfrac{t^{n+1}}{n+1}\ln t}_0^1 + \int_0^1 \dfrac{t^n}{n+1}\dif t = \intc{\dfrac{t^{n+1}}{\p{n+1^2}}}_0^1 = \dfrac{1}{\p{n+1}^2} = \O{n \to +\infty}{\frac{1}{n^2}}\]
        %
        Donc par comparaison à une série de \textsc{Riemann}, on a $\displaystyle\sum_{n = 0}^{+\infty} \int_0^1 \mod{f_n}$ est convergente.
        
        Par \emph{théorème d'interversion série-intégrale}, on a :
        %
        \[ \int_0^1 \sum_{n = 0}^{+\infty} f_n\p{t}\dif t = \int_0^1 \dfrac{\ln t}{t - 1}\dif t = \sum_{n = 0}^{+\infty} \int_0^1 f_n\p{t}\dif t = \sum_{n=0}^{+\infty} \dfrac{1}{\p{n+1}^2} = \sum_{n=1}^{+\infty} \dfrac{1}{n^2} = \zeta\p{2} = \dfrac{\pi^2}{6}\]
    }
    
    \subsection{}
    
    On pose, sous réserve d'existence, la fonction $F$ telle que $F\p{x} = \displaystyle\int_0^{+\infty} t^x \dfrac{e^{-t}}{1 + e^{-t}}\dif t$.
    
    \begin{enumerate}
        \item Montrer l'existence de $F\p{x}$ pour $x > -1$.
        
        \boxansconc{
            On a $t^{x + 2}\dfrac{e^{-t}}{1 + e^{-t}} = \O{t \to 0}$
        }
        
        \item Calculer $F\p{x}$ à l'aide de $\zeta\p{x + 1}$ et $\Gamma\p{x + 1}$ si $x > 0$.
        
        
    \end{enumerate}
    
    \subsection{}
    
    On rappelle que $\Gamma\p{x} = \displaystyle\int_0^{+\infty} t^{x-1}e^{-t}\dif t$. On pose $f_n\p{t} = \left\lbrace\begin{array}{cl}
        \p{1 - \dfrac{t}{n}}^n &\text{si} \ t \in \intc{0, n}  \\
        0 & \text{si} t > n
    \end{array}\right.$ pour 
    
    \subsection{}
    
    En posant $1 + t = e^u$ et en faisant une interversion série / intégrale, montrer que $\displaystyle\int_0^{+\infty} \dfrac{\ln{1 + t}}{t\sqrt{1 + t}}\dif t = \sum_{n \geq 0} \dfrac{1}{\p{2n + 1}^2}$.
    
    \subsection{TD exo 9}
    
    On pose $I_n = \displaystyle\int_0^{+\infty} \dfrac{ne^{-nt}}{1 + nt}\dif t$. On veut un DA de $I_n$.
    
    \chapter{Séries entières}
    
    \subsection{Rayon}
    
    Calculer le rayon $R$ de la série entière $\sum \sin{n\alpha}x^n$.
    
    \boxans{
        On a $R \geq 1$ car $\mod{\sin{n\alpha}}$.
        %
        \begin{enumerate}
            \item Pour $x = 1$, DV grossière ? Montrons que $\sin n\alpha \not\lima 0$ : si $\sin n\alpha \to 0$, alors
        \end{enumerate}
    }
    
    \subsection{Une série entière}
    
    $f\p{x} = \sum_{n \geq 1}$
    
    \begin{enumerate}
        \item $R = 1$
        
        \boxansconc{
            $1 \leq \ln n \leq n$ donc $1 \geq R \geq 1$
        }
        
        
    \end{enumerate}
    
    \newpage
    
    \subsection{Inverse}
    
    Soit $f\p{x} = \sum\limits_{n \geq 0} a_nx^n$ avec $a_0 \neq 0$ de rayon $R$. On prendra sans perte d'information $a_0 = 1$. But : $\dfrac{1}{f}$ DSE (Hors Programme).
    
    \begin{enumerate}
        \item $\exists ! \suite{b_n}$ tq $\forall n \in \bdN$, $\displaystyle\sum_{p +q = n} a_pb_q = \delta_{0,n}$.
        
        \boxansconc{
            Pour $n = 0$, il existe uniquement $b_q = \dfrac{1}{a_0} = 1$. (seul possibilité : $p = 0$ et $q = 0$.
            
            Ensuite pour $n = 1$, on a $a_0b_1 + a_1b_0 = 0$ d'où $b_1 = -\dfrac{a_1b_0}{a_0} = -a_1$.
            
            On suppose la propriété valide au rang $n \in \bdN$. Donc $\exists ! \p{b_0, b_1, \dots, b_n}$ qui satisfont la propriété.
            %
            Au rang $n+1$, on a $b_{n+1}a_0 + b_na_1 + \dots + b_0a_{n+1} = 0$ avec lequel on construit $b_{n+1}$. On a toujours :
            %
            \[ b_n = - \sum_{i = 0}^{n-1} b_ia_{n-i}\]
        }
        
        \item Si $R > 1$, montrer que $\sum b_nx^n$ est de rayon $R_b \neq 0$.
        
        
        
        \item Conclure.
        
        \boxansconc{
            Produit de \textsc{Cauchy}.
        }
    \end{enumerate}
    
    \section{Un DSE}
    
    Soit $p \in \bdR$, $f\p{x} = \p{x + \sqrt{1 + x^2}}^p$. On a :
    %
    \[f'\p{x} = p\p{1 + \dfrac{x}{\sqrt{1 + x^2}}}\p{x + \sqrt{1 + x^2}}^{p-1} = \dfrac{pf\p{x}}{\sqrt{x^2 + 1}}\]
    %
    Donc $\sqrt{x^2 + 1}f'\p{x} = pf\p{x}$, donc en dérivant :
    %
    \[ \dfrac{x}{\sqrt{1 + x^2}}f'\p{x} + \sqrt{x^2 + 1}f''\p{x} = pf'\p{x}\]
    %
    Ainsi $xf'\p{x} + \p{1 + x^2}f''\p{x} = p^2f\p{x}$, on a donc l'équation diff :
    %
    \[ \p{1 + x^2}y'' + xy' = p^2f\]
    
    
    \subsection{}

    \begin{equation}
        \dfrac{\partial^n}{\partial x^n}\left(f\circ g\right)(x) = \sum_{m_k} \dfrac{n!}{\prod m_k!} f^{(\sum m_k)}\circ g(x) \prod \left(\dfrac{g^{(j)}(x)}{j!}\right)^{m_j}
    \end{equation}
    
    \newpage
    
    \chapter{Espaces euclidiens}
    
    \subsection{}
    
    \underline{Rappel :} 
    %
    \[ \Tr{u} = 0 \iff \exists \bcB,\qquad \Mat_\bcB u = \begin{pNiceMatrix} 0 & & \star \\
     & \Ddots & \\
    \star & & 0\end{pNiceMatrix} \]
    %
    Pour $u \in \bcS$, montrer que $\Tr{u} = 0 \iff \exists \bcB \ \text{BON},\qquad \Mat_\bcB u =  \begin{pNiceMatrix} 0 & & \star \\
     & \Ddots & \\
    \star & & 0\end{pNiceMatrix}$.
    
    \newpage
    
    Posons $M $

    \chapter{Arithmétique}

    \subsection{Symbole de Pochhammer}

    Soit $\p{A, +, \times}$ un anneau commutatif. pour $x \in A$et $n \in \bdN$, on définit $x^{(n)}$ par :
    %
    \[ x^{(0)} = 1_A,\qquad x^{(n)} = x\p{x-1}\cdots\p{x - n + 1}\]

    Montrer que pour tous $x$ et $y \in A$ et tout $n \in \bdN$, on a :
    %
    \[ \p{x + y}^{(n)} = \sum_{k = 0}^n \binom{n}{k}x^{(k)}y^{(n-k)}\]

    \boxans{
        Soient $\p{x, y} \in A^2$. Pour $n \in \bdN$, on note $\bcP\p{n}$ le prédicat 
        %
        \[ \bcP\p{n} : \p{x + y}^{(n)} = \sum_{k = 0}^n \binom{n}{k}x^{(k)}y^{(n-k)}\]
        %
        On a $\bcP\p{0}$ car $\p{x + y}^{(0)} = 1_A$ et $\binom{0}{0}x^{(0)}y^{(0)} = 1_A1_A = 1_A$.
        
        Supposons alors que la propriété est vrai jusqu'au rang $n \in \bdN$. On a :
        %
        \begin{align*}
            \p{x + y}^{(n+1)} &= \p{x + y}^{(n)}\p{x + y - n} = \p{\sum_{k = 0}^n \binom{n}{k}x^{(k)}y^{(n-k)}}\p{x + y - n}\\
            &= \sum_{k = 0}^n \binom{n}{k}x^{(k)}y^{(n-k)}\p{x + y - n} = \sum_{k = 0}^n \binom{n}{k}x^{(k)}y^{(n-k)}\p{x -k + y - \p{n - k}}\\
            &= \sum_{k = 0}^n \binom{n}{k}x^{(k)}\p{x -k}y^{(n-k)} + \sum_{k = 0}^n \binom{n}{k}x^{(k)}y^{(n-k)}\p{y - \p{n -k }}\\
            &= \sum_{k = 0}^n \binom{n}{k}x^{(k + 1)}y^{(n-k)} + \sum_{k = 0}^n \binom{n}{k}x^{(k)}y^{(n-k + 1)}\\
            &= \sum_{k = 1}^{n+1} \binom{n}{k-1}x^{(k)}y^{(n-k+1)} + \sum_{k = 0}^n \binom{n}{k}x^{(k)}y^{(n-k + 1)}\\
            &= x^{(n+1)} + \sum_{k = 1}^{n} \binom{n+1}{k}x^{(k)}y^{(n+1-k)} + y^{(n+1)} = \sum_{k = 0}^{n+1} \binom{n+1}{k}x^{(k)}y^{(n+1-k)}
        \end{align*}
    }
    
    \newpage
    
    \[ \chi_{M\p{z}} = \begin{vNiceMatrix}X & 0 &-z\\-1 & X - 1 & 0\\ 0 & -1 & X\end{vNiceMatrix} = X\begin{vNiceMatrix}X - 1& 0\\-1 & X \end{vNiceMatrix} -z\begin{vNiceMatrix}-1 & X - 1\\0 & -1\end{vNiceMatrix} = X\p{X-1}X - z\p{-1}^2 = X^3 - X^2 - z\]
    
    On a $\chi_{M\p{z}}' = 3X^2 - 2X^2 = X\p{3X - 2}$ de racines $0$ et $\sfrac{2}{3}$.
    
    Ça c'est un $\chi$ et ça c'est un $\xi$
    
    \[ \rg\begin{pNiceMatrix}\sfrac{2}{3} & 0 & \sfrac{4}{27}\\ - 1 & -\sfrac{1}{3} & 0\\ 0 & -1 & \sfrac{2}{3}\end{pNiceMatrix} = \rg\begin{pNiceMatrix}\sfrac{2}{3} & 0 & \sfrac{4}{27}\\ 0 & -\sfrac{1}{3} & \sfrac{2}{9}\\ 0 & -1 & \sfrac{2}{3}\end{pNiceMatrix} = \rg\begin{pNiceMatrix}\sfrac{2}{3} & 0 & \sfrac{4}{27}\\ 0 & -\sfrac{1}{3} & \sfrac{2}{9}\\ 0 & 0 & 0\end{pNiceMatrix} \]
    
    \[ \bbP\p{Y = j} = \sum_{k \in \bdN^*} \bbP\p{X = k}\bbP\p{Y = j \enstq X = k} = \sum_{k=1}^j \dfrac{1}{n}\dfrac{1}{k} = \sum_{k=1}^j \dfrac{1}{kn} \]

    \[ \bdE\p{Y} = \dfrac{1}{n}\sum_{j=1}^n \sum_{k=1}^j \dfrac{j}{k} = \dfrac{1}{n}\sum_{1 \leq k \leq j \leq n} \dfrac{j}{k} = \dfrac{1}{n}\sum_{k=1}^n\dfrac{1}{k}\sum_{j=k}^n j = \dfrac{1}{n}\sum_{k=1}^n \dfrac{\p{n-k+1}\p{n+k}}{2k}\]


    \newpage 

    \[  \]
    
    
    \chapter{Calcul différentiel}
    
    \subsection{Polynômes}
    
    Soit $f: P \in \bdR\intc{X} \mapsto \displaystyle\int_0^1 P^3$. Déterminer $\dif f_P\p{Q}$.
    
    \boxans{
        \begin{align*}
        f\p{P + H} &= \int_0^1 \p{P + H}^3\\
        &= \int_0^1 P^3 + \int_0^1 3P^2 H + \int_0^1 PH^2 + \int_0^1 H^3\\
        &= f\p{P} + \int_0^1 3P^2 H + 3\int_0^1 PH^2 + \bco\p{\norm{H}}\\
        &= f\p{P} + \underbrace{\int_0^1 3P^2 H}_{\dif f_P\p{H}} + \bco\p{\norm{H}}
        \end{align*}
    }
    
    \subsection{Equation différentielle}
    
    Soient $g$ et $h$ de classe $\bcC^2$ et $f\p{x, y} = g\p{\dfrac{y}{x}} + xh\p{\dfrac{y}{x}}$.
    
    \begin{enumerate}
        \item Montrer que $x^2\dfrac{\partial^2 f}{\partial x^2} + 2xy\dfrac{\partial^2 f}{\partial x\partial y} + y^2\dfrac{\partial^2 f}{\partial y^2} = 0$
        
        \boxans{
            Posons $t = \dfrac{y}{x}$, d'où $\dfrac{\partial t}{\partial x} = -\dfrac{y}{x^2} = -\dfrac{t}{x}$ et $\dfrac{\partial t}{\partial y} = \dfrac{1}{x}$. On a d'une part :
            %
            \[ \dfrac{\partial f}{\partial x} = \dfrac{\partial t}{\partial x}\dfrac{\partial g}{\partial t}\p{t} + h\p{t} + x\dfrac{\partial t}{\partial x}\dfrac{\partial h}{\partial t}\p{t} = -\dfrac{y}{x^2}g'\p{\dfrac{y}{x}} + h\p{\dfrac{y}{x}} - \dfrac{y}{x}h'\p{\dfrac{y}{x}} \]
            %
            D'autre part :
            %
            \[ \dfrac{\partial f}{\partial y} = \dfrac{\partial t}{\partial y}\dfrac{\partial g}{\partial t}\p{t} + x\dfrac{\partial t}{\partial y}\dfrac{\partial h}{\partial t}\p{t} = \dfrac{1}{x}g'\p{\dfrac{y}{x}} + h'\p{\dfrac{y}{x}}\]
            %
            On obtient alors :
            %
            \[ \dfrac{\partial^2 f}{\partial x\partial y} = -\dfrac{1}{x^2}g'\p{t} + \dfrac{1}{x}\dfrac{\partial t}{\partial x}\dfrac{\partial g'}{\partial t}\p{t} + \dfrac{\partial t}{\partial x}\dfrac{\partial h'}{\partial t}\p{t} = -\dfrac{1}{x^2}g'\p{\dfrac{y}{x}} - \dfrac{y}{x^3}g''\p{\dfrac{y}{x}} -\dfrac{y}{x^2}h''\p{\dfrac{y}{x}}\]
        }
    \end{enumerate}
    
    
    \newpage
    
    \[ \dfrac{\partial f}{\partial x}\p{x, y} = -a\dfrac{\partial f}{\partial y}\p{x, y} \]
    
    Si $f\p{x, y} = -af\p{y, x}$

    \newpage
    
    \begin{align*}
        f\p{X + H} &= \phyavg{AX + AH \mid X + H} + \phyavg{U \mid X + H} = \phyavg{AX \mid X} + \phyavg{AX \mid H} + \phyavg{AH \mid X} + \phyavg{AH \mid H} + \phyavg{U \mid X} + \phyavg{U \mid H}\\
        &= f\p{X} + f\p{H} + \phyavg{AX \mid H} + \phyavg{AH \mid X}\\
        &= f\p{X} + \phyavg{2AX \mid H} + f\p{H} = f\p{X} + \phyavg{\textrm{Grad} f\p{X} \mid H} + \bco\p{\norm{H}}
    \end{align*}
    %
    Donc $\textrm{Grad} f\p{X} = 2AX$.
    \newpage
    
    \begin{form}{Un peu de topologie}{}
        \begin{center}
            \hg{Montrons que l'adhérence de $\suite{e^{\ii \sqrt{n}}}$ est le cercle unité $\bdU$.} 
        \end{center}
        %
        \medskip
        
        Soit $z \in \bdU$, on écrit $z = e^{\ii \theta}$. On pose $t_k = \p{2k\pi + \theta}^2$. La suite des $\p{t_k}_{k \in \bdN}$ est croissante et diverge vers $+\infty$, tout en vérifiant :
        %
        \[ \forall k \in \bdN,\qquad e^{\ii\sqrt{t_k}} = e^{\ii \theta} = z \]
        %
        Soit $k \in \bdN$. On pose $n = \floor{t_k} + 1$ l'entier supérieur le plus proche de $t_k$. Par la formule de l'arc moitié :
        %
        \[ \mod{e^{\ii\sqrt{n}} - e^{\ii \theta}} = \mod{e^{\ii\sqrt{n}} - e^{\ii \sqrt{t_k}}} = 2\mod{\sin{\sqrt{n} - \sqrt{t_k}}}\mod{e^{\ii\frac{\sqrt{n} + \sqrt{t_k}}{2}}} = 2\sin{\sqrt{n} - \sqrt{t_k}}\]
        %
        Par ailleurs le théorème des accroissements finis livre l'existence d'un $c_k \in \intc{t_k, n}$ tel que
        %
        \[ \sqrt{n} - \sqrt{t_k} = \dfrac{n - t_k}{2\sqrt{c_k}} \leq \dfrac{n - t_k}{2\sqrt{t_k}} \leq \dfrac{1}{2\sqrt{t_k}} \lima{k \to +\infty} 0 \]
        %
        Soient $\epsilon > 0$ et $n_0 \in \bdN$ On sait que $\sin{x} \lima{x \to 0} 0$ donc posons $\eta > 0$ tel que $\forall x \in \intc{0, \eta}$, $\sin{x} \leq \dfrac{\epsilon}{2}$. On pose 
        %
        \[ k_0 = \min\ens{k \in \bdN \enstq \dfrac{1}{2\sqrt{t_k}} \leq \eta \et t_k \geq n_0}\]
        %
        Dès lors $n \geq t_k \geq n_0$ et $\sqrt{n} - \sqrt{t_k} \leq \eta$ d'où :
        %
        \[ \mod{e^{\ii\sqrt{n}} - z} = \mod{e^{\ii\sqrt{n}} - e^{\ii \theta}} = 2\sin{\sqrt{n} - \sqrt{t_k}} \leq 2\sin{\eta} \leq 2\dfrac{\epsilon}{2} = \epsilon \]
        %
        Donc $z$ est valeur d'adhérence de la suite $\suite{e^{\ii \sqrt{n}}}$. Finalement 
        %
        \[ \hg{\overline{\ens{e^{\ii \sqrt{n}},\ n \in \bdN}} = \bdU}\]
    \end{form}
    
     \chapter{Oral}

     \subsection{510}

     \begin{enumerate}
         \item Calculer \(\p{u - \Id}v_n\).

         \boxansconc{
            \[ \p{u - \Id}v_n = \dfrac{1}{n+1}\p{u - \Id}\sum_{k=0}^n u^k = \dfrac{1}{n+1}\intc{\sum_{k=0}^n u^{k+1} - \sum_{k=0}^n u^k} = \dfrac{u^{n+1} - \id}{n+1}\]
         }

         \item Montrer que $\Ker\p{u - \Id}$ et $\Imm\p{u - \Id}$ sont en somme directe.

         \boxansconc{
            Soit $x \in \Ker\p{u - \Id}$ donc $u\p{x} = x$. On suppose de plus qu'il existe $y \in E$ tel que $u\p{y} - y = x$.

            \[ u\p{y} = x + y \qquad\text{donc}\qquad \norm{x + y} = \norm{u\p{y}} \leq \norm{y}\]

            \[ u\p{y} - u\p{x} = y \qquad\text{donc}\qquad \norm{y} = \norm{u\p{y} - u\p{x}} \leq \norm{y - x} \]


            \[ \norm{u(y)} = \norm{u(y) - y + y} \leq \norm{x} + \norm{y}\]
            u(x) <= u(x) - x + x <= || u(x) - (x) || + ||x||
         }
     \end{enumerate}
    
    
    
    

\end{document}


