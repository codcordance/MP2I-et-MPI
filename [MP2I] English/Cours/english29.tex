\documentclass[a4paper,english,bookmarks]{article}
\usepackage{./Structure/4PE18TEXTB}
\usepackage{marginnote}

\begin{document}

\stylizeDoc{English}{"An apple a day"}{Tuesday 29 March 2022}
\newboxans

\newcommand{\wordcount}[1]{\reversemarginpar\marginnote{\color{white5}#1}\normalmarginpar}

\hfill\newline

\section*{\centering\EBGaramond\Large Thème}

\begin{enumerate}
    \begin{minipage}[t]{0.45\linewidth}
        \item Trop d'informations et de conseils empêchent les étudiants de devenir autonomes. Laissons-les libres d'apprendre à leur rythme.
    
        \boxans{
            Too much information and advice prevent students from becoming independent. Let them be free to learn at their own pace.
        }
    
        \item Je regrette qu'il n'y ait pas plus de femmes dans les parlements européens, mais que peut-on faire ?
    
        \boxans{
            I regret that there are not more women in European parliaments, but what can we do?
        }
    
        \item Quand le pays aura surmonté\footnote{En anglais, présent dans une subordonnée lorsque le français utilise du futur} la crise économique, peut-être le gouvernement s'attaquera-t-il aux inégalités ?
    
        \boxans{
            When the country has overcome the economic crisis, perhaps will the government tackle the issue of inequalities?
        }
    \end{minipage}
    %
    \hfill
    %
    \begin{minipage}[t]{0.45\linewidth}
        \item Si vraiment ils avaient voulu réussir ce concours, ils se seraient davantage investis dans leurs études.
    
        \boxans{
            If they had really wanted to succeed in this competition, they would have invested more in their studies.
        }
    
        \item La crise des migrants est d'autant plus aiguë que l'Europe s'est saisie du problème assez tardivement.
    
        \boxans{
            If they had really wanted to succeed in this contest, they would have put more effort into their studies.
        }
    
        \item L'actionnaire britannique, dont j'ignore la ville d'origine, s'est fait parfaitement comprendre en français lors de la réunion.

        \boxans{
            The British shareholder, from which I do not know the city of origin, made himself perfectly understood in French during the meeting.
        }
    \end{minipage}
\end{enumerate}

\section*{\centering\EBGaramond\Large Questions}

\begin{enumerate}
    \item According to the article, is the Tory Party fated to be out of touch with the public opinion? 
    
    \boxans{
        According to the article, the Conservative Party, although it came to power through populism, is now defending values that are highly unpopular: they are against mandatory id cards, low-traffic neighbourhoods, when the people are for them; they debate issues that the people are not interested in, as the YouGov poll on the word 'woke' shows. The article seems to say that this difference is linked to the Tory party's specific worldview, surely condemning it to remain out of touch with public opinion, unless that public opinion changes on its own.
    }
    
    \item Which reforms would be desirable to improve the representation of the will of the people? Illustrate with examples taken from English speaking countries?
    
    \boxans{
        I think that referendums are perhaps one of the best options for people to express opinion, even when it is very controversial : on the contrary, it even allows us to look at the true position of the people, outside the prism of politicians and medias. This was for example the case for the referendum for the UK's exit from the European Union in 2016, where the majority response had only 51\% of the votes. Moreover, new technologies make it possible to greatly simplify the organisation for such an occasion. This was the case, for example, for the last US presidential election, that took place in the middle of the covid crisis.
    }
\end{enumerate}


\end{document}