\documentclass[a4paper,french,bookmarks]{article}
\usepackage{./Structure/4PE18TEXTB}
        
\begin{document}
\stylize{Mathématiques}{Devoir Maison 10 : Suites implicites}

\section*{Suites implicites}

On s’intéresse dans cet exercice aux points d’intersections d’abscisses positives entre la courbe de la fonction exponentielle et les courbes des fonctions puissances.

Soit $n \in \bdN^*$. On cherche donc à résoudre l'équation $(E_n) : e^x = x^n$.
\begin{enumerate}
    \item Montrer que pour tout $n \in \bdN^*$, l'équation $(E_n)$ d'inconnue $x$ dans $\bdRp$ est équivalente à l'équation 
    \[ (F_n) : \dfrac{\ln x}{x} = \dfrac{1}{n}\]
    
    \boxans{
    $\forall n \in \bdN^*$, on a les équivalences $e^x = x^n \iff x = n\ln{x} \iff \dfrac{\ln x}{x}=\dfrac{1}{n}$ donc \boxsol{$(E_n) \iff (F_n)$}.
    }
    
    \item Représenter le graphe de $f : x \mapsto \dfrac{\ln x}{x}$ \quad (on précisera les asymptotes et tangentes).
    
    \boxans{

        \pgfplotsset{width=12.5cm}
        \center \begin{tikzpicture}
            \begin{axis}[
                axis lines          = center,
                xmin                = -2,
                xmax                = 18,
                ymin                = -10,
                ymax                = 5,
                domain              = 0.1:18,
                grid                = both,
                grid style          = {line width = .1pt, draw = gray!30},
                major grid style    = {line width=.2pt,draw=gray!50},
                minor tick num      = 1,
                legend pos          = north east,
                xlabel=$x$,
                ylabel=$y$,
            ]
            
            \addplot[color=main1, line width=0.7mm,samples=200] {ln(x)/x};
            \addlegendentry{$f(x) = \frac{\ln x}{x}$};
            \addplot[dashed,color=main6,line width =0.5mm] coordinates{(0.1,-10) (0.1,2)};
            \addlegendentry{asymptote verticale en $0$}
            \addplot[color=main7,line width =0.5mm,latex'-latex'] coordinates{(1,1/e) (5,1/e)};
            \addlegendentry{tangente horizontale en $e$}
            \addplot[dashed,color=main8,line width =0.5mm] coordinates{(8,0) (18,0)};
            \addlegendentry{asymptote horizontale en $+\infty$}
            \addplot[
                color=main3,
                mark=*,
                mark size=2pt,
                point meta=explicit symbolic,
                nodes near coords,
                every node near coord/.append style={anchor=south, color=main3},
            ]
            coordinates {
                (e, 1/e) [$\left(e, \sfrac{1}{e}\right)$]
            };
            
            \end{axis}
        \end{tikzpicture}
    }
    
    \item Montrer que pour chaque entier $\mathbf{n \geq 3}$, l'équation $(F_n)$ admet exactement deux solutions réelles, notées $u_n$ et $v_n$ et telles que :
    \[ 1 < u_n < e < v_n\]
    
    \boxans{
        La fonction $f : x \mapsto \dfrac{\ln x}{x}$ sur $\bdRp$ étant un quotient de fonctions dérivables sur $\bdRp$, elle est dérivable sur $\bdRp$ :
        \[ \forall x \in \bdR, \quad f'(x) = \dfrac{1-\ln x}{x^2}\]
        On a $f'(x) > 0 \iff 1 - \ln x > 0 \iff x > e$ donc $f$ est strictement croissante sur $]0, e[$ et strictement décroissante sur $]e, +\infty[$. Par théorème de la bijection, on pose les réciproques $g_1$ et $g_2$ de $f$, avec 
        \[ g_1 : ]-\infty, \sfrac{1}{e}[ \to ]0, e[ \qquad \text{et} \qquad g_2 : ]0, \sfrac{1}{e}[ \to ]e, +\infty[\]
        avec $g_1$ strictement croissante et $g_2$ strictement décroissante.
        
        Soit $n \in \bdN$, $n \geq 3$ tel que $x \in \bdRp$ est solution de $(F_n)$. On a $n \geq 3 > e$ donc $\dfrac{1}{n} < \dfrac{1}{e}$. Donc :
        \[ x \in \bdRp \ \text{est solution de} \ (F_n) \iff f(x) = \dfrac{1}{n} \iff x = g_1\left(\dfrac{1}{n}\right) \quad \text{ou} \quad x = g_2\left(\dfrac{1}{n}\right)\]
        On a donc exactement deux solutions $u_n =  g_1\left(\dfrac{1}{n}\right)$ et $v_n =  g_2\left(\dfrac{1}{n}\right)$. Les ensemble d'arrivée de $g_1$ et $g_2$ livrent $u_n < e < v_n$. On remarque de plus que $f(1)=\dfrac{\ln 1}{1}=0$ donc $g_1(0) = 1$. Or $\dfrac{1}{n} \geq 0$ donc par stricte croissance de $g_1$, $u_n > 1$. Donc :
        
        \boxsol{$\forall n \in \llbracket3, +\infty\llbracket$, l'équation $(F_n)$ admet exactement deux solutions $u_n$ et $v_n$ telles que $1 < u_n < e < v_n$}.
    }
    
    \item Représenter sur un schéma les abscisses correspondant aux valeurs de $u_n$ et $u_{n+1}$ pour un entier $n \geq 3$.
    
    \textit{On prendra par exemple $n = 3$}.
    
    \boxans{

        \pgfplotsset{width=12.5cm}
        \center \begin{tikzpicture}
            \begin{axis}[
                axis lines          = center,
                xmin                = -1,
                xmax                = 4,
                ymin                = -0.2,
                ymax                = 0.6,
                domain              = 0.1:5,
                grid                = both,
                grid style          = {line width = .1pt, draw = gray!30},
                major grid style    = {line width=.2pt,draw=gray!50},
                minor tick num      = 1,
                legend pos          = north east,
                xtick               = {-1, 0, 1, 2, 3, 4},
                ytick               = {-0.2, 0, 0.2, 0.4,0.6},
                xlabel=$x$,
                ylabel=$y$,
            ]
            
            \addplot[color=main1, line width=0.7mm,samples=200,domain=0.1:e] {ln(x)/x};
            \addlegendentry{$f(x) = \frac{\ln x}{x}$};
            \addplot[color=main2!30!gray, dashed, line width=0.7mm,samples=200,domain=e:4] {ln(x)/x};
            
            \addplot[color=main5!50!gray, dashed, line width=0.5mm, -latex'] coordinates{(0,1/3) (1.857,1/3)};
            \addplot[color=main5!50!gray, dashed, line width=0.5mm, -latex'] coordinates{(1.857,1/3) (1.857,0)};
            
            \addplot[color=main9!50!gray, dashed, line width=0.5mm, -latex'] coordinates{(0,1/4) (1.43,1/4)};
            \addplot[color=main9!50!gray, dashed, line width=0.5mm, -latex'] coordinates{(1.43,1/4) (1.43,0)};
            
            \addplot[
                color=main5,
                mark=*,
                mark size=2pt,
                point meta=explicit symbolic,
                nodes near coords,
                every node near coord/.append style={anchor=east},
            ]
            coordinates {
                (0, 1/3) [$\frac{1}{3}$]
            };
            
            \addplot[
                color=main5,
                mark=x,
                mark size=2pt,
                point meta=explicit symbolic,
                nodes near coords,
                every node near coord/.append style={anchor=north},
            ]
            coordinates {
                (1.857,0) [$u_3$]
            };
            
            \addplot[
                color=main9,
                mark=*,
                mark size=2pt,
                point meta=explicit symbolic,
                nodes near coords,
                every node near coord/.append style={anchor=east},
            ]
            coordinates {
                (0, 1/4) [$\frac{1}{4}$]
            };
            
            \addplot[
                color=main9,
                mark=x,
                mark size=2pt,
                point meta=explicit symbolic,
                nodes near coords,
                every node near coord/.append style={anchor=north},
            ]
            coordinates {
                (1.43,0) [$u_4$]
            };
            
            \addplot[
                color=main3,
                mark=*,
                mark size=2pt,
                point meta=explicit symbolic,
                nodes near coords,
                every node near coord/.append style={anchor=south, color=main3},
            ]
            coordinates {
                (e, 1/e) [$\left(e, \sfrac{1}{e}\right)$]
            };
            
            \end{axis}
        \end{tikzpicture}
    }
    
    \item Étudier la monotonie de la suite $\left(u_n\right)_{n \geq 3}$.
    
    \boxans{
        Soit $n \in \bdN$, $n \geq 3$. On a $u_{n+1} - u_n = g_1\left(\dfrac{1}{n+1}\right) - g_1\left(\dfrac{1}{n}\right)$. Or $\dfrac{1}{n+1} < \dfrac{1}{n}$, donc par stricte croissance de $g_1$ on a $g_1\left(\dfrac{1}{n+1}\right) < g_1\left(\dfrac{1}{n}\right)$ donc $u_{n+1} < u_n$. Donc \boxsol{la suite $\left(u_n\right)_{n \geq 3}$ est strictement décroissante}.
    }
    
    \item Montrer que cette suite $\left(u_n\right)_{n \geq 3}$ converge et calculer sa limite.
    
    \boxans{
        La suite $\left(u_n\right)_{n \geq 3}$ est strictement décroissante et minorée par $1$ donc \boxsol{la suite $\left(u_n\right)_{n \geq 3}$ converge}.
        
        On a $\lim\limits_{n \to +\infty} u_n = \lim\limits_{n \to +\infty} g_1\left(\dfrac{1}{n}\right) = \lim\limits_{N \to 0^+} g_1(N) = 1$. Donc \boxsol{$\lim\limits_{n \to +\infty} u_n = 1$}.
    }
    
    \item Montrer que la suite $\left(v_n\right)_{n \geq 3}$ tend vers $+\infty$. \quad \textit{(On pourra calculer $f(n)$.)}
    
    \boxans{
        Soit $n \in \bdN$, $n \geq 3$. On a $v_{n+1} - v_n = g_2\left(\dfrac{1}{n+1}\right) - g_2\left(\dfrac{1}{n}\right)$. Comme précédemment, par stricte décroissance de $g_2$ on obtient $v_{n+1} > v_n$ donc la suite $\left(v_n\right)_{n \geq 3}$ est strictement croissante, donc elle converge ou diverge vers $+\infty$.
        On a $\lim\limits_{n \to +\infty} f(n) = \lim\limits_{n \to +\infty} \dfrac{\ln n}{n} = 0^+$ donc $\lim\limits_{n \to 0^+} g_2(n) = +\infty$. 
        
        Or $\lim\limits_{n \to +\infty} v_n = \lim\limits_{n \to +\infty} g_2\left(\dfrac{1}{n}\right) = \lim\limits_{N \to 0^+} g_2(N) = +\infty$ donc \boxsol{$\lim\limits_{n \to +\infty} v_n = +\infty$}.
    }
    
    \item Dans cette question, on cherche un développement asymptotique de la suite $u$.
    
    \begin{enumerate}
        \item En écrivant $u_n - 1= \epsilon_n$, \quad montrer que $\epsilon_n \asymp{+\infty}  \dfrac{1}{n}$.
        
        \boxans{
            On veut calculer $\lim\limits_{n \to +\infty} \dfrac{\epsilon_n}{\frac{1}{n}} = \lim\limits_{n \to +\infty} \dfrac{u_n - 1}{\frac{1}{n}} = \lim\limits_{n \to +\infty} \dfrac{g\left(\frac{1}{n}\right) -1}{\frac{1}{n}}=\lim\limits_{x \to 0} \dfrac{g(0+x) - g(0)}{x}$.
            
            Or $f'$ ne s'annule pas sur $]0, e[$ donc par dérivée de la réciproque, $g_1$ est dérivable et  tel que :
            \[ \forall x \in ]-\infty, \sfrac{1}{e}[,\quad g'_1(x) = \dfrac{1}{f'(g_1(x))} = \dfrac{1}{\frac{1}{g_1(x)^2}-\frac{x}{g_1(x)}} = \dfrac{g_1(x)^2}{1-xg_1(x)}\]
            
            Or $\lim\limits_{x \to 0} \dfrac{g(0+x) - g(0)}{x} = g_1'(0) = \dfrac{1^2}{1-0\times1}=1$ donc $\lim\limits_{n \to +\infty} \dfrac{\epsilon_n}{\frac{1}{n}} = 1$ donc \boxsol{$\epsilon_n \asymp{+\infty}  \dfrac{1}{n}$}.
        }
        
        \item  Rappeler le développement limité à tous ordres de $x \to \ln(1 + x)$ au voisinage de $0$.
        
        \boxans{
            Soit $x \in \bdR$ au voisinage de $0$ et $p \in \bdN$, on a le développement limité de $x \to \ln(1 + x)$ d'ordre $p$ :
            \[ \boxsol{$DL_p(0) : \ln(1+x) \eq{+\infty} x - \dfrac{x^2}{2} + \dfrac{x^3}{3} + \dots + (-1)^{n-1}\dfrac{x^p}{p}+o(x^p)$} \]
        }
        
        \item On pose alors $y_n = u_n - 1 - \dfrac{1}{n}$. \quad Donner un développement asymptotique à précision $\o{}{\frac{1}{n^2}}$ des quantités $\ln(u_n)$ et $\frac{u_n}{n}$ en fonction uniquement de $n$ et $y_n$. En déduire un équivalent de $y_n$.
        
        Justifier le développement asymptotique pour la suite $\left(u_n\right)_{n \geq 3}$ :
        \[ u_n \eq{+\infty} 1 + \dfrac{1}{n} + \dfrac{a}{n^2} + \o{}{\dfrac{1}{n^2}}\]
        On précisera la valeur du réel $a$.
        
        \boxans{
            On a $u_n - 1 = \epsilon_n$ et $\epsilon \asymp{+\infty} \dfrac{1}{n}$ donc $u_n - 1 \asymp{+\infty} \dfrac{1}{n}$ donc $u_n - 1 - \dfrac{1}{n} \eq{+\infty} \o{}{\dfrac{1}{n}}$ donc $y_n \eq{+\infty} \o{}{\dfrac{1}{n}}$.
            On a $y_n = u_n - 1 - \dfrac{1}{n}$ donc $u_n = 1 + y_n + \dfrac{1}{n}$. En utilisant le développement limité ci-dessus :
            \[ \boxsoll{$\ln(u_n)$} \eq{+\infty} y_n + \dfrac{1}{n} - \dfrac{\left(y_n + \frac{1}{n}\right)^2}{2}+\o{}{\left(y_n+\dfrac{1}{n}\right)^2} \boxsolr{$\eq{+\infty} y_n + \dfrac{1}{n} - \dfrac{1}{2n^2} +\o{}{\dfrac{1}{n^2}}$} \]
            De plus, on a :
            \[ \boxsoll{$\dfrac{u_n}{n}$} = \dfrac{1}{n} + \dfrac{y_n}{n} + \dfrac{1}{n^2} \eq{+\infty} \boxsolr{$\dfrac{1}{n} + \dfrac{1}{n^2} + \o{}{\dfrac{1}{n^2}}$} \]
            Or $f(u_n) = \dfrac{1}{n}$ donc $\dfrac{\ln(u_n)}{u_n} = \dfrac{1}{n}$ d'où $\ln(u_n) = \dfrac{u_n}{n}$. En combinant les deux développements ci-dessus :
            \[ y_n \eq{+\infty} \dfrac{1}{n} + \dfrac{1}{n^2} - \dfrac{1}{n} + \dfrac{1}{2n^2} + \o{}{\dfrac{1}{n^2}} \eq{+\infty} \dfrac{\sfrac{3}{2}}{n^2} +\o{}{\dfrac{1}{n^2}} \quad \text{donc} \ \boxsol{$y_n \asymp{+\infty} \dfrac{\sfrac{3}{2}}{n^2}$}\]
            Donc $u_n - 1 - \dfrac{1}{n}\eq{+\infty} \dfrac{\sfrac{3}{2}}{n^2} +\o{}{\dfrac{1}{n^2}}$ d'où \boxsol{$u_n \eq{+\infty} 1 + \dfrac{1}{n} + \dfrac{\sfrac{3}{2}}{n^2} + \o{}{\dfrac{1}{n^2}}$}
        }
        
    \end{enumerate}
    
    \item Dans cette question, on cherche un équivalent de la suite $v$.
    
    \begin{enumerate}
        \item Montrer que : \quad $\forall n > 3$, $v_n \geq n\ln{n}$.
        
        \boxans{
            Soit $n \in \bdN$, $n \geq 3$. On a $\dfrac{\ln v_n}{v_n} = \dfrac{1}{n}$ donc $v_n = n\ln(v_n)$. Or $n \geq 3 > e$ donc $\ln n \geq 1$ donc $\dfrac{\ln n}{n} \geq \dfrac{1}{n}$.
            
            Par décroissance de $g_2$, $n \leq g_2\left(\dfrac{1}{n}\right) = v_n$ donc $n\ln{v_n} \geq n\ln{n}$. Donc \boxsol{$v_n \geq n\ln(n)$}.
        }
        
        \item Soit $\epsilon$ un réel strictement positif fixé.
        \begin{enumerate}
            \item Calculer $f((1+\epsilon)n\ln n)$ et montrer que : $f((1+\epsilon)n\ln n) \asymp{+\infty} \dfrac{1}{(1+\epsilon)n}$
            
            \boxans{
            \[\boxsoll{$f((1+\epsilon)n\ln n)$} = \dfrac{\ln{(1+\epsilon)n\ln(n)}}{(1+\epsilon)n\ln n}=\boxsolr{$\dfrac{1}{(1+\epsilon )n}\left[1+\dfrac{\ln(1+\epsilon)+\ln{\ln{n}}}{\ln{n}}\right]$}\]
            
            Donc $\lim\limits_{n \to +\infty} \dfrac{f((1+\epsilon)n\ln n)}{\frac{1}{(1+\epsilon)n}} = \lim\limits_{n \to +\infty} \left(1+\dfrac{\ln(1+\epsilon)+\ln{\ln{n}}}{\ln{n}}\right) = 1$
            
            Donc \boxsol{$f((1+\epsilon)n\ln n) \asymp{+\infty} \dfrac{1}{(1+\epsilon)n}$}
            }
            
            \item En déduire l'existence d'un entier $n_0$ tel que pour tout $n \geq n_0$, on ait :
            \[ nf((1+\epsilon)n\ln n) \leq 1\]
            
            \boxans{
            On a $f((1+\epsilon)n\ln n) \asymp{+\infty} \dfrac{1}{(1+\epsilon)n}$ donc $\lim\limits_{n \to +\infty} nf((1+\epsilon)n\ln n) = \dfrac{1}{1+\epsilon}$. Par définition 
            \[ \exists n_0 \in \bdN,\ n_0 \geq 3,\ \forall n \in \bdN,\quad n \geq n_0 \implies \mod{nf((1+\epsilon)n\ln n) - \dfrac{1}{1+\epsilon}} \leq \dfrac{\epsilon}{1+\epsilon}\]
            Donc $\forall n \in \bdN$, $n\geq n_0 \implies nf((1+\epsilon)n\ln n) \leq \dfrac{\epsilon}{1+\epsilon} + \dfrac{1}{1+\epsilon} \implies f((1+\epsilon)n\ln n) \leq 1$.
            
            Donc \boxsol{il existe un entier $n_0$ tel que pour tout $n \geq n_0$, $nf((1+\epsilon)n\ln n) \leq 1$ }.
            }
            
            \item En déduire que pour tout $n \geq n_0$, on a :
            \[ n\ln n \leq v_n \leq (1+\epsilon)n \ln n\]
            
            \boxans{
                Soit $n \in \bdN$, $n \geq n_0$. On a $nf((1+\epsilon)n\ln n) \leq 1$ donc $f((1+\epsilon)n\ln n) \leq \dfrac{1}{n}$. Par décroissance de $g_2$, on a $(1+\epsilon)n\ln n \geq g_2\left(\dfrac{1}{n}\right) = v_n$. Donc \boxsol{$n\ln n \leq v_n \leq (1+\epsilon)n \ln n$}.
            }
        \end{enumerate}
        
        \item Donner un équivalent de la suite $\left(v_n\right)_{n \geq 3}$, lorsque $n$ tend vers l’infini.
        
        \boxans{
        \[ \forall \epsilon \in \bdRp,\ \exists n_o(\epsilon) \in \bdN,\ n_0 \geq 3,\ \forall n \in \bdN, \qquad n \geq n_0(\epsilon) \implies n\ln n \leq v_n \leq (1+\epsilon)n \ln n\]
        
        On peut donc considérer $\left(\epsilon_n\right)_{n \geq 3}$, $\epsilon_n \eq{+\infty} o(1)$ tel que
        \[ \forall n \in \bdN, \ n \geq 3,\quad u_n = (1+\epsilon)n\ln n \eq{+\infty} (1+\o{}{1})n\ln n \eq{+\infty} n\ln n + \o{}{n\ln n}\]
        On a donc \boxsol{$v_n \asymp{+\infty} n\ln n$}.
        
        On remarquera que la vitesse de converge est tout de même très lente, du fait de $\o{}{n\ln n}$.
        }

    \end{enumerate}


\end{enumerate}
\end{document}