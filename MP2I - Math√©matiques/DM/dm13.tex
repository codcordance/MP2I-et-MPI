\documentclass[a4paper,french,bookmarks]{article}

\usepackage{./Structure/4PE18TEXTB}

\renewcommand{\thesection}{\Roman{section}}
\renewcommand{\thesubsection}{\Alph{subsection}}

\begin{document}
\stylizeDoc{Mathématiques}{Devoir Maison 13}{Tchebychev et Block-Thielmann}

\section{Polynômes de Tchebychev}

\subsection{Polynômes de première espèce}

On définit les polynômes de Tchebychev de première espèce $\suite{T_n}$ par la relation :

\begin{equation}
     \forall n \in \bdN, \qquad \forall \theta \in \bdR,\qquad T_n(\cos \theta) = \cos{n\theta}
\end{equation}

On ne demande pas de justifier l'existence et l'unicité de la famille de polynômes définie par cette relation.

\begin{enumerate}
    \item\label{question:1} Déterminer $T_0$, $T_1$, $T_2$ et $T_3$.
    
    \boxans{
        \begin{enumerate}
            \ithand $\forall \theta \in \bdR$, on a $T_0(\cos \theta) = \cos{0\cdot\theta}$ donc $T_0(\cos \theta) = 1$. Or la fonction $\cos$ est bijective de $[0, \pi]$ dans $[-1, 1]$. On prend alors $\theta \in [0, \pi]$ et $x = \cos(\theta) \in [-1, 1]$ donc :
            
            \[ \forall x \in [-1, 1],\qquad T_0(x) = 1\]
            
            On pose le polynôme $Q_0 = T_0 - 1$. On a alors :
            
            \[ \forall x \in [-1, 1],\qquad Q_0(x) = T_0(x) - 1 = 1 - 1 = 0 \]
            
            Le polynôme $Q_0$ ayant une infinité de racine sur $[-1, 1]$, il est égal au polynôme nul : \quad $Q_0 = 0$.
            
            On en déduit donc que $T_1 - 1 = 0$ d'où \boxsol{$T_0 = 1$}.
        
            \ithand $\forall \theta \in \bdR$, on a $T_1(\cos \theta) = \cos {1\theta} = \cos \theta$ donc de même :
            
            \[ \forall x \in [-1, 1],\qquad T_1(x) = x\]
            
            On pose le polynôme $Q_1 = T_1 - X$. On a alors :
            
            \[\forall x \in [-1, 1],\qquad Q_1(x) = T_1(x) - x = x - x = 0 \]
            
            Comme précédemment, puisque le polynôme $Q_1 = T_1 - X$ a une infinité de racine, il est égal au polynôme nul. Donc $Q_1 = 0$ soit $T_1 - X = 0$, et donc \boxsol{$T_1 = X$}.
        
            \ithand $\forall \theta \in \bdR$, on a $T_2(\cos \theta) = \cos{2\theta} = 2\cos^2(\theta) - 1$, donc :
            
            \[ \forall x \in [-1, 1],\qquad T_2(x) = 2x^2 -1\]
            
            On pose le polynôme $Q_2 = 2X^2 - 1$. Comme ci-dessus, on a :
            \[ \forall x \in [-1, 1],\qquad Q_2(x) = T_2(x) - 2x^2-1 = 2x^2 - 1 - (2x^2 - 1) = 0 \]
            
            On obtient bien que $Q_2$ a une infinité de racine sur $[-1, 1]$ donc $Q = 0$, et donc \boxsol{$T_2 = 2X^2 - 1$}.
            
            \ithand Soit $\theta \in \bdR$. On a $T_3(\cos \theta) = \cos{3\theta}$. On développe alors $\cos(3\theta)$ :
            \begin{align*}
                \cos{3\theta} &= \cos{2\theta + \theta} \\
                &= \cos(2\theta)\cos(\theta) + \sin{2\theta}\sin{\theta}\\
                &= (2\cos^2(\theta) -1)\cos(\theta) - 2\cos{\theta}\sin^2(\theta)\\
                &= 2\cos^3(\theta) - \cos(\theta) - 2\cos{\theta}(1-\cos^2(\theta))\\
                &= 2\cos^3(\theta) - \cos(\theta) - 2\cos(\theta) +2\cos^3(\theta)\\
                &= 4\cos^3(\theta) - 3\cos(\theta)
            \end{align*}
            
            De manière analogue, $\forall x \in [-1, 1]$, on a $T_3(x) = 4x^3 - 3x$. On raisonne alors selon un polynôme $Q_3 = T_3 - 4X^3 - 3X$ comme on l'a fait ci-dessus, et en montrant qu'il a une infinité de racines sur le segment $[-1, 1]$, et qu'il est donc nul, on obtient finalement \boxsol{$T_3 = 4X^3 - 3X$}.
        \end{enumerate}
    }
    
    \item\label{question:2} En remarquant que pour tout réel $\theta$, on a $e^{in\theta} = \left(e^{i\theta}\right)^n$, montrer que
    
    \begin{equation}\label{eq:2}
         \forall n \in \bdN,\qquad T_n = \sum_{k=0}^{\left\lfloor \sfrac{n}{2}\right\rfloor} \binom{n}{2k} (X^2 -1)^k X^{n-2k}
    \end{equation}
    
    \boxans{
    
        Soit $n \in \bdN$ et $\theta \in \bdR$. On cherche en fait exprimer $\cos{n\theta}$ \textit{ - soit $\Re{e^{in\theta}}$ - } selon $\cos \theta$. On a :
        \begin{align*}
            e^{in\theta} &= \left(e^{i\theta}\right)^n = \left(\cos \theta + i\sin \theta\right)^n = \sum_{k=0}^n \binom{n}{k} (\cos{\theta})^{n-k}\left(i\sin{\theta}\right)^k\\
            &= \sum_{\substack{k=0\\k \equiv 0 \ [2]}}^n \binom{n}{k} (\cos{\theta})^{n-k}\left(i\sin{\theta}\right)^k + \sum_{\substack{k=0\\k \equiv 1 \ [2]}}^n \binom{n}{k} (\cos{\theta})^{n-k}\left(i\sin{\theta}\right)^k\\
            &= \sum_{j=0}^{\left\lfloor \sfrac{n}{2}\right\rfloor} \binom{n}{2j}(\cos \theta)^{n - 2j}(-1)^j(\underbrace{1-\cos^2\theta}_{ \sin^2 \theta})^j + \sum_{j=0}^{\left\lfloor \sfrac{n-1}{2}\right\rfloor}\binom{n}{2j+1}(\cos \theta)^{n+1-2j} \underbrace{i(-1)^j}_{i^{2j+1}}(\sin^2 \theta)^j\\
            &= \sum_{j=0}^{\left\lfloor \sfrac{n}{2}\right\rfloor} \binom{n}{2j}\left(\cos^2 \theta - 1\right)^j\cos^{n-2j} \theta + i\left[\underbrace{\ldots}_{\in \bdR}\right]
        \end{align*}
        
        On a bien $\displaystyle \cos{n\theta} = \Re{e^{in\theta}} = \sum_{k=0}^{\left\lfloor \sfrac{n}{2}\right\rfloor} \binom{n}{2k}\left(\cos^2 \theta - 1\right)^k\cos^{n-2k} \theta$. Comme à la question \enumref{question:1}, on a : 
        
        \[ \forall x \in [-1, 1],\qquad T_n(x) = \sum_{k=0}^{\left\lfloor \sfrac{n}{2}\right\rfloor} \binom{n}{2k}\left(X^2 - 1\right)^kX^{n-2k}\]
        
        En raisonnant de même selon un polynôme $Q_n$, on retrouve \boxsol{$\eqref{eq:2}: T_n = \displaystyle \sum_{k=0}^{\left\lfloor \sfrac{n}{2}\right\rfloor} \binom{n}{2k} (X^2 -1)^k X^{n-2k}$}.
    }
    
    \item\label{question:3} Montrer que la suite $(T_n)$ vérifie la relation de récurrence
    
    \begin{equation}\label{eq:3}
        \forall n \in \bdN, \qquad T_{n+2} = 2XT_{n+1} - T_n
    \end{equation}
    
    \boxans{
        Soit $n \in \bdN$ et $\theta \in \bdR$. On a :
        \begin{align*}
            2\cos{\theta}\cdot \cos{(n+1)\theta}-\cos(n\theta) &= \cos{\theta + (n+1)\theta} - \cos(\theta - (n-1)\theta) - \cos(n\theta)\\
            &= \cos{(n + 2)\theta} - \cos(-n\theta) + \cos(-n\theta)\\
            &= \cos{(n + 2)\theta}
        \end{align*}
        
        Donc $\forall x \in [-1, 1]$, $T_{n+2}(x) = 2xT_{n+1}(x) - T_n(x)$ d'où \boxsol{$\eqref{eq:3}: T_{n+2} = 2XT_{n+1} - T_n$}.
    }
    
    \item\label{question:4} En déduire, pour tout entier naturel $n$, le degré et le coefficient dominant de $T_n$.
    
    \boxans{
        On montre par récurrence d'ordre $2$ la propriété suivante :
        
        \[ \forall n \in \bdN^*,\qquad H(n) : \exists P_n \in \bdR_{n-1}[X],\qquad T_n = 2^{n-1}X^n + P_n\]
        
        Question \enumref{question:1} on a montré $H(1)$ et $H(2)$. Soit donc $n \in \bdN$, $n \geq 2$ tel que $H(n-1)$ et $H(n)$ sont vraies. On a :
        
        \[\exists P_n \in \bdR_{n-1}[X],\qquad  T_n = 2^{n-1}X^n + P_n \qquad\et\qquad \deg(T_{n-1}) = n-1\]
        
        Or $T_{n+1} = 2XT_n - T_{n-1}$ donc :
        
        \[T_{n+2} = 2X(2^{n-1}X^n + P_n) - T_{n-1} = 2^{n}X^{n+1} + 2XP_n + T_{n-1}\]
        
        On pose $P_{n+1} = 2XP_n + T_{n-1}$, ainsi $\deg P_{n+1} \leq n$. On a donc l'hérédité $H(n) \implies H(n+1)$.
        
        Or $H(1)$ et $H(2)$ sont vrais, donc par principe de récurrence, $H(n)$ est vrai pour tout $n \in \bdN^*$.
        
        Excepté le cas dégénéré \boxsol{$T_0 = 1$}, on a donc \boxsol{$T_n$ de degré $n$ et de coefficient dominant $2^{n-1}$}.
    }
    
    \item\label{question:5} Retrouver les résultats de la question précédente en utilisant uniquement l'expression de la question \enumref{question:2}, i.e. \eqref{eq:2}.
    
    \boxans{
        Soit $n \in \bdN$. On fait sortir les indices les plus opposés de la somme dans \eqref{eq:2}, puis par binôme de Newton on a :
        
        \[T_n = \sum_{k=0}^{\left\lfloor \sfrac{n}{2}\right\rfloor} \binom{n}{2k} (X^2 -1)^k X^{n-2k} = \sum_{k=0}^{\left\lfloor \sfrac{n}{2}\right\rfloor} \binom{n}{2k}X^{n-2k}\left[\sum_{j=0}^k\binom{k}{j} X^{2j}(-1)^{k-j}\right]\]
        
        On peut alors poser un nouveau polynôme $Q_k$ tel que $\deg Q_k \leq 2k-1$ tel que :
        
        \[ T_n = \sum_{k=0}^{\left\lfloor \sfrac{n}{2}\right\rfloor} \binom{n}{2k}X^{n-2k}\left[\binom{k}{k}X^{2k}(-1)^0 + Q_k\right] = \sum_{k=0}^{\left\lfloor \sfrac{n}{2}\right\rfloor} \binom{n}{2k}X^{n-2k}X^{2k} + \underbrace{P_k}_{\binom{n}{2k}X^{n-2k}Q_k \in \bdR_{n-1}[X]}\]
        
        Soit $R_n = \displaystyle \sum_{k=0}^{\left\lfloor \sfrac{n}{2}\right\rfloor} P_k$, de degré au plus $n-1$ tel que $T_n = \displaystyle R_n + X^n\sum_{k=0}^{\left\lfloor \sfrac{n}{2}\right\rfloor} \binom{n}{2k}$. Calculons ce coefficient :
        
        \begin{enumerate}
            \ithand Si $n = 0$, on a$\displaystyle \sum_{k=0}^0 \binom{k}{0} = 0$ devant $X^0 = 1$, on retrouve bien \boxsol{$T_0 = 1$}.
            
            \ithand Pour $n \in \bdN^*$, on pose $A = \displaystyle \sum_{k=0}^{\left\lfloor \sfrac{n}{2}\right\rfloor} \binom{n}{2k} = \sum_{\substack{k = 0\\k \equiv 0 \ [2]}}^{n} \binom{n}{k}$ et $A = \displaystyle \sum_{k=0}^{\left\lfloor \sfrac{(n-1)}{2}\right\rfloor} \binom{n}{2k} = \sum_{\substack{k = 0\\k \equiv 1 \ [2]}}^{n} \binom{n}{k}$. On a :
            
            \[ A + B = \displaystyle \sum_{\substack{k = 0\\k \equiv 1 \ [2]}}^{n} \binom{n}{k} + \sum_{\substack{k = 0\\k \equiv 1 \ [2]}}^{n} \binom{n}{k} = \sum_{k=0}^n \binom{n}{k} = (1+1)^n = 2^n \]
            
            De plus :
            \begin{align*}
                A - B &=  \displaystyle \sum_{\substack{k = 0\\k \equiv 1 \ [2]}}^{n} \binom{n}{k} -\sum_{\substack{k = 0\\k \equiv 1 \ [2]}}^{n} \binom{n}{k} = \sum_{\substack{k = 0\\k \equiv 1 \ [2]}}^{n} \binom{n}{k}(-1)^{k} +\sum_{\substack{k = 0\\k \equiv 1 \ [2]}}^{n}\binom{n}{k}(-1)^k \\
                &= \displaystyle \sum_{k=0}^n \binom{n}{k}(-1)^k = (1 - 1)^n = 0^n = 0 \qquad(\text{car} \ n \geq 1)
            \end{align*}
        
            %\[\left\lbrace\begin{array}{ll}
            %    A + B &= \displaystyle \sum_{\substack{k = 0\\k \equiv 1 \ [2]}}^{n} \binom{n}{k} + \sum_{\substack{k = 0\\k \equiv 1 \ [2]}}^{n} \binom{n}{k} = \sum_{k=0}^n \binom{n}{k} = (1+1)^n = 2^n  \\
            %    A - B &=  \displaystyle \sum_{\substack{k = 0\\k \equiv 1 \ [2]}}^{n} \binom{n}{k} -\sum_{\substack{k = 0\\k \equiv 1 \ [2]}}^{n} \binom{n}{k} = \sum_{\substack{k = 0\\k \equiv 1 \ [2]}}^{n} \binom{n}{k}(-1)^{k} +\sum_{\substack{k = 0\\k \equiv 1 \ [2]}}^{n}\binom{n}{k}(-1)^k \\
            %    &= \displaystyle \sum_{k=0}^n \binom{n}{k}(-1)^k = (1 - 1)^n = 0^n = 0 \qquad(\text{car} \ n \geq 1)
            %\end{array}\right.\]
            On a $A + B = 2^n$ et $A - B = 0$ donc $2A = 2^n$ d'où $A = 2^{n-1}$.
            
            \textit{In fine}, on retrouve bien que \boxsol{$T_n$ est de degré $n$ et de coefficient dominant $2^{n-1}$}. 
        \end{enumerate}
    
    }
    
    \item\label{question:6} Montrer que, pour tout entier naturel $n$, le polynôme $T_n$ est scindé sur $\bdR$, à racines simples appartenant à $]-1, 1[$.
    
    Déterminer les racines de $T_n$ ainsi que sa factorisation sur $\bdR$.
            
    \boxans{
        Soit $n \in \bdR$. Cherchons les racines de $T_n$. On a $T_n$ de degré $n$ donc $T_n$ a au plus $n$ racines.
        
        Cherchons alors $n$ racines distinctes dans $]-1, 1[$. Soit $\theta \in ]0; \pi[$, on a $\cos(n\theta) \in ]-1; 1[$.
        
        Cherchons les $\theta$ tels que $T_n(\cos \theta) = 0$, c'est-à-dire $\cos(n\theta) = 0$. On a alors :
        
        \[n\theta \equiv \dfrac{\pi}{2} \ [\pi] \qquad\text{donc}\qquad \exists k \in \bdN \qquad\text{donc}\qquad n\theta =  k\pi + \dfrac{\pi}{2} \qquad\text{donc}\qquad \theta = \dfrac{(2k+1)\pi}{2n}\]
        
        $\cos$ est bijective de $]0; \pi[$ sur $]-1; 1[$. Or :
        
        \[ \theta \in ]0; \pi[ \iff 0 < \dfrac{(2k+1)\pi}{2n} < \pi \iff -1 < k < n-1 \iff 0 \leq k \leq n-1\]
        
        Donc \boxsol{$\forall k \in \llbracket0, n-1\rrbracket$, $\cos{\dfrac{(2k+1)\pi}{2n}} \in ]-1;1[$ est racine de $T_n$}. De plus \boxsol{$T_n$ est scindé sur $\bdR$} car ces $n$ racines sont distinctes, et $T_n$ est de degré $n$. On obtient donc finalement la factorisation :
        
        \[ \boxsol{$T_n = 2^{n-1}\displaystyle\prod_{k=0}^{n-1}\left(X - \cos{\dfrac{(2k+1)\pi}{2n}}\right)$} \]
    }  
    
\end{enumerate}

\newpage
\subsection{Polynômes de deuxième espèce}

On définit les polynômes $\suite{U_n}$ de Tchebychev de deuxième espèce par :

\begin{equation}
     \forall n \in \bdN,\qquad U_n = \dfrac{1}{n+1}T'_{n+1}
\end{equation}

\begin{enumerate}[resume]
    \item Montrer que :
    
    \begin{equation}
        \forall n \in \bdN\qquad \forall \theta \in \bdR \backslash \pi \bdZ,\qquad U_n(\cos \theta) = \dfrac{\sin{(n+1)\theta}}{\sin \theta} 
    \end{equation}
    
    \boxans{
        Soit $n \in \bdN$ et $\theta \in \bdR \backslash \pi \bdZ$. On utilise ici la notation différentielle pour désigner la dérivée :
        
        \[ \dfrac{\dif \theta_n(\cos \theta)}{\dif \theta} = \dfrac{\dif \cos{n\theta}}{\dif \theta} = -n\sin{n\theta} \qquad\et\qquad \dfrac{\dif \theta_n(\cos \theta)}{\dif \theta}= -\sin(\theta)T'_n(\cos \theta)\]
        
        Donc $T'_n(\cos \theta) = \dfrac{-1}{\sin \theta}\dfrac{\dif \theta_n(\cos \theta)}{\dif \theta} = \dfrac{n\sin{n\theta}}{\sin \theta}$. Donc on a :
        
        \[U_n(\cos \theta) = \dfrac{1}{n+1}T'_{n+1}(\cos \theta) = \dfrac{n+1}{n+1}\dfrac{\sin{(n+1)\theta}}{\sin \theta} \qquad\text{donc}\qquad \boxsol{$U_n(\cos \theta) = \dfrac{\sin{(n+1)\theta}}{\sin \theta}$}\]
    
    }
    
    \item En déduire les propriétés suivantes :
    
    \begin{enumerate}
        \item La suite $\suite{U_n}$ vérifie la même relation de récurrence question \enumref{question:3} que la suite $\suite{T_n}$.
        
        \boxans{
            Soit $n \in \bdN$ et $\theta \in \bdR \backslash \pi \bdZ$. On procède comme à la question \enumref{question:3}.
            \begin{align*}
                2\cos(\theta)U_{n+1}(\cos \theta) - U_n(\cos \theta) &= \dfrac{2\cos \theta\sin{(n+1)\theta}}{\sin \theta} - \dfrac{\sin{n\theta}}{\sin \theta}\\
                &= \dfrac{\sin{(n+1)\theta + \theta} + \sin{(n+1)\theta - \theta) - \sin{n\theta}}}{\sin \theta}\\
                &= \dfrac{\sin{(n+2)\theta} + \sin{n\theta} - \sin{n\theta}}{\sin \theta} = \dfrac{\sin{(n+2)\theta}}{\sin \theta}
            \end{align*}
            
            Donc $\forall x \in [-1, 1]$, on a $U_{n+2}(x) = 2xU_{n+1}(x) - U_n(x)$. Le polynôme $Q = U_{n+2} + U_n - 2XU_{n-1}$ a une infinité de racines sur $[-1, 1]$ et donc $Q = 0$, d'où \boxsol{$ U_{n+2} = 2XU_{n-1} - U_n$}.
        }
        
        \item Pour tout entier naturel $n$, le polynôme $U_n$ est scindé sur $\bdR$ à racines simples appartenant à $]-1; 1[$.
    
        \boxans{
            Soit $n \in \bdN$. $T_{n+1}$ est scindé sur $\bdR$ et possède $n-1$ racines distinctes dans $]-1;1[$, que l'on trie dans l'ordre croissant et indexe par $i$ : $\alpha_0 < \alpha_1 < \ldots < \alpha_{n+1}$. Pour $i \in \llbracket 0; n-1\rrbracket$, on a $\alpha_i < \alpha_{i+1}$ et $T_{n+1}(\alpha_i) = T_{n+1}(\alpha_{i+1}) = 0$. $T_{n+1}$ est continue donc par théorème de Rolle :
            
            \[\forall i \in \llbracket 0; n-1\rrbracket,\qquad \exists \beta_i \in ]-\alpha_i;\alpha_{i+1}[,\qquad T'_{n+1}(\beta_i) = 0\]
            
            On a ainsi obtenu $n$ racines $\beta_i$, distinctes car sur des segments différents, et toutes dans $\left]-1;1\right[$.
            
            Or $T'_{n+1}$ est de degré au plus $n$ donc \boxsol{$U_n$ est scindé sur $\bdR$ à racines simples appartenant à $]-1; 1[$}.
        }
        
        \item Déterminer les racines de $U_n$ ainsi que la factorisation de $U_n$ sur $\bdR$.
        
        \boxans{
            Soit $n \in \bdR$. On procède comme à la question \enumref{question:6}. Soit $\theta \in ]0; \pi[$, on a les équivalences :
            
            \[U_n(\cos \theta) = 0 \iff \dfrac{\sin{(n+1)\theta}}{\sin \theta} = 0 \iff \sin{(n+1)\theta} = 0 \iff (n+1)\theta \equiv 0 \ [\pi]\]
            
            Donc $\exists k \in \bdZ$, tel que $(n+1)\theta = k\pi$. De plus $0 < \theta < \pi$ donc on obtient $1 \leq k \leq n$.
            
            Le coefficient dominant de $T_{n+1}$ est $2^n$. Or $T_{n+1}$ est un polynôme de degré $n+1$ donc en dérivant on multiplie par $n+1$ le coefficient de ce monôme dominant. Ainsi le coefficient dominant de $T'_{n+1}$ est $(n+1)2^n$, donc celui de $U_n$ est $2^n$. On obtient donc finalement \boxsol{$U_n = 2^n \displaystyle\prod_{k = 1}^{n} \left(X-\cos{\dfrac{k\pi}{n+1}}\right)$}.
        }
    \end{enumerate}
\end{enumerate}

\section{Le théorème de Block-Thielmann}

Dans cette partie, on munit l'ensemble $\bdR[X]$ des polynômes de la loi de composition interne associative donnée par la composition, notée $\circ$. Précisément, étant donné deux polynômes $(P, Q) \in \bdR[X]^2$, et en notant $\suite{p_n}$ la suite presque nulle des coefficients de $P$, on a :

\begin{equation}
    P \circ Q = \sum_{k=0}^{+\infty} p_kQ^k
\end{equation}

On dire que deux polynômes $P$ et $Q$ \textit{commutent} si $P \circ Q = Q \circ P$.

On appelle \textit{suite commutante} toute suite $\suiteZ{P_n}$ de polynômes de $\bdR[X]$ pour laquelle :

\begin{equation}
    \forall n \in \bdN^*, \qquad \deg P_n = n \qquad\et\qquad \forall (n, m) \in (\bdN^*)^2, \qquad P_m \circ P_n = P_n \circ P_m
\end{equation}

Enfin, on note $G$ l'ensemble des polynômes de $\bdR[X]$ de degré exactement $1$.

\textbf{Objectif du problème :} détermine l'ensemble des suites commutantes \textit{(Théorème de Block-Thielmann)}.

\begin{enumerate}
    \item\label{question2:1} Vérifier que $\suiteZ{X^n}$ et que $\suiteZ{T_n}$ sont deux suites commutantes.
    
    \boxans{
        \begin{enumerate}
            \ithand Soit $n \in \bdN^*$. On a $\deg X^n = n$. Soit aussi $m \in \bdN^*$, on a :
            
            \[ X^n \circ X^m = (X^m)^n = (X^n)^m = X^m \circ X^n\]
            
            Donc \boxsol{$\suiteZ{X^n}$ est une suite commutante}.
            
            \ithand Soit $n \in \bdN^*$. On a montré question \enumref{question:5} que $\deg T_n = n$. Soit aussi $m \in \bdN^*$.
            
            \[ \forall \theta \in \bdR,\qquad T_n(\cos \theta) = \cos{n\theta} \qquad\et\qquad T_m(\cos \theta) = \cos{m\theta}\]
            
            En posant $\theta' = n\theta$, on obtient :
            
            \[T_m(T_n(\cos \theta)) = T_m(\cos{n\theta}) = T_m(\cos(\theta')) = \cos{m\theta'} = \cos{mn\theta} = T_{m+n}(\cos \theta)\]
            
            Donc $\forall x \in [-1; 1]$, $T_m \circ T_n (x) = T_{m+n}(x)$. En posant $Q = T_m \circ T_n - T_{m+n}$, on trouve une infinité de racines pour $Q$ sur $[-1; 1]$ donc $Q = 0$, d'où $T_m \circ T_n = T_{m+n}$. Par commutativité de l'addition, on obtient alors :
            
            \[T_n \circ T_m = T_{n+m} = T_{m+n} = T_m \circ T_n\]
            
            Donc \boxsol{$\suiteZ{T_n}$ est une suite commutante}. 
        \end{enumerate}
    }
    
    \item\label{question2:2} Préciser le degré de $P \circ Q$ lorsque $P$ et $Q$ sont deux polynômes non constants de $\bdR[X]$.
    
    \boxans{
        Soient $(P,Q) \in \bdR[X]^2$ deux polynômes non constants, de degré respectifs $n \in \bdN^*$ et $m \in \bdN^*$. Alors il existe deux séquences $(p_i)_{0 \leq i \leq n}$ et $(q_k)_{0 \leq k \leq m}$ telles que:
        
        \[ P = \sum_{i=0}^n p_iX^i \qquad\et\qquad Q = \sum_{k=0}^m q_kX^k \qquad\text{donc} P \circ Q = \sum_{i=0}^n p_i\left(\sum_{k=0}^m q_kX^k\right)^i\]
        
        On considère alors le terme de la somme pour l'indice $i = n$, puis $k = m$ :
        
        \[ P \circ Q = p_n\left(\sum_{k=0}^m q_kX^k\right)^n + \underbrace{\dots}_{\in \bdR_{n+m-1}[X]} = p_n\left(q_mX^m + \dots \right)^n + \dots = p_nq_m^{nm}p^{nm} X^{nm} + \underbrace{\dots}_{\in \bdR_{n+m-1}[X]}\]
        
        $p_nq_m \neq 0$ donc le monomôme de plus grand degré est $p_nq_m^{nm}X^{nm}$ donc \boxsol{$\deg (P \circ Q) = \deg P \times \deg Q$}.\\
        
        \textbf{N. B.} Lorsque $P$ ou $Q$ est un polynôme constant non nul de degré égal à $0$, l'indéterminée $X$ disparaît dans la composition (dans les deux sens), ainsi la relation est toujours valide : 
        
        \[\deg (P\circ Q) = 0 = \deg P \times 0 = 0 \times \deg Q\]
        
        Lorsque $P$ ou $Q$ est le polynôme nul cependant, on n'obtient généralement pas un polynôme nul en composant, notamment lorsqu'on compose un polynôme constant non nul par le polynôme nul par exemple.
    }
    
    \item\label{question2:3} Montrer que $(G, \circ)$ est un groupe. Pour tout élément $U \in G$, on notera dans la suite $U^{-1}$ l'inverse de $U$ pour la loi $\circ$.
    
    \boxans{
        $\circ$ est toujours associative. $\forall P \in \bdR[X]$, on a $P \circ X = X \circ P = P$. Or $X = 1X + 0$ donc on a le neutre $X \in G$. Soit $P = aX + b \in G$ et $Q = cX + d \in G$ avec $a \neq 0$ et $b \neq 0$.
        \[ P \circ Q = a(cX + d) + b = acX + (ad + b) \in G \qquad\et\qquad Q \circ P = c(aX + b) + d = caX + (cb + d) \in G\]
        
        Donc $\circ$ est une loi de composition interne sur $G$.
        
        \[ P \circ Q = X \iff \left\lbrace\begin{array}{rl}
            ca &= 1  \\
            ad+b &= 0
        \end{array}\right.\iff  \left\lbrace\begin{array}{rl}
            c &= \sfrac{1}{a}  \\
            d &= -\sfrac{b}{a}
        \end{array}\right.\]
        
        Symétriquement, on a :
        
        \[Q \circ P = X \iff  \left\lbrace\begin{array}{rl}
            ca &= 1  \\
            cb+d &= 0
        \end{array}\right.\iff\left\lbrace\begin{array}{rl}
            c &= \sfrac{1}{a}  \\
            d &= -cb = -\sfrac{b}{a}
        \end{array}\right.\]
        
        $\forall U = aX + b\in G,\quad \exists U^{-1} = \dfrac{1}{a}X - \dfrac{b}{a}\in G,\quad U \circ U^{-1} = U^{-1} \circ U = X$, donc \boxsol{$(G, \circ)$ est un groupe}.
    }
    
    \item Soit $U = aX + b \in \bdR_1[X]$, avec $a \neq 0$.
    
    \begin{enumerate}
        \item Préciser $U^{-1}$ en fonction de $a$ et $b$.
        
        \boxans{
            On a montré à la question \enumref{question2:3} que \boxsol{$U^{-1} = \dfrac{1}{a}X - \dfrac{b}{a}$}.
        }
        
        \item Que valent $(U^{-1})^{-1}$ et $U \circ U^{-1}$, ainsi que $U^{-1} \circ U$ ?
        
        \boxans{
            Par propriété des groupes, on a \boxsol{$(U^{-1})^{-1} = U$}, \boxsol{$U \circ U^{-1} = X$} et \boxsol{ $U^{-1} \circ U = X$}.
        }
        
        \item Si $U$ et $V$ sont deux polynômes de degré $1$, exprimer $(U \circ V)^{-1}$ en fonction de $U^{-1}$ et $V^{-1}$.
        
        \boxans{
            Toujours par propriété des groupes, on a $(U \circ V)^{-1} = V^{-1} \circ U^{-1}$.
        }
        
        \item\label{question2:4:d} Montrer que, pour toute suite commutante $\suiteZ{P_n}$, la suite $\suiteZ{U^{-1} \circ P_n \circ U}$ est également commutante.
        
        \boxans{
            Soit une suite commutante $\suiteZ{P_n}$. Soit $n \in \bdN^*$, on a :
            
            \[\deg(U^{-1} \circ P_n \circ U) = \deg U^{-1} \times \deg P_n \times \deg U = 1 \times \deg P_n \times 1 = P_n = n\]
            
            Soit $m \in \bdN$. On a :
            \begin{align*}
                 \left(U^{-1} \circ P_n \circ U\right)\circ\left(U^{-1} \circ P_m \circ U\right) &= U^{-1} \circ P_n \circ \left(U\circ U^{-1}\right) \circ P_m \circ U\\
                 &= U^{-1} \circ P_n \circ P_m \circ U\\
                 &= U^{-1} \circ P_m \circ P_n \circ U\\
                 &= U^{-1} \circ P_m \circ \left(U\circ U^{-1}\right) \circ P_n \circ U\\
                 &= \left(U^{-1} \circ P_m \circ U\right)\circ\left(U^{-1} \circ P_n \circ U\right)
            \end{align*}
            
            Donc \boxsol{la suite $\suiteZ{U^{-1} \circ P_n \circ U}$ est également commutante}.\\
            
            \textbf{N. B.} Si la suite $\suiteZ{P_n}$ est commutante, alors $\suiteZ{Q_n} = \suiteZ{U^{-1} \circ P_n \circ U}$ l'est aussi. Dès lors pour $V \in G$, la $\suiteZ{V^{-1} \circ Q_n \circ V}$ est également commutante. 
            
            En prenant $V = U^{-1}$, on obtient que le sens réciproque : si $\suiteZ{U^{-1} \circ P_n \circ U}$ est commutante, alors $\suiteZ{U^{-1} \circ (U^{-1})^{-1} \circ P_n \circ U^{-1} \circ U} = \suiteZ{P_n}$ l'est aussi. Il s'agit donc en fait d'une équivalence.
        }
        
    \end{enumerate}
    
    \item\label{question2:5} Soit $P \in \bdR[X]$ non constant de degré $m$ et de coefficient dominant $\alpha$ et $Q \in \bdR[X]$ non constant de degré $n$ et de coefficient dominant $\beta$. Montrer que si $P$ et $Q$ commutent, alors $\beta^{m-1} = \alpha^{n-1}$.
    
    \boxans{
        On a montré question \enumref{question2:2} que les coefficients dominants de $P \circ Q$ et $Q \circ P$ sont respectivement $\alpha \beta^{nm}$ et $\alpha^{nm}\beta$.
        \[ P \et Q \ \text{commutent} \implies \alpha \beta^{nm} = \alpha^{nm}\beta \underset{\alpha \neq 0 \et \beta \neq 0}{\implies} \boxsol{$\beta^{m-1} = \alpha^{n-1}$}\]
    }
    
    \item\label{question2:6} Soit $\gamma \in \bdR$ fixé. On pose $\bsC = \{ P \in \bdR[X] \mid P \ \text{est non constant et} \ P \circ (X^2 + \gamma) = (X^2 + \gamma) \circ P\}$.
    
    \begin{enumerate}
        \item Montrer que tout élément de $\bsC$ est unitaire, \textit{c'est-à-dire de coefficient dominant $1$}.
        
        \boxans{
            Soit un polynôme $P \in \bsC$, on note $n \in \bdN^*$ son degré et $\alpha \in \bdR$ son coefficient dominant. Par définition $P$ commute avec $X^2 + \gamma$ de coefficient dominant $1$ et de degré $2$. On utilise le résultat de la question \enumref{question2:5} :
            
            \[ \alpha^{2-1} = 1^{n-1} \qquad\text{donc}\qquad \alpha = 1 \qquad\text{donc}\qquad \boxsol{tout élément de $\bsC$ est unitaire.}\]
        }
        
        \item\label{question2:6:b} On souhaite montrer que $\forall n \in \bdN^*$, $\bsC$ contient au plus un polynôme de degré $n$. On se donne $(Q_1, Q_2) \in \bsC^2$ de même degré. Montrer en étudiant le polynôme $(Q_1 - Q_2) \circ (X^2 + \gamma)$ que $Q_1 = Q_2$. \textit{On étudiera le degré}.
        
        \boxans{
            On a montré que $Q_1$ et $Q_2$ sont unitaires, ainsi $\deg (Q_1 - Q_2) < n$. On suppose alors par l'absurde que l'on peut poser $k \in \bdN$, tel que $k = \deg (Q_1 - Q_2)$, donc $k < n$. Alors :
            \begin{align*}
                 (Q_1 - Q_2) \circ (X^2 + \gamma) &= Q_1 \circ (X^2 + \gamma) - Q_2 \circ (X^2 + \gamma) = (X^2 + \gamma) \circ Q_1 + (X^2 + \gamma) \circ Q_2\\
                 & = Q_1^2 + \gamma - Q_2^2 - \gamma = Q_1^2 - Q_2^2 \eq{\substack{\text{anneau}\\(\bdR[X], +, \times)}} (Q_1 + Q_2)(Q_1 - Q_2)
            \end{align*}
            \[\text{Or} \ \left\lbrace\begin{array}{lll}
                \deg((Q_1 - Q_2) \circ (X^2 + \gamma)) &= \deg((Q_1 - Q_2)) \times \deg(X^2 + \gamma) &= 2k \\
                \deg((Q_1 + Q_2)(Q_1 - Q_2)) &= \deg ((Q_1 + Q_2)) + \deg((Q_1- Q_2)) &= n + k
            \end{array}\right.\]
            
            Donc $2k = k + n$ d'où $n = k$, ce qui est absurde puisque $k < n$. On a donc nécessairement $k = -\infty$, soit $Q_1 - Q_2 = 0$. Donc finalement \boxsol{$Q_1 = Q_2$}.
        }
        
        \item\label{question2:6:c} Montrer que si $\bsC$ contient un polynôme de degré $3$ alors $\gamma \in \{-2, 0\}$.
        
        \textit{On commencera par montrer que si $P$ est de degré $3$ dans $\bsC$, alors $P$ doit être impair, donc de la forme $X^3 + bX$}.
        
        \boxans{
            Soit $P \in \bsC$ de degré $3$. On note $P = X^3 + aX^2 + bX + c$ :
            
            \[ \forall x \in \bdR, \qquad \left[P \circ (X^2 + \gamma) \right](-x) = P((-x)^2 + \gamma) = P(x^2 + \gamma) = \left[P \circ (X^2 + \gamma) \right](x)\]
            
            Ainsi $P \circ (X^2 + \gamma)$ est pair, donc $(X^2 + \gamma) \circ P$ aussi. Donc pour tout réel $x \in \bdR$ :
            \[ \left[(X^2 + \gamma) \circ P\right](-x) = \left[P \circ (X^2 + \gamma) \right](x) \iff P(x)^2 +\gamma = P(-x)^2 + \gamma \iff P(-x)^2 = P(x)^2\]
            Puisque $P$ est continu, le facteur en $\pm$ est identique pour tout $x$ d'un même intervalle ouvert sur lequel $P$ ne s'annule pas. On peut alors considérer $]-\infty, -\delta[$ et $]\delta, +\infty[$, avec $\delta \gg 0$ très grand.
            
            Or $\lim\limits_{x \o -\infty} P(x) = -\infty$ et $\lim\limits_{x \to +\infty} P(x) = +\infty$, on a $P(-x) = -P(x)$, donc :
            \begin{align*}
                && x^3 + ax^2 + bx + c &= -((-x)^3 + a(-x)^2 + b(-x) +c) = x^3 - ax^2 + bx -c\\
                \text{donc}&& ax^2 + c &= -ax^2 - c \qquad \text{donc} \qquad ax^2 + c = 0
            \end{align*}
            
            On pose $Q = P - X^3 - bX$. $Q$ a une infinité de racines donc $Q = 0$ donc $P = X^3 + bX$.
            \begin{align*}
                \text{On a} && (X^3 + bX) \circ (X^2 + \gamma) &= (X^2 + \gamma) \circ (X^3 + bX)\\
                \text{donc} && (X^2 + \gamma)^3 + b(X^2 + \gamma) &= (X^3 + bX)^2 + \gamma\\
                \text{donc} && X^6 + 3\gamma X^4 + (3\gamma^2 + b)X^2 + (\gamma^3 + b\gamma) &= X^6 + 2bX^4 + b^2X^2 + \gamma
            \end{align*}
            
            On identifie alors pour le coefficient de $X^4$ que $3\gamma = 2b$ soit $b = \sfrac{3}{2}\gamma$. On identifie de plus devant $X^2$ :
            
            \[ 3\gamma^2 + b = b^2 \quad \text{donc} \quad 3\gamma^2 + \frac{3}{2}\gamma = \frac{9}{4}\gamma^2 \quad \text{donc} \quad 12\gamma^2 + 6\gamma = 9\gamma^2 \quad \text{donc} \quad 3\gamma^2 + 6\gamma = 0\]
            
            On a donc $\gamma(\gamma + 2) = 0$. On a donc bien \boxsol{$\gamma \in \{-2, 0\}$}.
            
        }
        
        \item\label{question2:6:d} Montrer que si $\gamma = 0$ alors $\bsC = \{X^n\}_{n \in \bdN^*}$.
        
        \boxans{
            Si $\gamma = 0$, alors $X^2 + \gamma = X^2$. Or on a montré question \enumref{question2:1} que $\suiteZ{X^n}$ est une suite commutante \textit{- donc $E = \{X^n\}_{n \in \bdN^*} \subset \bsC$ -} et question \enumref{question2:6:b} que $\bsC$ contenait au plus un polynôme de chaque degré. Ainsi ne peut-il y avoir autre élément de $\bsC$ qui ne soit dans $\{X^n\}_{n \in \bdN^*}$. Donc \boxsol{$\bsC = \{X^n\}_{n \in \bdN^*}$}.
            %Soit $P \in \bsC$. On sait que $P$ est unitaire et non constant donc :
            %\[ \text{On pose} \ n = \deg P \geq 1 \qquad\text{donc}\qquad \exists Q \in \bdR_{n-1}[X],\qquad P = X^n + Q\]
            %On a alors :
            %\[(X^2 + Q) \circ X^2 = X^2 \circ (X^n + Q) \quad\text{donc}\quad X^{2n} + Q = X^{2n} + 2QX^n + Q^2 \quad\text{donc}\quad Q = 2X^nQ + Q^2\]
            %Notons alors $k = \deg Q < n$ et $\alpha_k$ le coefficient dominant de $Q$. Regardons le monôme de degré $n+k > k$ dans l'égalité ci-dessus.
            %\begin{enumerate}
            %    \ithand $Q$ est de degré $k$, donc il n'y a pas de terme en $X^{n+k}$ (puisque $n \geq 1$, on a $n + k > k$).
            %    \ithand Dans $2X^nQ$, cela ne peut être que $2X^n\alpha_kX^k = 2\alpha_kX^{n+k}$.
            %    \ithand $Q$ est de degré $k$ donc $Q^2$ est de degré $2k$. Or $k < n$ donc $2k < n+k$. Il n'y a donc pas non plus de terme en $X^{n+k}$ dans $Q^2$.
            %\end{enumerate}
            %On a donc à $2\alpha_k = 0$ soit $\alpha_k = 0$. Ainsi $Q = 0$ donc $P = X^n$. Finalement, \boxsol{$\bsC = \{X^n\}_{n \in \bdN^*}$}.
        }
    \end{enumerate}
    
    \item \begin{enumerate}
        \item\label{question2:7:a} Soient $(a, b, c) \in \bdR^3$ avec $a \neq 0$. On pose $U = aX + \sfrac{b}{2}$. Montrer que le polynôme $U \circ (aX^2 + bX + c) \circ U^{-1}$ est de la forme $X^2 + \gamma$ avec un certain $\gamma \in \bdR$.
        
        \boxans{
            On a montré question \enumref{question2:3} que $U^{-1} = \dfrac{1}{a}X - \dfrac{b}{2a}$. Alors :
            \begin{align*}
                \boxsoll{$U \circ (aX^2 + bX + c) \circ U^{-1}$} &= \left(aX + \frac{b}{2}\right) \circ (aX^2 + bX +c) \circ \left(\dfrac{1}{a}X-\dfrac{b}{2a}\right)\\
                &= \left(aX + \frac{b}{2}\right) \circ \left(a\left(\dfrac{1}{a}X-\dfrac{b}{2a}\right)^2 + b\left(\dfrac{1}{a}X-\dfrac{b}{2a}\right) + c\right)\\
                &= \left(aX + \frac{b}{2}\right) \circ \left(\dfrac{1}{a}X^2 + \dfrac{b^2}{4} - \dfrac{b^2}{2a}+c\right)\\
                &= \boxsolr{$X^2 + \underbrace{\dfrac{b^2a}{4} - b^2 + ca + \dfrac{b}{2}}_{\gamma}$}
            \end{align*}
        }
        
        \item\label{question2:7:b} Trouver un polynôme $V \in \bdR[X]$ de degré $1$ pour lequel $V \circ (X^2 - 2) \circ V^{-1} = T_2$, où $T_2 = 2X^2 - 1$.
        
        \boxans{
            On pose $V = aX+ b$. On ne veut pas de terme en $X$ dans $V \circ (X^2 - 2) \circ V^{-1}$ donc $b = 0$. Or :
            \[ aX \circ (X^2 -2 ) \circ \dfrac{1}{a}X = aX \circ \left(\dfrac{1}{a^2}X^2 - 2\right) = \dfrac{1}{a}X^2 - 2a \qquad\text{donc} \boxsol{$V = \sfrac{1}{2}X$ est solution}\]
        }
    \end{enumerate}
    
    \item Déterminer toutes les suites commutantes. Précisément, si $\suiteZ{P_n}$ est une suite commutante, alors elle est de la forme :
    
    \begin{equation}
        \suiteZ{U^{-1} \circ X^n \circ U} \qquad\text{ou}\qquad \suiteZ{U^{-1} \circ T_n \circ U} \qquad\text{avec}\qquad U \in G
    \end{equation}
    
    \boxans{
        Soit $(a, b, c) \in \bdR^3$ tels que $P_2 = aX^2 + bX + c$. On pose $U = aX + \sfrac{b}{2} \in G$ et $V = \sfrac{1}{2}X \in G$.
        
        On pose $\suiteZ{Q_n} = \suiteZ{U \circ P_n \circ U^{-1}}$. On a montré question \enumref{question2:4:d} que puisque $\suiteZ{P_n}$ est une suite commutante, $\suiteZ{Q_n}$ aussi, et question \enumref{question2:7:a} que $Q_2 = U \circ P_2 \circ U^{-1} = X^2 + \gamma$ avec $\gamma \in \bdR$. Donc :
        \[ \forall n \in \bdN, \qquad Q_n \circ Q_2 = Q_2 \circ Q_n \qquad\text{soit}\qquad Q_n \circ (X^2 + \gamma) = (X^2 + \gamma) \circ Q_n \]
        Donc par définition, $\{Q_n\}_{n \in \bdN^*} \subset \bsC$. Or $\deg Q_n = n$, et par unicité du polynôme pour chaque degré dans $\bsC$ (question \enumref{question2:6:b}) on a $\bsC = \{Q_n\}_{n \in \bdN^*}$. On a $Q_3 \in \bsC$ et $\deg Q_3 = 3$, donc d'après la question \enumref{question2:6:c} $\gamma = 0$ ou $\gamma = -2$.
        
        \begin{enumerate}
            \ithand Si $\gamma = 0$, on a montré question \enumref{question2:6:d} que $\bsC = \{ X^n \}_{n \in \bdN^*}$ donc :
            \[\forall n \in \bdN,\qquad  U \circ P_n \circ U^{-1} = X^n \qquad\text{donc}\qquad \boxsol{$\suiteZ{P_n} = \suite{U^{-1} \circ X^n \circ U}$}\]
            \ithand Si $\gamma = -2$, on a montré question \enumref{question2:7:b} que $V \circ Q_2 \circ V^{-1} = T_2$. On pose $\suiteZ{R_n} = \suiteZ{V^{-1} \circ T_n \circ V}$. On a montré question \enumref{question2:1} que $\suiteZ{T_n}$ est une suite commutante, donc d'après la question \enumref{question2:4:d}, $R_n$ aussi.
            Or $R_2 = V^{-1} \circ T_2 \circ V = Q_2 = X^2 + \gamma$. Or :
            \[ \forall n \in \bdN,\qquad R_n \circ R_2 = R_2 \circ R_n \qquad\text{donc}\qquad R_2 \circ (X^2 + \gamma) = (X^2 + \gamma) \circ R_2\]
            De même, on a $\{R_n\}_{n \in \bdN^*} \subset \bsC$ et par unicité pour chaque degré (question \enumref{question2:6:b})  $\{R_n\}_{n \in \bdN^*} = \bsC = \{Q_n\}$.
            \[ \forall n \in \bdN,\qquad U \circ P_n \circ U^{-1} = V^{-1} \circ T_n \circ V \qquad\text{donc}\qquad U^{-1} \circ V^{-1} \circ T_n \circ V \circ U\]
            On pose $W = V \circ U$, or $U \in G$ et $V \in G$ donc puisque $(G, \circ)$ est un groupe (question \enumref{question2:3}) $W \in G$ et $W^{-1} = U^{-1} \circ V^{-1}$.
            Donc \boxsol{$\suiteZ{P_n} = \suite{W^{-1} \circ T_n \circ W}$}.
        \end{enumerate}
    }
\end{enumerate}

\end{document}