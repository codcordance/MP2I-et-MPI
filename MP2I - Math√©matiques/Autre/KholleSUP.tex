\documentclass[french,bookmarks]{article}

\usepackage{geometry}
\geometry{
    a4paper,
    total={170mm,257mm},
    left=20mm,
    top=20mm
}

\usepackage[french]{babel}
\usepackage[utf8]{inputenc}
\usepackage{fancyhdr}
 
\begin{document}

    \pagestyle{fancy}
    \fancyhf{}
    \rhead{2022-2023}
    \lhead{MP2I}
    \chead{Colle 0}
    \cfoot{\thepage}
    
    \section*{Révisions de Tle}
    
    \subsection*{Ensembles, combinatoire et dénombrement}
    
    \begin{enumerate}
        \item[$\bullet$] Raisonnement par récurrence. Récurrence double, forte.
        
        \item[$\bullet$] Nombre d'éléments d'une réunion d'ensembles, nombre d'éléments d'un produit cartésien. Nombre de $k$-uplets (ou $k$-listes) d'un ensemble à $n$ éléments.
        
        \item[$\bullet$] Définition de $n!$. Notion de permutation, arrangements ($k$-uplets d'éléments distincts d'un ensemble à $n$ éléments), combinaisons. Coefficient binomiaux et propriétés. Relation et triangle de Pascal.
    \end{enumerate}
    
    \subsection*{Limites de suites et de fonctions}
    
    \begin{enumerate}
        \item[$\bullet$] Limite infinie une suite $(u_n)$ tend vers $+\infty$ si tout intervalle de la forme $[a ; +\infty[$ contient toutes les valeurs $u_n$ à partir d'un certain rang. Suite tendant vers $-\infty$.
        
        \item[$\bullet$] Limite finie d'une suite : la suite $(u_n)$ converge vers le nombre réel $\ell$ si tout intervalle ouvert contenant $\ell$ contient toutes les valeurs $u_n$ à partir d'un certain rang.
        
        \item[$\bullet$] Théorèmes de comparaison, théorème des gendarmes et de convergence monotone (toute suite croissante majorée et décroissante minorée converge). Opérations sur les limites.
        
        \item[$\bullet$] Limite finie ou infinie d'une fonction, en $+\infty$, en $-\infty$, en un point. Théorèmes.
        
        \item [$\bullet$] Limites des fonctions de références étudiées en classe de première : puissances entières $x \mapsto x^n$, racine carrée $x \mapsto \sqrt{x}$, fonction exponentielle $x \mapsto e^x$. Croissance comparée de $x \mapsto e^x$ et $x \mapsto x^n$ en $+\infty$.
    \end{enumerate}
    
    \subsection*{Fonctions usuelles et dérivation}
    
    \begin{enumerate}
        \item[$\bullet$] Fonction logarithme népérien $\ln$, comme réciproque de la fonction exponentielle. Propriétés algébriques, dérivée, variations. Limite en $0$ et en $+\infty$, croissance comparée.
        
        \item[$\bullet$] Fonctions trigonométrique sinus et cosinus. Dérivées, variations, courbes représentatives.
        
        \item[$\bullet$] Composée de deux fonctions et dérivée.
        
        \item[$\bullet$] Dérivée seconde d'une fonction, convexité : définition par la position relative de la courbe représentative et des sécantes. Propriétés. Notion de point d'inflexion.
    \end{enumerate}
    
    \subsection*{Continuité, primitives, équations différentielles et calcul intégral}
    
    \begin{enumerate}
        \item[$\bullet$] Fonction continue en un point (définition par les limites), sur un intervalle. Toute fonction dérivable est continue. Théorème des valeurs intermédiaires.
        
        \item[$\bullet$] Equation différentielle $y' = f$. Notion de primitive d'une fonction continue sur un intervalle. Primitives des fonctions de référence. Equation différentielle $y' = ay + b$, où $a$ et $b$ sont des nombres réels.
        
        \item[$\bullet$] Définition de l'intégrale d'une fonction continue positive sur un segment $[a; b]$, comme aire sous la courbe représentative de $f$. 
        
        \item[$\bullet$] Si $f$ est une fonction continue positive sur $[a, b]$, alors la fonction $F_a$ définie sur $[a; b]$ par $F_a(x) = \displaystyle \int_a^x f(t)\textrm{d}t$ est la primitive de $f$ qui s'annule en $a$. Relation $F(b) - F(a) = \displaystyle\int_a^b f(x)\textrm{d}x$ où $F$ est une primitive de $f$.
            
        \item[$\bullet$] Propriétés de l'intégrale. Linéarité, positivité et intégration des inégalités. Relation de Chasles.
        
        \item[$\bullet$] Valeur moyenne d'une fonction. Intégration par parties.
    \end{enumerate}
    
    \section*{Questions / Exercices de cours}
    
    \begin{enumerate}
        \item Montrer par récurrence que pour tout entier naturel $n$ non nul, \qquad $1^3 + 2^3 + \dots + n^3 = \left(\displaystyle\frac{n(n+1)}{2}\right)^2$
        
        %\item Déterminer la limite de la suite $(a_n)$ telle que $a_0 = 1$ et pour tout entier $n$, \qquad $a_{n+1} = a_n + n^2$.
        
        \item Déterminer la limite en $+\infty$ et en $-1$ de la fonction $f$ définie sur $] -1; +\infty [$ par \qquad $f(x) = \displaystyle\frac{4x + 5}{x + 1}$.
        
        \item $f$ est la fonction définie sur $[0; 1]$ par \qquad $f(x) = 2 + 3e^{-x}$.
        
        Démontrer qu'il existe au moins un réel $c$ compris entre $0$ et $1$ et tel que $f(c) = 4$.
        
        \item Déterminer la solution $g$ de l'équation différentielle $y' = y + 1$ telle que $g(1) = 2$.
        
        \item Déterminer, à l'aide de l'intégration par partie, une primitive de $\ln$.
    \end{enumerate}
\end{document}