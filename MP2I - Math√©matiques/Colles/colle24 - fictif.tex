\documentclass[a4paper,french,bookmarks]{article}

\usepackage{./Structure/4PE18TEXTB}

\newboxans

\begin{document}

    \stylizeDoc{Mathématiques}{Programme de khôlle 24}{Énoncés et résolutions}
    
    \begin{tcolorbox}[
        enhanced,
        frame hidden,
        sharp corners,
        spread sidewards    = 5pt,
        halign              = center,
        valign              = center,
        interior style      = {left color=main4!15, right color=main2!12},
        arc                 = 0 cm
    ]
        \begin{minipage}{0.83\linewidth}
            \textbf{\sffamily Note :} Ce programme est \guill{fictif}, dans le sens où il est uniquement pour l'entraînement et qu'aucune khôlle n'a eu lieu cette semaine.
        \end{minipage}
    \end{tcolorbox}

    \section*{Espaces préhilbertiens réels}

    \questionsdecours

    \begin{enumerate}
        \item Définition d'un produit scalaire, d'un espace préhilbertien réel, d'un espace euclidien.
        
        Définition de la norme euclidienne et de la distance euclidienne.
        
        \boxans{
        
        }
        
        \item Montrer que sur $\bcM_n\p{\bdR}$, $\phyavg{A, B} = \Tr{\mtrans{A}B}$ défini un produit scalaire.
        
        Montrer que sur $\bcC^0\p{\intc{a, b}, \bdR}$, $\phyavg{f, g} = \displaystyle \int_a^b f\p{t}g\p{t}\dif t$ défini un produit scalaire.
        
        \boxans{
        
        }
        
        \item Montrer l'inégalité de \textsc{Cauchy}-\textsc{Swarchz} et le cas d'égalité.
        
        \boxans{
        
        }
        
        \item Montrer l'inégalité triangulaire sur les normes euclidiennes et le cas d'égalité.
        
        \boxans{
        
        }
        
        \item Quelles sont les formules de polarisation ?
        
        \boxans{
        
        }
        
        \item Montrer qu'une famille orthogonale sans vecteur nul est libre.
        
        \boxans{
        
        }
        
        \item Montrer le théorème de \textsc{Pythagore}.
        
        \boxans{
        
        }
        
        \item Décrire dans le cas général l'algorithme d'orthonormalisation de \textsc{Gram}-\textsc{Schmidt}.
        
        \boxans{
        
        }
        
    \end{enumerate}
\end{document}
