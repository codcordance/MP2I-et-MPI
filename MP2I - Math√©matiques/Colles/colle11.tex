\documentclass[a4paper,french,bookmarks]{article}
\usepackage{./Structure/4PE18TEXTB}

\begin{document}
\stylizeDoc{Mathématiques}{Programme de khôlle 11}{Énoncés et résolutions}

\section*{Analyse Asymptotique}

\subsection*{Comparaison des suites}

Suites dominées, négligeables, équivalentes. Notations de Landau $\o$, $\O$, $\asymp$. Croissances comparées. Opérations sur les équivalents : signe, limite, produit, quotient, puissances. Limites de fonctions usuelles et équivalents.

\subsection*{Comparaison des fonctions}

Voisinage. Fonctions dominées, négligeables, équivalentes.

Propriétés et Opérations. Changement de variable. Croissances comparées. Équivalents usuels.

\subsection*{Développements Limités}

\begin{enumerate}
    \ithand Définition d’une fonction admettant un développement limité à l’ordre $n$ au voisinage de $a \in \bdR$.
    
    \ithand Partie régulière d’un DL, unicité des coefficients d’un DL. Application aux fonctions paires, impaires.
    
    \ithand Troncature d’un DL à l’ordre $p < n$.
    
    \ithand Équivalence entre continuité et existence d’un DL à l’ordre $0$ ; entre dérivabilité et existence d’un DL à l’ordre $1$. Contre-exemple pour l’ordre $n \geq 2$.
    
    \ithand Formule de Taylor-Young (admise pour le moment) pour une fonction de classe $\mathcal{C}^n$ en $a \in \bdR$. DL usuels en $0$ à connaître : $e^x$, $\ch x$, $ \sh x, \cos x$, $ \sin x$, $ \left(1+x\right)^a$, $ \ln{1+x}, \dfrac{1}{1-x}$, $ \dfrac{1}{1+x}$ à tous ordres ; $\tan x $ et $\th x$ à l'ordre  $5$.

    \ithand Opérations sur les DL : somme, produit (avec forme normalisée).Méthode de composition de DL sur des exemples. Inverse et quotient de DL. Primitivation d'un DL.
    
    \ithand Développements asymptotiques : exemples et applications.
    
    \ithand Application à l'étude d'une fonction au voisinage d'un point $a \in \bdR$ ou en $\pm \infty$.
\end{enumerate}

\section*{Questions / Exercices de cours / Savoir faire}

\begin{enumerate}
    \item Donner les DL en $0$ à tous ordres de quelques fonctions usuelles.
    
    \boxans{
    DL usuels en $0$ à tous ordres :
    
    \begin{align*}
        \text{\ding{43}} && e^x &\eq{x \to 0} 1 + x + \dfrac{x^2}{2} + \dfrac{x^3}{3!} + \dots + \dfrac{x^n}{n!} + \o{}{x^n}\\
        \text{\ding{43}} && \ch x &\eq{x \to 0} 1 + \dfrac{x^2}{2!} + \dfrac{x^4}{4!} + \dots + \dfrac{x^{2n}}{(2n)!} + \o{}{x^{2n+1}}\\
        \text{\ding{43}} && \sh x &\eq{x \to 0} x + \dfrac{x^3}{3!} + \dfrac{x^5}{5!} + \dots + \dfrac{x^{2n+1}}{(2n+1)!} + \o{}{x^{2n+2}}\\
        \text{\ding{43}} && \cos x &\eq{x \to 0} 1 - \dfrac{x^2}{2!} + \dfrac{x^4}{4!} - \dots + (-1)^n\dfrac{x^{2n}}{(2n)!} + \o{}{x^{2n+1}}\\
        \text{\ding{43}} && \sin x &\eq{x \to 0} x - \dfrac{x^3}{3!} + \dfrac{x^5}{5!} - \dots + (-1)^n\dfrac{x^{2n+1}}{(2n+1)!} + \o{}{x^{2n+2}} \\
        \text{\ding{43}} && (1+x)\alpha  &\eq{x \to 0} 1 + \alpha x + \dfrac{\alpha(\alpha-1)}{2}x^2 + \dots + \dfrac{\alpha(\alpha-1)\dots(\alpha-n+1)}{n!}x^n + \o{}{x^n}\\
        \text{\ding{43}} && \dfrac{1}{1+x} &\eq{x \to 0} 1 - x + x^2 - x^3 - \dots + (-1)^nx^n + \o{}{x^n}\\
        \text{\ding{43}} && \dfrac{1}{1-x} &\eq{x \to 0} 1 + x + x^2 + x^3 + \dots + x^n + \o{}{x^n}\\
        \text{\ding{43}} && \ln{1+x} &\eq{x \to 0} x - \dfrac{x^2}{2} + \dfrac{x^3}{3} - \dfrac{x^4}{4} + \dots + (-1)^{n-1}\dfrac{x^n}{n} + \o{}{x^n}
    \end{align*}
    
    \newpage DL de $\tan x$ et de $\th x$ en $0$ à l'ordre 5 :
    
    \begin{multicols}{2}
        \[\text{\ding{43}} \tan x \eq{x \to 0} x + \dfrac{x^3}{3} + \dfrac{2}{15}x^5 + \o{}{x^5}\]
        
        \[\text{\ding{43}} \th x \eq{x \to 0} x - \dfrac{x^3}{3} + \dfrac{2}{15}x^5 + \o{}{x^5}\]
        
    \end{multicols}
    
    \text{}
    
    }
    
    \item  Énoncer la formule de Taylor-Young. Application au DL en $0$ de $x \mapsto e^x$ ou $x \mapsto \cos x$.
    
    \begin{form*}{Formule de Taylor-Young}{}
        Soit $f$ une fonction de classe $\mathcal{C}^n$ sur un intervalle $I$. Alors $f$ admet un $DL_n(a)$ en tout point $a$ de $I$ et on a :
        
        \[ f(x) = f(a) + f'(a)(x-a) + \dots + \dfrac{f^{(n)}(a)}{n!}(x-a)^n + \o{x \to a}{(x-a)^n} = \sum_{k=0}^n \dfrac{f^{(k)}(a)}{k!}(x-a)^k + \o{x \to a}{(x-a)^n}\]
        
        Ou encore, $\displaystyle f(a+h)=f(a)+f'(a)h + \dots + \dfrac{f^{(n)}(a)}{n!}h^n + \o{h \to 0}{h^n} = \sum_{k=0}^n \dfrac{f^{(k)}(a)}{k!}h^k + \o{h \to 0}{h^n}$.
    \end{form*}
    
    \boxans{
        \textit{Application.} Puisque $\exp^{(k)} = \exp$ et $\exp(0) = 1$ on a $\exp^{(k)}(0) = 1$. Ainsi :
        
        \[ \boxsol{$\exp(x) \eq{x \to 0} 1 + x + \dfrac{x^2}{2} + \dfrac{x^3}{3!} + \dots + \dfrac{x^n}{n!} + \o{}{x^n}$}\]
        
        On a $\cos' = -\sin$ et $\sin' = \cos$. Donc $\cos^{(k)}(x) =    (-1)^{\lfloor\sfrac{k+1}{2}\rfloor} \times \left\lbrace\begin{array}{ll}
            \cos(x) &\text{si} \  k \equiv 0 \ [2]  \\
            \sin(x) &\text{si} \ k \equiv 1 \ [2]
        \end{array}\right.$
    
        Puisque $\cos(0) = 1$ et $\sin(0) = 0$, on a $\cos^{(k)} = \left\lbrace\begin{array}{ll}
            (-1)^{\sfrac{k}{2}} &\text{si} \ k \equiv 0 \ [2] \\
            0 &\text{sinon} 
        \end{array}\right.$ donc :
        
        \[ \boxsol{$\cos(x) \eq{x \to 0} 1 - \dfrac{x^2}{2} + \dfrac{x^4}{4!} + \dots + (-1)^n\dfrac{x^2n}{(2n)!} + \o{}{x^{2n+1}}$}\]
    
    }
    
    \item Montrer que la fonction $f : x \mapsto x^3\sin{\dfrac{1}{x}}$ admet un DL en $0$ à l'ordre $2$, mais n'est pas deux fois dérivable en $0$.
    
    \boxans{
        On a $\dfrac{f(x)}{x^2} = x\sin{\dfrac{1}{x}} \lima{x \to 0} 0$ donc $f(x) \eq{x \to 0} \o{}{x^2}$. Ainsi, \boxsol{$f$ admet un $DL_2(0): f(x) \eq{x \to 0} \o{}{x^2}$}.
    
        Par troncature, $f$ admet un $DL_1(0): f(x) \eq{x \to 0} \o{}{x}$. Par opérations, $f$ est dérivable sur $\bdR^*$ et tel que :
        
        \[ \forall x \in \bdR^*,\ f'(x) = 3x^2\sin{\dfrac{1}{x}} + x^3\left(\dfrac{-1}{x^2}\right)\cos{\dfrac{1}{x}} = 3x^2\sin{\dfrac{1}{x}} - x\cos{\dfrac{1}{x}}\]
        
        On remarque que $\cos{\dfrac{1}{x}}$ n'a pas de limite en $0$ donc $f'$ n'est pas dérivable en $0$.
    
        Donc \boxsol{$f$ n'est pas deux fois dérivable en $0$.}
    }
    
    \item Retrouver les DL en $0$ à tous ordres de $\arcsin x$ et $\arctan x$ par primitivation. 
    
    \boxans{
        On a $\arctan'(x) = \dfrac{1}{1+x^2}$. Or $\dfrac{1}{1+x^2} \eq{x \to 0} 1 - x^2 + x^4 - x^6 + \dots + (-1)^nx^{2n} + \o{}{x^{2n}}$.
        
        On primitive : $\arctan(x) = \arctan(0) + x - \dfrac{x^3}{3} + \dfrac{x^5}{5} - \dots + \dfrac{x^{2n+1}}{2n+1} + \o{}{x^{2n+1}}$. Donc :
        \[ \boxsol{$\arctan(x) \eq{x \o 0} x - \dfrac{x^3}{3} + \dfrac{x^5}{5} - \dfrac{x^7}{7} - \dots + \dfrac{x^{2n+1}}{2n+1} + \o{}{x^{2n+1}}$}\]
        
       On a $\arcsin'(x) = \dfrac{1}{\sqrt{1-x^2}} = (1-x^2)^{-\frac{1}{2}} \eq{u \to 0} 1 - \dfrac{u}{2}+\dfrac{3}{2}u^2-\dots+\dfrac{(-1)^n(2n)!}{(2^n\times n!)^2}u^n + \o{}{u^n}$.
       
       En prenant $u = -x^2$ et en primitivant, on trouve :
       \[ \boxsol{$\arcsin(x) \eq{x \o 0} x + \dfrac{x^3}{6} + \dots + \dfrac{(2n)!}{(2^n\times n!)^2}\times\dfrac{x^{2n+1}}{2n+1}+\o{}{x^{2n+1}}$}\]
    }
    
    \item Retrouver le DL de $\tan$ en $0$ à l’ordre $5$ par la formule $\tan' = 1 + \tan^2$.
    
    \boxans{
        On a $\tan' = 1 + \tan^2$ et $\tan(x) \eq{x \to 0} x + \o{x^2}$ car $\tan$ est impaire.
        
        Donc $\tan' \eq{x \to 0} 1 + (x + \o{}{x^2})^2 = 1 + x^2 + \o{x^3}$. En primitivant, $\tan(x) \eq{x \to 0} \tan(0) + x + \dfrac{x^3}{3} + \o{}{x^4}$.
        
        Donc $\tan' \eq{x \to 0} 1 + (x + \dfrac{x^3}{3} + \o{}{x^4})^2 = 1 + x^2 + \dfrac{2}{3}x^4 + \o{}{x^5}$. En primitivant, on trouve finalement :
        \[ \boxsol{$\tan(x) \eq{x \to 0} x + \dfrac{x^3}{3} + \dfrac{2}{15}x^5 + \o{}{x^6}$}\]
    }

    \item Montrer qu'il existe un unique réel $x_n \in \left]n\pi - \dfrac{\pi}{2}; n\pi + \dfrac{\pi}{2}\right[$ tel que $\tan(x_n) = x_n$ puis montrer que 
    
    \[x_n \eq{+\infty} n\pi +\dfrac{\pi}{2}-\dfrac{1}{n\pi}+\o{}{\dfrac{1}{n}}\]
    
    \boxans{
        \begin{enumerate}
        
            \item On pose la fonction $f : x \mapsto \tanh{x} -x$ sur $\bdR \ \left\{n\pi + \dfrac{\pi}{2} | n \in \bdZ\right\}$.
        
            $f$ est dérivable sur chaque intervalle $I_n$, telle que $\forall n \in \bdN$, $\forall x \in I_n$, $f'(x) = \tan^2(x) \geq 0$, donc $f$ est croissante sur chaque intervalle $I_n$. 
        
            Soit $n \in \bdN$, $f$ est continue et strictement monotone sur $I_n$ donc d'après le théorème de la bijection continue $f$ est une bijection de $I_n$ dans $f(I_n) = \bdR$.
            Donc \boxsol{$\exists ! x_n \in I_n$ tel que $f(x_n) = 0$.}
        
            \item On a $\forall n \in \bdN^*$ $n\pi < x_n < n\pi + \dfrac{\pi}{2}$. Donc $\forall n \in \bdN$*, $ 1 < \dfrac{x_n}{n\pi} < 1 + \dfrac{1}{2n}$.
        
            Par théorème d'encadrement, $\dfrac{x_n}{n\pi} \lima{n \to +\infty} 1$, donc $x_n \asymp{+\infty} n\pi$ donc $x_n \eq{+\infty n\pi + o(n)}$.
        
            Pour en savoir plus, on pose $y_n = x_n - n\pi$, on a donc $y_n = o(n)$.
        
            Or $\tan{y_n} = \tan{x_n - n\pi} = \tan{x_n} = x_n$ donc $\arctan{\tan{y_n}} = \arctan{x_n}$.
        
            Or $n\pi \leq x_n < n\pi + \dfrac{\pi}{2}$ donc $0 \leq y_n < \dfrac{\pi}{2}$, donc $\arctan{\tan{y_n}} = y_n$ donc $y_n = \arctan{x_n}$.
        
            Or $x_n \lima{n \to +\infty} +\infty$ (car $x_n \asymp{+\infty} n\pi$ donc $y_n \lima{n \to +\infty} \dfrac{\pi}{2}$ d'où $y_n \asymp{+\infty} \dfrac{\pi}{2}$ soit $y_n \eq{+\infty} \dfrac{\pi}{2} + o(1)$.
        
            A ce stade, on a donc $x_n = n\pi + y_n$ soit $x_n \eq{+\infty} n\pi + \dfrac{\pi}{2} + o(1)$.
            On continue en posant $z_n = x_n - n\pi - \dfrac{\pi}{2}$, on a donc $z_n \eq{+\infty} o(1)$, donc $z_n \lima{n \to +\infty} 0$ d'où $\tan{z_n} \asymp{+\infty} z_n$.
        
            Or $\tan{z_n} = \tan{x_n - n\pi - \dfrac{\pi}{2}} = \tan{x_n - \dfrac{\pi}{2}} = -\dfrac{1}{\tan{x_n}} = -\dfrac{1}{x_n}$.
        
            Or $x_n \asymp{+\infty} n\pi$ donc $-\dfrac{1}{x_n} \asymp{+\infty} - \dfrac{1}{n\pi}$ donc $z_n \asymp{+\infty} -\dfrac{1}{n\pi}$, d'où $z_n \eq{+\infty} -\dfrac{1}{n\pi} + o\left(\dfrac{1}{n}\right)$.
         
            En conclusion, $x_n = n\pi + \dfrac{\pi}{2} + z_n$ donc \boxsol{$x_n \eq{+\infty} n\pi + \dfrac{\pi}{2} - \dfrac{1}{n\pi} + o\left(\dfrac{1}{n}\right)$.}
        \end{enumerate}
    }
    
\end{enumerate}

\end{document}