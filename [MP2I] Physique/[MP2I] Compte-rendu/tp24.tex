\documentclass[a4paper,french,bookmarks]{article}
\usepackage{./Structure/4PE18TEXTB}

\newboxans

\begin{document}

\stylizeDoc{Physique}{Compte-rendu du TP 24}{Calorimétrie}

\boxhtp{
    \itstar Mettre en œuvre un capteur de température.
    \itstar Effectuer des bilans d'énergie et mettre en œuvre une technique de calorimétrie.
    \itstar Mesurer des grandeurs thermodynamiques énergétiques : capacité thermique, enthalpie de changement d'états.
}{
    \itstar Calorimètre, thermomètre, balance, plaque chauffante.
    \itstar Solides à faire chauffer (aluminium, laiton, fer, zinc, plomb)
}

\text{}\hfill\\

\textbf{Remarques importantes concernant les mesures en thermodynamique}
% Curieux, besoin de cet espace vide pour que le premier que "triangle" de l'enum 
% s'affiche
\hfill
%
\begin{minipage}{0.35\linewidth}
	\centering
	\begin{tabular}{c|c}
	\end{tabular}
\end{minipage}
%
\begin{enumerate}
    \itt En thermodynamique, il n'est pas rare d'obtenir des incertitudes-type de l'ordre \SI{50}{\percent}. Ce n'est toutefois pas une raison pour manipuler n'importe comment, ne serait-ce que pour la sécurité lorsque l'on manipule des éléments chauds.
    
    \itt Attention à bien peser et mesurer les températures des fluides utilisés : on pourra par exemple poser le calorimètre contenant uniquement les instruments sur la balance et tarer celle-ci, les valeurs affichées par la balance correspondant alors à la masse du contenu.
\end{enumerate}

On donne $c_\text{eau} = \SI{4.18e3}{\joule \cdot \K^{-1} \cdot \kg^{-1}}$ la capacité thermique massique de l'eau liquide, supposée globalement indépendante de la température.


\section{Mesure de la valeur en eau du calorimètre (cf. TP précédent)}

\subsection{Le calorimètre}

Vous disposez d'un calorimètre, dispositif conçu de manière à pouvoir raisonnablement négliger, sur la durée d'une expérience, les échanges thermiques avec l'extérieur, de sorte que $Q \approx 0$.

\begin{enumerate}
    \item Lister les dispositifs permettant de limiter les transferts thermiques.
    
    \boxans{
        Les dispositifs du calorimètre permettant de limiter les transferts thermiques sont :
        %
        \begin{enumerate}
            \itt la paroi métallique réfléchissante à l'intérieur du calorimètre ;
            
            \itt la double paroi, séparant l'intérieur de l'extérieur du calorimètre avec du vide ou de l'air ;
            
            \itt le couvercle recouvrant presque la totalité du calorimètre.
        \end{enumerate}
    }
\end{enumerate}

Les expériences réalisées dans un calorimètre sont effectuées en contact avec l'atmosphère.

\begin{enumerate}[resume]
    \item Quelle forme prend alors le premier principe ?
    
    \boxans{
        Le premier principe livre :
        %
        \[\Delta U = \Delta Q + \Delta W \]
        %
        Or ici on a $Q = 0$ d'où $\Delta Q = 0$ Le premier principe se réécrit donc :
        %
        \[ \Delta U. = \Delta W\]
    }
\end{enumerate}

Le système thermodynamique $\Sigma$ étudié est

\begin{equation}
    \Sigma = \left\{\text{calorimètre} + \text{instruments} + \text{contenu}\right\}
\end{equation}

Il faut donc considérer la capacité thermique $C_\text{calo}$ du calorimètre et des instruments lorsque l'on applique le premier principe, puisque celle-ci n'est généralement pas négligeable devant celle du contenu. On se propose ainsi de déterminer la valeur en eau $µ$ du calorimètre qui vérifie

\begin{equation}
    C_\text{calo} = \mu c_\text{eau}
\end{equation}

\subsection{Méthode des mélanges}

\noafter
%
\boxans{
    \begin{experience}{Mesure de la valeur en eau du calorimètre}{}
        \begin{enumerate}
            \ithand Placer une masse $m_1$ d'eau \guill{froide}, attendre l'équilibre thermiqueà la température $T_1$ dans le calorimètre.
    
            \ithand Placer la sonde de température dans le calorimètre et la relier à la carte d'acquisition. Attendre quelques minutes que l'eau et le calorimètre soient alors à la même température et la noter.
            
            \ithand Ajouter une masse $m_2$ d'eau \guill{chaude} à la température $T_2$.
            
            \ithand Fermer le couvercle, agiter, et relever l'évolution de la température au cours du temps. Déterminer la température finale $T_\text f$ lorsque l'équilibre est atteint dans le calorimètre.
        \end{enumerate}
    \end{experience}
}
%
\nobefore\yesafter
%
\begin{expcom} \text{}
    \begin{enumerate}
        \itt On place $m_1 = \SI{320}{\g}$ d'eau, et on atteint l'équilibre thermique à la température $T_1 = \SI{22}{\celsius}$.
        \itt On ajoute $m_2 = \SI{270}{\g}$ à la température $T_2 = \SI{68}{\celsius}$
        \itt On atteint la température finale $T_\text f = \SI{40.5}{\celsius}$
    \end{enumerate}
\end{expcom}
%
\yesbefore

\begin{enumerate}[resume]
    \item Montrer que la valeur en eau du calorimètre est donnée par
    
    \begin{equation}
        \mu = \dfrac{m_1\left(T_1 - T_\text f\right) + m_2\left(T_2 - T_\text f\right)}{T_f - T_1}
    \end{equation}
    
    faire l'application numérique avec vos valeurs.
    
    \boxans{    
        La transformation subie par un système placé dans un calorimètre est monobare, en équilibre mécanique à l'état intial et final, et adiabatique. Ainsi $\Delta H = 0$. Or $\Delta H = \Delta H_\text{calorimètre} + \Delta H_\text{instruments} + \Delta H_\text{contenu}$.
        
        \begin{enumerate}
        
            \itt On a $\Delta H_\text{calorimètre} = C_\text{calo}\left(T_f - T_1\right) = \mu c_\text{eau}\left(T_f - T_1\right)$
        
            \itt On néglige, du fait de leur taille, l'influence des instruments, donc $\Delta H_\text{instruments} = 0$.
            
            
            \itt Le contenu du calorimètre n'étant que de l'eau, on a $\Delta H_\text{contenu} = \Delta H_\text{eau froide} + \Delta H_\text{eau chaude}$, d'où :
            %
            \[ \Delta H_\text{contenu} = c_\text{eau}m_1\left(T_\text f - T_1\right) + c_\text{eau}m_2\left(T_\text f - T_2\right) = c_\text{eau}\left(m_1\left(T_\text f - T_1\right) + m_2\left(T_\text f - T_2\right)\right)\]
        \end{enumerate}
        %
        On a donc :
        %
        \[ 0 = \mu c_\text{eau}\left(T_f - T_1\right) + c_\text{eau}m_1\left(T_\text f - T_1\right) + c_\text{eau}m_2\left(T_\text f - T_2\right) = c_\text{eau}\left(m_1\left(T_\text f - T_1\right) + m_2\left(T_\text f - T_2\right)\right) \]
        %
        On divise par $c_\text{eau}$, donc on a :
        %
        \[\mu\left(T_ 1 - T_ \text f\right) = m_1\left(T_\text f - T_1\right) + m_2\left(T_\text f - T_2\right) \qquad\text{donc}\qquad \mu = \dfrac{m_1\left(T_1 - T_\text f\right) + m_2\left(T_2 - T_\text f\right)}{T_f - T_1}\]
        
        L'application numérique livre $\mu = \SI{0.081}{\kg} = \SI{81}{\g}$.
    }
\end{enumerate}

Avec les calorimètres utilisés la valeur attendue est généralement de l'ordre de $\mu^\text{attendue} = \SI{15}{\g}$.

\noafter
%
\boxans{
    \begin{experience}{Comparaison avec la valeur attendue}{}
        \begin{enumerate}
            \ithand Estimer l'incertitude-type $u\left(\mu\right)$ sur la valeur de $\mu$ en réalisant une simulation Monte-Carlo.
            
            \ithand Comparer alors valeur mesurée et valeur attendue. Conclure.
        \end{enumerate}
    \end{experience}
}
%
\nobefore\yesafter
%
\begin{expcom} \text{}
    \begin{enumerate}
        \itt Après avoir effectué la simulation de Monte-Carlo, on obtient $\overline{\mu} = \SI{0.081}{\kg} = \SI{81}{\g}$ et $u\left(\mu\right) = \SI{0.01}{\kg} = \SI{10}{\g}$.
        
        \itt On calcule alors le z-score selon :
        %
        \[ z = \dfrac{\mod{\overline{\mu} - \mu^\text{attendue}}}{u\left(\mu\right)}\]
        %
        On obtient $z = 6.6$. Ce score n'est pas particulièrement bon, mais les remarques importantes données en introduction semblent le nuancer.
    \end{enumerate}
\end{expcom}
%
\yesbefore

\section{Enthalpie massique de transition de phase}

On cherche à déterminer $L_{S \to L}$.

\begin{enumerate}[resume]
    \item Concevoir un protocole pour mesurer $L_{S \to L}$ à l'aide d'eau, d'un glaçon à $T_0 = \SI{0}{\celsius}$ et d'un calorimètre.
    
    \boxans{
        \begin{experience}{Mesure d'une enthalpie massique de transition de phase}{}
            \begin{enumerate}
                \ithand Placer dans le calorimètre une masse $m_1$ d'eau, et attendre la température d'équilibre $T_1$.
                
                \ithand Ajouter un glaçon de masse $m_2$ et de température $T_0 = \SI{0}{\celsius}$ (donc à l'équilibre dans de l'eau liquide).
            
                \ithand Attendre que le glaçon fonde, et mesurer la température finale $T_f$.
            
                \ithand Déterminer alors $L_{S \to L}$.
            \end{enumerate}
        \end{experience}
    }
    
    \item Modéliser la situation et donner l'expression de $L_{S \to L}$ en fonction des autres variables.
    
    \boxans{
        On modélise la transformation (fictive) selon :
        %
        \[ \left[\begin{array}{cc}
            \text{eau liquide} & m_1, T_1 \\
            \text{glaçon} & m_2, T_0 \\ 
            \text{calorimètre} & \mu, T_1
        \end{array}\right] \overset{(1)}{\to} \left[\begin{array}{cc}
            \text{L} & m_1, T_\text f \\
            \text{S} & m_2, T_0 \\ 
            \text{S} & \mu, T_1
        \end{array}\right] \overset{(2)}{\to} \left[\begin{array}{cc}
            \text{L} & m_1, T_\text f \\
            \text{S} & m_2, T_0 \\ 
            \text{S} & \mu, T_\text f
        \end{array}\right] \overset{(3)}{\to} \left[\begin{array}{cc}
            \text{L} & m_1, T_\text f\\
            \text{L} & m_2, T_0 \\ 
            \text{S} & \mu, T_\text f
        \end{array}\right] \overset{(4)}{\to} \left[\begin{array}{cc}
            \text{L} & m_1, T_\text f \\
            \text{L} & m_2, T_\text f \\ 
            \text{S} & \mu, T_\text f
        \end{array}\right]\]
        %
        \begin{enumerate}
            \itt Pour la transformation $(1)$ : L'eau liquide est une PC2I donc $\Delta H_1 = C\Delta T = m_1c_\text{eau}\left(T_\text f - T_1\right)$.
            
            \itt Pour la transformation $(2)$ : Le calorimètre est une PC2I donc $\Delta H_2 = C\Delta T = \mu c_\text{eau}\left(T_\text f - T_1\right)$.
             
             \itt Pour la transformation $(3)$ : Le galçon fond doonc $\Delta H_3 = m_2 \times L_{S \to L}$.
             
             \itt Pour la transformation $(4)$ : Le glaçon fondu (liquide) est une PC2I donc $\Delta H_4 = m_2c_\text{eau}\left(T_f - T_0\right)$.
        \end{enumerate}
        
        La transformation subie par un système placé dans un calorimètre est monobare, en équilibre mécanique à l'EI et l'EF, et adiabatique donc $\Delta H = 0$.
        %
        \[ -m_2\times L_{S \to L} = m_1c_\text{eau}\left(T_\text f - T_1\right) + \mu c_\text{eau}\left(T_\text f - T_1\right) + m_2c_\text{eau}\left(T_f - T_0\right)\]
        %
        Donc :
        %
        \[ L_{S \to L} = -\dfrac{c_\text{eau}}{m_2}\left(\left(m_1 + \mu\right)\left(T_f - T_1\right) + m_2\left(T_f - T_0\right)\right) = c_\text{eau}\left(\dfrac{m_1 + \mu}{m_2}\left(T_1 - T_f\right) + T_0 - T_f\right)\]
    }
    
    \item Réaliser l'expérience et déterminer $L_{S \to L}$. On a $L_{S \to L}^\text{attendu} = \SI{334}{\kilo\joule \cdot \kelvin^{-1}}$.
    
    \begin{expcom} \text{}
        \begin{enumerate}
            \itt On a $T_1 \SI{37}{\celsius}$, $T_f = \SI{32}{\celsius}$, $m_1 = \SI{579}{\g}$ et $m_2 = \SI{21}{\g}$.
            
            \itt On obtient avec l'expression ci-dessus $L_{S \to L} = \SI{388}{\kilo\joule \cdot \kelvin^{-1}}$.
            
            \itt Manque de temps pour les incertitudes.
        \end{enumerate}
    \end{expcom}
    
\end{enumerate}

\end{document}