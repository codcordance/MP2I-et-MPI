\documentclass[a4paper,french,bookmarks]{article}
\usepackage{./Structure/4PE18TEXTB}

\newboxans

\begin{document}
\stylizeDoc{Physique}{Compte-rendu du TP 19}{Introduction à l'analyse spectrale
}

\boxhtp{
    \itstar  Mettre en œuvre un dispositif expérimental permettant d'analyser le spectre d'un signal acoustique.
}{
    \itstar un micro, un diapason avec une caisse de résonance
}

L'\textbf{analyse spectrale} consiste à déterminer le spectre d'un signal temporel.

\section{Étude du diapason}

\subsection{Le micro}

Un micro est un \textit{\color{main1} transducteur électromagnétique}, c'est-à-dire un dispositif qui transforme un signal sonore (une onde de pression) en un signal électrique.

\begin{experience}{Mise en place}{}
    \begin{enumerate}
        \ithand Allumer le micro. Le connecter à la plaquette octogonale sur l'entrée \texttt{EA0}, et vérifier que la plaquette est connectée à l'ordinateur préalablement allumé.
        
        \ithand Lancer le logiciel \texttt{Latis-Pro}, cliquer sur le bouton \texttt{EA0} dans la fenêtre \texttt{Acquisition temporelle}.

        \ithand Régler le nombre $N$ de points (\texttt{Points}) à la valeur maximale possible correspondant à une puissance de 2.
        
        \ithand Choisir la durée d'échantillonnage \texttt{Total} correspondant à $\Delta t$ de la fiche \bf{\sffamily Échantillonnage}. Vérifier que cela influe sur la \hg{\itshape période d'échantillonnage} \texttt{Te}.
    \end{enumerate}
\end{experience}

\boxexp{
    On prend $N = 256000$, $\texttt{Te} = \SI{100}{\ns}$ donc $\texttt{Total} = \SI{25.6}{\ms}$.
}

\subsection{Analyse temporelle du signal}\label{subsec:1.2}

\begin{experience}{Acquisition du son d'un diapason}{}
    \begin{enumerate}
        \ithand Approcher du micro le diapason muni de sa caisse de résonance.
        
        \ithand Exciter le diapason avec la baguette dédiée et lancer l'acquisition avec le bouton $\triangleright$ ou la touche \texttt{F10}.
    \end{enumerate}
\end{experience}

\boxexp{
    On obtient le signal suivant :
    
    \begin{center}
        \begin{minipage}[c]{0.48\linewidth}
            \pgfplotsset{width=10cm}
            \begin{tikzpicture}[scale = 0.7]
                \begin{axis}[
                    axis lines = left,
                    xlabel=$\textsf{Fréquence (Hz)}$,
                    ylabel=$\textsf{Temps (\mu s)}$,
                    %xmin=0,
                    %xmax=10,
                    %ymin=0,
                    %ymax=500,
                    %xtick distance={200},
                    %ytick distance={100},
                    xticklabel={$\mathsf{\pgfmathprintnumber{\tick}}$},
                    yticklabel={$\mathsf{\pgfmathprintnumber{\tick}}$},
                    %minor x tick num=2,
                    %minor y tick num=2,
                    x tick label style={/pgf/number format/1000 sep=\,},
                    font=\footnotesize,
                    grid = both,
                    grid style = {line width = .1pt, draw = gray!30},
                    major grid style = {line width=.2pt,draw=gray!50},
                ]
                    %\addplot[mark=none, main1, line width=0.6mm] coordinates
                    
                    %\addplot[dashed] table[x=Temps,y=EA0,col sep=comma] {diapason.csv};
                \end{axis}
            \end{tikzpicture}
            
             \textbf{\sffamily \color{main21} DONNES NON TROUVEES.}
        \end{minipage}
    \end{center}
}

\newpage

\begin{enumerate}
    \item Décrire le signal obtenu ; estimer sa valeur moyenne. Pourquoi cette allure évolue-t-elle au cours du temps ?
    
    \boxans{
        Le signal obtenu est de forme sinusoïdal, à une seule harmonique. L'allure du signal évolue légèrement au cours du temps, notamment au niveau de l'amplitude. Ceci s'explique par les conditions réelles de l'acquisition du signal, puisqu'il peut y avoir un mouvementent du micro, un 
    }
    
    \item Donner une estimation grossière de la période $T$ du signal.
    
    \boxans{
        On estime grossièrement que $T \approx \SI{2.3}{\ms}$.
    }
    
    \item En déduire la fréquence $f$ du signal sonore. Cela correspond-il à ce qui est indiqué sur le diapason ?
    
    \boxans{
        On a $f = \dfrac{1}{T}$. L'application numérique donne $f = \SI{435}{\hertz}$.
    }
\end{enumerate}

\subsection{Analyse spectrale du signal}

\begin{experience}{Visualisation spectrale du signal}{}
    \begin{enumerate}
        \ithand Aller dans l'onglet \texttt{Traitements}, puis \texttt{Calculs spécifiques}, et enfin \texttt{Analyse de Fourier}.

        \ithand Dans la fenêtre qui s'ouvre, sélectionner à l'onglet \texttt{Fenêtre de pondération} le choix \texttt{Hamming}, puis tracer le spectre. Une nouvelle fenêtre s'ouvre.
        
        \ithand Si les fenêtres se chevauchent, la touche \texttt{F5} permet un passage automatique en mosaïque.
    \end{enumerate}
\end{experience}

\boxexp{
    On obtient le spectre en amplitude suivant :
    
    \begin{center}
        \begin{minipage}[c]{0.48\linewidth}
            \pgfplotsset{width=10cm}
            \begin{tikzpicture}[scale = 0.7]
                \begin{axis}[
                    axis lines = left,
                    xlabel=$\textsf{Fréquence (Hz)}$,
                    ylabel=$\textsf{Amplitude (mV)}$,
                    xmin=0,
                    xmax=1000,
                    ymin=0,
                    ymax=500,
                    xtick distance={200},
                    ytick distance={100},
                    xticklabel={$\mathsf{\pgfmathprintnumber{\tick}}$},
                    yticklabel={$\mathsf{\pgfmathprintnumber{\tick}}$},
                    minor x tick num=2,
                    minor y tick num=2,
                    x tick label style={/pgf/number format/1000 sep=\,},
                    font=\footnotesize,
                    grid = both,
                    grid style = {line width = .1pt, draw = gray!30},
                    major grid style = {line width=.2pt,draw=gray!50},
                ]
                    \addplot[mark=none, main1, line width=0.6mm] coordinates {(440,0) (440,352)};
                \end{axis}
            \end{tikzpicture}
        \end{minipage}
    \end{center}
}

\begin{enumerate}[resume]
    \item Commentez l'allure du spectre : que voit-on et pourquoi ? Est-ce en accord avec les observations en \textbf{\sffamily \ref{subsec:1.2}} ?
    
    \boxans{
        On obtient une seule harmonique de fréquence proche de $\SI{440}{\hertz}$, ce qui est en accord avec les observations effectuées en \textbf{\sffamily \ref{subsec:1.2}}.
    }
\end{enumerate}

\section{Étude d'une corde du ukulélé}

\begin{center}
    \begin{minipage}{0.9\linewidth}
    \begin{tcolorbox}[
        breakable,
        enhanced,
        interior style      = {color=colexp!10},
        borderline north    = {.5pt}{0pt}{colexp!10},
        borderline south    = {.5pt}{0pt}{colexp!10},
        borderline west     = {.5pt}{0pt}{colexp!10},
        borderline east     = {.5pt}{0pt}{colexp!10},
        sharp corners       = downhill,
        arc                 = 0 cm,
        boxrule             = 0.5pt,
        drop fuzzy shadow   = black!50!white
    ]
        \centering Pour le bien des oreilles de chacun, n'excitez une corde du ukulélé que lorsque le protocole l'exige.
    \end{tcolorbox}
\end{minipage}

\end{center}

\subsection{Analyse préliminaire}

Une corde vibrante fixée à ces extrémités ne peut produire que certaines fréquences de vibrations appelées \textit{fréquences propres}. Nous avons montré lors du chapitre sur les phénomènes ondulatoires que ces fréquences s'écrivent :

\begin{equation}\label{eq:1}
    f_n = n\dfrac{c}{2L} \qquad \text{où} \quad n \in \bdN^*
\end{equation}

avec $c$ la vitesse de propagation des ondes mécaniques le long de la corde. La note associée à une corde correspond à sa \textit{fréquence fondamentale} $f_1$.

\begin{enumerate}[resume]
    \item Sans jouer, comment peut-on déterminer la corde qui donnera la note la plus grave ?

    \boxans{
        Le facteur $c$ dépend de la masse linéique de la corde, qui dépend elle-même de la masse. Puisque toutes les cordes ont à peu près la même longueur, c'est donc la masse de chaque corde qui va influencer sa tessiture. Plus la masse est élevée, plus $c$ sera faible et donc plus $f_1$ sera faible. La corde la plus grave sera donc celle avec la plus grande masse, visuellement celle la plus épaisse.
    }
\end{enumerate}

\subsection{Analyse temporelle du signal}

\begin{experience}{}{}
    \begin{enumerate}
        \ithand Remplacer le diapason par le ukulélé. Exciter au choix l’une des cordes en faisant en sorte que les autres ne vibrent pas.
        
        \ithand Lancer l’acquisition du signal par l’ordinateur
    \end{enumerate}
\end{experience}

\boxexp{
    On rejoue un \textsc{La} sur le ukulélé, de manière à pouvoir garder la configuration du début. On obtient le signal suivant :
    
    - figure
}

\begin{enumerate}[resume]
    \item Décrire le signal obtenu et estimer sa valeur moyenne.
    
    \boxans{
        On obtient un signal de moyenne $\phyavg{s} = \SI{0}{\mV}$ et de forme sinusoïdale, quoique bien plus \guill{perturbée}.
    }
    
    \item Donner une estimation grossière de la période $T$ du signal et en déduire la valeur de la fréquence de sa première harmonique de Fourier.
    
    \boxans{
        On estime grossièrement que $T \approx \SI{2.4}{\ms}$. On a $f = \dfrac{1}{T}$ L'application numérique donne $f = \SI{417}{\hertz}$.
    }
    
    \item A partir de ces observations, que peut-on déjà en déduire quand à la composition spectrale du signal ?
    
    \boxans{
        Le signal est ici plus \guill{complexe} qu'une simple sinusoïde, on en déduit donc que sa composition spectrale sera faite de plusieurs harmoniques.
    }
\end{enumerate}

\subsection{Analyse spectrale du signal}

\begin{experience}{Visualisation spectrale du signal}{}
    \begin{enumerate}
        \ithand Réaliser à nouveau l’analyse spectrale du signal
    \end{enumerate}
\end{experience}

\boxexp{
    On obtient le spectre en amplitude suivant :
}

\begin{enumerate}
    \ithand Commentez l’allure du spectre. Que voit-on ? Pourquoi ?
    
    \boxans{
        On observe plusieurs \guill{pics} d'amplitude, dont les fréquences sont des multiples de la fréquence fondamentale $f_1 = \SI{440}{\hertz}$. Ceci s'explique par la formule \eqref{eq:1}.
    }
\end{enumerate}
\end{document}