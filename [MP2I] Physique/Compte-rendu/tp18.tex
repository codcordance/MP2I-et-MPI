\documentclass[a4paper,french,bookmarks]{article}
\usepackage{./Structure/4PE18TEXTB}
    
\begin{document}
\stylizeDoc{Physique}{Compte-rendu du TP 18}{Le filtre RC}

\boxhtp{
    \itstar Mettre en œuvre un dispositif expérimental pour tracer le diagramme de Bode du filtre RC
}{
    \itstar un GBF, un oscilloscope, boîte à décades de résistances, de capacités, une bobine.
}

\begin{experience}{Diagramme de Bode du filtre RC}{}
    \begin{enumerate}
        \ithand Choisir une valeur de capacité (typiquement $C =\SI{10}{nF}$) et une valeur de résistance (typiquement $R = \SI{1}{k\Omega}$).
        
        \ithand Tracer le diagramme de Bode du filtre RC.
    \end{enumerate}
\end{experience}

\boxexp{
    \begin{center}
        \begin{circuitikz}
        \draw (0, 0) 
            to[vsourcesin, v=$\underline e(t)$] ++(0, 2.4) coordinate (m0)
            to[R, l=$R \eq \SI{1}{k\Omega}$] ++(5,0) coordinate (m1)
            to[C, v=$\underline s(t)$, l=$C \eq \SI{10}{nF}$] ++(0,-2.4)
            to[short, i=$i$] ++(-5,0) node[eground, rotate = -45]{}
            to[short, -o] ++(0,0);
            --(0,0);
                
            \draw[->] (m0) to[short, o-]  ++(-1,1) node[label={west:\texttt{Voie 1}}]{};
                
            \draw[->] (m1) to[short, o-]  ++(1,1) node[label={east:\texttt{Voie 2}}]{};
        \end{circuitikz}
            
        \begin{tabular}{|>{\centering\columncolor{colexp!20}}p{0.1\linewidth}|c!{\color{colexp!50!gray}\vrule}|}\hline
        \rowcolor{colexp!20} $\vphantom{\dfrac{1}{2}}f_G (\SI{}{kHz})$ & $1$\\\hline
        $\vphantom{\dfrac{1}{2}}A_s (\SI{}{V})$ & $9.8$\\\hline
        $\vphantom{\dfrac{1}{2}}\phi_s (\SI{}{{}^\circ)}$ & $1.0$\\\hline
    \end{tabular}
    \end{center}
        
    On a de plus $\omega_c = \dfrac{1}{\omega_0} = \dfrac{1}{RC}$, d'où $f_C = \dfrac{1}{2\pi RC}$. L'application numérique donne $f = \SI{15915}{Hz}$.
}

\end{document}