\documentclass[a4paper,french,bookmarks]{article}
\usepackage{./Structure/4PE18TEXTB}

\newboxans

\begin{document}
\stylizeDoc{Physique}{Compte-rendu du TP 23}{Calorimétrie}

\boxhtp{
    \itstar Mettre en œuvre un capteur de température.
    \itstar Effectuer des bilans d'énergie et mettre en œuvre une technique de calorimétrie.
    \itstar Mesurer des grandeurs thermodynamiques énergétiques : capacité thermique, enthalpie de changement d'états.
}{
    \itstar Calorimètre, thermomètre, balance, plaque chauffante.
    \itstar Solides à faire chauffer (aluminium, laiton, fer, zinc, plomb)
}

\text{}\hfill\\

\textbf{Remarques importantes concernant les mesures en thermodynamique}
% Curieux, besoin de cet espace vide pour que le premier que "triangle" de l'enum 
% s'affiche
\hfill
%
\begin{minipage}{0.35\linewidth}
	\centering
	\begin{tabular}{c|c}
	\end{tabular}
\end{minipage}
%
\begin{enumerate}
    \itt En thermodynamique, il n'est pas rare d'obtenir des incertitudes-type de l'ordre \SI{50}{\percent}. Ce n'est toutefois pas une raison pour manipuler n'importe comment, ne serait-ce que pour la sécurité lorsque l'on manipule des éléments chauds.
    
    \itt Attention à bien peser et mesurer les températures des fluides utilisés : on pourra par exemple poser le calorimètre contenant uniquement les instruments sur la balance et tarer celle-ci, les valeurs affichées par la balance correspondant alors à la masse du contenu.
\end{enumerate}

On donne $c_\text{eau} = \SI{4.18e3}{\joule \cdot \K^{-1} \cdot \kg^{-1}}$ la capacité thermique massique de l'eau liquide, supposée globalement indépendante de la température.


\section{Mesure de la valeur en eau du calorimètre}

\subsection{Le calorimètre}

Vous disposez d'un calorimètre, dispositif conçu de manière à pouvoir raisonnablement négliger, sur la durée d'une expérience, les échanges thermiques avec l'extérieur, de sorte que $Q \approx 0$.

\begin{enumerate}
    \item Lister les dispositifs permettant de limiter les transferts thermiques.
    
    \boxans{
        Les dispositifs du calorimètre permettant de limiter les transferts thermiques sont :
        %
        \begin{enumerate}
            \itt la paroi métallique réfléchissante à l'intérieur du calorimètre ;
            
            \itt la double paroi, séparant l'intérieur de l'extérieur du calorimètre avec du vide ou de l'air ;
            
            \itt le couvercle recouvrant presque la totalité du calorimètre.
        \end{enumerate}
    }
\end{enumerate}

Les expériences réalisées dans un calorimètre sont effectuées en contact avec l'atmosphère.

\begin{enumerate}[resume]
    \item Quelle forme prend alors le premier principe ?
    
    \boxans{
        Le premier principe livre :
        %
        \[\Delta U = \Delta Q + \Delta W \]
        %
        Or ici on a $Q = 0$ d'où $\Delta Q = 0$ Le premier principe se réécrit donc :
        %
        \[ \Delta U. = \Delta W\]
    }
\end{enumerate}

Le système thermodynamique $\Sigma$ étudié est

\begin{equation}
    \Sigma = \left\{\text{calorimètre} + \text{instruments} + \text{contenu}\right\}
\end{equation}

Il faut donc considérer la capacité thermique $C_\text{calo}$ du calorimètre et des instruments lorsque l'on applique le premier principe, puisque celle-ci n'est généralement pas négligeable devant celle du contenu. On se propose ainsi de déterminer la valeur en eau $µ$ du calorimètre qui vérifie

\begin{equation}
    C_\text{calo} = \mu c_\text{eau}
\end{equation}

\subsection{Méthode des mélanges}

\noafter
%
\boxans{
    \begin{experience}{Mesure de la valeur en eau du calorimètre}{}
        \begin{enumerate}
            \ithand Placer une masse $m_1$ d'eau \guill{froide} à la température $T_1$ dans le calorimètre.
    
            \ithand Placer la sonde de température dans le calorimètre et la relier à la carte d'acquisition. Attendre quelques minutes que l'eau et le calorimètre soient alors à la même température et la noter.
            
            \ithand Ajouter une masse $m_2$ d'eau \guill{chaude} à la température $T_2$.
            
            \ithand Fermer le couvercle, agiter, et relever l'évolution de la température au cours du temps. Déterminer la température finale $T_\text f$ lorsque l'équilibre est atteint dans le calorimètre.
        \end{enumerate}
    \end{experience}
}
%
\nobefore\yesafter
%
\begin{expcom} \text{}
    \begin{enumerate}
        \itt On a $m_1 = \SI{560}{\g}$ à la température $T_1' = \SI{18}{\celsius}$.
        \itt On atteint la température $T_1 = \SI{15}{\celsius}$.
        \itt On ajoute $m_2 = \SI{238}{\g}$ à la température $T_2 = \SI{60}{\celsius}$
        \itt On atteint la température finale $T_\text f = \SI{25.4}{\celsius}$
    \end{enumerate}
\end{expcom}
%
\yesbefore

\begin{enumerate}[resume]
    \item Montrer que la valeur en eau du calorimètre est donnée par
    
    \begin{equation}
        \mu = \dfrac{m_1\left(T_1 - T_\text f\right) + m_2\left(T_2 - T_\text f\right)}{T_f - T_1}
    \end{equation}
    
    faire l'application numérique avec vos valeurs.
    
    \boxans{    
        La transformation subie par un système placé dans un calorimètre est monobare, en équilibre mécanique à l'état intial et final, et adiabatique. Ainsi $\Delta H = 0$. Or $\Delta H = \Delta H_\text{calorimètre} + \Delta H_\text{instruments} + \Delta H_\text{contenu}$.
        
        \begin{enumerate}
        
            \itt On a $\Delta H_\text{calorimètre} = C_\text{calo}\left(T_f - T_1\right) = \mu c_\text{eau}\left(T_f - T_1\right)$
        
            \itt On néglige, du fait de leur taille, l'influence des instruments, donc $\Delta H_\text{instruments} = 0$.
            
            
            \itt Le contenu du calorimètre n'étant que de l'eau, on a $\Delta H_\text{contenu} = \Delta H_\text{eau froide} + \Delta H_\text{eau chaude}$, d'où :
            %
            \[ \Delta H_\text{contenu} = c_\text{eau}m_1\left(T_\text f - T_1\right) + c_\text{eau}m_2\left(T_\text f - T_2\right) = c_\text{eau}\left(m_1\left(T_\text f - T_1\right) + m_2\left(T_\text f - T_2\right)\right)\]
        \end{enumerate}
        %
        On a donc :
        %
        \[ 0 = \mu c_\text{eau}\left(T_f - T_1\right) + c_\text{eau}m_1\left(T_\text f - T_1\right) + c_\text{eau}m_2\left(T_\text f - T_2\right) = c_\text{eau}\left(m_1\left(T_\text f - T_1\right) + m_2\left(T_\text f - T_2\right)\right) \]
        %
        On divise par $c_\text{eau}$, donc on a :
        %
        \[\mu\left(T_ 1 - T_ \text f\right) = m_1\left(T_\text f - T_1\right) + m_2\left(T_\text f - T_2\right) \qquad\text{donc}\qquad \mu = \dfrac{m_1\left(T_1 - T_\text f\right) + m_2\left(T_2 - T_\text f\right)}{T_f - T_1}\]
        
        L'application numérique livre $\mu = \SI{0.23}{\kg}$.
    }
\end{enumerate}

Avec les calorimètres utilisés la valeur attendue est généralement de l'ordre de $\mu^\text{attendue} = \SI{15}{\g}$.

\noafter
%
\boxans{
    \begin{experience}{Comparaison avec la valeur attendue}{}
        \begin{enumerate}
            \ithand Estimer l'incertitude-type $u\left(\mu\right)$ sur la valeur de $\mu$ en réalisant une simulation Monte-Carlo.
            
            \ithand Comparer alors valeur mesurée et valeur attendue. Conclure.
        \end{enumerate}
    \end{experience}
}
%
\nobefore\yesafter
%
\begin{expcom} \text{}
    \begin{enumerate}
        \itt Après avoir effectué la simulation de Monte-Carlo, on obtient $\overline{\mu} = \SI{0.23}{\kg}$ et $u\left(\mu\right) = \SI{0.02}{\kg}$.
        
        \itt On calcule alors le z-score selon :
        %
        \[ z = \dfrac{\mod{\overline{\mu} - \mu^\text{attendue}}}{u\left(\mu\right)}\]
        %
        On obtient $z = 10$. Ce score n'est pas particulièrement bon, mais les remarques importantes données en introduction semblent le nuancer.
    \end{enumerate}
\end{expcom}
%
\yesbefore

\section{Capacité thermique des métaux - loi de Dulong et Petit}

On peut estimer la capacité thermique d'un métal par la loi de Dulong et Petit.

\begin{property*}{Loi de Dulong et Petit}{}
    Pour une température \guill{suffisamment} grande (température de Debye, qui dépend du métal), \hg{la capacité thermique molaire d'un métal vaut $C_\text m = 3R$} et cela quelque soit le métal.
\end{property*}

On donne ci-dessus plusieurs valeurs tabulées pour les métaux disponibles au laboratoire.

\begin{center}
    	\begin{tabular}{c||c|c|c|c|c|c}
		Métal 				& Cuivre & Fer & Laiton & Zinc & Plomb & Aluminium	\\ \hline\hline
		$c \left(\SI{}{\joule \cdot \kelvin^{-1} \cdot \kg^{-1}}\right)$ 	& 385 & 44 & 377 & 380 & 129 & 897 \\
		$M \left(\SI{}{\g \cdot \mol^{-1}}\right)$ & 63,5 & 55,8 & - & 65,4 & 207 & 27,0 \\
		$\rho \left(\SI{}{\kg \cdot \L^{-1}}\right)$ & 8,96 & 7,87 & - & 7,13 & 11,4 & 2,7
	\end{tabular}
\end{center}

\begin{enumerate}[resume]
    \item Pourquoi la masse molaire du laiton ainsi que sa masse volumique ne sont-elles pas indiquées ?
    
    \boxans{
        Car il s'agit d'un alliage, ces valeurs dépendent donc des proportions des différentes espèces, qui varient selon les échantillons.
    }
    
    \item Les valeurs numériques données dans le tableau sont-elles cohérentes avec la loi de Dulong et Petit ?
    
    \boxans{
        On fait le produit des deux premières lignes, pour obtenir $C_\text m = cM$ :
        \begin{center}
            \begin{tabular}{c||c|c|c|c|c|c}
		Métal 				& Cuivre & Fer & Laiton & Zinc & Plomb & Aluminium	\\ \hline\hline
		$C_\text m \left(\SI{}{\joule \cdot \kelvin^{-1} \cdot \mol^{-1}}\right)$ & 24,4 & 24,8 & - & 24,9 & 26,7 & 24,2
	\end{tabular}
        \end{center}
        
	On remarque les valeurs obtenues semblent être les mêmes qu'importe les métaux, ce qui est cohérent avec la loi de Dulong et Petit.
    }
\end{enumerate}

On souhaite mesurer la capacité thermique d'un métal. On utilise une méthode similaire à la partie précédente.

\noafter
%
\boxans{
    \begin{experience}{Mesure de la capacité thermique d'un morceau de métal}{}
        \begin{enumerate}
            \ithand Verser une masse $m_\text{eau}$ d'eau dans le calorimètre. Attendre l'équilibre et noter $T_1$ la température dans le calorimètre.
            
            \ithand Prendre et identifier un morceau de métal qui chauffe au bain-marie sur la paillasse centrale. \bf{Attention à ne pas vous brûler}.
            
            \ithand On note $m$ sa masse, $C = mc$ sa capacité thermique et $T_2$ sa température initiale.
            
            \ithand Plonger rapidement ce morceau de métal dans le calorimètre et suivre l'évolution de la température en homogénéisant régulièrement le mélange.
            
            \ithand Estimer $T_\text f$ la température finale dans le calorimètre lorsque l'équilibre thermique est atteint.
        \end{enumerate}
    \end{experience}
}
%
\nobefore\yesafter
%
\begin{expcom} \text{}
\begin{enumerate}
    \itt On verse $m_\text{eau} = \SI{640}{\g}$ d'eau, de température à l'équilibre $T_1 = \SI{17}{\celsius}$.
    
    \itt La masse du métal (qu'on identifie à de l'aluminium) est $m = \SI{130}{\g}$ et sa température $T_2 = \SI{85}{\celsius}$.
    
    \itt On atteint la température finale $T_f = \SI{20}{\celsius}$
\end{enumerate}
\end{expcom}
%
\yesbefore

\begin{enumerate}[resume]
    \item Montrer que la capacité thermique du métal est donnée par
    
    \begin{equation}
        C = \left(m_\text{eau} + \mu\right) c_\text{eau} \dfrac{T_\text f - T_1}{T_2 - T_f}
    \end{equation}
    
    \boxans{
        On a $\Delta H = 0$ et $\Delta H = \Delta_\text{calorimètre} + \Delta_\text{eau} + \Delta_\text{métal}$.
        %
        \begin{enumerate}
            \begin{multicols}{3}
                    \itt $\Delta H_\text{calorimètre} = \mu c_\text{eau}\left(T_\text f - T_1\right)$.
                    \itt $\Delta H_\text{eau} = m_\text{eau}c_\text{eau}\left(T_\text f - T_1\right)$.
                    \itt $\Delta H_\text{métal} = C\left(T_\text f - T_2\right)$.
            \end{multicols}

        \end{enumerate}
        %
        Donc $C\left(T_\text f - T_2\right) + \mu c_\text{eau}\left(T_f - T_1\right) + \Delta H_\text{métal} = C\left(T_f - T_2\right)$ d'où $C = \left(m_\text{eau} + \mu\right) c_\text{eau} \dfrac{T_\text f - T_1}{T_2 - T_f}$.
    }
    
    \item Faire l'application numérique et comparer avec la valeur attendue et déterminant les incertitudes-type.
    
    \boxans{
        L'application numérique avec l'expression ci-dessus $C = \SI{126}{\joule \cdot \kelvin^{-1}}$.
        
        La valeur attendue est $C = mc = \SI{117}{\joule \cdot \kelvin^{-1}}$.
        
        On obtient donc des résultats assez cohérents. Manque de temps pour le calcul des incertitudes et du z-score.
    }
\end{enumerate}

\end{document}