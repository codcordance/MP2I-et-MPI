\documentclass[a4paper,french,bookmarks]{article}
\usepackage{./Structure/4PE18TEXTB}

\newboxans

\begin{document}
\stylizeDoc{Physique}{Compte-rendu du TP 20}{Étude d'un filtre inconnu}

\boxhtp{
    \itstar Déterminer la nature d'un filtre inconnu
}{
    \itstar Un filtre inconnu, un oscilloscope, un GBF
}

Vous disposez sur votre paillasse d'un filtre inconnu inclus dans une \guill{boîte noire}, d'un oscilloscope et d'un GBF.

\noafter
\boxans{
    
    \begin{experience}{Diagramme de Bode}{}
        \begin{enumerate}
            \ithand Faire une analyse qualitative du filtre et déterminer son type : passe-bas, passe-haut, passe-bande, etc.
    
            \ithand Déterminer expérimentalement le gain maximal du filtre muni de son incertitude. Donner aussi la pulsation $\omega_\text{max}$ associée à ce gain (cette pulsation peut être égale à $0$ ou $+\infty$).
        
            \ithand Déterminer expérimentalement la ou les pulsations de coupure du filtre munies de leurs incertitudes.
        
            \ithand Tracer expérimentalement le diagramme de Bode du filtre. Resserrer les points autour des pulsations pertinentes.
        \end{enumerate}
    \end{experience}
    
    \begin{center}
        Appeler le professeur pour montrer le diagramme de Bode.
    \end{center}
}
%
\nobefore\yesafter
%
\begin{expcom}
    Il s'agit du filtre inconnu 3. On effectue le montage suivant :
    
    \begin{center}
	    \begin{tikzpicture}
		  \draw(0,2) to (4,2) to[R,l_=$\text{Filtre inconnu} \ $,v^<={$u_\text{s}$}] (4,0) node[cground,scale=0.8] {} ;
		  \draw (0,0) node[cground,scale=0.8] {} to[open,v^>={$u_\text{e}$}] (0,2) ;
	    \end{tikzpicture}
    \end{center}
    
    En faisant varier la fréquence $u_\text{e}$ du GBF, et en observant la fréquence $u_\text{s}$ de sortie sur l'oscilloscope, on comprend que le filtre est un passe-haut. On obtient le diagramme de Bode suivant :
    %
    \begin{center}
        \begin{minipage}{0.48\linewidth}
            \pgfplotsset{width=\textwidth}
            \begin{tikzpicture}
                \begin{axis}[
                    axis lines = left,
                    xlabel=$\mathsf{f \ \textsf{(Hz)}}$,
                    ylabel=$\mathsf{G}_\textsf{dB} \ \textsf{(dB)}$,
                    domain=0.08:12,
                    xmode=log,
                    log basis x={10},
                    xmin=10,
                    xmax=1000000,
                    ymin=-42,
                    ymax=1,
                    log ticks with fixed point,
                  xticklabel={$\mathsf{10^{\pgfmathprintnumber{\tick}}}$},
                    yticklabel={$\mathsf{\pgfmathprintnumber{\tick}}$},
                    font=\footnotesize,
                    grid = both,
                    grid style = {line width = .1pt, draw = gray!30},
                    major grid style = {line width=.2pt,draw=gray!50},
                ]
                    \addplot[mark=+, main1, line width=0.2mm, only marks] coordinates {(1, -39.82) (5, -36.30)  (10, -33.89) (15, -30.97) (20, -29.03) (30, -26.85) (50, -24.35) (100, -18.24) (200, -13.98) (300, -10.55) (400, -8.42) (500, -6.56) (600, -5.39) (700, -4.67) (800, -4.06) (900, -3.34) (1000, -2.87) (2000, -0.87) (3000, -0.59) (4000, -0.44) (5000, -0.15) (6000, -0.15) (7000, -0.15) (8000, -0.07) (9000, -0.07) (10000, -0.04) (100000, -0.15) (1000000, -0.22)};
                \end{axis}
            \end{tikzpicture}
        \end{minipage}
        %
        \hfill
        %
        \begin{minipage}{0.48\linewidth}
            \pgfplotsset{width=\textwidth}
            \begin{tikzpicture}
                \begin{axis}[
                    axis lines = left,
                    xlabel=$\mathsf{f} \ \textsf{(Hz)}$,
                    ylabel=$\mathsf{\varphi} \ \textsf{(}{}^{\circ}\textsf{)}$,
                    domain=0.08:12,
                    xmode=log,
                    log basis x={10},
                    xmin=10,
                    xmax=1000000,
                    ymin=0,
                    ymax=100,
                    log ticks with fixed point,
                  xticklabel={$\mathsf{10^{\pgfmathprintnumber{\tick}}}$},
                    yticklabel={$\mathsf{\pgfmathprintnumber{\tick}}$},
                    font=\footnotesize,
                    grid = both,
                    grid style = {line width = .1pt, draw = gray!30},
                    major grid style = {line width=.2pt,draw=gray!50},
                ]
                    \addplot[mark=+, main1, line width=0.2mm, only marks] coordinates {(10, 90) (15, 90) (20, 89) (30, 89) (50, 88) (100, 85) (200, 78) (300, 72) (400, 65) (500, 61) (600, 55) (700, 52) (800, 49) (900, 46) (1000, 43) (2000, 25) (3000, 16) (4000, 12) (5000, 9) (6000, 8) (7000, 7) (8000, 6) (9000, 5.5) (10000, 5) (100000, 0) (1000000, 0)};
                \end{axis}
            \end{tikzpicture}
        \end{minipage}
    \end{center}
\end{expcom}
%
\yesbefore

\begin{enumerate}
    \item Discuter du caractère intégrateur ou dérivateur du filtre en fonction de la fréquence.
    
    \boxans{
        On se place à \SI{100}{\hertz} et on envoie un signal triangle. On remarque alors qu'on obtient un signal créneau en sortie, ce qui montre le caractère dérivateur du filtre à basse fréquence.
    }

    \item Proposer un circuit possible pour votre filtre inconnu, sachant qu'il n'est composé que de dipôles linéaires d'ordre 1 (résistances, condensateurs, bobines).
    
    \boxans{
        \begin{minipage}{0.45\linewidth}
            Ce peut être un filtre $RL$ :
            
            \begin{tikzpicture}
		        \draw(0,2) to[R,l=$R$] (4,2) to[L,l_=$L$,v^<={$u_\text{s}$}] (4,0) node[cground,scale=0.8] {} ;
		        \draw (0,0) node[cground,scale=0.8] {} to[open,v^>={$u_\text{e}$}] (0,2) ;
	        \end{tikzpicture}
        \end{minipage}
        %
        \hfill
        %
        \begin{minipage}{0.45\linewidth}
            Ou un filtre $CR$ :
            
            \begin{tikzpicture}
		        \draw(0,2) to[C,l=$C$] (4,2) to[R,l_=$R$,v^<={$u_\text{s}$}] (4,0) node[cground,scale=0.8] {} ;
		        \draw (0,0) node[cground,scale=0.8] {} to[open,v^>={$u_\text{e}$}] (0,2) ;
	        \end{tikzpicture}
        \end{minipage}
        
        \hfill\\
        Après vérification, il s'agit d'un filtre $CR$.
    }
\end{enumerate}


\end{document}