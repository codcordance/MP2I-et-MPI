\documentclass[a4paper,french,bookmarks]{article}
\usepackage{./Structure/4PE18TEXTB}

\begin{document}

\stylizeDoc{Français}{L'enfance}{Dissertation n° 2}


\section*{\centering\EBGaramond\Large\itshape Dissertation n° 2}

\qquad Le biologiste et psychologue Jean Piaget s'est employé à décrire les stades de développement de l'enfant et en a fait un sujet d'étude à part entière, tant la psychologie infantile, celle d'un individu qui ne peut normalement communiquer, s'avère ardue à saisir. Le philosophe Roger-Pol Droit donne la définition suivante : \guill{ce qui définit l'enfance, c'est le silence – le terme enfance dit, y insiste, ne cesse de le rappeler, quand bien même nous ne l'entendons plus. L'enfance est de l'autre côté des mots, hors du langage. Dans le mutisme, l'incapacité de parler.} Le penseur fait ici référence à l'étymologie latine du mot, \guill{infans}, qui signifie incapable de parler, sans éloquence. Le vocable \guill{enfant} est ainsi en lui-même porteur d'un sens que l'usage fait oublier (\guill{nous ne l'entendons plus}), mais qui, pour Droit, est l'essence même de celui qu'il décrit : le \guill{mutisme}, cette existence silencieuse \guill{hors du langage}, voilà \guill{ce qui définit l'enfance}. Le langage devient alors un mode d'expression exclusif à celui qui n'est pas enfant - l'adulte - caractérisé par l'usage des mots. Et pourtant, hors du langage parlé car \guill{de l'autre côté des mots}, l'enfant est-il dépourvu de toute forme de langage, de communication ? Ne possède-t-il pas, au-delà de ce mutisme essentiel, un mode d'expression qui lui est propre ? De plus, il est question d'une césure nette entre enfant et adulte suggérée par la maîtrise du langage parlé, mais l'aptitude à user des mots marque-t-elle nettement la sortie de l'enfance et l'entrée dans l'existence adulte ? 

\qquad Pour répondre à ces interrogations, nous montrerons dans un premier temps que l'existence infantile est caractérisée par l'incapacité à user des mots. Néanmoins nous verrons, dans un deuxième temps, que l'enfant possède tout de même certains modes de communication qui lui sont propre, presque sa propre grammaire. Enfin, on montrera dans un dernier temps qu'enfant et adulte diffèrent dans leur rapport au langage, non seulement par l'usage du lexique et de la syntaxe, mais aussi par celui de la sémantique, où réside leur compréhension des concepts, leur vision du monde. En vue de présenter des exemples concrets durant le développement, nous nous appuierons sur \underline{\itshape L'Émile} de Jean-Jacques Rousseau, sur les \underline{\itshape Contes} d'Hans Christian Andersen, ainsi que sur \underline{\itshape Aké, les années d'enfance} de Wole Soyinka.  

\text{}\\[20pt]

\qquad Si le langage est pour nous autres adultes aujourd'hui une évidence, et que même dans nos esprits nous pensons avec des mots, il ne faut oublier que l'on vient au monde sans cela. Il nous a fallu un jour apprendre à lire les mots, et même avant à les prononcer. L'enfant ne maîtrise donc pas, initialement, le langage parlé, et les mots qui lui sont inhérents. Dans le premier livre de \underline{\itshape L'Émile}, Jean-Jacques Rousseau le précise bien : \guill{Toutes nos langues sont des ouvrages de l'art}. Le philosophe explique d'ailleurs bien que les mots que prononcent les nourrices aux enfants \guill{sont parfaitement inutiles}, que ces derniers ne peuvent en \guill{entendre le sens}. Mais Rousseau va plus loin, et explique même que la seule manière de communiquer dont dispose l'enfant est d'afficher ses émotions : les grimaces, les sourires qui passent sur son visage, expriment ses sentiments et besoins aux adultes qui l'entourent. Ce n'est pourtant pas là une volonté consciente de communiquer, tout cela se faisant de manière inconsciente et naturelle chez l'enfant. L'auteur de \underline{\itshape L'Émile} s'attarde particulièrement sur un élément, celui des pleurs. Ces derniers sont particulièrement utiles à l'enfant pour faire connaître ses besoins, car avec eux il importune les adultes autour jusqu'à satisfaction. Ils ne sont toutefois pas le remède le plus efficace : quand il a faim, soif, qu'il veut dormir ou qu'il est sujet à tout autre mal, l'enfant pleure, et c'est donc bien une tare pour les adultes autour de savoir de quoi exactement il souffre. C'est en fait l'unique remède de l'enfant, et nul doute ne fait que s'il avait les mots, il emploierait plutôt ces derniers pour décrire ses besoins. Les pleurs sont en cela symptomatiques de la non connaissance du langage chez l'enfant. Dans l'ouvrage de Wole Soyinka, c'est le cas de Folasade, la petite sœur de la famille, qui un jour \guill{se mit à crier […] tenant toute la maison en éveil}, sans que la famille ne puisse rien y faire. L'élément des larmes est aussi présent dans les \underline{\itshape Contes} d'Andersen, notamment avec \textit{La Petite Sirène}, où le personnage du conte éponyme ne pourra communiquer par les pleurs bien qu'elle le veuille, cela étant une dernière voie d'expression naturelle ayant perdu sa faculté de parler, car \guill{une sirène n'a point de larme}.  

\qquad L'existence de l'enfant par-delà les mots se voit également par l'incompréhension des adultes qui interagissent avec lui. Puisque l'enfant existe par définition \guill{hors} du langage, et qu'est adulte ce qui n'est pas enfant, l'existence \guill{dans} le langage devient une condition \textit{sine qua non} de l'adulte. Ce dernier ne parvient alors à saisir l'enfant, et inversement : les deux sont dans l'incompréhension de l'autre. La sœur de Wole par exemple, ne peut témoigner par les mots de la cause de son mal, et sa mort entraîne l'incompréhension de ceux qui pensent avec les mots. Rousseau donne sinon l'exemple d'une nourrice qui, devant un enfant qui pleure et qu'elle ne parvient à faire taire (par incompréhension de son mal qu'il ne peut exprimer par les mots) le frappe.  

\qquad On voit donc apparaître, dans la question du langage, l'opposition entre l'adulte et l'enfant : le deuxième ne peut interagir avec le premier que grâce aux grimaces, aux sourires, aux pleurs, des stratégies qui lui permettent de satisfaire une partie de ces besoins mais qui ne sont pas parfaites, laissant donc un certain fossé d'incompréhension entre les deux, qui ne peut être comblé que l'apprentissage des mots. L'enfant est donc caractérisé par sa non-maîtrise du langage parlé.

\text{}\\[20pt]

\qquad Néanmoins, est-ce là dire que l'enfant est totalement dépourvu de langage ? Pour être précis, nous avons surtout montré que l'enfant se caractérise par sa non-maîtrise du langage des adultes, dit \guill{langage-parlé}, mais est-ce là tout ce qu'on appelle langage ? On a justement montré que l'enfant existe dans la non-perfection de sa communication avec l'adulte. Mais ceci n'implique-t-il pas que l'enfant possède justement un moyen de communication qui lui est propre ? L'enfant qui pleure n'emploie-t-il justement pas certains codes et modes d'expression ? Lorsque Rousseau écrit dans le livre I de \underline{\itshape L'Émile} : \guill{Le malheureux suffoquait de colère, je le vis devenir violet […] tous les signes du ressentiment étaient dans ses accents}, alors qu'il décrit un nourrisson frappé par sa nourrice irritée par ses cris, il apparaît que l'enfant est parcouru d'un sentiment et l'exprime avec ses moyens. Le corps, en plus des cris et des vocalises variées, occupe ainsi une grande place dans la compréhension et l'expression de l'enfant avant qu'il ne parle. C'est ainsi que l'enfant, d'abord impotent, incapable de subvenir à ses besoins et même à sa propre curiosité, émet très tôt le désir de s'approcher d'un tel objet et gesticulant, ayant conscience que ses gestes sont compris par son entourage. Comprenant que les enfants peuvent aisément glisser vers la tyrannie, commandant par les cris et les gestes à ceux qui les soignent, Rousseau préconise à ce titre d'être très attentif à l'attitude de l'enfant et aux moindres signes d'un tel caprice. \guill{Mais quand il se plaint et crie en tendant la main, il commande à l'objet de s'approcher, ou à vous de le lui apporter}.  Ainsi, loin de lancer ses balbutiements de manière hasardeuse, ou dans l'unique cas d'une douleur, implorant un secours à qui puisse l'entendre, l'enfant a en conscience le potentiel destinataire, et sait adapter son expression, ses tons, ses gestes, comme le serait une discussion ponctuée, et dont les mots seraient choisis, ce qui saurait constituer une forme de langage. L'auteur de \underline{\itshape De l'éducation} écrit même à ce sujet \guill{il existe un premier langage commun aux adultes et aux enfants} que l'usage des mots a fait oublier, mais que les nourrices comprennent, ce qui leur permet de communiquer avec les jeunes enfants, par leurs expressions et les tons. 

\qquad Mais si l'enfant comprend aussi finement les modalités de son environnement encore à un jeune âge, c'est encore par sa sensibilité émotionnelle et sa capacité à s'imprégner et à s'attacher eux personnes qui l'entourent. Le personnage de Wole, dans \underline{\itshape Aké, les années d'enfance}, bien qu'il soit jugé sans doute trop jeune pour endurer un discours si triste annonçant la mort de sa sœur, comprend de lui-même quasi-instantanément la situation. \guill{Mais qu'est-ce qu'il peut comprendre ?} répète alors sa mère, stupéfaite de la sensibilité émotionnelle de son fils, lui permettant de saisir la situation par le seul contexte et la lourdeur sentimentale. Il est également question de nécessité, alors que l'enfant se développe, qu'il puisse s'attacher et trouver des piliers émotionnels au sein des membres de sa famille. C'est ainsi que Rousseau donne \guill{Point de mère, point d'enfant}, et insiste sur le fait qu'il est capital que le jeune sache ce qu'est l'appréciation pour sa mère avant que l'on ne lui apprenne la politesse, et l'obligation qu'il a d'aimer ses parents par bienséance. Ainsi l'enfant possède un sens de l'émotion qu'il développe par la nécessité de reconnaître son entourage, et qui lui sert à percevoir au-delà des mots qui ne lui sont pas dits, ou qu'il ne peut comprendre. Cette prévalence de la sensation dans le monde de l'enfant et dans sa compréhension du monde transparaît dans \textit{La Petite Fille aux Allumettes} d'Andersen, où l'allumette \guill{qui brûla, qui brilla} inspire à la fillette \guill{une oie rôtie} ou \guill{milles bougies sur les branches vertes}, la narration s'exclame du point de vue interne \guill{Le feu y brûlait si magnifique, il chauffait si bien !}, formant un éclatant contraste avec le froid et la blancheur de l'hiver. Les émotions et les images qu'inspirent une simple allumette semble un monde à part pour la jeune enfant, retranscrit par un lexique des sensations et de vives expressions.  

\qquad Par ailleurs, cette sensibilité accrue ne confère pas à l'enfant qu'une simple faculté d'effleurer les concepts et les relations simples, instantanées. L'auteur de \underline{\itshape L'Émile} fait mention d'une conception que se fait le jeune enfant de l'injustice, et des rapports de domination intra-personnels qui lui sont connus par les relations que les adultes entretiennent avec lui. Des notions aussi complexes ne sont pas consciemment assimilées par l'enfant, mais sont ressenties dans son comportement et dans ses stratégies de communication, ce qui porte Rousseau à formuler des mises en garde. Pour le penseur, \guill{les premiers cris d'un enfant sont des implorations, mais si l'on prend trop de peine à les écouter, ils deviennent des injonctions}. Ainsi l'enfant et est capable, bien que dénué de langage adulte, d'élaborer par l'expérience des stratégies comportementales pour prendre l'axe dominant de la relation, ce qui peut permettre d'affirmer l'existence d'un biais efficace de compréhension de son entourage chez l'enfant, différent de celui de l'adulte qui réfléchit avec les mots. 

\qquad L'enfant est donc doté d'un mode d'expression qui lui est propre, passant par les expressions du corps et une variété de vocalises, qui forment le lexique, la grammaire de ce \guill{premier langage} comme l'appelle Rousseau. Mais il possède un mode de réflexion propre, où l'approche sentimentale et empirique lui fournit suffisamment d'informations sur son entourage pour adapter son comportement et intégrer des notions complexes.  

\text{}\\[20pt]

\qquad Toutefois, la précédente polarisation entre enfant et adulte supposant la maîtrise ou non du langage parlé tend à supposer que les mots sont la principale sinon l'unique différence entre eux. Mais n'existe-t-il pas d'autres atypismes concernant l'enfant, si ce n'est son incapacité à user des mots, du lexique ? Ayant montré que l'enfant peut partiellement compenser le manque des mots par d'autres biais d'expression, dans une optique purement fonctionnelle concernant le contenant du langage, il s'agirait maintenant de déterminer s'il possède sa propre approche des concepts et de la sémantique, soit du contenu du langage. En un sens assez léger, les enfants qui s'approprient de nouveaux termes présentent de nombreuses fantaisies, que les adultes prennent le loisir de corriger ou dont ils s'amusent. Wole Soyinka, personnage central d'\underline{\itshape Aké, les années d'enfance}, associe de manière surprenante les objets qui l'entourent, à savoir le Chanoine du village, dont l'apparence massive lui fait penser \guill{C'est à cause de sa tête, elle ressemblait à un boulet de canon. C'est pour cela que Papa l'appelait canon}. Ces rapprochements fréquents chez l'enfant sont pourtant sérieux chez lui, et semblent faire part intégrante de son appropriation des notions, qu'il relie par quelque motif afin de cartographier le monde qui l'entoure.  

\qquad En revanche, bien que cet atypisme puisse être relié à un avancement de l'enfant dans sa compréhension du monde, d'autres caractéristiques peuvent découler d'une réelle incapacité à raisonner, à saisir le sens d'une expression renvoyant à un concept que seul un avancement suffisant en âge peut en permettre l'assimilation chez le jeune. Il est question chez Rousseau, comme chez d'autres penseurs de l'éducation et de la formation des jeunes, d'un âge dit de raison, ou de conscience, où l'enfant est suffisamment mur pour discuter de sujets plus abstraits, ou philosophiques, demandant une conscientisation de notions détachées du monde concret. Mais avant cet âge approximatif et avant tout théorique dans son stade de développement, l'enfant semble hermétique à certaines notions dont certains exemples sont illustrés dans \underline{\itshape L'Émile}. Ainsi les enfants qui découvrent les Fables de La Fontaine - Rousseau prend l'exemple de la fable sur \textit{Le Corbeau et le Renard} - n'en saisissent pas le sens profond et \guill{n'aimant point à s'humilier, ils prennent toujours le beau rôle ; c'est le choix très naturel de l'amour propre}. La morale fait ainsi partie d'un socle de notions que l'enfant ne peut pas même approcher avant un certain âge, en ce qu'elle implique une structure de réflexion détachée de soi qui n'est pas propre à la jeunesse. De même, Rousseau met en scène un échange caricatural entre Le Maître et l'Enfant, sommant le précepteur d'épargner à leurs élèves, dont la raison n'est pas adaptée à l'exercice, toute discussion d'ordre moral, \guill{un cercle inévitable. Sortez-en, l'enfant ne vous entend plus. Connaître le bien et le mal n'est pas l'affaire de l'enfant}. 

\qquad Cette incapacité de l'enfant, contrairement à l'adulte, de porter son esprit, pourtant à cet âge habitué à l'usage des mots, à des questions abstraites ou détachées d'un vécu propre, va de pair avec un manque d'expérience et une forme d'extravagance dans la manière dont il se figure les relations entre les concepts connus. Il en est question dans \underline{\itshape Aké, les années d'enfance}, où le jeune Soyinka se questionne et trépigne d'inquiétude à l'idée que sa blessure à la tête amoindrisse ses capacités intellectuelles et le force à redoubler. Il va jusqu'à demander à ses parents s'ils ont \guill{remis le sang de son dansiki dans sa tête}, ce à quoi son père, comprenant, amusé, les rapprochements de son jeune fils, répond sérieusement à l'affirmative. Sur ce cas, le manque d'érudition du jeune Wole est la cause de sa représentation erronée de la situation, mais l'imagination et la capacité à s'abstraire de la réalité peut conduire chez l'enfant à un travestissement similaire de la réalité. L'onirisme est présent dans l'univers des \underline{\itshape Contes} d'Andersen, et \textit{La Petite Fille aux Allumettes}, mourant de froid une nuit d'hiver, est décrite comme \guill{Ayant les joues rouges, et le sourire aux lèvres} dans le glaçant parallèle du spectacle de son cadavre observé par les passants. L'image seule du feu de l'allumette, alliée au souvenir et à la force de l'attachement, ici à sa grand-mère, ont permis à l'enfant d'outre-passer une réalité pourtant aux prises avec la souffrance, chose que les adultes, incarnés dans le conte par les passants peinent à comprendre. \guill{Personne ne sut quelles belles choses elle avait vues et au milieu de quelle splendeur elle était entrée}, indique la narration. \guill{Elle a voulue se réchauffer}, souffle un passant dont l'esprit adulte, bien plus ancré dans le pragmatisme et la rationalité, cherche une explication, un scénario éloigné de la spiritualité réelle dont a fait preuve la petite fille.

\qquad Ainsi, l'enfant envisage le monde d'une manière radicalement différente de l'adulte, en ce que son esprit modelable et en cours de développement tisse des liens entre des concepts peu récents pour lui, laissant une grande place au rêve et à l'imagination qui transcende un réel pragmatique si cher à la vision adulte. Sa différence avec l'adulte s'opère dans le langage, non seulement par le contenant (les mots, le lexique) mais aussi par le contenu (le sens, la sémantique). 

\text{}\\[20pt]

\qquad En conclusion, l'étude des œuvres a mené à confirmer l'existence d'une différence notable entre les modes d'expression des adultes, le langage parlé usant des mots, et celui dit des enfants, qui en est dépourvu. De cette différence naît un clivage impliquant des difficultés notables de communication entre enfant et adultes, ce qui peut expliquer l'expression de Pol Droit, à savoir que le silence et le mutisme définissaient l'enfance. De nombreux exemples ont pu montrer l'existence d'une réelle volonté chez l'enfant de communiquer avec l'extérieur par une voie non verbale, usant de tout ce qu'il peut : pleurs, cris, autres vocalises et gesticulations aux significations variées. De plus, cette approche infantile de la communication permet à l'enfant de saisir une quantité non négligeable d'informations, ayant jusqu'à comprendre les subtilités des relations intra-personnelles qui le concerne et à internaliser des concepts complexes. C'est là tout un autre lexique, toute une autre grammaire qui vient caractériser l'enfant. On comprend que le fossé entre l'enfant et l'adulte, qui permet la distinction ontologique entre ces deux être, réside dans la langue. Il s'avère d'ailleurs qu'enfants et adultes diffèrent par autre choses que l'usage des mots, et donc que le silence de Pol Droit qui définit l'enfance, revêt un sens linguistique encore plus grand. En effet, le fossé du langage entre adultes et enfants, déjà creusé par le lexique, est renforcé par la sémantique : ces derniers découvrant les concepts et apprivoisant un monde qu'ils connaissent mal hors d'eux-mêmes. Leurs lacunes concernant ces notions peuvent mener à un deuxième niveau d'incompréhension entre enfants et adultes, voire mener l'enfant à réserver une grande part à la rêverie et à l'imagination dans sa vision du monde, ce qui échappe à l'adulte, plus pragmatique et enraciné dans le réel de ses connaissances. Ce mode de communication de l'enfant, non-verbal avec les adultes qui l'entourent, de sens et de concepts particulier avec lui-même, est bien \guill{hors du langage} comme le dit Roger-Pol Droit. Assimilable pour les adultes à un mutisme, car sans les mots, c'est bien lui qui fait, essentiellement, l'enfant.

\end{document}