\documentclass[a4paper,french,bookmarks]{article}
\usepackage{./Structure/4PE18TEXTB}

\renewcommand{\thesection}{\Roman{section}}
\newboxans

\begin{document}

\stylizeDoc{Français}{Cours, deuxième ouvrage}{Contes, H.C. Andersen}

\section{Ttitre de la section}

L'étude de l'enfance : 
\hfill
\begin{enumerate}
	\itt Sélection de la corde à accorder (donc $f_\text{ac}$ est fixée).
	
	\itt Création d'un signal carré de référence de fréquence $f_\text{ac}$.
	
	\itt Enregistrement du signal $u_\text{e}(t)$ provenant de l'excitation de
	la corde à accorder : signal quelconque, d'amplitude assez faible,
	de fréquence $f_\text{co}$.
	
	\itt Amplification et filtrage de ce signal.
	
	\itt Extraction de la fondamentale du signal : obtention d'un signal
	sinusoïdal de fréquence $f_\text{co}$ par l'utilisation d'un filtre à
	fréquence caractéristique réglable par le signal extérieur de
	référence.
	
	\itt Mise en forme de ce signal : obtention d'un signal carré de
	fréquence $f_\text{co}$.
	
	\itt On a donc à disposition deux signaux carrés (signaux logiques)
	de fréquences respectives $f_\text{ac}$ et $f_\text{co}$. Dans les accordeurs
	récents le traitement est numérique : les signaux sont envoyés dans un
	calculateur numérique intégré qui calcule l'écart de fréquence et
	indique à l'utilisateur quand la corde est accordée, c'est-à-dire
	quand $f_\text{co}=f_\text{ac}$.
\end{enumerate}

\end{document}